
\documentclass{sebase}
%%%%%%%%%%%%%%%%%%%%%%%%%%%%%%%%%%%%%%%%%%%%%%%%%%%%%%%%%%%%%%%%%%%%%%%%%%%%%%%%%%%%%%%%%%%%%%%%%%%%%%%%%%%%%%%%%%%%%%%%%%%%%%%%%%%%%%%%%%%%%%%%%%%%%%%%%%%%%%%%%%%%%%%%%%%%%%%%%%%%%%%%%%%%%%%%%%%%%%%%%%%%%%%%%%%%%%%%%%%%%%%%%%%%%%%%%%%%%%%%%%%%%%%%%%%%
\usepackage{amssymb}
\usepackage{amsmath}
\usepackage{float}
\usepackage{makeidx}
\usepackage{SECALCUL}
\usepackage{lie}

\setcounter{MaxMatrixCols}{10}
%TCIDATA{OutputFilter=LATEX.DLL}
%TCIDATA{Version=5.50.0.2953}
%TCIDATA{<META NAME="SaveForMode" CONTENT="1">}
%TCIDATA{BibliographyScheme=Manual}
%TCIDATA{Created=Mon Dec 15 16:20:00 1997}
%TCIDATA{LastRevised=Tuesday, May 05, 2009 10:43:56}
%TCIDATA{<META NAME="ViewSettings" CONTENT="25">}
%TCIDATA{<META NAME="GraphicsSave" CONTENT="32">}
%TCIDATA{CSTFile=SECALCUL.cst}

\input tcilatex
\newenvironment{instructions}{\STARTINSTR}{\ENDINSTR}
\begin{document}


\chapter[2\quad Limits and Derivatives]{}

\section{2.1\quad The Tangent and Velocity Problems}

%TCIMACRO{\TeXButton{setET}{\renewcommand{\ET}{1}}}%
%BeginExpansion
\renewcommand{\ET}{1}%
%EndExpansion
%TCIMACRO{\TeXButton{noCCC}{\noCCC}}%
%BeginExpansion
\noCCC%
%EndExpansion
%TCIMACRO{\TeXButton{setRM}{\renewcommand{\RM}{1}}}%
%BeginExpansion
\renewcommand{\RM}{1}%
%EndExpansion
%TCIMACRO{\TeXButton{noRM}{\renewcommand{\RM}{0}}}%
%BeginExpansion
\renewcommand{\RM}{0}%
%EndExpansion
%TCIMACRO{\TeXButton{set page2}{\setcounter{page}{2}}}%
%BeginExpansion
\setcounter{page}{2}%
%EndExpansion

In this section we see how limits arise when we attempt to find the tangent
to a curve or the velocity of an object.

\subsection{THE TANGENT PROBLEM}

\marginpar{\hspace*{\fill}\FRAME{itbpFU}{134.5pt}{194.9375pt}{201.5pt}{\Qcb{%
\QTR{FigureNumber}{FIGURE 1}}}{}{6et020101m_00190.ai}{\special{language
"Scientific Word";type "GRAPHIC";maintain-aspect-ratio TRUE;display
"USEDEF";valid_file "F";width 134.5pt;height 194.9375pt;depth
201.5pt;original-width 1.8311in;original-height 2.6751in;cropleft
"0";croptop "0.9981";cropright "1.0015";cropbottom "0";filename
'graphics/6et020101m_00190.ai';file-properties "XNPEU";}}\hspace*{\fill}}The
word \textit{tangent} is derived from the Latin word\textit{\ tangens},
which means \textquotedblleft touching.\textquotedblright\ Thus a tangent to
a curve is a line that touches the curve. In other words, a tangent line
should have the same direction as the curve at the point of contact. How can
this idea be made precise?

For a circle we could simply follow Euclid and say that a tangent is a line
that intersects the circle once and only once, as in Figure 1(a). For more
complicated curves this definition is inadequate. Figure l(b) shows two
lines $l$ and $t$ passing through a point $P$ on a curve $C$. The line $l$
intersects $C$ only once, but it certainly does not look like what we think
of as a tangent. The line $t$, on the other hand, looks like a tangent but
it intersects $C$ twice.

To be specific, let's look at the problem of trying to find a tangent line $%
t $ to the parabola $y=x^{2}$ in the following example.%
%TCIMACRO{\TeXButton{longpage}{\enlargethispage{\baselineskip}}}%
%BeginExpansion
\enlargethispage{\baselineskip}%
%EndExpansion

\begin{Example}[1]
%TCIMACRO{\TeXButton{VIDEO}{\VIDEO}}%
%BeginExpansion
\VIDEO%
%EndExpansion
%TCIMACRO{%
%\hyperref{\fbox{{\footnotesize Video 5et 020101 / 019}}\quad }{}{\fbox{{\footnotesize Video 5et 020101 / 019}}\quad }{}}%
%BeginExpansion
\msihyperref{\fbox{{\footnotesize Video 5et 020101 / 019}}\quad }{}{\fbox{{\footnotesize Video 5et 020101 / 019}}\quad }{}%
%EndExpansion
Find an equation of the tangent line to the parabola $y=x^{2}$ at the point $%
P(1,1)$.
\end{Example}

\begin{Solution}
We will be able to find an equation of the tangent line $t$ as soon as we
know its slope $m$. The difficulty is that we know only one point, $P$, on $%
t $, whereas we need two points to compute the slope. But observe that we
can compute an approximation to $m$ by choosing a nearby point $Q(x,x^{2})$
on the parabola (as in Figure 2) 
\marginpar{\hspace*{\fill}\FRAME{itbpFU}{124.75pt}{95.8125pt}{87.625pt}{\Qcb{%
\QTR{FigureNumber}{FIGURE 2}}}{}{4e020102.wmf}{\special{language "Scientific
Word";type "GRAPHIC";maintain-aspect-ratio TRUE;display "USEDEF";valid_file
"F";width 124.75pt;height 95.8125pt;depth 87.625pt;original-width
1.7365in;original-height 1.3223in;cropleft "0";croptop "1.0021";cropright
"1.0024";cropbottom "0";filename 'graphics/4e020102.wmf';file-properties
"XNPEU";}}\hspace*{\fill}}and computing the slope $m_{PQ}$ of the secant
line $PQ$. [A \textbf{secant line}, from the Latin word \textit{secans},
meaning cutting, is a line that cuts (intersects) a curve more than once.]

We choose $x\neq 1$ so that $Q\neq P$. Then\\[6pt]
\hspace*{\fill}$m_{PQ}=\dfrac{x^{2}-1}{x-1}$\hspace*{\fill}\\[6pt]
For instance, for the point $Q(1.5,2.25)$ we have\\[6pt]
\hspace*{\fill}$m_{PQ}=\dfrac{2.25-1}{1.5-1}=\dfrac{1.25}{0.5}=2.5$\hspace*{%
\fill}\\[6pt]
\marginpar{\hspace*{\fill}$%
\begin{tabular}[t]{|c|c|}
\hline
$x$ & $m_{PQ}\rule[-5pt]{0pt}{12.24pt}$ \\ \hline
\multicolumn{1}{|l|}{$2$} & \multicolumn{1}{|l|}{$3$} \\ 
\multicolumn{1}{|l|}{$1.5$} & \multicolumn{1}{|l|}{$2.5$} \\ 
\multicolumn{1}{|l|}{$1.1$} & \multicolumn{1}{|l|}{$2.1$} \\ 
\multicolumn{1}{|l|}{$1.01$} & \multicolumn{1}{|l|}{$2.01$} \\ 
\multicolumn{1}{|l|}{$1.001$} & \multicolumn{1}{|l|}{$2.001$} \\ \hline
\end{tabular}%
$\hspace*{\fill}}The tables in the margin show the values of $m_{PQ}$ for
several values of $x$ close to 1. The closer $Q$ is to $P$, the closer $x$
is to 1 and, it appears from the tables, the closer $m_{PQ}$ is to~2. This
suggests that the slope of the tangent line $t$ should be $m=2$.

We say that the slope of the tangent line is the \textit{limit} of the
slopes of the secant lines, and we express this symbolically by
writing\bigskip

$\hfill \hspace{13pt}\lim\limits_{Q\rightarrow P}m_{PQ}=m\qquad $and$\qquad
\lim\limits_{x\rightarrow 1}\dfrac{x^{2}-1}{x-1}=2\hfill $\bigskip

\marginpar{\hspace*{\fill}$%
\begin{tabular}[t]{|c|c|}
\hline
$x$ & $m_{PQ}\rule[-5pt]{0pt}{12.24pt}$ \\ \hline
\multicolumn{1}{|l|}{$0$} & \multicolumn{1}{|l|}{$1$} \\ 
\multicolumn{1}{|l|}{$0.5$} & \multicolumn{1}{|l|}{$1.5$} \\ 
\multicolumn{1}{|l|}{$0.9$} & \multicolumn{1}{|l|}{$1.9$} \\ 
\multicolumn{1}{|l|}{$0.99$} & \multicolumn{1}{|l|}{$1.99$} \\ 
\multicolumn{1}{|l|}{$0.999$} & \multicolumn{1}{|l|}{$1.999$} \\ \hline
\end{tabular}%
$\hspace*{\fill}}Assuming that the slope of the tangent line is indeed 2, we
use the point-slope form of the equation of a line (see Appendix B) to write
the equation of the tangent line through $(1,1)$ as 
\begin{equation*}
y-1=2(x-1)\text{ \qquad or \qquad }y=2x-1
\end{equation*}

Figure 3 illustrates the limiting process that occurs in this example. As $Q$
approaches $P$ along the parabola, the corresponding secant lines rotate
about $P$ and approach the tangent line $t$.

%TCIMACRO{\TeXButton{graphicS}{\vspace{12pt}\hskip-190pt\hfil}}%
%BeginExpansion
\vspace{12pt}\hskip-190pt\hfil%
%EndExpansion
\FRAME{itbpFU}{6.7507in}{3.8647in}{0in}{\Qcb{\QTR{FigureNumber}{FIGURE 3}}}{%
}{6et020103_00192.ai}{\special{language "Scientific Word";type
"GRAPHIC";maintain-aspect-ratio TRUE;display "USEDEF";valid_file "F";width
6.7507in;height 3.8647in;depth 0in;original-width 6.7233in;original-height
3.8372in;cropleft "0";croptop "1";cropright "1";cropbottom "0";filename
'graphics/6et020103_00192.ai';file-properties "XNPEU";}}%
%TCIMACRO{\TeXButton{graphicE}{\vspace{12pt}\hfil}}%
%BeginExpansion
\vspace{12pt}\hfil%
%EndExpansion
\vspace{-24pt}

$\hfill \vspace{-12pt}\blacksquare $
\end{Solution}

\marginpar{\FRAME{itbpF}{0.3347in}{0.1531in}{0in}{}{}{tec.ai}{\special%
{language "Scientific Word";type "GRAPHIC";maintain-aspect-ratio
TRUE;display "USEDEF";valid_file "F";width 0.3347in;height 0.1531in;depth
0in;original-width 0.3053in;original-height 0.1254in;cropleft "0";croptop
"1";cropright "1";cropbottom "0";filename 'graphics/TEC.ai';file-properties
"XNPEU";}}\hspace{3pt}\QTR{Author}{In Visual 2.1 you can see how the process
in Figure 3 works for additional functions.}}Many functions that occur in
science are not described by explicit equations; they are defined by
experimental data. The next example shows how to estimate the slope of the
tangent line to the graph of such a function.

\begin{Example}[2]
%TCIMACRO{\TeXButton{VIDEO}{\VIDEO}}%
%BeginExpansion
\VIDEO%
%EndExpansion
%TCIMACRO{%
%\hyperref{\fbox{{\footnotesize Video 5et 020102 / 020}}\quad }{}{\fbox{{\footnotesize Video 5et 020102 / 020}}\quad }{}}%
%BeginExpansion
\msihyperref{\fbox{{\footnotesize Video 5et 020102 / 020}}\quad }{}{\fbox{{\footnotesize Video 5et 020102 / 020}}\quad }{}%
%EndExpansion
The flash unit on a camera operates by storing charge on a capacitor and
releasing it suddenly when the flash is set off. The data in the table
describe the charge \textit{Q} remaining on the capacitor (measured in
microcoulombs) at time $t$ (measured in seconds after the flash goes off).
Use the data to draw the graph of this function and estimate the slope of
the tangent%
\marginpar{
\qquad \qquad $%
\begin{tabular}{|c|c|}
\hline
$t$ & $Q$ \\ \hline
\multicolumn{1}{|r|}{$0.00$} & \multicolumn{1}{|r|}{$100.00$} \\ 
\multicolumn{1}{|r|}{$0.02$} & \multicolumn{1}{|r|}{$81.87$} \\ 
\multicolumn{1}{|r|}{$0.04$} & \multicolumn{1}{|r|}{$67.03$} \\ 
\multicolumn{1}{|r|}{$0.06$} & \multicolumn{1}{|r|}{$54.88$} \\ 
\multicolumn{1}{|r|}{$0.08$} & \multicolumn{1}{|r|}{$44.93$} \\ 
\multicolumn{1}{|r|}{$0.10$} & \multicolumn{1}{|r|}{$36.76$} \\ \hline
\end{tabular}%
\ $} line at the point where $t=0.04$. [\textit{Note:\ }The slope of the
tangent line represents the electric current flowing from the capacitor to
the flash bulb (measured in microamperes).]
\end{Example}

\begin{Solution}
In Figure 4 we plot the given data and use them to sketch a curve that
approximates the graph of the function.\\[8pt]
\hspace*{\fill}\FRAME{itbpFU}{244.9375pt}{132.125pt}{0pt}{\Qcb{%
\QTR{FigureNumber}{FIGURE 4}}}{}{6et020104_00193.ai}{\special{language
"Scientific Word";type "GRAPHIC";maintain-aspect-ratio TRUE;display
"USEDEF";valid_file "F";width 244.9375pt;height 132.125pt;depth
0pt;original-width 3.3612in;original-height 1.8011in;cropleft "0";croptop
"1";cropright "1";cropbottom "0";filename
'graphics/6et020104_00193.ai';file-properties "XNPEU";}}\hspace*{\fill}%
\pagebreak
\end{Solution}

Given the points $P(0.04,67.03)$ and $R(0.00,100.00)$ on the graph, we find
that the slope of the secant line $PR$ is\\[5pt]
\hspace*{\fill}$m_{PR}=\dfrac{100.00-67.03}{0.00-0.04}=-824.25$\hspace*{\fill%
}\\[5pt]
\marginpar{\hspace*{\fill}$%
\begin{tabular}[t]{|c|c|}
\hline
$R$ & $m_{PR}\rule[-5pt]{0pt}{12.24pt}$ \\ \hline
\multicolumn{1}{|l|}{$\left( 0.00,100.00\right) $} & \multicolumn{1}{|l|}{$%
-824.25$} \\ 
\multicolumn{1}{|l|}{$\left( 0.02,81.87\right) $} & \multicolumn{1}{|l|}{$%
-742.00$} \\ 
\multicolumn{1}{|l|}{$\left( 0.06,54.88\right) $} & \multicolumn{1}{|l|}{$%
-607.50$} \\ 
\multicolumn{1}{|l|}{$\left( 0.08,44.93\right) $} & \multicolumn{1}{|l|}{$%
-552.50$} \\ 
\multicolumn{1}{|l|}{$\left( 0.10,36.76\right) $} & \multicolumn{1}{|l|}{$%
-504.50$} \\ \hline
\end{tabular}%
$\hspace*{\fill}}The table at the left shows the results of similar
calculations for the slopes of other secant lines. From this table we would
expect the slope of the tangent line at $t=0.04$ to lie somewhere between $%
-742$ and $-607.5$. In fact, the average of the slopes of the two closest
secant lines is\\[8pt]
\hspace*{\fill}$\tfrac{1}{2}(-742-607.5)=-674.75$\hspace*{\fill}\\[8pt]
So, by this method, we estimate the slope of the tangent line to be $-675$.

Another method is to draw an appproximation to the tangent line at $P$ and
measure the sides of the triangle $ABC$, as in Figure 4. This gives an
estimate of the slope of the tangent line as%
\marginpar{\vspace{-0.3in}The physical meaning of the answer in Example 2 is
that the electric current flowing from the capacitor to the flash bulb after
0.04 second is about $-670$ microamperes.}\\[8pt]
$\hspace*{\fill}-\dfrac{\left\vert AB\right\vert }{\left\vert BC\right\vert }%
\approx -\dfrac{80.4-53.6}{0.06-0.02}=-670\hfill $%
\begin{tabular}{l}
\\ 
$\blacksquare $%
\end{tabular}%
\vspace{15pt}

\subsection{THE VELOCITY PROBLEM}

If you watch the speedometer of a car as you travel in city traffic, you see
that the nee\nolinebreak dle doesn't stay still for very long; that is, the
velocity of the car is not constant. We assume from watching the speedometer
that the car has a definite velocity at each moment, but how is the
\textquotedblleft instantaneous\textquotedblright\ velocity defined? Let's
investigate the ex\nolinebreak ample of a falling ball.

\begin{Example}[3]
%TCIMACRO{\TeXButton{VIDEO}{\VIDEO}}%
%BeginExpansion
\VIDEO%
%EndExpansion
%TCIMACRO{%
%\hyperref{\fbox{{\footnotesize Video 5et 020103 / 021}}\quad }{}{\fbox{{\footnotesize Video 5et 020103 / 021}}\quad }{}}%
%BeginExpansion
\msihyperref{\fbox{{\footnotesize Video 5et 020103 / 021}}\quad }{}{\fbox{{\footnotesize Video 5et 020103 / 021}}\quad }{}%
%EndExpansion
\marginpar{\vspace{6pt}\FRAME{itbpFU}{165.4375pt}{220.9375pt}{201.5pt}{\Qcb{%
{}}}{}{4c30201un01_01527.tif}{\special{language "Scientific Word";type
"GRAPHIC";maintain-aspect-ratio TRUE;display "USEDEF";valid_file "F";width
165.4375pt;height 220.9375pt;depth 201.5pt;original-width
2.2615in;original-height 3.0303in;cropleft "0";croptop "1";cropright
"1";cropbottom "0";filename 'graphics/4c30201un01_01527.tif';file-properties
"XNPEU";}}\vspace*{6pt}\newline
The CN Tower in Toronto was the tallest freestanding building in the world
for 32~years.}Suppose that a ball is dropped from the upper observation deck
of the CN Tower in Toronto, 450 m above the ground. Find the velocity of the
ball after 5~seconds.
\end{Example}

\begin{Solution}
Through experiments carried out four centuries ago, Galileo discovered that
the distance fallen by any freely falling body is proportional to the square
of the time it has been falling. (This model for free fall neglects air
resistance.) If the distance fallen after $t$ seconds is denoted by $s(t)$
and measured in meters, then Galileo's law is expressed by the equation\\[4pt%
]
\hspace*{\fill}$s(t)=4.9t^{2}$\hspace*{\fill}\vspace{4pt}

\hspace*{12pt}The difficulty in finding the velocity after 5 s is that we
are dealing with a single instant of time $(t=5)$, so no time interval is
involved. However, we can approximate the desired quantity by computing the
average velocity over the brief time interval of a tenth of a second from $%
t=5$ to $t=5.1$:%
\begin{eqnarray*}
\text{average velocity\hspace{3pt}} &=&\dfrac{\text{change in position}}{%
\text{time elapsed}} \\
&=&\dfrac{s(5.1)-s(5)}{0.1} \\
&=&\dfrac{4.9(5.1)^{2}-4.9(5)^{2}}{0.1}=49.49\text{ m}/\text{s}
\end{eqnarray*}%
The following table shows the results of similar calculations of the average
velocity over successively smaller time periods.\\[6pt]
\hspace*{\fill}$%
\begin{tabular}{|c|c|}
\hline
$\text{Time interval}$ & $\text{Average velocity (m}/\text{s)}$ \\ \hline
\multicolumn{1}{|l|}{$5\leq t\leq 6$} & $53.9\hspace{14pt}$ \\ 
\multicolumn{1}{|l|}{$5\leq t\leq 5.1$} & $49.49\hspace{9pt}$ \\ 
\multicolumn{1}{|l|}{$5\leq t\leq 5.05$} & $49.245\hspace{5pt}$ \\ 
\multicolumn{1}{|l|}{$5\leq t\leq 5.01$} & $49.049\hspace{4pt}$ \\ 
\multicolumn{1}{|l|}{$5\leq t\leq 5.001$} & $49.0049$ \\ \hline
\end{tabular}%
$\hspace*{\fill}\\[6pt]
It appears that as we shorten the time period, the average velocity is
becoming closer to 49 m$/$s. The \textbf{instantaneous velocity} when $t=5$
is defined to be the limiting value of these average velocities over shorter
and shorter time periods that start at $t=5$. Thus the (instantaneous)
velocity after 5 s is\\[6pt]
$\hspace*{\stretch{2}}v=49\ $m$/$s$\hfill \vspace{-15pt}\blacksquare $
\end{Solution}

You may have the feeling that the calculations used in solving this problem
are very similar to those used earlier in this section to find tangents. In
fact, there is a close connection between the tangent problem and the
problem of finding velocities. If we draw the graph of the distance function
of the ball (as in Figure 5) and we consider the points $P(a,4.9a^{2})$ and $%
Q(a+h,4.9(a+h)^{2})$ on the graph, then the slope of the secant line $PQ$ is%
\\[4pt]
\hspace*{\fill}$m_{PQ}=\dfrac{4.9(a+h)^{2}-4.9a^{2}}{(a+h)-a}$\hspace*{\fill}%
\\[6pt]
which is the same as the average velocity over the time interval $[a,a+h]$.
Therefore the velocity at time $t=a$ (the limit of these average velocities
as $h$ approaches 0) must be equal to the slope of the tangent line at $P$
(the limit of the slopes of the secant lines).\\[6pt]
\hspace*{\fill}\FRAME{itbpFU}{4.4157in}{1.772in}{0in}{\Qcb{%
\QTR{FigureNumber}{FIGURE 5}}}{}{4c3020105_00194.wmf}{\special{language
"Scientific Word";type "GRAPHIC";maintain-aspect-ratio TRUE;display
"USEDEF";valid_file "F";width 4.4157in;height 1.772in;depth
0in;original-width 4.3881in;original-height 1.7443in;cropleft "0";croptop
"1";cropright "1";cropbottom "0";filename
'graphics/4c3020105_00194.wmf';file-properties "XNPEU";}}\hspace*{\fill}%
\vspace*{9pt}

Examples 1 and 3 show that in order to solve tangent and velocity problems
we must be able to find limits. After studying methods for computing limits
in the next\ four\ sections, we will return to the problems of finding
tangents and velocities in Sec\nolinebreak tion~2.7.\vspace{-12pt}

\QTP{MultColDiv}
Exercises 2.1

\vspace{-6pt}%
%TCIMACRO{%
%\TeXButton{s2col}{\setlength{\columnsep}{24pt}
%\advance \leftskip by -165pt
%\advance\hsize by 165pt
%\advance\linewidth by 165pt
%\begin{multicols}{2}}}%
%BeginExpansion
\setlength{\columnsep}{24pt}
\advance \leftskip by -165pt
\advance\hsize by 165pt
\advance\linewidth by 165pt
\begin{multicols}{2}%
%EndExpansion

\begin{ExerciseList}
\item[\hfill 1.] A tank holds 1000 gallons of water, which drains from the
bottom of the tank in half an hour. The values in the table show the volume~$%
V$ of water remaining in the tank (in gallons) after $t$~minutes.\\[9pt]
$\hspace*{\fill}${\small $%
\begin{tabular}{|c|c|c|c|c|c|c|}
\hline
$t$ (min) & $5$ & $10$ & $15$ & $20$ & $25$ & $30$ \\ \hline
$V$ (gal) & $694$ & $444$ & $250$ & $111$ & $28$ & $0$ \\ \hline
\end{tabular}%
\ $}$\hfill $\vspace{9pt}

\begin{ExerciseList}
\item[(a)] If $P$ is the point $(15,250)$ on the graph of $V$, find the
slopes of the secant lines $PQ$ when $Q$ is the point on the graph with $t=5$%
, $10$, $20$, $25$, and $30$.

%TCIMACRO{%
%\hyperref{ANSWER}{}{\textbf{ANSWER:} $-44.4$, $-38.8$, $-27.8$, $-22.2$, $-16.\overline{6}$}{}}%
%BeginExpansion
\msihyperref{ANSWER}{}{\textbf{ANSWER:} $-44.4$, $-38.8$, $-27.8$, $-22.2$, $-16.\overline{6}$}{}%
%EndExpansion

%TCIMACRO{%
%\hyperref{\fbox{\textbf{master 00410}}}{}{\fbox{\textbf{master 00410}}}{}}%
%BeginExpansion
\msihyperref{\fbox{\textbf{master 00410}}}{}{\fbox{\textbf{master 00410}}}{}%
%EndExpansion

\item[(b)] Estimate the slope of the tangent line at $P$ by averaging the
slopes of two secant lines.

%TCIMACRO{\hyperref{ANSWER}{}{\textbf{ANSWER:} $-33.3$}{}}%
%BeginExpansion
\msihyperref{ANSWER}{}{\textbf{ANSWER:} $-33.3$}{}%
%EndExpansion

%TCIMACRO{%
%\hyperref{\fbox{\textbf{master 04482}}}{}{\fbox{\textbf{master 04482}}}{}}%
%BeginExpansion
\msihyperref{\fbox{\textbf{master 04482}}}{}{\fbox{\textbf{master 04482}}}{}%
%EndExpansion

\item[(c)] Use a graph of the function to estimate the slope of the tangent
line at~$P$. (This slope represents the rate at which the water is flowing
from the tank after 15 minutes.)

%TCIMACRO{%
%\hyperref{\fbox{\textbf{master 04483}}}{}{\fbox{\textbf{master 04483}}}{}}%
%BeginExpansion
\msihyperref{\fbox{\textbf{master 04483}}}{}{\fbox{\textbf{master 04483}}}{}%
%EndExpansion

%TCIMACRO{\hyperref{ANSWER}{}{\textbf{ANSWER:} $-33\tfrac{1}{3}$}{}}%
%BeginExpansion
\msihyperref{ANSWER}{}{\textbf{ANSWER:} $-33\tfrac{1}{3}$}{}%
%EndExpansion
\end{ExerciseList}

\item[\hfill 2.] A cardiac monitor is used to measure the heart rate of a
patient after surgery. It compiles the number of heartbeats after $t$%
~minutes. When the data in the table are graphed, the slope of the tangent
line represents the heart rate in beats per minute.\\[6pt]
$\hspace*{\fill}${\small $%
\begin{tabular}{|l|c|c|c|c|c|}
\hline
$t$ (min) & 36 & 38 & 40 & 42 & 44 \\ \hline
Heartbeats & \multicolumn{1}{|r|}{2530} & \multicolumn{1}{|r|}{2661} & 
\multicolumn{1}{|r|}{2806} & \multicolumn{1}{|r|}{2948} & 
\multicolumn{1}{|r|}{3080} \\ \hline
\end{tabular}%
\ $}$\hfill $\\[6pt]
The monitor estimates this value by calculating the slope of a secant line.
Use the data to estimate the patient's heart rate after 42 minutes using the
secant line between the points with the given values of \textit{t}.

\begin{ExerciseList}
\item[(a)] $t=36$\quad and\quad $t=42\hspace{18pt}$%
%TCIMACRO{%
%\hyperref{\fbox{\textbf{master 00411}}}{}{\fbox{\textbf{master 00411}}}{}}%
%BeginExpansion
\msihyperref{\fbox{\textbf{master 00411}}}{}{\fbox{\textbf{master 00411}}}{}%
%EndExpansion

\item[(b)] $t=38$\quad and\quad $t=42$\quad 
%TCIMACRO{%
%\hyperref{\fbox{\textbf{master 04484}}}{}{\fbox{\textbf{master 04484}}}{}}%
%BeginExpansion
\msihyperref{\fbox{\textbf{master 04484}}}{}{\fbox{\textbf{master 04484}}}{}%
%EndExpansion

\item[(c)] $t=40$\quad and\quad $t=42\hspace{18pt}$%
%TCIMACRO{%
%\hyperref{\fbox{\textbf{master 04485}}}{}{\fbox{\textbf{master 04485}}}{}}%
%BeginExpansion
\msihyperref{\fbox{\textbf{master 04485}}}{}{\fbox{\textbf{master 04485}}}{}%
%EndExpansion

\item[(d)] $t=42$\quad and\quad $t=44$\quad \vspace*{3pt}%
%TCIMACRO{%
%\hyperref{\fbox{\textbf{master 04486}}}{}{\fbox{\textbf{master 04486}}}{}}%
%BeginExpansion
\msihyperref{\fbox{\textbf{master 04486}}}{}{\fbox{\textbf{master 04486}}}{}%
%EndExpansion
\end{ExerciseList}

What are your conclusions?

\item[{\hfill \protect\fbox{3.\hspace{-2pt}}}] The point $P(2,-1)$ lies on
the curve $y=1/(1-x)$.

\begin{ExerciseList}
\item[(a)] If $Q$ is the point $(x,1/(1-x))$, use your calculator to find
the slope of the secant line $PQ$ (correct to six decimal places) for the
following values of $x$:\vspace{4pt}

%TCIMACRO{\TeXButton{table3pt}{\setlength\tabcolsep{3pt}}}%
%BeginExpansion
\setlength\tabcolsep{3pt}%
%EndExpansion
$%
\begin{tabular}{rlrlrl}
\textbf{(i)} & 1.5 & \textbf{(ii)} & 1.9 & \textbf{(iii)} & 1.99\vspace*{6pt}
\\ 
\textbf{(iv)} & 1.999$\hspace*{24pt}$ & \textbf{(v)} & 2.5 & \textbf{(vi)} & 
2.1\vspace*{6pt} \\ 
\textbf{(vii)} & 2.01 & \textbf{(viii)} & 2.001$\hspace*{24pt}$ &  & 
\vspace*{6pt}%
\end{tabular}%
$%
%TCIMACRO{\TeXButton{table6pt}{\setlength\tabcolsep{6pt}}}%
%BeginExpansion
\setlength\tabcolsep{6pt}%
%EndExpansion

\item[(b)] Using the results of part (a), guess the value of the slope of
the tangent line to the curve at $P(2,-1)$.

\item[(c)] Using the slope from part (b), find an equation of the tangent
line to the curve at $P(2,-1)$.

%TCIMACRO{\hyperref{\fbox{\textbf{NEW}}}{}{\fbox{\textbf{NEW}}}{}}%
%BeginExpansion
\msihyperref{\fbox{\textbf{NEW}}}{}{\fbox{\textbf{NEW}}}{}%
%EndExpansion
\end{ExerciseList}

\item[$\hfill $4.] The point $P(0.5,0)$ lies on the curve $y=\cos \pi x$.

\begin{ExerciseList}
\item[(a)] If $Q$ is the point $(x,\cos \pi x)$, use your calculator to find
the slope of the secant line $PQ$ (correct to six decimal places) for the
following values of $x$:\vspace{4pt}

%TCIMACRO{\TeXButton{table3pt}{\setlength\tabcolsep{3pt}}}%
%BeginExpansion
\setlength\tabcolsep{3pt}%
%EndExpansion
$%
\begin{tabular}{rlrlrl}
\textbf{(i)} & 0 & \textbf{(ii)} & 0.4 & \textbf{(iii)} & 0.49\vspace*{3pt}
\\ 
\textbf{(iv)} & 0.499$\hspace*{24pt}$ & \textbf{(v)} & 1 & \textbf{(vi)} & 
0.6\vspace*{3pt} \\ 
\textbf{(vii)} & 0.51 & \textbf{(viii)} & 0.501$\hspace*{24pt}$ &  & 
\vspace*{3pt}%
\end{tabular}%
$%
%TCIMACRO{\TeXButton{table6pt}{\setlength\tabcolsep{6pt}}}%
%BeginExpansion
\setlength\tabcolsep{6pt}%
%EndExpansion

%TCIMACRO{%
%\hyperref{\fbox{\textbf{master 81825}}}{}{\fbox{\textbf{master 81825}}}{}}%
%BeginExpansion
\msihyperref{\fbox{\textbf{master 81825}}}{}{\fbox{\textbf{master 81825}}}{}%
%EndExpansion

\item[(b)] Using the results of part (a), guess the value of the slope of
the tangent line to the curve at $P(0.5,0)$.

%TCIMACRO{%
%\hyperref{\fbox{\textbf{master 81826}}}{}{\fbox{\textbf{master 81826}}}{}}%
%BeginExpansion
\msihyperref{\fbox{\textbf{master 81826}}}{}{\fbox{\textbf{master 81826}}}{}%
%EndExpansion

\item[(c)] Using the slope from part (b), find an equation of the tangent
line to the curve at $P(0.5,0)$.

%TCIMACRO{%
%\hyperref{\fbox{\textbf{master 81827}}}{}{\fbox{\textbf{master 81827}}}{}}%
%BeginExpansion
\msihyperref{\fbox{\textbf{master 81827}}}{}{\fbox{\textbf{master 81827}}}{}%
%EndExpansion

\item[(d)] Sketch the curve, two of the secant lines, and the tangent line.

%TCIMACRO{%
%\hyperref{\fbox{\textbf{master 81828}}}{}{\fbox{\textbf{master 81828}}}{}}%
%BeginExpansion
\msihyperref{\fbox{\textbf{master 81828}}}{}{\fbox{\textbf{master 81828}}}{}%
%EndExpansion
\end{ExerciseList}

\item[{\hfill \protect\fbox{5.\hspace{-2pt}}}] If a ball is thrown into the
air with a velocity of 40 ft$/$s, its height in feet $t$~seconds later is
given by $y=40t-16t^{2}$.

\begin{ExerciseList}
\item[(a)] Find the average velocity for the time period beginning when $t=2$
and lasting\vspace{4pt}

%TCIMACRO{\TeXButton{table3pt}{\setlength\tabcolsep{3pt}}}%
%BeginExpansion
\setlength\tabcolsep{3pt}%
%EndExpansion
\begin{tabular}{rlrl}
\textbf{(i)} & 0.5~second & \textbf{(ii)} & 0.1~second\vspace*{3pt} \\ 
\textbf{(iii)} & 0.05~second$\hspace*{24pt}$ & \textbf{(iv)} & 0.01~second%
\vspace*{3pt}%
\end{tabular}%
%TCIMACRO{\TeXButton{table6pt}{\setlength\tabcolsep{6pt}}}%
%BeginExpansion
\setlength\tabcolsep{6pt}%
%EndExpansion

%TCIMACRO{%
%\hyperref{ANSWER}{}{\textbf{ANSWER:} (i) $-32$ ft$/$s\quad (ii) $-25.6$ ft$/$s\quad \newline
%(iii) $-24.8$ ft$/$s\quad (iv) $-24.16$ ft$/$s}{}}%
%BeginExpansion
\msihyperref{ANSWER}{}{\textbf{ANSWER:} (i) $-32$ ft$/$s\quad (ii) $-25.6$ ft$/$s\quad \newline
(iii) $-24.8$ ft$/$s\quad (iv) $-24.16$ ft$/$s}{}%
%EndExpansion

%TCIMACRO{%
%\hyperref{\fbox{\textbf{master 00414}}}{}{\fbox{\textbf{master 00414}}}{}}%
%BeginExpansion
\msihyperref{\fbox{\textbf{master 00414}}}{}{\fbox{\textbf{master 00414}}}{}%
%EndExpansion

\item[(b)] Estimate the instantaneous velocity when $t=2$.

%TCIMACRO{%
%\hyperref{ANSWER}{}{\textbf{ANSWER:}\hspace*{3pt}(b)~ $-24$ ft$/$s}{}}%
%BeginExpansion
\msihyperref{ANSWER}{}{\textbf{ANSWER:}\hspace*{3pt}(b)~ $-24$ ft$/$s}{}%
%EndExpansion

%TCIMACRO{%
%\hyperref{\fbox{\textbf{master 04499}}}{}{\fbox{\textbf{master 04499}}}{}}%
%BeginExpansion
\msihyperref{\fbox{\textbf{master 04499}}}{}{\fbox{\textbf{master 04499}}}{}%
%EndExpansion
\end{ExerciseList}

\item[\hfill 6.] If a rock is thrown upward on the planet Mars with a
velocity of 10~m$/$s, its height in meters $t$ seconds later is given by
\linebreak $y=10t-1.86t^{2}.$

\begin{ExerciseList}
\item[(a)] Find the average velocity over the given time intervals:\\[3pt]
%TCIMACRO{\TeXButton{table3pt}{\setlength\tabcolsep{3pt}}}%
%BeginExpansion
\setlength\tabcolsep{3pt}%
%EndExpansion
$%
\begin{tabular}[t]{rlrlrl}
\textbf{(i)} & $[1,2]$ & \textbf{(ii)} & $[1,1.5]$ & \textbf{(iii)} & $%
[1,1.1]$.\vspace*{3pt} \\ 
\textbf{(iv)} & $[1,1.01]\hspace*{24pt}$ & \textbf{(v)} & $[1,1.001]\hspace*{%
24pt}$ & \vspace*{3pt} & 
\end{tabular}%
$%
%TCIMACRO{\TeXButton{table6pt}{\setlength\tabcolsep{6pt}}}%
%BeginExpansion
\setlength\tabcolsep{6pt}%
%EndExpansion

%TCIMACRO{%
%\hyperref{\fbox{\textbf{master 80046}}}{}{\fbox{\textbf{master 80046}}}{}}%
%BeginExpansion
\msihyperref{\fbox{\textbf{master 80046}}}{}{\fbox{\textbf{master 80046}}}{}%
%EndExpansion

\item[(b)] Estimate the instantaneous velocity when $t=1$.

%TCIMACRO{%
%\hyperref{\fbox{\textbf{master 80051}}}{}{\fbox{\textbf{master 80051}}}{}}%
%BeginExpansion
\msihyperref{\fbox{\textbf{master 80051}}}{}{\fbox{\textbf{master 80051}}}{}%
%EndExpansion
\end{ExerciseList}

\item[\hfill 7.] The table shows the position of a cyclist.\\[6pt]
$\hspace*{\fill}${\small $%
\begin{tabular}{|l|c|c|c|c|c|c|}
\hline
$t\text{ (seconds)}$ & $0$ & $1$ & $2$ & $3$ & $4$ & $5$ \\ \hline
$s\text{ (meters)}$ & $0$ & $1.4$ & $5.1$ & $10.7$ & $17.7$ & $25.8$ \\ 
\hline
\end{tabular}%
\ $}$\hspace*{\fill}$\vspace{12pt}

\begin{ExerciseList}
\item[(a)] Find the average velocity for each time period:\vspace{3pt}

%TCIMACRO{\TeXButton{table3pt}{\setlength\tabcolsep{3pt}}}%
%BeginExpansion
\setlength\tabcolsep{3pt}%
%EndExpansion
$%
\begin{tabular}[t]{rlrl}
\textbf{(i)} & $[1,3]$ & \textbf{(ii)} & $[2,3]$\vspace*{6pt} \\ 
\textbf{(iii)} & $[3,5]$\hspace*{36pt} & \textbf{(iv)} & $[3,4]$\vspace*{6pt}%
\end{tabular}%
$%
%TCIMACRO{\TeXButton{table6pt}{\setlength\tabcolsep{6pt}}}%
%BeginExpansion
\setlength\tabcolsep{6pt}%
%EndExpansion

%TCIMACRO{%
%\hyperref{ANSWER}{}{\textbf{ANSWER:} (a)~(i)~4.65~m$/$s\quad (ii)~5.6~m$/$s\quad \newline
%(iii)~7.55~m$/$s\quad (iv)~7~m$/$s}{}}%
%BeginExpansion
\msihyperref{ANSWER}{}{\textbf{ANSWER:} (a)~(i)~4.65~m$/$s\quad (ii)~5.6~m$/$s\quad \newline
(iii)~7.55~m$/$s\quad (iv)~7~m$/$s}{}%
%EndExpansion

%TECHARTS_DUMMY_BEGIN_TAG

%TCIMACRO{\hyperref{(i) = }{}{(i) = }{}}%
%BeginExpansion
\msihyperref{(i) = }{}{(i) = }{}%
%EndExpansion
%TCIMACRO{%
%\hyperref{\fbox{\textbf{master 30062}}\quad }{}{\fbox{\textbf{master 30062}}\quad }{}}%
%BeginExpansion
\msihyperref{\fbox{\textbf{master 30062}}\quad }{}{\fbox{\textbf{master 30062}}\quad }{}%
%EndExpansion
%TCIMACRO{\hyperref{(ii) = }{}{(ii) = }{}}%
%BeginExpansion
\msihyperref{(ii) = }{}{(ii) = }{}%
%EndExpansion
%TCIMACRO{%
%\hyperref{\fbox{\textbf{master 30212}}}{}{\fbox{\textbf{master 30212}}}{}}%
%BeginExpansion
\msihyperref{\fbox{\textbf{master 30212}}}{}{\fbox{\textbf{master 30212}}}{}%
%EndExpansion

%TCIMACRO{\hyperref{(iii) = }{}{(iii) = }{}}%
%BeginExpansion
\msihyperref{(iii) = }{}{(iii) = }{}%
%EndExpansion
%TCIMACRO{%
%\hyperref{\fbox{\textbf{master 30213}}\quad }{}{\fbox{\textbf{master 30213}}\quad }{}}%
%BeginExpansion
\msihyperref{\fbox{\textbf{master 30213}}\quad }{}{\fbox{\textbf{master 30213}}\quad }{}%
%EndExpansion
%TCIMACRO{\hyperref{(iv) = }{}{(iv) = }{}}%
%BeginExpansion
\msihyperref{(iv) = }{}{(iv) = }{}%
%EndExpansion
%TCIMACRO{%
%\hyperref{\fbox{\textbf{master 30214}}}{}{\fbox{\textbf{master 30214}}}{}}%
%BeginExpansion
\msihyperref{\fbox{\textbf{master 30214}}}{}{\fbox{\textbf{master 30214}}}{}%
%EndExpansion

%TECHARTS_DUMMY_END_TAG

\item[(b)] Use the graph of $s$ as a function of $t$ to estimate the
instantaneous velocity when $t=3$.

%TCIMACRO{\hyperref{ANSWER}{}{\textbf{ANSWER:} (b)~6.3~m$/$s}{}}%
%BeginExpansion
\msihyperref{ANSWER}{}{\textbf{ANSWER:} (b)~6.3~m$/$s}{}%
%EndExpansion

%TCIMACRO{%
%\hyperref{\fbox{\textbf{master 30215}}}{}{\fbox{\textbf{master 30215}}}{}}%
%BeginExpansion
\msihyperref{\fbox{\textbf{master 30215}}}{}{\fbox{\textbf{master 30215}}}{}%
%EndExpansion
\end{ExerciseList}

\item[\hfill 8.] The displacement (in centimeters) of a particle moving back
and forth along a straight line is given by the equation of motion $s=2\sin
\pi t+3\cos \pi t$, where $t$ is measured in seconds.

\begin{ExerciseList}
\item[(a)] Find the average velocity during each time period:\vspace{3pt}

%TCIMACRO{\TeXButton{table3pt}{\setlength\tabcolsep{3pt}}}%
%BeginExpansion
\setlength\tabcolsep{3pt}%
%EndExpansion
$%
\begin{tabular}[t]{rlrl}
\textbf{(i)} & $[1,2]$ & \textbf{(ii)} & $[1,1.1]$\vspace{3pt}\vspace*{6pt}
\\ 
\textbf{(iii)} & $[1,1.01]$\hspace*{36pt} & \textbf{(iv)} & $[1,1.001]$%
\vspace*{6pt}%
\end{tabular}%
$%
%TCIMACRO{\TeXButton{table6pt}{\setlength\tabcolsep{6pt}}}%
%BeginExpansion
\setlength\tabcolsep{6pt}%
%EndExpansion

%TECHARTS_DUMMY_BEGIN_TAG

%TCIMACRO{\hyperref{(i) = }{}{(i) = }{}}%
%BeginExpansion
\msihyperref{(i) = }{}{(i) = }{}%
%EndExpansion
%TCIMACRO{%
%\hyperref{\fbox{\textbf{master 30063}}\quad }{}{\fbox{\textbf{master 30063}}\quad }{}}%
%BeginExpansion
\msihyperref{\fbox{\textbf{master 30063}}\quad }{}{\fbox{\textbf{master 30063}}\quad }{}%
%EndExpansion
%TCIMACRO{\hyperref{(ii) = }{}{(ii) = }{}}%
%BeginExpansion
\msihyperref{(ii) = }{}{(ii) = }{}%
%EndExpansion
%TCIMACRO{%
%\hyperref{\fbox{\textbf{master 30216}}}{}{\fbox{\textbf{master 30216}}}{}}%
%BeginExpansion
\msihyperref{\fbox{\textbf{master 30216}}}{}{\fbox{\textbf{master 30216}}}{}%
%EndExpansion

%TCIMACRO{\hyperref{(iii) = }{}{(iii) = }{}}%
%BeginExpansion
\msihyperref{(iii) = }{}{(iii) = }{}%
%EndExpansion
%TCIMACRO{%
%\hyperref{\fbox{\textbf{master 30217}}\quad }{}{\fbox{\textbf{master 30217}}\quad }{}}%
%BeginExpansion
\msihyperref{\fbox{\textbf{master 30217}}\quad }{}{\fbox{\textbf{master 30217}}\quad }{}%
%EndExpansion
%TCIMACRO{\hyperref{(iv) = }{}{(iv) = }{}}%
%BeginExpansion
\msihyperref{(iv) = }{}{(iv) = }{}%
%EndExpansion
%TCIMACRO{%
%\hyperref{\fbox{\textbf{master 30218}}}{}{\fbox{\textbf{master 30218}}}{}}%
%BeginExpansion
\msihyperref{\fbox{\textbf{master 30218}}}{}{\fbox{\textbf{master 30218}}}{}%
%EndExpansion

%TECHARTS_DUMMY_END_TAG
\item[(b)] Estimate the instantaneous velocity of the particle when $t=1$.

%TCIMACRO{%
%\hyperref{\fbox{\textbf{master 30219}}}{}{\fbox{\textbf{master 30219}}}{}}%
%BeginExpansion
\msihyperref{\fbox{\textbf{master 30219}}}{}{\fbox{\textbf{master 30219}}}{}%
%EndExpansion
\end{ExerciseList}

\item[{\hfill \protect\fbox{9.\hspace{-2pt}}}] The point $P\left( 1,0\right) $
lies on the curve $y=\sin \left( 10\pi /x\right) $.

\begin{ExerciseList}
\item[(a)] If $Q$ is the point $\left( x,\sin \left( 10\pi /x\right) \right) 
$, find the slope of the secant line $PQ$ (correct to four decimal places)
for \newline
$x=2$, $1.5$, $1.4$, $1.3$, $1.2$, $1.1$, $0.5$, $0.6$, $0.7$, $0.8$, and $%
0.9$. Do the slopes appear to be approaching a limit?\\[3pt]
%TCIMACRO{%
%\hyperref{ANSWER}{}{\textbf{ANSWER:} (a)~$0$, $1.7321$, $-1.0847$, $-2.7433$, $4.3301$, \newline
%$-2.8173$, $0$, $-2.1651$, $-2.6061$, $-5$, $3.4202$; no}{}}%
%BeginExpansion
\msihyperref{ANSWER}{}{\textbf{ANSWER:} (a)~$0$, $1.7321$, $-1.0847$, $-2.7433$, $4.3301$, \newline
$-2.8173$, $0$, $-2.1651$, $-2.6061$, $-5$, $3.4202$; no}{}%
%EndExpansion

%TCIMACRO{%
%\hyperref{\fbox{\textbf{master 00418}}}{}{\fbox{\textbf{master 00418}}}{}}%
%BeginExpansion
\msihyperref{\fbox{\textbf{master 00418}}}{}{\fbox{\textbf{master 00418}}}{}%
%EndExpansion

\item[(b)] 
%TCIMACRO{\TeXButton{GCALCXT}{\GCALCXT}}%
%BeginExpansion
\GCALCXT%
%EndExpansion
Use a graph of the curve to explain why the slopes of the secant lines in
part (a) are not close to the slope of the tangent line at $P$.\\[3pt]
%TCIMACRO{%
%\hyperref{ANSWER}{}{\textbf{ANSWER:} (b) \textit{no text answer}}{}}%
%BeginExpansion
\msihyperref{ANSWER}{}{\textbf{ANSWER:} (b) \textit{no text answer}}{}%
%EndExpansion

%TCIMACRO{%
%\hyperref{\fbox{\textbf{master 04505}}}{}{\fbox{\textbf{master 04505}}}{}}%
%BeginExpansion
\msihyperref{\fbox{\textbf{master 04505}}}{}{\fbox{\textbf{master 04505}}}{}%
%EndExpansion

\item[(c)] By choosing appropriate secant lines, estimate the slope of the
tangent line at $P$.\\[3pt]
%TCIMACRO{\hyperref{ANSWER}{}{\textbf{ANSWER:} (c)~$-31.4$}{}}%
%BeginExpansion
\msihyperref{ANSWER}{}{\textbf{ANSWER:} (c)~$-31.4$}{}%
%EndExpansion

%TCIMACRO{%
%\hyperref{\fbox{\textbf{master 04506}}}{}{\fbox{\textbf{master 04506}}}{}}%
%BeginExpansion
\msihyperref{\fbox{\textbf{master 04506}}}{}{\fbox{\textbf{master 04506}}}{}%
%EndExpansion
\end{ExerciseList}
\end{ExerciseList}

%TCIMACRO{%
%\TeXButton{e2col}{\end{multicols}
%\advance \leftskip by 165pt
%\advance\hsize by -165pt
%\advance\linewidth by -165pt
%}}%
%BeginExpansion
\end{multicols}
\advance \leftskip by 165pt
\advance\hsize by -165pt
\advance\linewidth by -165pt
%
%EndExpansion

\end{document}
