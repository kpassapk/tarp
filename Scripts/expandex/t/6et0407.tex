
\documentclass{sebase}
%%%%%%%%%%%%%%%%%%%%%%%%%%%%%%%%%%%%%%%%%%%%%%%%%%%%%%%%%%%%%%%%%%%%%%%%%%%%%%%%%%%%%%%%%%%%%%%%%%%%%%%%%%%%%%%%%%%%%%%%%%%%%%%%%%%%%%%%%%%%%%%%%%%%%%%%%%%%%%%%%%%%%%%%%%%%%%%%%%%%%%%%%%%%%%%%%%%%%%%%%%%%%%%%%%%%%%%%%%%%%%%%%%%%%%%%%%%%%%%%%%%%%%%%%%%%
\usepackage{amssymb}
\usepackage{float}
\usepackage{makeidx}
\usepackage{SECALCUL}
\usepackage{lie}

%TCIDATA{OutputFilter=LATEX.DLL}
%TCIDATA{Version=5.50.0.2953}
%TCIDATA{<META NAME="SaveForMode" CONTENT="1">}
%TCIDATA{BibliographyScheme=Manual}
%TCIDATA{Created=Mon Dec 15 16:20:00 1997}
%TCIDATA{LastRevised=Monday, August 10, 2009 12:09:59}
%TCIDATA{<META NAME="ViewSettings" CONTENT="16">}
%TCIDATA{<META NAME="GraphicsSave" CONTENT="32">}
%TCIDATA{CSTFile=SECALCUL.cst}

\input tcilatex
\newenvironment{instructions}{\STARTINSTR}{\ENDINSTR}
\begin{document}


\chapter[4\quad Applications of Differentiation]{}

\section{4.7\quad Optimization Problems}

%TCIMACRO{\TeXButton{noCCC}{\noCCC}}%
%BeginExpansion
\noCCC%
%EndExpansion
%TCIMACRO{\TeXButton{setET}{\renewcommand{\ET}{1}}}%
%BeginExpansion
\renewcommand{\ET}{1}%
%EndExpansion
%TCIMACRO{\TeXButton{noRM}{\renewcommand{\RM}{0}}}%
%BeginExpansion
\renewcommand{\RM}{0}%
%EndExpansion
%TCIMACRO{\TeXButton{setRM}{\renewcommand{\RM}{1}}}%
%BeginExpansion
\renewcommand{\RM}{1}%
%EndExpansion
%TCIMACRO{%
%\TeXButton{set page 70/57}{\ifnum \RM=1\setcounter{page}{70}\else \setcounter{page}{57}\fi}}%
%BeginExpansion
\ifnum \RM=1\setcounter{page}{70}\else \setcounter{page}{57}\fi%
%EndExpansion

The methods we have learned in this chapter for finding extreme values have
practical applications in many areas of life. A businessperson wants to
minimize costs and maximize profits. A traveler wants to minimize
transportation time. Fermat's Principle in optics states that light follows
the path that takes the least time. In this section and the next we solve
such problems as maximizing areas, volumes, and profits and minimizing
distances, times, and costs.

In solving such practical problems the greatest challenge is often to
convert the word problem into a mathematical optimization problem by setting
up the function that is to be maximized or minimized. Let's recall the
problem-solving principles discussed on page 
%TCIMACRO{\TeXButton{76}{76} }%
%BeginExpansion
76
%EndExpansion
and adapt them to this situation:\vspace{12pt}

\noindent \QTR{CaseHead}{STEPS IN SOLVING OPTIMIZATION PROBLEMS}

\begin{enumerate}
\item[1.] \textit{Understand the Problem}%
%TCIMACRO{\TeXButton{en}{\enskip}}%
%BeginExpansion
\enskip%
%EndExpansion
The first step is to read the problem carefully until it is clearly
understood. Ask yourself: What is the unknown? What are the given
quantities? What are the given conditions?

\item[2.] \textit{Draw a Diagram}%
%TCIMACRO{\TeXButton{en}{\enskip}}%
%BeginExpansion
\enskip%
%EndExpansion
In most problems it is useful to draw a diagram and identify the given and
required quantities on the diagram.

\item[3.] \textit{Introduce Notation}%
%TCIMACRO{\TeXButton{en}{\enskip}}%
%BeginExpansion
\enskip%
%EndExpansion
Assign a symbol to the quantity that is to be maximized or minimized (let's
call it $Q$ for now). Also select symbols $(a,b,c,\ldots ,x,y)$ for other
unknown quantities and label the diagram with these symbols. It may help to
use initials as suggestive symbols---for example, $A$ for area, $h$ for
height, $t$~for time.

\item[4.] Express $Q$ in terms of some of the other symbols from Step 3.

\item[5.] If $Q$ has been expressed as a function of more than one variable
in Step 4, use the given information to find relationships (in the form of
equations) among these variables. Then use these equations to eliminate all
but one of the variables in the expression for $Q$. Thus $Q$ will be
expressed as a function of \textit{one} variable $x$, say, $Q=f(x)$. Write
the domain of this function.

\item[6.] Use the methods of Sections 
%TCIMACRO{\TeXButton{4.1}{4.1}}%
%BeginExpansion
4.1%
%EndExpansion
\ and 
%TCIMACRO{\TeXButton{4.3}{4.3}}%
%BeginExpansion
4.3%
%EndExpansion
\ to find the \textit{absolute} maximum or minimum value of $f$. In
particular, if the domain of $f$ is a closed interval, then the Closed
Interval Method in Section 
%TCIMACRO{\TeXButton{4.1}{4.1}}%
%BeginExpansion
4.1%
%EndExpansion
\ can be used.
\end{enumerate}

\begin{Example}[1]
A farmer has 2400 ft of fencing and wants to fence off a rectangular field
that borders a straight river. He needs no fence along the river. What are
the dimensions of the field that has the largest area?
\end{Example}

\begin{Solution}
\marginpar{$\hspace*{24pt}$%
%TCIMACRO{\TeXButton{SQR}{\hskip .5em\protect\rule{4pt}{4pt}\hskip .5em}}%
%BeginExpansion
\hskip .5em\protect\rule{4pt}{4pt}\hskip .5em%
%EndExpansion
Understand the problem\\[-1pt]
$\hspace*{24pt}$%
%TCIMACRO{\TeXButton{SQR}{\hskip .5em\protect\rule{4pt}{4pt}\hskip .5em}}%
%BeginExpansion
\hskip .5em\protect\rule{4pt}{4pt}\hskip .5em%
%EndExpansion
Analogy: Try special cases\\[-1pt]
$\hspace*{24pt}$%
%TCIMACRO{\TeXButton{SQR}{\hskip .5em\protect\rule{4pt}{4pt}\hskip .5em}}%
%BeginExpansion
\hskip .5em\protect\rule{4pt}{4pt}\hskip .5em%
%EndExpansion
Draw diagrams}In order to get a feeling for what is happening in this
problem, let's experiment with some special cases. Figure 1 (not to scale)
shows three possible ways of laying out the 2400 ft of fencing. 
%TCIMACRO{\TeXButton{longpage}{\enlargethispage{12pt}}}%
%BeginExpansion
\enlargethispage{12pt}%
%EndExpansion

%TCIMACRO{\TeXButton{graphicS}{\vspace{12pt}\hskip-160pt\hfil}}%
%BeginExpansion
\vspace{12pt}\hskip-160pt\hfil%
%EndExpansion
\FRAME{itbpFU}{7.1537in}{1.3725in}{0in}{\Qcb{\QTR{FigureNumber}{FIGURE 1}}}{%
}{4e040701.wmf}{\special{language "Scientific Word";type
"GRAPHIC";maintain-aspect-ratio TRUE;display "USEDEF";valid_file "F";width
7.1537in;height 1.3725in;depth 0in;original-width 7.1537in;original-height
1.3725in;cropleft "0";croptop "1";cropright "1";cropbottom "0";filename
'graphics/4e040701.wmf';file-properties "XNPEU";}}%
%TCIMACRO{\TeXButton{graphicE}{\vspace{12pt}\hfil}}%
%BeginExpansion
\vspace{12pt}\hfil%
%EndExpansion

We see that when we try shallow, wide fields or deep, narrow fields, we get
relatively small areas. It seems plausible that there is some intermediate
configuration that produces the largest area.

Figure 2 illustrates the general case. We wish to maximize the area $A$ of
the rect-\linebreak angle. 
\marginpar{$\hspace*{24pt}$%
%TCIMACRO{\TeXButton{SQR}{\hskip .5em\protect\rule{4pt}{4pt}\hskip .5em}}%
%BeginExpansion
\hskip .5em\protect\rule{4pt}{4pt}\hskip .5em%
%EndExpansion
Introduce notation\vspace*{9pt}\FRAME{dtbpFU}{2.0003in}{1.0464in}{0pt}{\Qcb{%
\QTR{FigureNumber}{FIGURE 2}}}{}{4e040702.wmf}{\special{language "Scientific
Word";type "GRAPHIC";maintain-aspect-ratio TRUE;display "USEDEF";valid_file
"F";width 2.0003in;height 1.0464in;depth 0pt;original-width
2.0003in;original-height 1.0464in;cropleft "0";croptop "1";cropright
"1";cropbottom "0";filename 'graphics/4e040702.wmf';file-properties "XNPEU";}%
}}Let $x$ and $y$ be the depth and width of the rectangle (in feet). Then we
express $A$ in terms of $x$ and $y$:\\[6pt]
\hspace*{\fill}$A=xy$\hspace*{\fill}\\[9pt]
We want to express $A$ as a function of just one variable, so we eliminate $%
y $ by expressing it in terms of $x$. To do this we use the given
information that the total length of the fencing is 2400 ft. Thus\\[6pt]
\hspace*{\fill}$2x+y=2400$\hspace*{\fill}\\[9pt]
From this equation we have $y=2400-2x$, which gives\\[6pt]
\hspace*{\fill}$A=x(2400-2x)=2400x-2x^{2}$\hspace*{\fill}\\[9pt]
Note that $x\geq 0$ and $x\leq 1200$ (otherwise $A<0$). So the function that
we wish to maximize is\\[6pt]
\hspace*{\fill}$A(x)=2400x-2x^{2}\qquad \qquad 0\leq x\leq 1200$\hspace*{%
\fill}\\[9pt]
The derivative is $A^{\prime }\left( x\right) =2400-4x$, so to find the
critical numbers we solve the equation\\[6pt]
\hspace*{\fill}$2400-4x=0$\hspace*{\fill}\\[9pt]
which gives $x=600$. The maximum value of $A$ must occur either at this
critical number or at an endpoint of the interval. Since $A(0)=0$, $%
A(600)=720$,$000$,\linebreak and $A(1200)=0$, the Closed Interval Method
gives the maximum value as \newline
$A(600)=720$,$000$.

[Alternatively, we could have observed that $A^{\prime \prime }(x)=-4<0$ for
all $x$, so $A$ is always concave downward and the local maximum at $x=600$
must be an absolute maximum.]

Thus the rectangular field should be 600 ft deep and 1200 ft wide.\vspace*{%
-21pt}$\blacksquare $
\end{Solution}

\begin{Example}[2]
%TCIMACRO{\TeXButton{VIDEO}{\VIDEO}}%
%BeginExpansion
\VIDEO%
%EndExpansion
%TCIMACRO{%
%\hyperref{\frame{{\footnotesize video 5et 040702}}\quad }{}{\frame{{\footnotesize video 5et 040702}}\quad }{}}%
%BeginExpansion
\msihyperref{\frame{{\footnotesize video 5et 040702}}\quad }{}{\frame{{\footnotesize video 5et 040702}}\quad }{}%
%EndExpansion
A cylindrical can is to be made to hold 1 L of oil. Find the dimensions that
will minimize the cost of the metal to manufacture the can.
\end{Example}

\begin{Solution}
\marginpar{\vspace*{-24pt}\FRAME{dtbpFU}{1.0421in}{1.3033in}{0pt}{\Qcb{%
\QTR{FigureNumber}{FIGURE 3}}}{}{4e040703.wmf}{\special{language "Scientific
Word";type "GRAPHIC";maintain-aspect-ratio TRUE;display "USEDEF";valid_file
"F";width 1.0421in;height 1.3033in;depth 0pt;original-width
1.0421in;original-height 1.3033in;cropleft "0";croptop "1";cropright
"1";cropbottom "0";filename 'graphics/4e040703.wmf';file-properties "XNPEU";}%
}\FRAME{dtbpFU}{1.8887in}{1.6674in}{0pt}{\Qcb{\QTR{FigureNumber}{FIGURE 4}}}{%
}{4e040704.wmf}{\special{language "Scientific Word";type
"GRAPHIC";maintain-aspect-ratio TRUE;display "USEDEF";valid_file "F";width
1.8887in;height 1.6674in;depth 0pt;original-width 1.8887in;original-height
1.6674in;cropleft "0";croptop "1";cropright "1";cropbottom "0";filename
'graphics/4e040704.wmf';file-properties "XNPEU";}}}Draw the diagram as in
Figure 3, where $r$ is the radius and $h$ the height (both in centimeters).
In order to minimize the cost of the metal, we minimize the total surface
area of the cylinder (top, bottom, and sides). From Figure 4 we see that the
sides are made from a rectangular sheet with dimensions $2\pi r$ and $h$. So
the surface area is\\[9pt]
\hspace*{\fill}$A=2\pi r^{2}+2\pi rh$\hspace*{\fill}\vspace*{6pt}

To eliminate $h$ we use the fact that the volume is given as 1 L, which we
take to be 1000 $\mathrm{cm}^{3}$. Thus\\[6pt]
\hspace*{\fill}$\pi r^{2}h=1000$\hspace*{\fill}\\[9pt]
which gives $h=1000/(\pi r^{2})$. Substitution of this into the expression
for $A$ gives\\[6pt]
\hspace*{\fill}$A=2\pi r^{2}+2\pi r\left( \dfrac{1000}{\pi r^{2}}\right)
=2\pi r^{2}+\dfrac{2000}{r}$\hspace*{\fill}\\[9pt]
Therefore, the function that we want to minimize is \\[6pt]
\hspace*{\fill}$A(r)=2\pi r^{2}+\frac{2000}{r}\qquad r>0$\hspace*{\fill}%
\pagebreak \\[6pt]
\marginpar{
In the Applied Project on page 
%TCIMACRO{\TeXButton{333}{333}}%
%BeginExpansion
333%
%EndExpansion
\ we investigate the most economical shape for a can by taking into account
other manufacturing costs.\vspace*{9pt}\FRAME{dtbpFU}{1.7365in}{1.8057in}{0pt%
}{\Qcb{\QTR{FigureNumber}{FIGURE 5}}}{}{4e040705.wmf}{\special{language
"Scientific Word";type "GRAPHIC";maintain-aspect-ratio TRUE;display
"USEDEF";valid_file "F";width 1.7365in;height 1.8057in;depth
0pt;original-width 1.7365in;original-height 1.8057in;cropleft "0";croptop
"1";cropright "1";cropbottom "0";filename
'graphics/4e040705.wmf';file-properties "XNPEU";}}}To find the critical
numbers, we differentiate:\\[6pt]
\hspace*{\fill}$A^{\prime }(r)=4\pi r-\dfrac{2000}{r^{2}}=\dfrac{4(\pi
r^{3}-500)}{r^{2}}$\hspace*{\fill}\\[6pt]
Then $A^{\prime }(r)=0$ when $\pi r^{3}=500$, so the only critical number is 
$r=\sqrt[3]{500/\pi }$.

Since the domain of $A$ is $(0,\infty )$, we can't use the argument of 
%TCIMACRO{\hyperref{Example 1}{Example 1}{}{EX00240} }%
%BeginExpansion
\msihyperref{Example 1}{Example 1}{}{EX00240}
%EndExpansion
concerning endpoints. But we can observe that $A^{\prime }(r)<0$ for $r<\sqrt%
[3]{500/\pi }${}\thinspace and $A^{\prime }(r)>0$ for $r>\sqrt[3]{500/\pi }$%
, so $A$ is decreasing for \textit{all} $r$ to the left of the critical
number and increasing for \textit{all} $r$ to the right. Thus $r=\sqrt[3]{%
500/\pi }$ must give rise to an \textit{absolute} minimum.

[Alternatively, we could argue that $A(r)\rightarrow \infty $ as $%
r\rightarrow 0^{+}$ and $A(r)\rightarrow \infty $ as $r\rightarrow \infty $,
so there must be a minimum value of $A(r)$, which must occur at the critical
number. See Figure 5.]

The value of $h$ corresponding to $r=\sqrt[3]{500/\pi }$ is\\[6pt]
\hspace*{\fill}$h=\dfrac{1000}{\pi r^{2}}=\dfrac{1000}{\pi (500/\pi )^{2/3}}%
=2\sqrt[3]{\dfrac{500}{\pi }}=2r$\hspace*{\fill}\\[6pt]
Thus, to minimize the cost of the can, the radius should be $\sqrt[3]{%
500/\pi }${}\thinspace cm and the height should be equal to twice the
radius, namely, the diameter.$\blacksquare $
\end{Solution}

\marginpar{\vspace*{60pt}\FRAME{itbpF}{0.3347in}{0.1531in}{0in}{}{}{tec.ai}{%
\special{language "Scientific Word";type "GRAPHIC";maintain-aspect-ratio
TRUE;display "USEDEF";valid_file "F";width 0.3347in;height 0.1531in;depth
0in;original-width 0.3053in;original-height 0.1254in;cropleft "0";croptop
"1";cropright "1";cropbottom "0";filename 'graphics/TEC.ai';file-properties
"XNPEU";}}\hspace{3pt}\QTR{Author}{Module 4.7 takes you through six
additional optimization problems, including animations of the physical
situations.}}\QTR{Notehead}{NOTE 1}\quad The argument used in 
%TCIMACRO{\hyperref{Example 2}{Example 2}{}{EX00241} }%
%BeginExpansion
\msihyperref{Example 2}{Example 2}{}{EX00241}
%EndExpansion
to justify the absolute minimum is a variant of the First Derivative Test
(which applies only to \textit{local} maximum or minimum values) and is
stated here for future reference.

%TCIMACRO{\TeXButton{BOXSTART}{\STARTBOX}}%
%BeginExpansion
\STARTBOX%
%EndExpansion
\vspace{3pt}

\QTR{BOXHEAD}{FIRST DERIVATIVE TEST FOR ABSOLUTE EXTREME VALUES}\vspace{3pt}

Suppose that $c$ is a critical number of a continuous function $f$ defined
on an interval.

\begin{enumerate}
\item[(a)] If $f^{\prime }(x)>0$ for all $x<c$ and $f^{\prime }(x)<0$ for
all $x>c$, then $f(c)$ is the\newline
absolute maximum value of $f$.

\item[(b)] If $f^{\prime }(x)<0$ for all $x<c$ and $f^{\prime }(x)>0$ for
all $x>c$, then $f(c)$ is the\newline
absolute minimum value of $f$.
\end{enumerate}

%TCIMACRO{\TeXButton{BOXEND}{\ENDBOX}}%
%BeginExpansion
\ENDBOX%
%EndExpansion

\QTR{Notehead}{NOTE 2\quad }An alternative method for solving optimization
problems is to use implicit differentiation. Let's look at Example 2 again
to illustrate the method. We work with the same equations\\[6pt]
\hspace*{\fill}$A=2\pi r^{2}+2\pi rh\qquad \qquad \pi r^{2}h=100$\hspace*{%
\fill}\\[6pt]
but instead of eliminating $h$, we differentiate both equations implicitly
with respect\linebreak to $r$:\\[6pt]
\hspace*{\fill}$A^{\prime }=4\pi r+2\pi h+2\pi rh^{\prime }\qquad \qquad
2\pi rh+\pi r^{2}h^{\prime }=0$\hspace*{\fill}\\[6pt]
The minimum occurs at a critical number, so we set $A^{\prime }=0$,
simplify, and arrive at the equations\\[6pt]
\hspace*{\fill}$2r+h+rh^{\prime }=0\qquad \qquad 2h+rh^{\prime }=0$\hspace*{%
\fill}\\[6pt]
and subtraction gives $2r-h=0$ or $h=2r$.

\begin{Example}[3]
%TCIMACRO{\TeXButton{VIDEO}{\VIDEO}}%
%BeginExpansion
\VIDEO%
%EndExpansion
%TCIMACRO{%
%\hyperref{\frame{{\footnotesize video 5et 040703}}\quad }{}{\frame{{\footnotesize video 5et 040703}}\quad }{}}%
%BeginExpansion
\msihyperref{\frame{{\footnotesize video 5et 040703}}\quad }{}{\frame{{\footnotesize video 5et 040703}}\quad }{}%
%EndExpansion
Find the point on the parabola $y^{2}=2x$ that is closest to the point $%
(1,4) $.
\end{Example}

\begin{Solution}
The distance between the point $(1,4)$ and the point $(x,y)$ is\\[6pt]
\hspace*{\fill}$d=\sqrt{(x-1)^{2}+(y-4)^{2}}$\hspace*{\fill}\\[6pt]
\marginpar{\FRAME{dtbpFU}{1.7218in}{1.8057in}{0pt}{\Qcb{\QTR{FigureNumber}{%
FIGURE 6}}}{}{4e040706.wmf}{\special{language "Scientific Word";type
"GRAPHIC";maintain-aspect-ratio TRUE;display "USEDEF";valid_file "F";width
1.7218in;height 1.8057in;depth 0pt;original-width 1.7218in;original-height
1.8057in;cropleft "0";croptop "1";cropright "1";cropbottom "0";filename
'graphics/4e040706.wmf';file-properties "XNPEU";}}}(See Figure 6.) But if $%
(x,y)$ lies on the parabola, then $x=\tfrac{1}{2}y^{2}$, so the expression
for $d$ becomes\\[6pt]
\hspace*{\fill}$d=\sqrt{\left( \tfrac{1}{2}y^{2}-1\right) ^{2}+(y-4)^{2}}$%
\hspace*{\fill}\\[6pt]
(Alternatively, we could have substituted $y=\sqrt{2x}$ to get $d$ in terms
of $x$ alone.) Instead of minimizing $d$, we minimize its square:\\[6pt]
\hspace*{\fill}$d^{2}=f(y)=\left( \tfrac{1}{2}y^{2}-1\right) ^{2}+(y-4)^{2}$%
\hspace*{\fill}\\[6pt]
(You should convince yourself that the minimum of $d$ occurs at the same
point as the minimum of $d^{2}$, but $d^{2}$ is easier to work with.)
Differentiating, we obtain\\[6pt]
\hspace*{\fill}$f^{\prime }(y)=2\!\left( \tfrac{1}{2}y^{2}-1\right)
\!y+2(y-4)=y^{3}-8$\hspace*{\fill}\\[6pt]
so $f^{\prime }(y)=0$ when $y=2$. Observe that $f^{\prime }(y)<0$ when $y<2$
and $f^{\prime }(y)>0$ when $y>2$, so by the First Derivative Test for
Absolute Extreme Values, the absolute minimum occurs when $y=2$. (Or we
could simply say that because of the geometric nature of the problem, it's
obvious that there is a closest point but not a farthest point.) The
corresponding value of $x$ is $x=\tfrac{1}{2}y^{2}=2$. Thus the point on $%
y^{2}=2x$ closest to $(1,4)$ is $(2,2)$.\vspace*{-18pt}$\blacksquare $
\end{Solution}

\begin{Example}[4]
\marginpar{\FRAME{dtbpFU}{1.1251in}{2.642in}{0pt}{\Qcb{\QTR{FigureNumber}{%
FIGURE 7}}}{}{4e040707.wmf}{\special{language "Scientific Word";type
"GRAPHIC";maintain-aspect-ratio TRUE;display "USEDEF";valid_file "F";width
1.1251in;height 2.642in;depth 0pt;original-width 1.1251in;original-height
2.642in;cropleft "0";croptop "1";cropright "1";cropbottom "0";filename
'graphics/4e040707.wmf';file-properties "XNPEU";}}}A man launches his boat
from point $A$ on a bank of a straight river, 3 km wide, and wants to reach
point $B$, 8 km downstream on the opposite bank, as quickly as possible (see
Figure 7). He could row his boat directly across the river to point $C$ and
then run to $B$, or he could row directly to $B$, or he could row to some
point $D$ between $C$ and $B$ and then run to $B$. If he can row 6 km$/$h
and run 8 km$/$h, where should he land to reach $B$ as soon as possible? (We
assume that the speed of the water is negligible compared with the speed at
which the man rows.)
\end{Example}

\begin{Solution}
If we let $x$ be the distance from $C$ to $D$, then the running distance
is\linebreak $\left\vert DB\right\vert =8-x$ and the Pythagorean Theorem
gives the rowing distance as \newline
$\left\vert AD\right\vert =\sqrt{x^{2}+9}$. We use the equation 
\[
\text{time}=\frac{\text{distance}}{\text{rate}} 
\]%
Then the rowing time is $\sqrt{x^{2}+9}/6$ and the running time is $(8-x)/8$%
, so the total time $T$ as a function of $x$ is 
\[
T(x)=\frac{\sqrt{x^{2}+9}}{6}+\frac{8-x}{8} 
\]%
The domain of this function $T$ is $[0,8]$. Notice that if $x=0$, he rows to 
$C$ and if $x=8$, he rows directly to $B$. The derivative of $T$ is 
\[
T^{\prime }(x)=\frac{x}{6\sqrt{x^{2}+9}}-\frac{1}{8} 
\]%
Thus, using the fact that $x\geq 0$, we have%
%TCIMACRO{\TeXButton{longpage}{\enlargethispage{\baselineskip}}}%
%BeginExpansion
\enlargethispage{\baselineskip}%
%EndExpansion
\\[6pt]
\hspace*{\fill}$%
\begin{array}{lllll}
T^{\prime }(x)=0 & \qquad \Leftrightarrow \qquad & \dfrac{x}{6\,\sqrt{x^{2}+9%
}}=\dfrac{1}{8}\vspace{6pt} & \quad \Leftrightarrow \quad & 4x=3\,\sqrt{%
x^{2}+9} \\ 
& \qquad \Leftrightarrow \qquad & 16x^{2}=9(x^{2}+9) & \quad \Leftrightarrow
\quad & 7x^{2}=81\vspace{6pt} \\ 
& \qquad \Leftrightarrow \qquad & x=\dfrac{9}{\sqrt{7}} &  & 
\end{array}%
$\hspace*{\fill}\pagebreak \\[6pt]
\marginpar{\vspace{-12pt}\FRAME{dtbpFU}{1.9164in}{1.5714in}{0pt}{\Qcb{%
\QTR{FigureNumber}{FIGURE 8}}}{}{4e040708.wmf}{\special{language "Scientific
Word";type "GRAPHIC";maintain-aspect-ratio TRUE;display "USEDEF";valid_file
"F";width 1.9164in;height 1.5714in;depth 0pt;original-width
1.9164in;original-height 1.5714in;cropleft "0";croptop "1";cropright
"1";cropbottom "0";filename 'graphics/4e040708.wmf';file-properties "XNPEU";}%
}}The only critical number is $x=9/\sqrt{7}$. To see whether the minimum
occurs at this critical number or at an endpoint of the domain $[0,8]$, we
evaluate $T$ at all three points: 
\[
T(0)=1.5\qquad T\left( \frac{9}{\sqrt{7}}\right) =1+\frac{\sqrt{7}}{8}%
\approx 1.33\qquad T(8)=\frac{\sqrt{73}}{6}\approx 1.42 
\]%
Since the smallest of these values of $\,T$ occurs when $x=9/\sqrt{7}$, the
absolute minimum value of $\,T$ must occur there. Figure 8 illustrates this
calculation by showing the graph of $\,T$.

Thus the man should land the boat at a point $9/\sqrt{7}$ km ($\approx \!3.4$
km) downstream from his starting point.\vspace*{-18pt}$\blacksquare $
\end{Solution}

\begin{Example}[5]
%TCIMACRO{\TeXButton{VIDEO}{\VIDEO}}%
%BeginExpansion
\VIDEO%
%EndExpansion
%TCIMACRO{%
%\hyperref{\frame{{\footnotesize video 5et 040705}}\quad }{}{\frame{{\footnotesize video 5et 040705}}\quad }{}}%
%BeginExpansion
\msihyperref{\frame{{\footnotesize video 5et 040705}}\quad }{}{\frame{{\footnotesize video 5et 040705}}\quad }{}%
%EndExpansion
Find the area of the largest rectangle that can be inscribed in a semicircle
of radius $r$.
\end{Example}

\begin{SolutionOne}[1]
\marginpar{\FRAME{dtbpFU}{1.9303in}{1.3085in}{0pt}{\Qcb{\QTR{FigureNumber}{%
FIGURE 9}}}{}{4e040709.wmf}{\special{language "Scientific Word";type
"GRAPHIC";maintain-aspect-ratio TRUE;display "USEDEF";valid_file "F";width
1.9303in;height 1.3085in;depth 0pt;original-width 1.9303in;original-height
1.3085in;cropleft "0";croptop "1";cropright "1";cropbottom "0";filename
'graphics/4e040709.wmf';file-properties "XNPEU";}}}Let's take the semicircle
to be the upper half of the circle $x^{2}+\,y^{2}=r^{2}$ with center the
origin. Then the word \textit{inscribed} means that the rectangle has two
vertices on the semicircle and two vertices on the $x$-axis as shown in
Figure 9.

Let $(x,y)$ be the vertex that lies in the first quadrant. Then the
rectangle has sides of lengths $2x$ and $y$, so its area is\\[6pt]
\hspace*{\fill}$A=2xy$\hspace*{\fill}\\[6pt]
To eliminate $y$ we use the fact that $(x,y)$ lies on the circle $%
x^{2}+y^{2}=r^{2}$ and so $y=\sqrt{r^{2}-x^{2}}$. Thus\\[6pt]
\hspace*{\fill}$A=2x\sqrt{r^{2}-x^{2}}$\hspace*{\fill}\\[6pt]
The domain of this function is $0\leq x\leq r$. Its derivative is \\[6pt]
\hspace*{\fill}$A^{\prime }=2\sqrt{r^{2}-x^{2}}-\dfrac{2x^{2}}{\sqrt{%
r^{2}-x^{2}}}=\dfrac{2(r^{2}-2x^{2})}{\sqrt{r^{2}-x^{2}}}$\hspace*{\fill}\\[%
6pt]
which is 0 when $2x^{2}=r^{2}$, that is, $x=r/\sqrt{2}$ (since $x\geq 0$).
This value of $x$ gives a maximum value of $A$ since $A(0)=0$ and $A(r)=0$.
Therefore the area of the largest inscribed rectangle is \\[5pt]
\hspace*{\fill}$A\!\left( \dfrac{r}{\sqrt{2}}\right) =2\,\dfrac{r}{\sqrt{2}}%
\sqrt{r^{2}-\dfrac{r^{2}}{2}}=r^{2}$\hspace*{\fill}$\vspace{-21pt}$
\end{SolutionOne}

\begin{SolutionOne}[2]
\marginpar{\FRAME{dtbpFU}{1.9726in}{1.1147in}{0pt}{\Qcb{\QTR{FigureNumber}{%
FIGURE 10}}}{}{4e040710.wmf}{\special{language "Scientific Word";type
"GRAPHIC";maintain-aspect-ratio TRUE;display "USEDEF";valid_file "F";width
1.9726in;height 1.1147in;depth 0pt;original-width 1.9726in;original-height
1.1147in;cropleft "0";croptop "1";cropright "1";cropbottom "0";filename
'graphics/4e040710.wmf';file-properties "XNPEU";}}}A simpler solution is
possible if we think of using an angle as a variable. Let $\theta $ be the
angle shown in Figure 10. Then the area of the rectangle is 
\[
A(\theta )=(2r\cos \theta )(r\sin \theta )=r^{2}(2\sin \theta \cos \theta
)=r^{2}\sin 2\theta 
\]%
\noindent We know that $\sin \,2\theta $ has a maximum value of 1 and it
occurs when $2\theta =\pi /2$. So $A(\theta )$ has a maximum value of $r^{2}$
and it occurs when $\theta =\pi /4$.

Notice that this trigonometric solution doesn't involve differentiation. In
fact, we didn't need to use calculus at all.$\vspace{-12pt}\blacksquare $
\end{SolutionOne}

\subsection{APPLICATIONS TO BUSINESS AND ECONOMICS\protect\vspace*{-6pt}}

%TCIMACRO{\TeXButton{longpage}{\enlargethispage{12pt}}}%
%BeginExpansion
\enlargethispage{12pt}%
%EndExpansion
In Section 3.7\ we introduced the idea of marginal cost. Recall that if $%
C(x) $, the \textbf{cost function}, is the cost of producing $x$ units of a
certain product, then the \textbf{marginal cost} is the rate of change of $C$
with respect to $x$. In other words, the marginal cost function is the
derivative, $C^{\prime }(x)$, of the cost function.

Now let's consider marketing. Let $p(x)$ be the price per unit that the
company can charge if it sells $x$ units. Then $p$ is called the \textbf{%
demand function} (or \textbf{price function}) and we would expect it to be a
decreasing function of $x$. If $x$ units are sold and the \pagebreak
\linebreak price per unit is $p(x)$, then the total revenue is\\[6pt]
\hspace*{\fill}$R(x)=xp(x)$\hspace*{\fill}\\[6pt]
and $R$ is called the \textbf{revenue function}. The derivative $R^{\prime }$
of the revenue function is called the \textbf{marginal revenue function} and
is the rate of change of revenue with respect to the number of units sold.

If $x$ units are sold, then the total profit is \\[6pt]
\hspace*{\fill}$P(x)=R(x)-C(x)$\hspace*{\fill}\\[6pt]
and $P$ is called the \textbf{profit function}.\textbf{\ }The \textbf{%
marginal profit function} is $P^{\prime }$, the derivative of the profit
function. In Exercises 53--58 you are asked to use the marginal cost,
revenue, and profit functions to minimize costs and maximize revenues and
profits.$\vspace{-6pt}$

\begin{Example}[6]
%TCIMACRO{\TeXButton{VIDEO}{\VIDEO}}%
%BeginExpansion
\VIDEO%
%EndExpansion
A store has been selling $200$ DVD burners a week at $\$350$ each. A market
survey indicates that for each $\$10$ rebate offered to buyers, the number
of units sold will increase by $20$ a week. Find the demand function and the
revenue function. How large a rebate should the store offer to maximize its
revenue?
\end{Example}

\begin{Solution}
If $x$ is the number of DVD burners sold per week, then the weekly increase
in sales is $x-200$. For each increase of $20$ units sold, the price is
decreased by $\$10$. So for each additional unit sold, the decrease in price
will be $\frac{1}{20}\times 10$ and the demand function is\\[6pt]
\hspace*{\fill}$p(x)=350-\tfrac{10}{20}(x-200)=450-\tfrac{1}{2}x$\hspace*{%
\fill}\\[6pt]
The revenue function is \\[6pt]
\hspace*{\fill}$R(x)=xp(x)=450x-\tfrac{1}{2}x^{2}$\hspace*{\fill}\\[6pt]
Since $R^{\prime }(x)=450-x$, we see that $R^{\prime }(x)=0$ when $x=450$.
This value of $x$ gives an absolute maximum by the First Derivative Test (or
simply by observing that the graph of $R$ is a parabola that opens
downward). The corresponding price is\\[6pt]
\hspace*{\fill}$p(450)=450-\tfrac{1}{2}(450)=225$\hspace*{\fill}\\[6pt]
and the rebate is $350-225=125$. Therefore, to maximize revenue, the store
should offer a rebate of \$125.\vspace*{-27pt}$\blacksquare $
\end{Solution}

\QTP{MultColDiv}
Exercises 4.7

\vspace*{-9pt}%
%TCIMACRO{%
%\TeXButton{s2col}{\setlength{\columnsep}{24pt}
%\advance \leftskip by -165pt
%\advance\hsize by 165pt
%\advance\linewidth by 165pt
%\begin{multicols}{2}}}%
%BeginExpansion
\setlength{\columnsep}{24pt}
\advance \leftskip by -165pt
\advance\hsize by 165pt
\advance\linewidth by 165pt
\begin{multicols}{2}%
%EndExpansion

\begin{ExerciseList}
\item[\hfill 1.] Consider the following problem: Find two numbers whose sum
is 23 and whose product is a maximum.

\begin{ExerciseList}
\item[(a)] Make a table of values, like the following one, so that the sum
of the numbers in the first two columns is always 23. On the basis of the
evidence in your table, estimate the answer to the problem.\vspace{6pt}

\begin{tabular}{|l|l|l|}
\hline
First number & Second number & Product \\ \hline
\multicolumn{1}{|c|}{1} & \multicolumn{1}{|c|}{22} & \multicolumn{1}{|c|}{22}
\\ 
\multicolumn{1}{|c|}{2} & \multicolumn{1}{|c|}{21} & \multicolumn{1}{|c|}{42}
\\ 
\multicolumn{1}{|c|}{3} & \multicolumn{1}{|c|}{20} & \multicolumn{1}{|c|}{60}
\\ 
\multicolumn{1}{|c|}{$\vdots $} & \multicolumn{1}{|c|}{$\vdots $} & 
\multicolumn{1}{|c|}{$\vdots $} \\ \hline
\end{tabular}

%TCIMACRO{\hyperref{ANSWER}{}{\textbf{ANSWER:} (a) $11$, $12$}{}}%
%BeginExpansion
\msihyperref{ANSWER}{}{\textbf{ANSWER:} (a) $11$, $12$}{}%
%EndExpansion

%TCIMACRO{%
%\hyperref{\fbox{\textbf{master 02533}}}{}{\fbox{\textbf{master 02533}}}{}}%
%BeginExpansion
\msihyperref{\fbox{\textbf{master 02533}}}{}{\fbox{\textbf{master 02533}}}{}%
%EndExpansion
\vspace{6pt}

\item[(b)] Use calculus to solve the problem and compare with your answer to
part (a).

%TCIMACRO{\hyperref{ANSWER}{}{\textbf{ANSWER:} (b) $11.5$, $11.5$}{}}%
%BeginExpansion
\msihyperref{ANSWER}{}{\textbf{ANSWER:} (b) $11.5$, $11.5$}{}%
%EndExpansion

%TCIMACRO{%
%\hyperref{\fbox{\textbf{master 05074}}}{}{\fbox{\textbf{master 05074}}}{}}%
%BeginExpansion
\msihyperref{\fbox{\textbf{master 05074}}}{}{\fbox{\textbf{master 05074}}}{}%
%EndExpansion
\end{ExerciseList}

\item[\hfill 2.] Find two numbers whose difference is 100 and whose product
is a minimum.

%TCIMACRO{%
%\hyperref{\fbox{\textbf{master 02534}}}{}{\fbox{\textbf{master 02534}}}{}}%
%BeginExpansion
\msihyperref{\fbox{\textbf{master 02534}}}{}{\fbox{\textbf{master 02534}}}{}%
%EndExpansion

\item[\hfill 3.] Find two positive numbers whose product is 100 and whose
sum is a minimum.

%TCIMACRO{\hyperref{ANSWER}{}{\textbf{ANSWER:} $10$, $10$}{}}%
%BeginExpansion
\msihyperref{ANSWER}{}{\textbf{ANSWER:} $10$, $10$}{}%
%EndExpansion

%TCIMACRO{%
%\hyperref{\fbox{\textbf{master 02535}}}{}{\fbox{\textbf{master 02535}}}{}}%
%BeginExpansion
\msihyperref{\fbox{\textbf{master 02535}}}{}{\fbox{\textbf{master 02535}}}{}%
%EndExpansion

\item[\hfill 4.] Find a positive number such that the sum of the number and
its reciprocal is as small as possible.

%TCIMACRO{%
%\hyperref{\fbox{\textbf{master 02536}}}{}{\fbox{\textbf{master 02536}}}{}}%
%BeginExpansion
\msihyperref{\fbox{\textbf{master 02536}}}{}{\fbox{\textbf{master 02536}}}{}%
%EndExpansion

\item[\hfill 5.] Find the dimensions of a rectangle with perimeter $100$ m
whose area is as large as possible.

%TCIMACRO{\hyperref{ANSWER}{}{\textbf{ANSWER:} $25$ m by $25$ m}{}}%
%BeginExpansion
\msihyperref{ANSWER}{}{\textbf{ANSWER:} $25$ m by $25$ m}{}%
%EndExpansion

%TCIMACRO{%
%\hyperref{\fbox{\textbf{master 02537}}}{}{\fbox{\textbf{master 02537}}}{}}%
%BeginExpansion
\msihyperref{\fbox{\textbf{master 02537}}}{}{\fbox{\textbf{master 02537}}}{}%
%EndExpansion

\item[\hfill 6.] Find the dimensions of a rectangle with area $1000$ m$^{2}$
whose perimeter is as small as possible.

%TCIMACRO{%
%\hyperref{\fbox{\textbf{master 02538}}}{}{\fbox{\textbf{master 02538}}}{}}%
%BeginExpansion
\msihyperref{\fbox{\textbf{master 02538}}}{}{\fbox{\textbf{master 02538}}}{}%
%EndExpansion

\item[\hfill 7.] A model used for the yield $Y$ of an agricultural crop as a
function of the nitrogen level $N$ in the soil (measured in appropriate
units) is\\[6pt]
\hspace*{\fill}$Y=\dfrac{kN}{1+N^{2}}$\hspace*{\fill}\\[6pt]
where $k$ is a positive constant. What nitrogen level gives the best yield?

%TCIMACRO{\hyperref{ANSWER}{}{\textbf{ANSWER:} $N=1$}{}}%
%BeginExpansion
\msihyperref{ANSWER}{}{\textbf{ANSWER:} $N=1$}{}%
%EndExpansion

%TCIMACRO{%
%\hyperref{\fbox{\textbf{master 80365}}}{}{\fbox{\textbf{master 80365}}}{}}%
%BeginExpansion
\msihyperref{\fbox{\textbf{master 80365}}}{}{\fbox{\textbf{master 80365}}}{}%
%EndExpansion

\item[\hfill 8.] The rate (in mg $\mathrm{carbon}/\mathrm{m}^{3}/\mathrm{h}$%
) at which photosynthesis takes place for a species of phytoplankton is
modeled by the function\\[6pt]
\hspace*{\fill}$P=\dfrac{100I}{I^{2}+I+4}$\hspace*{\fill}\\[6pt]
where $I$ is the light intensity (measured in thousands of foot-candles).
For what light intensity is $P$ a maximum?

%TCIMACRO{%
%\hyperref{\fbox{\textbf{master 80366}}}{}{\fbox{\textbf{master 80366}}}{}}%
%BeginExpansion
\msihyperref{\fbox{\textbf{master 80366}}}{}{\fbox{\textbf{master 80366}}}{}%
%EndExpansion

\item[\hfill 9.] Consider the following problem:\hspace{6pt}A farmer with
750 ft of fencing wants to enclose a rectangular area and then divide it
into four pens with fencing parallel to one side of the rectangle. What is
the largest possible total area of the four pens?

\begin{ExerciseList}
\item[(a)] Draw several diagrams illustrating the situation, some with
shallow, wide pens and some with deep, narrow pens. Find the total areas of
these configurations. Does it appear that there is a maximum area? If so,
estimate it.

%TCIMACRO{%
%\hyperref{ANSWER}{}{\textbf{ANSWER:} \newline
%(a) \FRAME{itbpF}{2.6247in}{1.2254in}{1.1338in}{}{}{4a040701a.wmf}{\special{language "Scientific Word";type "GRAPHIC";maintain-aspect-ratio TRUE;display "USEDEF";valid_file "F";width 2.6247in;height 1.2254in;depth 1.1338in;original-width 6.4627in;original-height 3.0156in;cropleft "0";croptop "1";cropright "1";cropbottom "0";filename 'graphics/4a040701a.wmf';file-properties "XNPEU";}}\newline
%}{}}%
%BeginExpansion
\msihyperref{ANSWER}{}{\textbf{ANSWER:} \newline
(a) \FRAME{itbpF}{2.6247in}{1.2254in}{1.1338in}{}{}{4a040701a.wmf}{\special{language "Scientific Word";type "GRAPHIC";maintain-aspect-ratio TRUE;display "USEDEF";valid_file "F";width 2.6247in;height 1.2254in;depth 1.1338in;original-width 6.4627in;original-height 3.0156in;cropleft "0";croptop "1";cropright "1";cropbottom "0";filename 'graphics/4a040701a.wmf';file-properties "XNPEU";}}\newline
}{}%
%EndExpansion

%TCIMACRO{%
%\hyperref{\fbox{\textbf{master 02539}}}{}{\fbox{\textbf{master 02539}}}{}}%
%BeginExpansion
\msihyperref{\fbox{\textbf{master 02539}}}{}{\fbox{\textbf{master 02539}}}{}%
%EndExpansion

\item[(b)] Draw a diagram illustrating the general situation. Introduce
notation and label the diagram with your symbols.

%TCIMACRO{%
%\hyperref{ANSWER}{}{\textbf{ANSWER:} \newline
%(b) \FRAME{itbpF}{1.2635in}{0.7264in}{0.6028in}{}{}{4a040701b.wmf}{\special{language "Scientific Word";type "GRAPHIC";maintain-aspect-ratio TRUE;display "USEDEF";valid_file "F";width 1.2635in;height 0.7264in;depth 0.6028in;original-width 6.4627in;original-height 3.6988in;cropleft "0";croptop "1";cropright "1";cropbottom "0";filename 'graphics/4a040701b.wmf';file-properties "XNPEU";}}\newline
%}{}}%
%BeginExpansion
\msihyperref{ANSWER}{}{\textbf{ANSWER:} \newline
(b) \FRAME{itbpF}{1.2635in}{0.7264in}{0.6028in}{}{}{4a040701b.wmf}{\special{language "Scientific Word";type "GRAPHIC";maintain-aspect-ratio TRUE;display "USEDEF";valid_file "F";width 1.2635in;height 0.7264in;depth 0.6028in;original-width 6.4627in;original-height 3.6988in;cropleft "0";croptop "1";cropright "1";cropbottom "0";filename 'graphics/4a040701b.wmf';file-properties "XNPEU";}}\newline
}{}%
%EndExpansion

%TCIMACRO{%
%\hyperref{\fbox{\textbf{master 05075}}}{}{\fbox{\textbf{master 05075}}}{}}%
%BeginExpansion
\msihyperref{\fbox{\textbf{master 05075}}}{}{\fbox{\textbf{master 05075}}}{}%
%EndExpansion

\item[(c)] Write an expression for the total area.

%TCIMACRO{\hyperref{ANSWER}{}{\textbf{ANSWER:} (c) $A=xy$}{}}%
%BeginExpansion
\msihyperref{ANSWER}{}{\textbf{ANSWER:} (c) $A=xy$}{}%
%EndExpansion

%TCIMACRO{%
%\hyperref{\fbox{\textbf{master 05076}}}{}{\fbox{\textbf{master 05076}}}{}}%
%BeginExpansion
\msihyperref{\fbox{\textbf{master 05076}}}{}{\fbox{\textbf{master 05076}}}{}%
%EndExpansion

\item[(d)] Use the given information to write an equation that relates the
variables.

%TCIMACRO{\hyperref{ANSWER}{}{\textbf{ANSWER:} (d) $5x+2y=750$}{}}%
%BeginExpansion
\msihyperref{ANSWER}{}{\textbf{ANSWER:} (d) $5x+2y=750$}{}%
%EndExpansion

%TCIMACRO{%
%\hyperref{\fbox{\textbf{master 05077}}}{}{\fbox{\textbf{master 05077}}}{}}%
%BeginExpansion
\msihyperref{\fbox{\textbf{master 05077}}}{}{\fbox{\textbf{master 05077}}}{}%
%EndExpansion

\item[(e)] Use part (d) to write the total area as a function of one
variable.

%TCIMACRO{%
%\hyperref{ANSWER}{}{\textbf{ANSWER:} (e)~$A(x)=375x-\tfrac{5}{2}x^{2}$}{}}%
%BeginExpansion
\msihyperref{ANSWER}{}{\textbf{ANSWER:} (e)~$A(x)=375x-\tfrac{5}{2}x^{2}$}{}%
%EndExpansion

%TCIMACRO{%
%\hyperref{\fbox{\textbf{master 05078}}}{}{\fbox{\textbf{master 05078}}}{}}%
%BeginExpansion
\msihyperref{\fbox{\textbf{master 05078}}}{}{\fbox{\textbf{master 05078}}}{}%
%EndExpansion

\item[(f)] Finish solving the problem and compare the answer with your
estimate in part (a).

%TCIMACRO{%
%\hyperref{ANSWER}{}{\textbf{ANSWER:} (f) $14$,$062.5$ ft$^{2}$}{}}%
%BeginExpansion
\msihyperref{ANSWER}{}{\textbf{ANSWER:} (f) $14$,$062.5$ ft$^{2}$}{}%
%EndExpansion

%TCIMACRO{%
%\hyperref{\fbox{\textbf{master 05079}}}{}{\fbox{\textbf{master 05079}}}{}}%
%BeginExpansion
\msihyperref{\fbox{\textbf{master 05079}}}{}{\fbox{\textbf{master 05079}}}{}%
%EndExpansion
\end{ExerciseList}

\item[\hfill 10.] Consider the following problem:\hspace{6pt}A box with an
open top is to be constructed from a square piece of cardboard, $3$ ft wide,
by cutting out a square from each of the four corners and bending up the
sides. Find the largest volume that such a box can have.

\begin{ExerciseList}
\item[(a)] Draw several diagrams to illustrate the situation, some short
boxes with large bases and some tall boxes with small bases. Find the
volumes of several such boxes. Does it appear that there is a maximum
volume? If so, estimate it.

%TCIMACRO{%
%\hyperref{\fbox{\textbf{master 02540}}}{}{\fbox{\textbf{master 02540}}}{}}%
%BeginExpansion
\msihyperref{\fbox{\textbf{master 02540}}}{}{\fbox{\textbf{master 02540}}}{}%
%EndExpansion

\item[(b)] Draw a diagram illustrating the general situation. Introduce
notation and label the diagram with your symbols.

%TCIMACRO{%
%\hyperref{\fbox{\textbf{master 05080}}}{}{\fbox{\textbf{master 05080}}}{}}%
%BeginExpansion
\msihyperref{\fbox{\textbf{master 05080}}}{}{\fbox{\textbf{master 05080}}}{}%
%EndExpansion

\item[(c)] Write an expression for the volume.

%TCIMACRO{%
%\hyperref{\fbox{\textbf{master 05081}}}{}{\fbox{\textbf{master 05081}}}{}}%
%BeginExpansion
\msihyperref{\fbox{\textbf{master 05081}}}{}{\fbox{\textbf{master 05081}}}{}%
%EndExpansion

\item[(d)] Use the given information to write an equation that relates the
variables.

%TCIMACRO{%
%\hyperref{\fbox{\textbf{master 05082}}}{}{\fbox{\textbf{master 05082}}}{}}%
%BeginExpansion
\msihyperref{\fbox{\textbf{master 05082}}}{}{\fbox{\textbf{master 05082}}}{}%
%EndExpansion

\item[(e)] Use part (d) to write the volume as a function of one \linebreak
variable.

%TCIMACRO{%
%\hyperref{\fbox{\textbf{master 05083}}}{}{\fbox{\textbf{master 05083}}}{}}%
%BeginExpansion
\msihyperref{\fbox{\textbf{master 05083}}}{}{\fbox{\textbf{master 05083}}}{}%
%EndExpansion

\item[(f)] Finish solving the problem and compare the answer with your
estimate in part (a).

%TCIMACRO{%
%\hyperref{\fbox{\textbf{master 05084}}}{}{\fbox{\textbf{master 05084}}}{}}%
%BeginExpansion
\msihyperref{\fbox{\textbf{master 05084}}}{}{\fbox{\textbf{master 05084}}}{}%
%EndExpansion
\end{ExerciseList}

\item[\hfill 11.] A farmer wants to fence an area of 1.5 million square feet
in a rectangular field and then divide it in half with a fence parallel to
one of the sides of the rectangle. How can he do this so as to minimize the
cost of the fence?

%TCIMACRO{\hyperref{ANSWER}{}{\textbf{ANSWER:} 1000 ft by 1500 ft}{}}%
%BeginExpansion
\msihyperref{ANSWER}{}{\textbf{ANSWER:} 1000 ft by 1500 ft}{}%
%EndExpansion

%TCIMACRO{%
%\hyperref{\fbox{\textbf{master 02541}}}{}{\fbox{\textbf{master 02541}}}{}}%
%BeginExpansion
\msihyperref{\fbox{\textbf{master 02541}}}{}{\fbox{\textbf{master 02541}}}{}%
%EndExpansion

\item[\hfill 12.] A box with a square base and open top must have a volume
of 32,000 $\mathrm{cm}^{3}$. Find the dimensions of the box that minimize
the amount of material used.

%TCIMACRO{%
%\hyperref{\fbox{\textbf{master 02542}}}{}{\fbox{\textbf{master 02542}}}{}}%
%BeginExpansion
\msihyperref{\fbox{\textbf{master 02542}}}{}{\fbox{\textbf{master 02542}}}{}%
%EndExpansion

\item[{\hfill {\protect\fbox{\hspace{-2pt}13.\hspace{-2pt}}}}] If 1200 $%
\mathrm{cm}^{2}$ of material is available to make a box with a square base
and an open top, find the largest possible volume of the box.

%TCIMACRO{\hyperref{ANSWER}{}{\textbf{ANSWER:} 4000 cm$^{3}$}{}}%
%BeginExpansion
\msihyperref{ANSWER}{}{\textbf{ANSWER:} 4000 cm$^{3}$}{}%
%EndExpansion

%TCIMACRO{%
%\hyperref{\fbox{\textbf{master 02543}}}{}{\fbox{\textbf{master 02543}}}{}}%
%BeginExpansion
\msihyperref{\fbox{\textbf{master 02543}}}{}{\fbox{\textbf{master 02543}}}{}%
%EndExpansion

\item[\hfill 14.] A rectangular storage container with an open top is to
have a volume of 10 $\mathrm{m}^{3}$. The length of its base is twice the
width. Material for the base costs $\$10$ per square meter. Material for the
sides costs $\$6$ per square meter. Find the cost of materials for the
cheapest such container.

%TCIMACRO{%
%\hyperref{\fbox{\textbf{master 02544}}}{}{\fbox{\textbf{master 02544}}}{}}%
%BeginExpansion
\msihyperref{\fbox{\textbf{master 02544}}}{}{\fbox{\textbf{master 02544}}}{}%
%EndExpansion

\item[\hfill 15.] Do Exercise 14 assuming the container has a lid that is
made from the same material as the sides.

%TCIMACRO{\hyperref{ANSWER}{}{\textbf{ANSWER:} \$191.28}{}}%
%BeginExpansion
\msihyperref{ANSWER}{}{\textbf{ANSWER:} \$191.28}{}%
%EndExpansion

%TCIMACRO{%
%\hyperref{\fbox{\textbf{master 02545}}}{}{\fbox{\textbf{master 02545}}}{}}%
%BeginExpansion
\msihyperref{\fbox{\textbf{master 02545}}}{}{\fbox{\textbf{master 02545}}}{}%
%EndExpansion

\item[{\hfill {\protect\fbox{\hspace{-2pt}16.\hspace{-2pt}}}}] 

\begin{ExerciseList}
\item[(a)] Show that of all the rectangles with a given area, the one with
smallest perimeter is a square.

%TCIMACRO{%
%\hyperref{\fbox{\textbf{master 02546}}}{}{\fbox{\textbf{master 02546}}}{}}%
%BeginExpansion
\msihyperref{\fbox{\textbf{master 02546}}}{}{\fbox{\textbf{master 02546}}}{}%
%EndExpansion

\item[(b)] Show that of all the rectangles with a given perimeter, the one
with greatest area is a square.

%TCIMACRO{%
%\hyperref{\fbox{\textbf{master 05085}}}{}{\fbox{\textbf{master 05085}}}{}}%
%BeginExpansion
\msihyperref{\fbox{\textbf{master 05085}}}{}{\fbox{\textbf{master 05085}}}{}%
%EndExpansion
\end{ExerciseList}

\item[{\hfill {\protect\fbox{\hspace{-2pt}17.\hspace{-2pt}}}}] Find the point
on the line $y=4x+7$ that is closest to the origin.

%TCIMACRO{%
%\hyperref{ANSWER}{}{\textbf{ANSWER:} $\left( -\tfrac{28}{17},\tfrac{7}{17}\right) $}{}}%
%BeginExpansion
\msihyperref{ANSWER}{}{\textbf{ANSWER:} $\left( -\tfrac{28}{17},\tfrac{7}{17}\right) $}{}%
%EndExpansion

%TCIMACRO{%
%\hyperref{\fbox{\textbf{master 02547}}}{}{\fbox{\textbf{master 02547}}}{}}%
%BeginExpansion
\msihyperref{\fbox{\textbf{master 02547}}}{}{\fbox{\textbf{master 02547}}}{}%
%EndExpansion

\item[\hfill 18.] Find the point on the line $6x+y=9$ that is closest to the
point $(-3,1)$.

%TCIMACRO{%
%\hyperref{\fbox{\textbf{master 02548}}}{}{\fbox{\textbf{master 02548}}}{}}%
%BeginExpansion
\msihyperref{\fbox{\textbf{master 02548}}}{}{\fbox{\textbf{master 02548}}}{}%
%EndExpansion

\item[\hfill 19.] Find the points on the ellipse $4x^{2}+y^{2}=4$ that are
farthest away from the point $(1,0)$.

%TCIMACRO{%
%\hyperref{ANSWER}{}{\textbf{ANSWER:} $\left( -\tfrac{1}{3},\pm \tfrac{4}{3}\sqrt{2}\right) $}{}}%
%BeginExpansion
\msihyperref{ANSWER}{}{\textbf{ANSWER:} $\left( -\tfrac{1}{3},\pm \tfrac{4}{3}\sqrt{2}\right) $}{}%
%EndExpansion

%TCIMACRO{%
%\hyperref{\fbox{\textbf{master 02549}}}{}{\fbox{\textbf{master 02549}}}{}}%
%BeginExpansion
\msihyperref{\fbox{\textbf{master 02549}}}{}{\fbox{\textbf{master 02549}}}{}%
%EndExpansion

\item[\hfill 20.] 
%TCIMACRO{\TeXButton{GCALCX}{\GCALCX}}%
%BeginExpansion
\GCALCX%
%EndExpansion
Find, correct to two decimal places, the coordinates of the point on the
curve $y=\tan x$ that is closest to the point $(1,1)$.

%TCIMACRO{%
%\hyperref{\fbox{\textbf{master 02550}}}{}{\fbox{\textbf{master 02550}}}{}}%
%BeginExpansion
\msihyperref{\fbox{\textbf{master 02550}}}{}{\fbox{\textbf{master 02550}}}{}%
%EndExpansion

\item[\hfill 21.] Find the dimensions of the rectangle of largest area that
can be inscribed in a circle of radius $r$.

%TCIMACRO{%
%\hyperref{ANSWER}{}{\textbf{ANSWER:} Square, side $\sqrt{2}\,r$}{}}%
%BeginExpansion
\msihyperref{ANSWER}{}{\textbf{ANSWER:} Square, side $\sqrt{2}\,r$}{}%
%EndExpansion

%TCIMACRO{%
%\hyperref{\fbox{\textbf{master 02551}}}{}{\fbox{\textbf{master 02551}}}{}}%
%BeginExpansion
\msihyperref{\fbox{\textbf{master 02551}}}{}{\fbox{\textbf{master 02551}}}{}%
%EndExpansion

\item[{\hfill {\protect\fbox{\hspace{-2pt}22.\hspace{-2pt}}}}] Find the area
of the largest rectangle that can be inscribed in the ellipse $%
x^{2}/a^{2}+y^{2}/b^{2}=1$.

%TCIMACRO{%
%\hyperref{\fbox{\textbf{master 02552}}}{}{\fbox{\textbf{master 02552}}}{}}%
%BeginExpansion
\msihyperref{\fbox{\textbf{master 02552}}}{}{\fbox{\textbf{master 02552}}}{}%
%EndExpansion

\item[\hfill 23.] Find the dimensions of the rectangle of largest area that
can be inscribed in an equilateral triangle of side $L$ if one side of the
rectangle lies on the base of the triangle.

%TCIMACRO{\hyperref{ANSWER}{}{\textbf{ANSWER:} $L/2$, $\sqrt{3}L/4$}{}}%
%BeginExpansion
\msihyperref{ANSWER}{}{\textbf{ANSWER:} $L/2$, $\sqrt{3}L/4$}{}%
%EndExpansion

%TCIMACRO{%
%\hyperref{\fbox{\textbf{master 02553}}}{}{\fbox{\textbf{master 02553}}}{}}%
%BeginExpansion
\msihyperref{\fbox{\textbf{master 02553}}}{}{\fbox{\textbf{master 02553}}}{}%
%EndExpansion

\item[\hfill 24.] Find the dimensions of the rectangle of largest area that
has its base on the $x$-axis and its other two vertices above the $x$-axis
and lying on the parabola $y=8-x^{2}$.

%TCIMACRO{%
%\hyperref{\fbox{\textbf{master 02554}}}{}{\fbox{\textbf{master 02554}}}{}}%
%BeginExpansion
\msihyperref{\fbox{\textbf{master 02554}}}{}{\fbox{\textbf{master 02554}}}{}%
%EndExpansion

\item[\hfill 25.] Find the dimensions of the isosceles triangle of largest
area that can be inscribed in a circle of radius $r$.

%TCIMACRO{%
%\hyperref{ANSWER}{}{\textbf{ANSWER:} Base $\sqrt{3}$\thinspace $r$, height $3r/2$}{}}%
%BeginExpansion
\msihyperref{ANSWER}{}{\textbf{ANSWER:} Base $\sqrt{3}$\thinspace $r$, height $3r/2$}{}%
%EndExpansion

%TCIMACRO{%
%\hyperref{\fbox{\textbf{master 02555}}}{}{\fbox{\textbf{master 02555}}}{}}%
%BeginExpansion
\msihyperref{\fbox{\textbf{master 02555}}}{}{\fbox{\textbf{master 02555}}}{}%
%EndExpansion

\item[\hfill 26.] Find the area of the largest rectangle that can be
inscribed in a right triangle with legs of lengths 3 cm and 4 cm if two
sides of the rectangle lie along the legs.

%TCIMACRO{%
%\hyperref{\fbox{\textbf{master 02556}}}{}{\fbox{\textbf{master 02556}}}{}}%
%BeginExpansion
\msihyperref{\fbox{\textbf{master 02556}}}{}{\fbox{\textbf{master 02556}}}{}%
%EndExpansion

\item[\hfill 27.] A right circular cylinder is inscribed in a sphere of
radius $r$. Find the largest possible volume of such a cylinder.

%TCIMACRO{%
%\hyperref{ANSWER}{}{\textbf{ANSWER:} $4\pi r^{3}/\left( 3\sqrt{3}\right) $}{}}%
%BeginExpansion
\msihyperref{ANSWER}{}{\textbf{ANSWER:} $4\pi r^{3}/\left( 3\sqrt{3}\right) $}{}%
%EndExpansion

%TCIMACRO{%
%\hyperref{\fbox{\textbf{master 02557}}}{}{\fbox{\textbf{master 02557}}}{}}%
%BeginExpansion
\msihyperref{\fbox{\textbf{master 02557}}}{}{\fbox{\textbf{master 02557}}}{}%
%EndExpansion

\item[\hfill 28.] A right circular cylinder is inscribed in a cone with
height $h$ and base radius $r$. Find the largest possible volume of such a
cylinder.

%TCIMACRO{%
%\hyperref{\fbox{\textbf{master 02558}}}{}{\fbox{\textbf{master 02558}}}{}}%
%BeginExpansion
\msihyperref{\fbox{\textbf{master 02558}}}{}{\fbox{\textbf{master 02558}}}{}%
%EndExpansion

\item[\hfill 29.] A right circular cylinder is inscribed in a sphere of
radius $r$. Find the largest possible surface area of such a cylinder.

%TCIMACRO{%
%\hyperref{ANSWER}{}{\textbf{ANSWER:} $\pi r^{2}\left( 1+\sqrt{5}\right) $}{}}%
%BeginExpansion
\msihyperref{ANSWER}{}{\textbf{ANSWER:} $\pi r^{2}\left( 1+\sqrt{5}\right) $}{}%
%EndExpansion

%TCIMACRO{%
%\hyperref{\fbox{\textbf{master 02560}}}{}{\fbox{\textbf{master 02560}}}{}}%
%BeginExpansion
\msihyperref{\fbox{\textbf{master 02560}}}{}{\fbox{\textbf{master 02560}}}{}%
%EndExpansion

\item[{\hfill {\protect\fbox{\hspace{-2pt}30.\hspace{-2pt}}}}] A Norman
window has the shape of a rectangle surmounted by a semicircle. (Thus the
diameter of the semicircle is equal to the width of the rectangle. See
Exercise 56 on page~%
%TCIMACRO{\TeXButton{23}{23}}%
%BeginExpansion
23%
%EndExpansion
.) If the perimeter of the window is 30 ft, find the dimensions of the
window so that the greatest possible amount of light is admitted.

%TCIMACRO{%
%\hyperref{\fbox{\textbf{master 02561}}}{}{\fbox{\textbf{master 02561}}}{}}%
%BeginExpansion
\msihyperref{\fbox{\textbf{master 02561}}}{}{\fbox{\textbf{master 02561}}}{}%
%EndExpansion

\item[\hfill 31.] The top and bottom margins of a poster are each 6 cm and
the side margins are each 4 cm. If the area of printed material on the
poster is fixed at 384 $\mathrm{cm}^{2}$, find the dimensions of the poster
with the smallest area.

%TCIMACRO{\hyperref{ANSWER}{}{\textbf{ANSWER:} 24 cm, 36 cm}{}}%
%BeginExpansion
\msihyperref{ANSWER}{}{\textbf{ANSWER:} 24 cm, 36 cm}{}%
%EndExpansion

%TCIMACRO{%
%\hyperref{\fbox{\textbf{master 02562}}}{}{\fbox{\textbf{master 02562}}}{}}%
%BeginExpansion
\msihyperref{\fbox{\textbf{master 02562}}}{}{\fbox{\textbf{master 02562}}}{}%
%EndExpansion

\item[\hfill 32.] A poster is to have an area of 180 $\mathrm{in}^{2}$ with
1-inch margins at the bottom and sides and a 2-inch margin at the top. What
dimensions will give the largest printed area?

%TCIMACRO{%
%\hyperref{\fbox{\textbf{master 02563}}}{}{\fbox{\textbf{master 02563}}}{}}%
%BeginExpansion
\msihyperref{\fbox{\textbf{master 02563}}}{}{\fbox{\textbf{master 02563}}}{}%
%EndExpansion

\item[{\hfill {\protect\fbox{\hspace{-2pt}33.\hspace{-2pt}}}}] A piece of
wire 10 m long is cut into two pieces. One piece is bent into a square and
the other is bent into an equilateral tri-\linebreak angle. How should the
wire be cut so that the total area enclosed is \QTR{PartLetter}{(a)} a
maximum? \QTR{PartLetter}{(b)} A minimum?

%TCIMACRO{%
%\hyperref{ANSWER}{}{\textbf{ANSWER:} (a) Use all of the wire for the square\quad \newline
%(b) $40\sqrt{3}/\left( 9+4\sqrt{3}\right) $ m for the square}{}}%
%BeginExpansion
\msihyperref{ANSWER}{}{\textbf{ANSWER:} (a) Use all of the wire for the square\quad \newline
(b) $40\sqrt{3}/\left( 9+4\sqrt{3}\right) $ m for the square}{}%
%EndExpansion

%
%TECHARTS_DUMMY_BEGIN_TAG
%TCIMACRO{\hyperref{(a) = }{}{(a) = }{}}%
%BeginExpansion
\msihyperref{(a) = }{}{(a) = }{}%
%EndExpansion
%TCIMACRO{%
%\hyperref{\fbox{\textbf{master 02564}}\quad }{}{\fbox{\textbf{master 02564}}\quad }{}}%
%BeginExpansion
\msihyperref{\fbox{\textbf{master 02564}}\quad }{}{\fbox{\textbf{master 02564}}\quad }{}%
%EndExpansion
%TCIMACRO{\hyperref{(b) = }{}{(b) = }{}}%
%BeginExpansion
\msihyperref{(b) = }{}{(b) = }{}%
%EndExpansion
%TCIMACRO{%
%\hyperref{\fbox{\textbf{master 05086}}}{}{\fbox{\textbf{master 05086}}}{}}%
%BeginExpansion
\msihyperref{\fbox{\textbf{master 05086}}}{}{\fbox{\textbf{master 05086}}}{}%
%EndExpansion
%
%
%
%TECHARTS_DUMMY_END_TAG

\item[\hfill 34.] Answer Exercise 33 if one piece is bent into a square and
the other into a circle.

%TCIMACRO{%
%\hyperref{\fbox{\textbf{master 02565}}}{}{\fbox{\textbf{master 02565}}}{}}%
%BeginExpansion
\msihyperref{\fbox{\textbf{master 02565}}}{}{\fbox{\textbf{master 02565}}}{}%
%EndExpansion

\item[\hfill 35.] A cylindrical can without a top is made to contain $V$ $%
\mathrm{cm}^{3}$ of liquid. Find the dimensions that will minimize the cost
of the metal to make the can.

%TCIMACRO{%
%\hyperref{ANSWER}{}{\textbf{ANSWER:} $\mathrm{Height}=\mathrm{radius}=\sqrt[3]{V/\pi }$ cm}{}}%
%BeginExpansion
\msihyperref{ANSWER}{}{\textbf{ANSWER:} $\mathrm{Height}=\mathrm{radius}=\sqrt[3]{V/\pi }$ cm}{}%
%EndExpansion

%TCIMACRO{%
%\hyperref{\fbox{\textbf{master 02566}}}{}{\fbox{\textbf{master 02566}}}{}}%
%BeginExpansion
\msihyperref{\fbox{\textbf{master 02566}}}{}{\fbox{\textbf{master 02566}}}{}%
%EndExpansion

\item[\hfill 36.] A fence 8 ft tall runs parallel to a tall building at a
distance of 4 ft from the building. What is the length of the shortest
ladder that will reach from the ground over the fence to the wall of the
building?

%TCIMACRO{%
%\hyperref{\fbox{\textbf{master 02567}}}{}{\fbox{\textbf{master 02567}}}{}}%
%BeginExpansion
\msihyperref{\fbox{\textbf{master 02567}}}{}{\fbox{\textbf{master 02567}}}{}%
%EndExpansion

\item[\hfill 37.] A cone-shaped drinking cup is made from a circular piece
of paper of radius $R$ by cutting out a sector and joining the edges $CA$
and $CB$. Find the maximum capacity of such a cup.\\[4pt]
\hspace*{\fill}\FRAME{itbpF}{1.0888in}{1.126in}{0in}{}{}{4e0407x35.wmf}{%
\special{language "Scientific Word";type "GRAPHIC";maintain-aspect-ratio
TRUE;display "USEDEF";valid_file "F";width 1.0888in;height 1.126in;depth
0in;original-width 1.1805in;original-height 1.222in;cropleft "0";croptop
"1";cropright "1";cropbottom "0";filename
'graphics/4e0407x35.wmf';file-properties "XNPEU";}}\hspace*{\fill}

%TCIMACRO{%
%\hyperref{ANSWER}{}{\textbf{ANSWER:} $V=2\pi R^{3}/\left( 9\sqrt{3}\right) $}{}}%
%BeginExpansion
\msihyperref{ANSWER}{}{\textbf{ANSWER:} $V=2\pi R^{3}/\left( 9\sqrt{3}\right) $}{}%
%EndExpansion

%TCIMACRO{%
%\hyperref{\fbox{\textbf{master 02568}}}{}{\fbox{\textbf{master 02568}}}{}}%
%BeginExpansion
\msihyperref{\fbox{\textbf{master 02568}}}{}{\fbox{\textbf{master 02568}}}{}%
%EndExpansion

\item[\hfill 38.] A cone-shaped paper drinking cup is to be made to hold 27
cm$^{3}$ of water. Find the height and radius of the cup that will use the
smallest amount of paper.

%TCIMACRO{%
%\hyperref{\fbox{\textbf{master 02569}}}{}{\fbox{\textbf{master 02569}}}{}}%
%BeginExpansion
\msihyperref{\fbox{\textbf{master 02569}}}{}{\fbox{\textbf{master 02569}}}{}%
%EndExpansion

\item[\hfill 39.] A cone with height $h$ is inscribed in a larger cone with
height $H$ so that its vertex is at the center of the base of the larger
cone. Show that the inner cone has maximum volume when $h=\frac{1}{3}H$.

%TCIMACRO{\hyperref{ANSWER}{}{\textbf{ANSWER:} \textit{No text answer}}{}}%
%BeginExpansion
\msihyperref{ANSWER}{}{\textbf{ANSWER:} \textit{No text answer}}{}%
%EndExpansion

%TCIMACRO{%
%\hyperref{\fbox{\textbf{master 02570}}}{}{\fbox{\textbf{master 02570}}}{}}%
%BeginExpansion
\msihyperref{\fbox{\textbf{master 02570}}}{}{\fbox{\textbf{master 02570}}}{}%
%EndExpansion

\item[\hfill 40.] An object with weight $W$ is dragged along a horizontal
plane by a force acting along a rope attached to the object. If the rope
makes an angle $\theta $ with the plane, then the magnitude of the force is 
\\[6pt]
\hspace*{\fill}$F=\dfrac{\mu W}{\mu \,\sin \theta +\cos \theta }$\hspace*{%
\fill}\\[6pt]
where $\mu $ is a constant called the coefficient of friction. For what
value of $\theta $ is $F$ smallest?

%TCIMACRO{%
%\hyperref{\fbox{\textbf{master 80367}}}{}{\fbox{\textbf{master 80367}}}{}}%
%BeginExpansion
\msihyperref{\fbox{\textbf{master 80367}}}{}{\fbox{\textbf{master 80367}}}{}%
%EndExpansion

\item[\hfill 41.] If a resistor of $R$ ohms is connected across a battery of 
$E$ volts with internal resistance $r$ ohms, then the power (in watts) in
the external resistor is\\[6pt]
\hspace*{\fill}$P=\dfrac{E^{2}R}{(R+r)^{2}}$\hspace*{\fill}\\[6pt]
If $E$ and $r$ are fixed but $R$ varies, what is the maximum value of the
power?

%TCIMACRO{\hyperref{ANSWER}{}{\textbf{ANSWER:} $E^{2}/(4r)$}{}}%
%BeginExpansion
\msihyperref{ANSWER}{}{\textbf{ANSWER:} $E^{2}/(4r)$}{}%
%EndExpansion

%TCIMACRO{%
%\hyperref{\fbox{\textbf{master 30196}}}{}{\fbox{\textbf{master 30196}}}{}}%
%BeginExpansion
\msihyperref{\fbox{\textbf{master 30196}}}{}{\fbox{\textbf{master 30196}}}{}%
%EndExpansion

\item[\hfill 42.] For a fish swimming at a speed $v$ relative to the water,
the energy expenditure per unit time is proportional to $v^{3}$. It is
believed that migrating fish try to minimize the total energy required to
swim a fixed distance. If the fish are swimming against a current $u$ $(u<v)$%
, then the time required to swim a distance $L$ is $L/(v-u)$ and the total
energy $E$ required to swim the distance is given by\\[6pt]
\hspace*{\fill}$E(v)=av^{3}\cdot \dfrac{L}{v-u}$\hspace*{\fill}\\[6pt]
where $a$ is the proportionality constant.

\begin{ExerciseList}
\item[(a)] Determine the value of $v$ that minimizes $E$.

%TCIMACRO{%
%\hyperref{\fbox{\textbf{master 02571}}}{}{\fbox{\textbf{master 02571}}}{}}%
%BeginExpansion
\msihyperref{\fbox{\textbf{master 02571}}}{}{\fbox{\textbf{master 02571}}}{}%
%EndExpansion

\item[(b)] Sketch the graph of $E$.

%TCIMACRO{%
%\hyperref{\fbox{\textbf{master 05087}}}{}{\fbox{\textbf{master 05087}}}{}}%
%BeginExpansion
\msihyperref{\fbox{\textbf{master 05087}}}{}{\fbox{\textbf{master 05087}}}{}%
%EndExpansion
\vspace{6pt}
\end{ExerciseList}

\textit{Note:}\quad This result has been verified experimentally; migrating
fish swim against a current at a speed $50\%$ greater than the current speed.

\item[\hfill 43.] In a beehive, each cell is a regular hexagonal prism, open
at one end with a trihedral angle at the other end as in the figure. It is
believed that bees form their cells in such a way as to minimize the surface
area for a given volume, thus using the least amount of wax in cell
construction. Examination of these cells has shown that the measure of the
apex angle $\theta $ is amazingly consistent. Based on the geometry of the
cell, it can be shown that the surface area $S$ is given by\\[6pt]
\hspace*{\fill}$S=6sh-\tfrac{3}{2}s^{2}\cot \theta +\left( 3s^{2}\sqrt{3}%
/2\right) \csc \theta $\hspace*{\fill}\\[6pt]
where $s$, the length of the sides of the hexagon, and $h$, the height, are
constants.

\begin{ExerciseList}
\item[(a)] Calculate $dS/d\theta $.

%TCIMACRO{%
%\hyperref{ANSWER}{}{\textbf{ANSWER:} $\tfrac{3}{2}s^{2}\,\csc \theta \left( \csc \theta -\sqrt{3}\cot \theta \right) $}{}}%
%BeginExpansion
\msihyperref{ANSWER}{}{\textbf{ANSWER:} $\tfrac{3}{2}s^{2}\,\csc \theta \left( \csc \theta -\sqrt{3}\cot \theta \right) $}{}%
%EndExpansion

%TCIMACRO{%
%\hyperref{\fbox{\textbf{master 02572}}}{}{\fbox{\textbf{master 02572}}}{}}%
%BeginExpansion
\msihyperref{\fbox{\textbf{master 02572}}}{}{\fbox{\textbf{master 02572}}}{}%
%EndExpansion

\item[(b)] What angle should the bees prefer?

%TCIMACRO{%
%\hyperref{ANSWER}{}{\textbf{ANSWER:} $\cos ^{-1}\left( 1/\sqrt{3}\right) \approx 55\,^{\circ }$}{}}%
%BeginExpansion
\msihyperref{ANSWER}{}{\textbf{ANSWER:} $\cos ^{-1}\left( 1/\sqrt{3}\right) \approx 55\,^{\circ }$}{}%
%EndExpansion

%TCIMACRO{%
%\hyperref{\fbox{\textbf{master 05088}}}{}{\fbox{\textbf{master 05088}}}{}}%
%BeginExpansion
\msihyperref{\fbox{\textbf{master 05088}}}{}{\fbox{\textbf{master 05088}}}{}%
%EndExpansion

\item[(c)] Determine the minimum surface area of the cell (in terms of $s$
and $h$).

%TCIMACRO{%
%\hyperref{ANSWER}{}{\textbf{ANSWER:} $6s\left[ h+s/\left( 2\sqrt{2}\right) \right] $}{}}%
%BeginExpansion
\msihyperref{ANSWER}{}{\textbf{ANSWER:} $6s\left[ h+s/\left( 2\sqrt{2}\right) \right] $}{}%
%EndExpansion

%TCIMACRO{%
%\hyperref{\fbox{\textbf{master 05089}}}{}{\fbox{\textbf{master 05089}}}{}}%
%BeginExpansion
\msihyperref{\fbox{\textbf{master 05089}}}{}{\fbox{\textbf{master 05089}}}{}%
%EndExpansion
\vspace{4pt}
\end{ExerciseList}

\textit{Note:}\quad Actual measurements of the angle $\theta $ in beehives
have been made, and the measures of these angles seldom differ from the
calculated value by more than $2\,^{\circ }$.\\[6pt]
\hspace*{\fill}\FRAME{itbpF}{1.3335in}{1.7478in}{0in}{}{}{4e0407x37.wmf}{%
\special{language "Scientific Word";type "GRAPHIC";maintain-aspect-ratio
TRUE;display "USEDEF";valid_file "F";width 1.3335in;height 1.7478in;depth
0in;original-width 1.3335in;original-height 1.7478in;cropleft "0";croptop
"1";cropright "1";cropbottom "0";filename
'graphics/4e0407x37.wmf';file-properties "XNPEU";}}\hspace*{\fill}

%TCIMACRO{%
%\hyperref{ANSWER}{}{\textbf{ANSWER:} (a) $\tfrac{3}{2}s^{2}\csc \,\theta \left( \csc \,\theta -\sqrt{3}\cot \,\theta \right) $\quad \newline
%(b) $\cos ^{-1}\left( 1/\sqrt{3}\right) \approx 55\,^{\circ }$\quad (c) $6s\left[ h+s/\left( 2\sqrt{2}\right) \right] $}{}}%
%BeginExpansion
\msihyperref{ANSWER}{}{\textbf{ANSWER:} (a) $\tfrac{3}{2}s^{2}\csc \,\theta \left( \csc \,\theta -\sqrt{3}\cot \,\theta \right) $\quad \newline
(b) $\cos ^{-1}\left( 1/\sqrt{3}\right) \approx 55\,^{\circ }$\quad (c) $6s\left[ h+s/\left( 2\sqrt{2}\right) \right] $}{}%
%EndExpansion

\item[\hfill 44.] A boat leaves a dock at 2:00 \textsc{pm\thinspace }and
travels due south at a speed of 20 km$/$h. Another boat has been heading due
east at 15 km$/$h and reaches the same dock at 3:00 \textsc{pm}. At what
time were the two boats closest together?

%TCIMACRO{%
%\hyperref{\fbox{\textbf{master 02573}}}{}{\fbox{\textbf{master 02573}}}{}}%
%BeginExpansion
\msihyperref{\fbox{\textbf{master 02573}}}{}{\fbox{\textbf{master 02573}}}{}%
%EndExpansion

\item[\hfill 45.] Solve the problem in Example 4 if the river is 5 km wide
and point $B$ is only 5 km downstream from $A$.

%TCIMACRO{\hyperref{ANSWER}{}{\textbf{ANSWER:} Row directly to $B$}{}}%
%BeginExpansion
\msihyperref{ANSWER}{}{\textbf{ANSWER:} Row directly to $B$}{}%
%EndExpansion

%TCIMACRO{%
%\hyperref{\fbox{\textbf{master 02574}}}{}{\fbox{\textbf{master 02574}}}{}}%
%BeginExpansion
\msihyperref{\fbox{\textbf{master 02574}}}{}{\fbox{\textbf{master 02574}}}{}%
%EndExpansion
%TCIMACRO{\TeXButton{longpage}{\enlargethispage{12pt}}}%
%BeginExpansion
\enlargethispage{12pt}%
%EndExpansion

\item[\hfill 46.] A woman at a point $A$ on the shore of a circular lake
with radius 2 mi wants to arrive at the point $C$ diametrically opposite $A$
on the other side of the lake in the shortest possible time. She can walk at
the rate of 4 mi$/$h and row a boat at 2 mi$/$h. How should she proceed?\\[%
6pt]
\hspace*{\fill}\FRAME{itbpF}{1.4477in}{1.2306in}{0in}{}{}{4e0407x40.wmf}{%
\special{language "Scientific Word";type "GRAPHIC";maintain-aspect-ratio
TRUE;display "USEDEF";valid_file "F";width 1.4477in;height 1.2306in;depth
0in;original-width 1.5281in;original-height 1.2938in;cropleft "0";croptop
"1";cropright "0.9986";cropbottom "0";filename
'graphics/4e0407x40.wmf';file-properties "XNPEU";}}\hspace*{\fill}

%TCIMACRO{%
%\hyperref{\fbox{\textbf{master 02575}}}{}{\fbox{\textbf{master 02575}}}{}}%
%BeginExpansion
\msihyperref{\fbox{\textbf{master 02575}}}{}{\fbox{\textbf{master 02575}}}{}%
%EndExpansion

\item[\hfill 47.] An oil refinery is located on the north bank of a straight
river that is 2 km wide. A pipeline is to be constructed from the refinery
to storage tanks located on the south bank of the river 6 km east of the
refinery. The cost of laying pipe is \$400,000$/$km over land to a point $P$
on the north bank and \$800,000$/$km under the river to the tanks. To
minimize the cost of the pipeline, where should $P$ be located?

%TCIMACRO{%
%\hyperref{ANSWER}{}{\textbf{ANSWER:} $\approx 4.85$ km east of the refinery}{}}%
%BeginExpansion
\msihyperref{ANSWER}{}{\textbf{ANSWER:} $\approx 4.85$ km east of the refinery}{}%
%EndExpansion

%TCIMACRO{%
%\hyperref{\fbox{\textbf{master 80368}}}{}{\fbox{\textbf{master 80368}}}{}}%
%BeginExpansion
\msihyperref{\fbox{\textbf{master 80368}}}{}{\fbox{\textbf{master 80368}}}{}%
%EndExpansion

\item[\hfill 48.] 
%TCIMACRO{\TeXButton{GCALCX}{\GCALCX}}%
%BeginExpansion
\GCALCX%
%EndExpansion
Suppose the refinery in Exercise 47 is located 1 km north of the river.
Where should $P$ be located?

%TCIMACRO{%
%\hyperref{\fbox{\textbf{master 80369}}}{}{\fbox{\textbf{master 80369}}}{}}%
%BeginExpansion
\msihyperref{\fbox{\textbf{master 80369}}}{}{\fbox{\textbf{master 80369}}}{}%
%EndExpansion

\item[{\hfill {\protect\fbox{\hspace{-2pt}49.\hspace{-2pt}}}}] The
illumination of an object by a light source is directly proportional to the
strength of the source and inversely proportional to the square of the
distance from the source. If two light sources, one three times as strong as
the other, are placed 10 ft apart, where should an object be placed on the
line between the sources so as to receive the least illumination?

%TCIMACRO{%
%\hyperref{ANSWER}{}{\textbf{ANSWER:} $10\sqrt[3]{3}/\left( 1+\sqrt[3]{3}\right) $ ft from the stronger source}{}}%
%BeginExpansion
\msihyperref{ANSWER}{}{\textbf{ANSWER:} $10\sqrt[3]{3}/\left( 1+\sqrt[3]{3}\right) $ ft from the stronger source}{}%
%EndExpansion

%TCIMACRO{%
%\hyperref{\fbox{\textbf{master 02576}}}{}{\fbox{\textbf{master 02576}}}{}}%
%BeginExpansion
\msihyperref{\fbox{\textbf{master 02576}}}{}{\fbox{\textbf{master 02576}}}{}%
%EndExpansion

\item[{\hfill {\protect\fbox{\hspace{-2pt}50.\hspace{-2pt}}}}] Find an
equation of the line through the point $(3,5)$ that cuts off the least area
from the first quadrant.

%TCIMACRO{%
%\hyperref{\fbox{\textbf{master 02577}}}{}{\fbox{\textbf{master 02577}}}{}}%
%BeginExpansion
\msihyperref{\fbox{\textbf{master 02577}}}{}{\fbox{\textbf{master 02577}}}{}%
%EndExpansion

\item[\hfill 51.] Let $a$ and $b$ be positive numbers. Find the length of
the shortest line segment that is cut off by the first quadrant and passes
through the point $(a,b)$.

%TCIMACRO{%
%\hyperref{ANSWER}{}{\textbf{ANSWER:} $\left( a^{2/3}+b^{2/3}\right) ^{3/2}$}{}}%
%BeginExpansion
\msihyperref{ANSWER}{}{\textbf{ANSWER:} $\left( a^{2/3}+b^{2/3}\right) ^{3/2}$}{}%
%EndExpansion

%TCIMACRO{%
%\hyperref{\fbox{\textbf{master 02578}}}{}{\fbox{\textbf{master 02578}}}{}}%
%BeginExpansion
\msihyperref{\fbox{\textbf{master 02578}}}{}{\fbox{\textbf{master 02578}}}{}%
%EndExpansion

\item[\hfill 52.] At which points on the curve $y=1+40x^{3}-3x^{5}$ does the
tangent line have the largest slope?

%TCIMACRO{%
%\hyperref{\fbox{\textbf{master 02579}}}{}{\fbox{\textbf{master 02579}}}{}}%
%BeginExpansion
\msihyperref{\fbox{\textbf{master 02579}}}{}{\fbox{\textbf{master 02579}}}{}%
%EndExpansion

\item[\hfill 53.] 

\begin{ExerciseList}
\item[(a)] If $C(x)$ is the cost of producing $x$ units of a commodity, then
the \textbf{average cost} per unit is $c(x)=C(x)/x$. Show that if the
average cost is a minimum, then the marginal cost equals the average cost.

%TCIMACRO{%
%\hyperref{\fbox{\textbf{master 40579}}}{}{\fbox{\textbf{master 40579}}}{}}%
%BeginExpansion
\msihyperref{\fbox{\textbf{master 40579}}}{}{\fbox{\textbf{master 40579}}}{}%
%EndExpansion

\item[(b)] If $C(x)=16$,$000+200x+4x^{3/2}$, in dollars, find{\small \ }(i)\
the cost, average cost, and marginal cost at a production level of 1000
units; (ii)\ the production level that will minimize the average cost; and
(iii)\ the minimum average cost.
\end{ExerciseList}

%TCIMACRO{%
%\hyperref{ANSWER}{}{\textbf{ANSWER:} (b) (i) $\$342$,$491$; $\$342/$unit; $\$390/$unit\quad \newline
%(ii) 400\quad (iii) $\$320/$unit}{}}%
%BeginExpansion
\msihyperref{ANSWER}{}{\textbf{ANSWER:} (b) (i) $\$342$,$491$; $\$342/$unit; $\$390/$unit\quad \newline
(ii) 400\quad (iii) $\$320/$unit}{}%
%EndExpansion

%
%TECHARTS_DUMMY_BEGIN_TAG
%TCIMACRO{\hyperref{(i) = }{}{(i) = }{}}%
%BeginExpansion
\msihyperref{(i) = }{}{(i) = }{}%
%EndExpansion
%TCIMACRO{%
%\hyperref{\fbox{\textbf{master 02608}}\qquad }{}{\fbox{\textbf{master 02608}}\qquad }{}}%
%BeginExpansion
\msihyperref{\fbox{\textbf{master 02608}}\qquad }{}{\fbox{\textbf{master 02608}}\qquad }{}%
%EndExpansion
%TCIMACRO{\hyperref{(ii) = }{}{(ii) = }{}}%
%BeginExpansion
\msihyperref{(ii) = }{}{(ii) = }{}%
%EndExpansion
%TCIMACRO{%
%\hyperref{\fbox{\textbf{master 05106}}}{}{\fbox{\textbf{master 05106}}}{}}%
%BeginExpansion
\msihyperref{\fbox{\textbf{master 05106}}}{}{\fbox{\textbf{master 05106}}}{}%
%EndExpansion

%TCIMACRO{\hyperref{(iii) = }{}{(iii) = }{}}%
%BeginExpansion
\msihyperref{(iii) = }{}{(iii) = }{}%
%EndExpansion
%TCIMACRO{%
%\hyperref{\fbox{\textbf{master 05107}}}{}{\fbox{\textbf{master 05107}}}{}}%
%BeginExpansion
\msihyperref{\fbox{\textbf{master 05107}}}{}{\fbox{\textbf{master 05107}}}{}%
%EndExpansion
%
%
%
%TECHARTS_DUMMY_END_TAG

\item[\hfill 54.] 

\begin{ExerciseList}
\item[(a)] Show that if the profit $P(x)$ is a maximum, then the marginal
revenue equals the marginal cost.

%TCIMACRO{%
%\hyperref{\fbox{\textbf{master 70077}}}{}{\fbox{\textbf{master 70077}}}{}}%
%BeginExpansion
\msihyperref{\fbox{\textbf{master 70077}}}{}{\fbox{\textbf{master 70077}}}{}%
%EndExpansion

\item[(b)] If $C(x)=16$,$000+500x-1.6x^{2}+0.004x^{3}$ is the cost function
and $p(x)=1700-7x$ is the demand function, find the production level that
will maximize profit.

%TCIMACRO{%
%\hyperref{\fbox{\textbf{master 70112}}}{}{\fbox{\textbf{master 70112}}}{}}%
%BeginExpansion
\msihyperref{\fbox{\textbf{master 70112}}}{}{\fbox{\textbf{master 70112}}}{}%
%EndExpansion
\end{ExerciseList}

\item[{\hfill {\protect\fbox{\hspace{-2pt}55.\hspace{-2pt}}}}] A baseball
team plays in a stadium that holds 55,000 spectators. With ticket prices at
\$10, the average attendance had been 27,000. When ticket prices were
lowered to \$8, the average attendance rose to 33,000.

\begin{ExerciseList}
\item[(a)] Find the demand function, assuming that it is linear.

%TCIMACRO{%
%\hyperref{ANSWER}{}{\textbf{ANSWER:} (a) $p\left( x\right) =19-\tfrac{1}{3000}x$}{}}%
%BeginExpansion
\msihyperref{ANSWER}{}{\textbf{ANSWER:} (a) $p\left( x\right) =19-\tfrac{1}{3000}x$}{}%
%EndExpansion

%TCIMACRO{%
%\hyperref{\fbox{\textbf{master 02621}}}{}{\fbox{\textbf{master 02621}}}{}}%
%BeginExpansion
\msihyperref{\fbox{\textbf{master 02621}}}{}{\fbox{\textbf{master 02621}}}{}%
%EndExpansion

\item[(b)] How should ticket prices be set to maximize revenue?

%TCIMACRO{\hyperref{ANSWER}{}{\textbf{ANSWER:} (b) \$9.50}{}}%
%BeginExpansion
\msihyperref{ANSWER}{}{\textbf{ANSWER:} (b) \$9.50}{}%
%EndExpansion

%TCIMACRO{%
%\hyperref{\fbox{\textbf{master 05118}}}{}{\fbox{\textbf{master 05118}}}{}}%
%BeginExpansion
\msihyperref{\fbox{\textbf{master 05118}}}{}{\fbox{\textbf{master 05118}}}{}%
%EndExpansion
\end{ExerciseList}

\item[\hfill 56.] During the summer months Terry makes and sells necklaces
on the beach. Last summer he sold the necklaces for \$10 each and his sales
averaged 20 per day. When he increased the price by \$1, he found that the
average decreased by two sales per day.

\begin{ExerciseList}
\item[(a)] Find the demand function, assuming that it is linear.

%TCIMACRO{%
%\hyperref{\fbox{\textbf{master 02622}}}{}{\fbox{\textbf{master 02622}}}{}}%
%BeginExpansion
\msihyperref{\fbox{\textbf{master 02622}}}{}{\fbox{\textbf{master 02622}}}{}%
%EndExpansion

\item[(b)] If the material for each necklace costs Terry \$6, what should
the selling price be to maximize his profit?

%TCIMACRO{%
%\hyperref{\fbox{\textbf{master 05119}}}{}{\fbox{\textbf{master 05119}}}{}}%
%BeginExpansion
\msihyperref{\fbox{\textbf{master 05119}}}{}{\fbox{\textbf{master 05119}}}{}%
%EndExpansion
\end{ExerciseList}

\item[\hfill 57.] A manufacturer has been selling 1000 television sets a
week at \$450 each. A market survey indicates that for each \$10 rebate
offered to the buyer, the number of sets sold will increase by 100 per week.

\begin{ExerciseList}
\item[(a)] Find the demand function.

%TCIMACRO{%
%\hyperref{ANSWER}{}{\textbf{ANSWER:} (a) $p\left( x\right) =550-\tfrac{1}{10}x$}{}}%
%BeginExpansion
\msihyperref{ANSWER}{}{\textbf{ANSWER:} (a) $p\left( x\right) =550-\tfrac{1}{10}x$}{}%
%EndExpansion

%TCIMACRO{%
%\hyperref{\fbox{\textbf{master 02623}}}{}{\fbox{\textbf{master 02623}}}{}}%
%BeginExpansion
\msihyperref{\fbox{\textbf{master 02623}}}{}{\fbox{\textbf{master 02623}}}{}%
%EndExpansion

\item[(b)] How large a rebate should the company offer the buyer in order to
maximize its revenue?

%TCIMACRO{\hyperref{ANSWER}{}{\textbf{ANSWER:} (b) \$175}{}}%
%BeginExpansion
\msihyperref{ANSWER}{}{\textbf{ANSWER:} (b) \$175}{}%
%EndExpansion

%TCIMACRO{%
%\hyperref{\fbox{\textbf{master 05120}}}{}{\fbox{\textbf{master 05120}}}{}}%
%BeginExpansion
\msihyperref{\fbox{\textbf{master 05120}}}{}{\fbox{\textbf{master 05120}}}{}%
%EndExpansion

\item[(c)] If its weekly cost function is $C(x)=68$,$000+150x$, how should
the manufacturer set the size of the rebate in order to maximize its profit?

%TCIMACRO{\hyperref{ANSWER}{}{\textbf{ANSWER:} (c) \$100}{}}%
%BeginExpansion
\msihyperref{ANSWER}{}{\textbf{ANSWER:} (c) \$100}{}%
%EndExpansion

%TCIMACRO{%
%\hyperref{\fbox{\textbf{master 05121}}}{}{\fbox{\textbf{master 05121}}}{}}%
%BeginExpansion
\msihyperref{\fbox{\textbf{master 05121}}}{}{\fbox{\textbf{master 05121}}}{}%
%EndExpansion
\end{ExerciseList}

\item[\hfill 58.] The manager of a 100-unit apartment complex knows from
experience that all units will be occupied if the rent is \$800 per month. A
market survey suggests that, on average, one additional unit will remain
vacant for each \$10 increase in rent. What rent should the manager charge
to maximize revenue?

%TCIMACRO{%
%\hyperref{\fbox{\textbf{master 02624}}}{}{\fbox{\textbf{master 02624}}}{}}%
%BeginExpansion
\msihyperref{\fbox{\textbf{master 02624}}}{}{\fbox{\textbf{master 02624}}}{}%
%EndExpansion

\item[\hfill 59.] Show that of all the isosceles triangles with a given
perimeter, the one with the greatest area is equilateral.

%TCIMACRO{\hyperref{ANSWER}{}{\textbf{ANSWER:} \textit{No answer}}{}}%
%BeginExpansion
\msihyperref{ANSWER}{}{\textbf{ANSWER:} \textit{No answer}}{}%
%EndExpansion

%TCIMACRO{%
%\hyperref{\fbox{\textbf{master 02580}}}{}{\fbox{\textbf{master 02580}}}{}}%
%BeginExpansion
\msihyperref{\fbox{\textbf{master 02580}}}{}{\fbox{\textbf{master 02580}}}{}%
%EndExpansion

\item[\hfill 60.] 
%TCIMACRO{\TeXButton{CASX}{\CASX}}%
%BeginExpansion
\CASX%
%EndExpansion
The frame for a kite is to be made from six pieces of wood. The four
exterior pieces have been cut with the lengths indicated in the figure. To
maximize the area of the kite, how long should the diagonal pieces be?\\[6pt]
\hspace*{\fill}\FRAME{itbpF}{1.5835in}{1.0257in}{0in}{}{}{5et0407x48.ai}{%
\special{language "Scientific Word";type "GRAPHIC";maintain-aspect-ratio
TRUE;display "USEDEF";valid_file "F";width 1.5835in;height 1.0257in;depth
0in;original-width 1.5835in;original-height 1.0257in;cropleft "0";croptop
"1";cropright "1";cropbottom "0";filename
'graphics/5et0407x48.ai';file-properties "XNPEU";}}\hspace*{\fill}

%TCIMACRO{%
%\hyperref{\fbox{\textbf{master 02581}}}{}{\fbox{\textbf{master 02581}}}{}}%
%BeginExpansion
\msihyperref{\fbox{\textbf{master 02581}}}{}{\fbox{\textbf{master 02581}}}{}%
%EndExpansion

\item[\hfill 61.] 
%TCIMACRO{\TeXButton{GCALCX}{\GCALCX}}%
%BeginExpansion
\GCALCX%
%EndExpansion
A point $P$ needs to be located somewhere on the line $AD$ so that the total
length $L$ of cables linking $P$ to the points $A$, $B$, and $C$ is
minimized (see the figure). Express $L$ as a function of $x=\left\vert
AP\right\vert $ and use the graphs of $L$ and $dL/dx$ to estimate the
minimum value.\\[6pt]
\hspace*{\fill}\FRAME{itbpF}{1.6527in}{1.4589in}{0in}{}{}{4e0407x45.wmf}{%
\special{language "Scientific Word";type "GRAPHIC";maintain-aspect-ratio
TRUE;display "USEDEF";valid_file "F";width 1.6527in;height 1.4589in;depth
0in;original-width 1.6527in;original-height 1.4589in;cropleft "0";croptop
"1";cropright "1";cropbottom "0";filename
'graphics/4e0407x45.wmf';file-properties "XNPEU";}}\hspace*{\fill}

%TCIMACRO{\hyperref{ANSWER}{}{\textbf{ANSWER:} $9.35$ m}{}}%
%BeginExpansion
\msihyperref{ANSWER}{}{\textbf{ANSWER:} $9.35$ m}{}%
%EndExpansion

%TCIMACRO{%
%\hyperref{\fbox{\textbf{master 02582}}}{}{\fbox{\textbf{master 02582}}}{}}%
%BeginExpansion
\msihyperref{\fbox{\textbf{master 02582}}}{}{\fbox{\textbf{master 02582}}}{}%
%EndExpansion

\item[\hfill 62.] The graph shows the fuel consumption $c$ of a car
(measured in gallons per hour) as a function of the speed $v$ of the car. At
very low speeds the engine runs inefficiently, so initially $c$ decreases as
the speed increases. But at high speeds the fuel consumption increases. You
can see that $c(v)$ is minimized for this car when $v\approx 30$ mi$/$h.
However, for fuel efficiency, what must be minimized is not the consumption
in gallons per hour but rather the fuel consumption in gallons \textit{per
mile.} Let's call this consumption $G$. Using the graph, estimate the speed
at which $G$ has its minimum value.\\[6pt]
\hspace*{\fill}\FRAME{itbpF}{1.5558in}{1.1416in}{0in}{}{}{4e0407x46.wmf}{%
\special{language "Scientific Word";type "GRAPHIC";maintain-aspect-ratio
TRUE;display "USEDEF";valid_file "F";width 1.5558in;height 1.1416in;depth
0in;original-width 1.5558in;original-height 1.1416in;cropleft "0";croptop
"1";cropright "1";cropbottom "0";filename
'graphics/4e0407x46.wmf';file-properties "XNPEU";}}\hspace*{\fill}

%TCIMACRO{%
%\hyperref{\fbox{\textbf{master 02583}}}{}{\fbox{\textbf{master 02583}}}{}}%
%BeginExpansion
\msihyperref{\fbox{\textbf{master 02583}}}{}{\fbox{\textbf{master 02583}}}{}%
%EndExpansion

\item[{\hfill {\protect\fbox{\hspace{-2pt}63.\hspace{-2pt}}}}] Let $v_{1}$ be
the velocity of light in air and $v_{2}$ the velocity of light in water.
According to Fermat's Principle, a ray of light will travel from a point $A$
in the air to a point $B$ in the water by a path $ACB$ that minimizes the
time taken. Show that\\[6pt]
\hspace*{\fill}$\dfrac{\sin \theta _{1}}{\sin \theta _{2}}=\dfrac{v_{1}}{%
v_{2}}$\hspace*{\fill}\\[6pt]
where $\theta _{1}$ (the angle of incidence) and $\theta _{2}$ (the angle of
refraction) are as shown. This equation is known as Snell's Law.\\[6pt]
\hspace*{\fill}\FRAME{itbpF}{2.0003in}{1.4053in}{0in}{}{}{4e0407x47.wmf}{%
\special{language "Scientific Word";type "GRAPHIC";maintain-aspect-ratio
TRUE;display "USEDEF";valid_file "F";width 2.0003in;height 1.4053in;depth
0in;original-width 2.0003in;original-height 1.4053in;cropleft "0";croptop
"1";cropright "1";cropbottom "0";filename
'graphics/4e0407x47.wmf';file-properties "XNPEU";}}\hspace*{\fill}

%TCIMACRO{\hyperref{ANSWER}{}{\textbf{ANSWER:} \textit{no answer}}{}}%
%BeginExpansion
\msihyperref{ANSWER}{}{\textbf{ANSWER:} \textit{no answer}}{}%
%EndExpansion

%TCIMACRO{%
%\hyperref{\fbox{\textbf{master 02584}}}{}{\fbox{\textbf{master 02584}}}{}}%
%BeginExpansion
\msihyperref{\fbox{\textbf{master 02584}}}{}{\fbox{\textbf{master 02584}}}{}%
%EndExpansion

\item[\hfill 64.] Two vertical poles $PQ$ and $ST$ are secured by a rope $%
PRS $ going from the top of the first pole to a point $R$ on the ground
between the poles and then to the top of the second pole as in the figure.
Show that the shortest length of such a rope occurs when $\theta _{1}=\theta
_{2}$.\\[6pt]
\hspace*{\fill}\FRAME{itbpF}{2.1525in}{1.4607in}{0in}{}{}{4e0407x48.wmf}{%
\special{language "Scientific Word";type "GRAPHIC";maintain-aspect-ratio
TRUE;display "USEDEF";valid_file "F";width 2.1525in;height 1.4607in;depth
0in;original-width 2.1525in;original-height 1.4607in;cropleft "0";croptop
"1";cropright "1";cropbottom "0";filename
'graphics/4e0407x48.wmf';file-properties "XNPEU";}}\hspace*{\fill}

%TCIMACRO{%
%\hyperref{\fbox{\textbf{master 02585}}}{}{\fbox{\textbf{master 02585}}}{}}%
%BeginExpansion
\msihyperref{\fbox{\textbf{master 02585}}}{}{\fbox{\textbf{master 02585}}}{}%
%EndExpansion

\item[\hfill 65.] The upper right-hand corner of a piece of paper, 12 in. by
8 in., as in the figure, is folded over to the bottom edge. How would you
fold it so as to minimize the length of the fold? In other words, how would
you choose $x$ to minimize $y$?\\[6pt]
\hspace*{\fill}\FRAME{itbpF}{1.8706in}{1.241in}{0in}{}{}{5et0407x53.wmf}{%
\special{language "Scientific Word";type "GRAPHIC";maintain-aspect-ratio
TRUE;display "USEDEF";valid_file "F";width 1.8706in;height 1.241in;depth
0in;original-width 1.8334in;original-height 1.2064in;cropleft "0";croptop
"1";cropright "1";cropbottom "0";filename
'graphics/5et0407x53.wmf';file-properties "XNPEU";}}\hspace*{\fill}

%TCIMACRO{\hyperref{ANSWER}{}{\textbf{ANSWER:} $x=6$ in.}{}}%
%BeginExpansion
\msihyperref{ANSWER}{}{\textbf{ANSWER:} $x=6$ in.}{}%
%EndExpansion

%TCIMACRO{%
%\hyperref{\fbox{\textbf{master 02586}}}{}{\fbox{\textbf{master 02586}}}{}}%
%BeginExpansion
\msihyperref{\fbox{\textbf{master 02586}}}{}{\fbox{\textbf{master 02586}}}{}%
%EndExpansion

\item[\hfill 66.] A steel pipe is being carried down a hallway 9 ft wide. At
the end of the hall there is a right-angled turn into a narrower hallway 6
ft wide. What is the length of the longest pipe that can be carried
horizontally around the corner?\\[6pt]
\hspace*{\fill}\FRAME{itbpF}{1.7167in}{1.1571in}{0in}{}{}{4e0407x50.ai}{%
\special{language "Scientific Word";type "GRAPHIC";maintain-aspect-ratio
TRUE;display "USEDEF";valid_file "F";width 1.7167in;height 1.1571in;depth
0in;original-width 1.6803in;original-height 1.1234in;cropleft "0";croptop
"1";cropright "1";cropbottom "0";filename
'../../6eSWfiles/6eCh04/graphics/4e0407x50.ai';file-properties "XNPEU";}}%
\hspace*{\fill}

%TCIMACRO{%
%\hyperref{\fbox{\textbf{master 02587}}}{}{\fbox{\textbf{master 02587}}}{}}%
%BeginExpansion
\msihyperref{\fbox{\textbf{master 02587}}}{}{\fbox{\textbf{master 02587}}}{}%
%EndExpansion

\item[\hfill 67.] An observer stands at a point $P$, one unit away from a
track. Two runners start at the point $S$ in the figure and run along the
track. One runner runs three times as fast as the other. Find the maximum
value of the observer's angle of sight $\theta $ between the runners. [%
\textit{Hint:} Maximize $\tan \theta $.]\\[6pt]
\hspace*{\fill}\FRAME{itbpF}{1.2081in}{1.2358in}{0in}{}{}{3c04rx42n.wmf}{%
\special{language "Scientific Word";type "GRAPHIC";maintain-aspect-ratio
TRUE;display "USEDEF";valid_file "F";width 1.2081in;height 1.2358in;depth
0in;original-width 1.2081in;original-height 1.2358in;cropleft "0";croptop
"1";cropright "1";cropbottom "0";filename
'graphics/3c04rx42n.wmf';file-properties "XNPEU";}}\hspace*{\fill}

%TCIMACRO{\hyperref{ANSWER}{}{\textbf{ANSWER:} $\pi /6$}{}}%
%BeginExpansion
\msihyperref{ANSWER}{}{\textbf{ANSWER:} $\pi /6$}{}%
%EndExpansion

%TCIMACRO{%
%\hyperref{\fbox{\textbf{master 02588}}}{}{\fbox{\textbf{master 02588}}}{}}%
%BeginExpansion
\msihyperref{\fbox{\textbf{master 02588}}}{}{\fbox{\textbf{master 02588}}}{}%
%EndExpansion

\item[\hfill 68.] A rain gutter is to be constructed from a metal sheet of
width 30~cm by bending up one-third of the sheet on each side through an
angle $\theta $. How should $\theta $ be chosen so that the gutter will
carry the maximum amount of water?\\[6pt]
\hspace*{\fill}\FRAME{itbpF}{2.0141in}{0.7143in}{0in}{}{}{4e0407x52.wmf}{%
\special{language "Scientific Word";type "GRAPHIC";maintain-aspect-ratio
TRUE;display "USEDEF";valid_file "F";width 2.0141in;height 0.7143in;depth
0in;original-width 2.0141in;original-height 0.7143in;cropleft "0";croptop
"1";cropright "1";cropbottom "0";filename
'graphics/4e0407x52.wmf';file-properties "XNPEU";}}\hspace*{\fill}

%TCIMACRO{%
%\hyperref{\fbox{\textbf{master 02589}}}{}{\fbox{\textbf{master 02589}}}{}}%
%BeginExpansion
\msihyperref{\fbox{\textbf{master 02589}}}{}{\fbox{\textbf{master 02589}}}{}%
%EndExpansion

\item[{\hfill {\protect\fbox{\hspace{-2pt}69.\hspace{-2pt}}}}] Where should
the point $P$ be chosen on the line segment $AB$ so as to maximize the angle 
$\theta $?\\[6pt]
\hspace*{\fill}\FRAME{itbpF}{1.7919in}{1.2202in}{0in}{}{}{5et0407x57.wmf}{%
\special{language "Scientific Word";type "GRAPHIC";maintain-aspect-ratio
TRUE;display "USEDEF";valid_file "F";width 1.7919in;height 1.2202in;depth
0in;original-width 1.7919in;original-height 1.2202in;cropleft "0";croptop
"1";cropright "1";cropbottom "0";filename
'graphics/5et0407x57.wmf';file-properties "XNPEU";}}\hspace*{\fill}

%TCIMACRO{%
%\hyperref{ANSWER}{}{\textbf{ANSWER:} At a distance $5-2\sqrt{5}$ from $A$}{}}%
%BeginExpansion
\msihyperref{ANSWER}{}{\textbf{ANSWER:} At a distance $5-2\sqrt{5}$ from $A$}{}%
%EndExpansion

%TCIMACRO{%
%\hyperref{\fbox{\textbf{master 02590}}}{}{\fbox{\textbf{master 02590}}}{}}%
%BeginExpansion
\msihyperref{\fbox{\textbf{master 02590}}}{}{\fbox{\textbf{master 02590}}}{}%
%EndExpansion

\item[\hfill 70.] A painting in an art gallery has height $h$ and is hung so
that its lower edge is a distance $d$ above the eye of an observer (as in
the figure). How far from the wall should the observer stand to get the best
view? (In other words, where should the observer stand so as to maximize the
angle $\theta $ subtended at his eye by the painting?)\\[6pt]
\hspace*{\fill}\FRAME{itbpF}{1.8196in}{0.742in}{0in}{}{}{5et0407x58.wmf}{%
\special{language "Scientific Word";type "GRAPHIC";maintain-aspect-ratio
TRUE;display "USEDEF";valid_file "F";width 1.8196in;height 0.742in;depth
0in;original-width 1.8196in;original-height 0.742in;cropleft "0";croptop
"1";cropright "1";cropbottom "0";filename
'graphics/5et0407x58.wmf';file-properties "XNPEU";}}\hspace*{\fill}

%TCIMACRO{%
%\hyperref{\fbox{\textbf{master 02591}}}{}{\fbox{\textbf{master 02591}}}{}}%
%BeginExpansion
\msihyperref{\fbox{\textbf{master 02591}}}{}{\fbox{\textbf{master 02591}}}{}%
%EndExpansion

\item[\hfill 71.] Find the maximum area of a rectangle that can be
circumscribed about a given rectangle with length $L$ and width $W$. [%
\textit{Hint:}%
%TCIMACRO{\TeXButton{en}{\enskip}}%
%BeginExpansion
\enskip%
%EndExpansion
Express the area as a function of an angle $\theta $.]

%TCIMACRO{\hyperref{ANSWER}{}{\textbf{ANSWER:} $\tfrac{1}{2}(L+W)^{2}$}{}}%
%BeginExpansion
\msihyperref{ANSWER}{}{\textbf{ANSWER:} $\tfrac{1}{2}(L+W)^{2}$}{}%
%EndExpansion

%TCIMACRO{%
%\hyperref{\fbox{\textbf{master 02592}}}{}{\fbox{\textbf{master 02592}}}{}}%
%BeginExpansion
\msihyperref{\fbox{\textbf{master 02592}}}{}{\fbox{\textbf{master 02592}}}{}%
%EndExpansion

\item[\hfill 72.] The blood vascular system consists of blood vessels
(arteries, arterioles, capillaries, and veins) that convey blood from the
heart to the organs and back to the heart. This system should work so as to
minimize the energy expended by the heart in pumping the blood. In
particular, this energy is reduced when the resistance of the blood is
lowered. One of Poiseuille's Laws gives the resistance $R$ of the blood as\\[%
6pt]
\hspace*{\fill}$R=C\dfrac{L}{r^{4}}$\hspace*{\fill}\\[6pt]
where $L$ is the length of the blood vessel, $r$ is the radius, and $C$ is a
positive constant determined by the viscosity of the blood. (Poiseuille
established this law experimentally, but it also follows from Equation 
%TCIMACRO{\TeXButton{9.4.2/ 8.4.2}{\ifnum\ET=0 9.4.2\else 8.4.2\fi}}%
%BeginExpansion
\ifnum\ET=0 9.4.2\else 8.4.2\fi%
%EndExpansion
.) The figure shows a main blood vessel with radius $r_{1}$ branching at an
angle $\theta $ into a smaller vessel with radius $r_{2}$.\\[6pt]
\hspace*{\fill}\FRAME{itbpF}{3.1799in}{1.6553in}{0in}{}{}{4e0407x54.wmf}{%
\special{language "Scientific Word";type "GRAPHIC";maintain-aspect-ratio
TRUE;display "USEDEF";valid_file "F";width 3.1799in;height 1.6553in;depth
0in;original-width 3.1799in;original-height 1.6553in;cropleft "0";croptop
"1";cropright "1";cropbottom "0";filename
'graphics/4e0407x54.wmf';file-properties "XNPEU";}}\hspace*{\fill}

\begin{ExerciseList}
\item[(a)] Use Poiseuille's Law to show that the total resistance of the
blood along the path $ABC$ is\\[6pt]
\hspace*{\fill}$R=C\left( \dfrac{a-b\cot \theta }{r_{1}^{4}}+\dfrac{b\csc
\theta }{r_{2}^{4}}\right) $\hspace*{\fill}\\[6pt]
where $a$ and $b$ are the distances shown in the figure.

%TCIMACRO{%
%\hyperref{\fbox{\textbf{master 02593}}}{}{\fbox{\textbf{master 02593}}}{}}%
%BeginExpansion
\msihyperref{\fbox{\textbf{master 02593}}}{}{\fbox{\textbf{master 02593}}}{}%
%EndExpansion

\item[(b)] Prove that this resistance is minimized when\\[6pt]
\hspace*{\fill}$\cos \theta =\dfrac{r_{2}^{4}}{r_{1}^{4}}$\hspace*{\fill}

%TCIMACRO{%
%\hyperref{\fbox{\textbf{master 05093}}}{}{\fbox{\textbf{master 05093}}}{}}%
%BeginExpansion
\msihyperref{\fbox{\textbf{master 05093}}}{}{\fbox{\textbf{master 05093}}}{}%
%EndExpansion

\item[(c)] Find the optimal branching angle (correct to the nearest degree)
when the radius of the smaller blood vessel is two-thirds the radius of the
larger vessel.

%TCIMACRO{%
%\hyperref{\fbox{\textbf{master 05094}}}{}{\fbox{\textbf{master 05094}}}{}}%
%BeginExpansion
\msihyperref{\fbox{\textbf{master 05094}}}{}{\fbox{\textbf{master 05094}}}{}%
%EndExpansion
\vspace*{6pt}
\end{ExerciseList}

\hspace*{\fill}\FRAME{itbpF}{200.4375pt}{120.1875pt}{0pt}{}{}{6et0407x72.tif%
}{\special{language "Scientific Word";type "GRAPHIC";maintain-aspect-ratio
TRUE;display "USEDEF";valid_file "F";width 200.4375pt;height
120.1875pt;depth 0pt;original-width 197.6961bp;original-height
117.746bp;cropleft "0";croptop "1";cropright "1";cropbottom "0";filename
'graphics/6et0407x72.tif';file-properties "XNPEU";}}\hspace*{\fill}

\item[\hfill 73.] Ornithologists have determined that some species of birds
tend to avoid flights over large bodies of water during daylight hours. It
is believed that more energy is required to fly over water than over land
because air generally rises over land and falls over water during the day. A
bird with these tendencies is released from an island that is 5 km from the
nearest point $B$ on a straight shoreline, flies to a point $C$ on the
shoreline, and then flies along the shoreline to its nesting area $D$.
Assume that the bird instinctively chooses a path that will minimize its
energy expenditure. Points $B$ and $D$ are 13 km apart.

\begin{ExerciseList}
\item[(a)] In general, if it takes 1.4 times as much energy to fly over
water as it does over land, to what point $C$ should the bird fly in order
to minimize the total energy expended in returning to its nesting area?

%TCIMACRO{%
%\hyperref{ANSWER}{}{\textbf{ANSWER:} (a) About 5.1 km from $B$}{}}%
%BeginExpansion
\msihyperref{ANSWER}{}{\textbf{ANSWER:} (a) About 5.1 km from $B$}{}%
%EndExpansion

%TCIMACRO{%
%\hyperref{\fbox{\textbf{master 02594}}}{}{\fbox{\textbf{master 02594}}}{}}%
%BeginExpansion
\msihyperref{\fbox{\textbf{master 02594}}}{}{\fbox{\textbf{master 02594}}}{}%
%EndExpansion

\item[(b)] Let $W$ and $L$ denote the energy (in joules) per kilometer flown
over water and land, respectively. What would a large value of the ratio $%
W/L $ mean in terms of the bird's flight? What would a small value mean?
Determine the ratio $W/L$ corresponding to the minimum expenditure of energy.

%TCIMACRO{%
%\hyperref{ANSWER}{}{\textbf{ANSWER:} (b) $C$ is close to $B$; $C$ is close to $D$; $W/L=\sqrt{25+x^{2}}/x$, where $x=\left\vert BC\right\vert $}{}}%
%BeginExpansion
\msihyperref{ANSWER}{}{\textbf{ANSWER:} (b) $C$ is close to $B$; $C$ is close to $D$; $W/L=\sqrt{25+x^{2}}/x$, where $x=\left\vert BC\right\vert $}{}%
%EndExpansion

%TCIMACRO{%
%\hyperref{\fbox{\textbf{master 05090}}}{}{\fbox{\textbf{master 05090}}}{}}%
%BeginExpansion
\msihyperref{\fbox{\textbf{master 05090}}}{}{\fbox{\textbf{master 05090}}}{}%
%EndExpansion

\item[(c)] What should the value of $W/L$ be in order for the bird to fly
directly to its nesting area $D$? What should the value of $W/L$ be for the
bird to fly to $B$ and then along the shore to $D$?

%TCIMACRO{%
%\hyperref{ANSWER}{}{\textbf{ANSWER:} (c) $\approx 1.07$; no such value}{}}%
%BeginExpansion
\msihyperref{ANSWER}{}{\textbf{ANSWER:} (c) $\approx 1.07$; no such value}{}%
%EndExpansion

%TCIMACRO{%
%\hyperref{\fbox{\textbf{master 05091}}}{}{\fbox{\textbf{master 05091}}}{}}%
%BeginExpansion
\msihyperref{\fbox{\textbf{master 05091}}}{}{\fbox{\textbf{master 05091}}}{}%
%EndExpansion

\item[(d)] If the ornithologists observe that birds of a certain species
reach the shore at a point 4~km from $B$, how many times more energy does it
take a bird to fly over water than over land?

%TCIMACRO{%
%\hyperref{ANSWER}{}{\textbf{ANSWER:} (d) $\sqrt{41}/4\approx 1.6$}{}}%
%BeginExpansion
\msihyperref{ANSWER}{}{\textbf{ANSWER:} (d) $\sqrt{41}/4\approx 1.6$}{}%
%EndExpansion

%TCIMACRO{%
%\hyperref{\fbox{\textbf{master 05092}}}{}{\fbox{\textbf{master 05092}}}{}}%
%BeginExpansion
\msihyperref{\fbox{\textbf{master 05092}}}{}{\fbox{\textbf{master 05092}}}{}%
%EndExpansion
\vspace*{6pt}
\end{ExerciseList}

\hspace*{\fill}\FRAME{itbpF}{2.1932in}{1.484in}{0in}{}{}{6et0407x73_00566.wmf%
}{\special{language "Scientific Word";type "GRAPHIC";maintain-aspect-ratio
TRUE;display "USEDEF";valid_file "F";width 2.1932in;height 1.484in;depth
0in;original-width 2.1664in;original-height 1.4563in;cropleft "0";croptop
"1";cropright "1";cropbottom "0";filename
'graphics/6et0407x73_00566.wmf';file-properties "XNPEU";}}\hspace*{\fill}

\item[\hfill 74.] 
%TCIMACRO{\TeXButton{GCALCX}{\GCALCX}}%
%BeginExpansion
\GCALCX%
%EndExpansion
Two light sources of identical strength are placed 10 m apart. An object is
to be placed at a point $P$ on a line $\ell $ parallel to the line joining
the light sources and at a distance $d$ meters from it (see the figure). We
want to locate $P$ on $\ell $ so that the intensity of illumination is
minimized. We need to use the fact that the intensity of illumination for a
single source is directly proportional to the strength of the source and
inversely proportional to the square of the distance from the source.

\begin{ExerciseList}
\item[(a)] Find an expression for the intensity $I(x)$ at the point $P$.

%TCIMACRO{%
%\hyperref{\fbox{\textbf{master 02595}}}{}{\fbox{\textbf{master 02595}}}{}}%
%BeginExpansion
\msihyperref{\fbox{\textbf{master 02595}}}{}{\fbox{\textbf{master 02595}}}{}%
%EndExpansion

\item[(b)] If $d=5$ m, use graphs of $I(x)$ and $I^{\prime }(x)$ to show
that the intensity is minimized when $x=5$ m, that is, when $P$ is at the
midpoint of $\ell $.

%TCIMACRO{%
%\hyperref{\fbox{\textbf{master 05097}}}{}{\fbox{\textbf{master 05097}}}{}}%
%BeginExpansion
\msihyperref{\fbox{\textbf{master 05097}}}{}{\fbox{\textbf{master 05097}}}{}%
%EndExpansion

\item[(c)] If $d=10$ m, show that the intensity (perhaps surprisingly) is 
\textit{not} minimized at the midpoint.

%TCIMACRO{%
%\hyperref{\fbox{\textbf{master 05098}}}{}{\fbox{\textbf{master 05098}}}{}}%
%BeginExpansion
\msihyperref{\fbox{\textbf{master 05098}}}{}{\fbox{\textbf{master 05098}}}{}%
%EndExpansion

\item[(d)] Somewhere between $d=5$ m and $d=10$ m there is a transitional
value of $d$ at which the point of minimal illumination abruptly changes.
Estimate this value of $d$ by graphical methods. Then find the exact value
of $d$.

%TCIMACRO{%
%\hyperref{\fbox{\textbf{master 05099}}}{}{\fbox{\textbf{master 05099}}}{}}%
%BeginExpansion
\msihyperref{\fbox{\textbf{master 05099}}}{}{\fbox{\textbf{master 05099}}}{}%
%EndExpansion
\vspace*{6pt}
\end{ExerciseList}

\hspace*{\fill}\FRAME{itbpF}{2.6385in}{1.0092in}{0in}{}{}{6et0407x74_00567.ai%
}{\special{language "Scientific Word";type "GRAPHIC";maintain-aspect-ratio
TRUE;display "USEDEF";valid_file "F";width 2.6385in;height 1.0092in;depth
0in;original-width 2.6109in;original-height 0.9816in;cropleft "0";croptop
"1";cropright "1";cropbottom "0";filename
'graphics/6et0407x74_00567.ai';file-properties "XNPEU";}}\hspace*{\fill}
\end{ExerciseList}

%TCIMACRO{%
%\TeXButton{e2col}{\end{multicols}
%\advance \leftskip by 165pt
%\advance\hsize by -165pt
%\advance\linewidth by -165pt
%}}%
%BeginExpansion
\end{multicols}
\advance \leftskip by 165pt
\advance\hsize by -165pt
\advance\linewidth by -165pt
%
%EndExpansion

\end{document}
