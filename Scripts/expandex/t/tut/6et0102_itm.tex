
\documentclass{sebase}
%%%%%%%%%%%%%%%%%%%%%%%%%%%%%%%%%%%%%%%%%%%%%%%%%%%%%%%%%%%%%%%%%%%%%%%%%%%%%%%%%%%%%%%%%%%%%%%%%%%%%%%%%%%%%%%%%%%%%%%%%%%%%%%%%%%%%%%%%%%%%%%%%%%%%%%%%%%%%%%%%%%%%%%%%%%%%%%%%%%%%%%%%%%%%%%%%%%%%%%%%%%%%%%%%%%%%%%%%%%%%%%%%%%%%%%%%%%%%%%%%%%%%%%%%%%%
\usepackage{amssymb}
\usepackage{amsmath}
\usepackage{SECALCUL}
\usepackage{lie}

\setcounter{MaxMatrixCols}{10}
%TCIDATA{OutputFilter=LATEX.DLL}
%TCIDATA{Version=5.50.0.2953}
%TCIDATA{<META NAME="SaveForMode" CONTENT="1">}
%TCIDATA{BibliographyScheme=Manual}
%TCIDATA{Created=Mon Dec 15 16:20:00 1997}
%TCIDATA{LastRevised=Tuesday, March 25, 2008 11:30:47}
%TCIDATA{<META NAME="ViewSettings" CONTENT="0">}
%TCIDATA{<META NAME="GraphicsSave" CONTENT="32">}
%TCIDATA{CSTFile=SECALCUL.cst}

\input tcilatex
\newenvironment{instructions}{\STARTINSTR}{\ENDINSTR}
\begin{document}


\chapter{1\quad Functions and Models}

\section{1.2\quad Mathematical Models:\ A Catalog of Essential Functions}

%TCIMACRO{\TeXButton{noCCC}{\noCCC}}%
%BeginExpansion
\noCCC%
%EndExpansion
\setET%
%TCIMACRO{\TeXButton{noRM}{\renewcommand{\RM}{0}}}%
%BeginExpansion
\renewcommand{\RM}{0}%
%EndExpansion
%TCIMACRO{\TeXButton{setRM}{\renewcommand{\RM}{1}}}%
%BeginExpansion
\renewcommand{\RM}{1}%
%EndExpansion
%TCIMACRO{%
%\TeXButton{set page 18/ 15}{\ifnum\RM=1 \setcounter{page}{18}\else \setcounter{page}{15}\fi}}%
%BeginExpansion
\ifnum\RM=1 \setcounter{page}{18}\else \setcounter{page}{15}\fi%
%EndExpansion

A \textbf{mathematical model} is a mathematical description (often by means
of a function or an equation) of a real-world phenomenon such as the size of
a population, the demand for a product, the speed of a falling object, the
concentration of a product in a chemical reaction, the life expectancy of a
person at birth, or the cost of emission reductions. The purpose of the
model is to understand the phenomenon and perhaps to make predictions about
future behavior.

Figure 1 illustrates the process of mathematical modeling. Given a
real-world problem, our first task is to formulate a mathematical model by
identifying and naming the independent and dependent variables and making
assumptions that simplify the phenomenon enough to make it mathematically
tractable. We use our knowledge of the physical situation and our
mathematical skills to obtain equations that relate the variables. In
situations where there is no physical law to guide us, we may need to
collect data (either from a library or the Internet or by conducting our own
experiments) and examine the data in the form of a table in order to discern
patterns. From this numerical representation of a function we may wish to
obtain a graphical representation by plotting the data. The graph might even
suggest a suitable algebraic formula in some cases.

%TCIMACRO{\TeXButton{graphicS}{\vspace{12pt}\hskip-190pt\hfil}}%
%BeginExpansion
\vspace{12pt}\hskip-190pt\hfil%
%EndExpansion
\FRAME{itbpFU}{7.1915in}{0.8225in}{0in}{\Qcb{\QTR{FigureNumber}{FIGURE 1}%
\quad The modeling process}}{\Qlb{FIGURE 1 The modeling process}}{%
6et010201_00911.ai}{\special{language "Scientific Word";type
"GRAPHIC";maintain-aspect-ratio TRUE;display "USEDEF";valid_file "F";width
7.1915in;height 0.8225in;depth 0in;original-width 7.0638in;original-height
0.7831in;cropleft "0";croptop "1";cropright "1";cropbottom "0";filename
'graphics/6et010201_00911.ai';file-properties "XNPEU";}}%
%TCIMACRO{\TeXButton{graphicE}{\vspace{12pt}\hfil}}%
%BeginExpansion
\vspace{12pt}\hfil%
%EndExpansion

The second stage is to apply the mathematics that we know (such as the
calculus that will be developed throughout this book) to the mathematical
model that we have formulated in order to derive mathematical conclusions.
Then, in the third stage, we take those mathematical conclusions and
interpret them as information about the origi\-nal real-world phenomenon by
way of offering explanations or making predictions. The final step is to
test our predictions by checking against new real data. If the predictions
don't compare well with reality, we need to refine our model or to formulate
a new model and start the cycle again.

A mathematical model is never a completely accurate representation of a
physical situation---it is an \textit{idealization}. A good model simplifies
reality enough to permit mathematical calculations but is accurate enough to
provide valuable conclusions. It is important to realize the limitations of
the model. In the end, Mother Nature has the final say.

There are many different types of functions that can be used to model
relationships observed in the real world. In what follows, we discuss the
behavior and graphs of these functions and give examples of situations
appropriately modeled by such functions.%
%TCIMACRO{\TeXButton{longpage}{\enlargethispage{\baselineskip}}}%
%BeginExpansion
\enlargethispage{\baselineskip}%
%EndExpansion

\subsection{LINEAR MODELS}

When we say that $y$ is a \textbf{linear function} of $x$, 
\marginpar{
The coordinate geometry of lines is reviewed in Appendix B.}we mean that the
graph of the function is a line, so we can use the slope-intercept form of
the equation of a line to write a formula for the function as 
\begin{equation*}
y=f(x)=mx+b
\end{equation*}
where $m$ is the slope of the line and $b$ is the $y$-intercept.

A characteristic feature of linear functions is that they grow at a constant
rate. For instance, Figure 2 shows a graph of the linear function $f(x)=3x-2$
and a table of sample values. Notice that whenever \textit{x} increases by
0.1, the value of $f(x)$ increases by 0.3. So $f(x)$ increases three times
as fast as \textit{x}. Thus the slope of the graph $y=3x-2$, namely 3, can
be interpreted as the rate of change of \textit{y} with respect to \textit{x}%
.\bigskip

$\hfill $\FRAME{itbpFU}{136.5pt}{112.5625pt}{48.1875pt}{\Qcb{%
\QTR{FigureNumber}{FIGURE 2}}}{}{3c010202.wmf}{\special{language "Scientific
Word";type "GRAPHIC";maintain-aspect-ratio TRUE;display "USEDEF";valid_file
"F";width 136.5pt;height 112.5625pt;depth 48.1875pt;original-width
136.5pt;original-height 112.5625pt;cropleft "0";croptop "1";cropright
"1";cropbottom "0";filename 'graphics/3c010202.wmf';file-properties "XNPEU";}%
}\hspace*{12pt}\qquad 
\begin{tabular}{|c|c|}
\hline
\hspace*{12pt}$x$\hspace*{12pt} & \hspace*{12pt}$f(x)=3x-2%
\rule[-6pt]{0pt}{17pt}$\hspace*{12pt} \\ \hline
1.0 & 1.0 \\ 
1.1 & 1.3 \\ 
1.2 & 1.6 \\ 
1.3 & 1.9 \\ 
1.4 & 2.2 \\ 
1.5 & 2.5 \\ \hline
\end{tabular}
$\hfill $

\begin{Example}[1]
%TECHARTS_DUMMY_ITEM_TAG-(1)
%TCIMACRO{\TeXButton{VIDEO}{\VIDEO}}%
%BeginExpansion
\VIDEO%
%EndExpansion
%TCIMACRO{%
%\hyperref{\fbox{{\footnotesize Video 5et 010201 / 004}}\quad }{}{\fbox{{\footnotesize Video 5et 010201 / 004}}\quad }{}}%
%BeginExpansion
\msihyperref{\fbox{{\footnotesize Video 5et 010201 / 004}}\quad }{}{\fbox{{\footnotesize Video 5et 010201 / 004}}\quad }{}%
%EndExpansion
\thinspace \vspace{-6pt}

\begin{enumerate}
\item[(a)] As dry air moves upward, it expands and cools. If the ground
temperature is $20^{\circ }$C and the temperature at a height of 1 km is $%
10^{\circ }$C, express the temperature $T$ (in $^{\circ }$C) as a function
of the height $h$ (in kilometers), assuming that a linear model is
appropriate.

\item[(b)] Draw the graph of the function in part (a). What does the slope
represent?

\item[(c)] What is the temperature at a height of 2.5 km?
\end{enumerate}
\end{Example}

\begin{Solution}
\thinspace \vspace{-6pt}

\begin{enumerate}
\item[(a)] Because we are assuming that $T$ is a linear function of $h$, we
can write 
\begin{equation*}
T=mh+b
\end{equation*}
We are given that $T=20$ when $h=0$, so 
\begin{equation*}
20=m\cdot 0+b=b
\end{equation*}
\marginpar{\vspace{24pt}\FRAME{dtbpFU}{1.5558in}{1.5835in}{0pt}{\Qcb{%
\QTR{FigureNumber}{FIGURE 3}}}{}{3c010203.wmf}{\special{language "Scientific
Word";type "GRAPHIC";maintain-aspect-ratio TRUE;display "USEDEF";valid_file
"F";width 1.5558in;height 1.5835in;depth 0pt;original-width
1.5558in;original-height 1.5835in;cropleft "0";croptop "1";cropright
"1";cropbottom "0";filename 'graphics/3c010203.wmf';file-properties "XNPEU";}%
}}In other words, the $y$-intercept is $b=20$.

\quad We are also given that $T=10$ when $h=1$, so 
\begin{equation*}
10=m\cdot 1+20
\end{equation*}
The slope of the line is therefore $m=10-20=-10$ and the required linear
function is 
\begin{equation*}
T=-10h+20
\end{equation*}

\item[(b)] The graph is sketched in Figure 3. The slope is $m=-10^{\circ }$C$%
/$km, and this represents the rate of change of temperature with respect to
height.

\item[(c)] At a height of $h=2.5$ km, the temperature is\bigskip

$\hfill T=-10(2.5)+20=-5^{\circ }$C$\hfill \blacksquare $
\end{enumerate}
\end{Solution}

If there is no physical law or principle to help us formulate a model, we
construct an \textbf{empirical model}, which is based entirely on collected
data. We seek a curve that ``fits'' the data in the sense that it captures
the basic trend of the data points.\pagebreak

\begin{Example}[2]
%TECHARTS_DUMMY_ITEM_TAG-(2)
%TCIMACRO{\TeXButton{VIDEO}{\VIDEO}}%
%BeginExpansion
\VIDEO%
%EndExpansion
%TCIMACRO{%
%\hyperref{\fbox{{\footnotesize Video 5et 010202 / 005}}\quad }{}{\fbox{{\footnotesize Video 5et 010202 / 005}}\quad }{}}%
%BeginExpansion
\msihyperref{\fbox{{\footnotesize Video 5et 010202 / 005}}\quad }{}{\fbox{{\footnotesize Video 5et 010202 / 005}}\quad }{}%
%EndExpansion
Table 1 lists the average carbon dioxide level in the atmosphere, meas-ured
in parts per million at Mauna Loa Observatory from 1980 to 2002. Use the
data in Table 1 to find a model for the carbon dioxide level.
\end{Example}

\begin{Solution}
We use the data in Table 1 to make the scatter plot in Figure 4, where $t$
represents time (in years) and $C$ represents CO$_{2}$ level (in parts per
million, ppm).

%TCIMACRO{\TeXButton{graphicS}{\vspace{12pt}\hskip-100pt\hfil}}%
%BeginExpansion
\vspace{12pt}\hskip-100pt\hfil%
%EndExpansion
$%
\begin{tabular}{|c|c||c|c|}
\multicolumn{4}{c}{\QTR{FigureNumber}{TABLE 1}} \\ \hline
& CO$_{2}$ level &  & CO$_{2}$ level \\ 
Year & (in ppm) & Year & (in ppm) \\ \hline
1980 & 338.7 & 1992 & 356.4 \\ 
1982 & 341.1 & 1994 & 358.9 \\ 
1984 & 344.4 & 1996 & 362.6 \\ 
1986 & 347.2 & 1998 & 366.6 \\ 
1988 & 351.5 & 2000 & 369.4 \\ 
1990 & 354.2 & 2002 & 372.9 \\ \hline
\end{tabular}%
\hspace{36pt}$\FRAME{itbpFU}{3.1399in}{1.9409in}{0.946in}{\Qcb{%
\QTR{FigureNumber}{FIGURE 4\quad }Scatter plot for the average CO$_{2}$ level%
}}{}{3c010204.wmf}{\special{language "Scientific Word";type
"GRAPHIC";maintain-aspect-ratio TRUE;display "USEDEF";valid_file "F";width
3.1399in;height 1.9409in;depth 0.946in;original-width
3.1393in;original-height 1.9415in;cropleft "0";croptop "1";cropright
"1";cropbottom "0";filename 'graphics/3c010204.wmf';file-properties "XNPEU";}%
}\vspace{-15pt}%
%TCIMACRO{\TeXButton{graphicE}{\vspace{12pt}\hfil}}%
%BeginExpansion
\vspace{12pt}\hfil%
%EndExpansion
\end{Solution}

Notice that the data points appear to lie close to a straight line, so it's
natural to choose a linear model in this case. But there are many possible
lines that approximate these data points, so which one should we use? From
the graph, it appears that one possibility is the line that passes through
the first and last data points. The slope of this line is \\[6pt]
\hspace*{\fill}$\dfrac{372.9-338.7}{2002-1980}=\dfrac{34.2}{22}\approx
1.5545 $\hspace*{\fill}\\[6pt]
and its equation is \\[6pt]
\hspace*{\fill}$C-338.7=1.5545(t-1980)$\hspace*{\fill}\\[6pt]
or\medskip

\QTR{BOXHEAD}{(1)}$\hspace{\stretch{2}}C=1.5545t-2739.21\hspace{\stretch{2}}$%
\medskip

Equation 1 gives one possible linear model for the carbon dioxide level; it
is graphed in Figure 5.%
%TCIMACRO{\TeXButton{longpage}{\enlargethispage{\baselineskip}}}%
%BeginExpansion
\enlargethispage{\baselineskip}%
%EndExpansion
\vspace{9pt}\FRAME{dtbpFU}{3.1393in}{1.9415in}{0pt}{\Qcb{\QTR{FigureNumber}{%
FIGURE 5}\quad Linear model through first and last data points}}{}{%
3c010205.wmf}{\special{language "Scientific Word";type
"GRAPHIC";maintain-aspect-ratio TRUE;display "USEDEF";valid_file "F";width
3.1393in;height 1.9415in;depth 0pt;original-width 3.1393in;original-height
1.9415in;cropleft "0";croptop "1";cropright "1";cropbottom "0";filename
'graphics/3c010205.wmf';file-properties "XNPEU";}}\vspace{9pt}

Although%
\marginpar{\vspace{-36pt}A computer or graphing calculator finds the
regression line by the method of \textbf{least squares}, which is to
minimize the sum of the squares of the vertical distances between the data
points and the line. The details are explained in Section 14.7.} our model
fits the data reasonably well, it gives values higher than most of the
actual CO$_{2}$ levels. A better linear model is obtained by a procedure
from statistics called \textit{linear regression}. If we use a graphing
calculator, we enter the data from Table 1 into the data editor and choose
the linear regression command. (With Maple we use the fit[leastsquare]
command in the stats package; with Mathematica we use the Fit command.) The
machine gives the slope and $y$-intercept of the regression line as \\[6pt]
\hspace*{\fill}$m=1.55192\qquad \qquad b=-2734.55$\hspace*{\fill}\\[6pt]
So our least squares model for the CO$_{2}$ level is\medskip

\QTR{BOXHEAD}{(2)}$\hfill C=1.55192t-2734.55\hfill $\medskip

In Figure 6 we graph the regression line as well as the data points.
Comparing with Figure 5, we see that it gives a better fit than our previous
linear model.\vspace{9pt}\newline
\hspace*{\fill}\FRAME{itbpFU}{3.1393in}{1.9415in}{0in}{\Qcb{%
\QTR{FigureNumber}{FIGURE 6}\quad The regression line}}{}{3c010206.wmf}{%
\special{language "Scientific Word";type "GRAPHIC";maintain-aspect-ratio
TRUE;display "USEDEF";valid_file "F";width 3.1393in;height 1.9415in;depth
0in;original-width 3.1393in;original-height 1.9415in;cropleft "0";croptop
"1";cropright "1";cropbottom "0";filename
'graphics/3c010206.wmf';file-properties "XNPEU";}}\hspace*{\fill}$%
\blacksquare $

\begin{Example}[3]
%TECHARTS_DUMMY_ITEM_TAG-(3)
%TCIMACRO{\TeXButton{VIDEO}{\VIDEO}}%
%BeginExpansion
\VIDEO%
%EndExpansion
%TCIMACRO{%
%\hyperref{\fbox{{\footnotesize Video 5et 010203 / 006}}\quad }{}{\fbox{{\footnotesize Video 5et 010203 / 006}}\quad }{}}%
%BeginExpansion
\msihyperref{\fbox{{\footnotesize Video 5et 010203 / 006}}\quad }{}{\fbox{{\footnotesize Video 5et 010203 / 006}}\quad }{}%
%EndExpansion
Use the linear model given by Equation 2 to estimate the average CO$_{2}$
level for 1987 and to predict the level for the year 2010. According to this
model, when will the CO$_{2}$ level exceed 400 parts per million?
\end{Example}

\begin{Solution}
Using Equation 2 with $t=1987,$ we estimate that the average CO$_{2}$ level
in 1987 was\\[6pt]
\hspace*{\fill}$C(1987)=(1.55192)(1987)-2734.55\approx 349.12$\hspace*{\fill}%
\\[6pt]
This is an example of \textit{interpolation} because we have estimated a
value \textit{between} observed values. (In fact, the Mauna Loa Observatory
reported that the average CO$_{2}$ level in 1987 was 348.93 ppm, so our
estimate is quite accurate.)

With $t=2010$, we get\\[6pt]
\hspace*{\fill}$C(2010)=(1.55192)(2010)-2734.55\approx 384.81$\hspace*{\fill}%
\\[6pt]
So we predict that the average CO$_{2}$ level in the year 2010 will be 384.8
ppm. This is an example of \textit{extrapolation} because we have predicted
a value \textit{outside} the region of observations. Consequently, we are
far less certain about the accuracy of our prediction.

Using Equation 2, we see that the CO$_{2}$ level exceeds 400 ppm when\\[6pt]
\hspace*{\fill}$1.55192t-2734.55>400$\hspace*{\fill}\\[6pt]
Solving this inequality, we get \\[6pt]
\hspace*{\fill}$t>\dfrac{3134.55}{1.55192}\approx 2019.79$\hspace*{\fill}\\[%
6pt]
We therefore predict that the CO$_{2}$ level will exceed 400 ppm by the year
2019. This prediction is somewhat risky because it involves a time quite
remote from our observations.$\blacksquare $
\end{Solution}

\subsection{POLYNOMIALS}

\noindent A function $P$ is called a \textbf{polynomial} if\\[6pt]
\hspace*{\fill}$P(x)=a_{n}x^{n}+a_{n-1}x^{n-1}+\cdots
+a_{2}x^{2}+a_{1}x+a_{0}$\hspace*{\fill}\\[6pt]
where $n$ is a nonnegative integer and the numbers $a_{0},a_{1},a_{2},$ $%
...,a_{n}$ are constants called the \textbf{coefficients} of the polynomial.
The domain of any polynomial is $\mathbb{R}=(-\infty ,\infty ).$ If the
leading coefficient $a_{n}\neq 0$, then the \textbf{degree }of the
polynomial is $n$. For example, the function\\[6pt]
\hspace*{\fill}$P\!\left( x\right) =2x^{6}-x^{4}+\tfrac{2}{5}x^{3}+\sqrt{2}$%
\hspace*{\fill}\\[6pt]
is a polynomial of degree 6.

A polynomial of degree 1 is of the form $P(x)=mx+b$ and so it is a linear
function. A polynomial of degree 2 is of the form $P(x)=ax^{2}+bx+c$ and is
called a \textbf{quadratic function}. Its graph is always a parabola
obtained by shifting the parabola $y=ax^{2}$, as we will see in the next
section. The parabola opens upward if $a>0$ and downward if $a<0$. (See
Figure 7.)\\[6pt]
\hspace*{\fill}\FRAME{itbpFU}{3.3055in}{1.5387in}{0in}{\Qcb{%
\QTR{FigureNumber}{FIGURE 7}\quad The graphs of quadratic functions are
parabolas.}}{}{6et010207_00059.ai}{\special{language "Scientific Word";type
"GRAPHIC";maintain-aspect-ratio TRUE;display "USEDEF";valid_file "F";width
3.3055in;height 1.5387in;depth 0in;original-width 3.278in;original-height
1.5112in;cropleft "0";croptop "1";cropright "1";cropbottom "0";filename
'graphics/6et010207_00059.ai';file-properties "XNPEU";}}\hspace*{\fill}%
\vspace*{9pt}

A polynomial of degree 3 is of the form\\[6pt]
\hspace*{\fill}$P(x)=ax^{3}+bx^{2}+cx+d\qquad \left( a\neq 0\right) $%
\hspace*{\fill}\\[6pt]
and is called a \textbf{cubic function}. Figure 8 shows the graph of a cubic
function in part~(a) and graphs of polynomials of degrees 4 and 5 in
parts~(b) and (c). We will see later why the graphs have these shapes.%
\vspace*{6pt}

%TCIMACRO{\TeXButton{graphicS}{\vspace{12pt}\hskip-50pt\hfil}}%
%BeginExpansion
\vspace{12pt}\hskip-50pt\hfil%
%EndExpansion
\FRAME{itbpFU}{4.7504in}{1.5714in}{0in}{\Qcb{\QTR{FigureNumber}{FIGURE 8}}}{%
}{3c010208.wmf}{\special{language "Scientific Word";type
"GRAPHIC";maintain-aspect-ratio TRUE;display "USEDEF";valid_file "F";width
4.7504in;height 1.5714in;depth 0in;original-width 365.4375pt;original-height
118.4375pt;cropleft "0";croptop "1";cropright "1";cropbottom "0";filename
'graphics/3c010208.wmf';file-properties "XNPEU";}}%
%TCIMACRO{\TeXButton{graphicE}{\vspace{12pt}\hfil}}%
%BeginExpansion
\vspace{12pt}\hfil%
%EndExpansion

Polynomials are commonly used to model various quantities that occur in the
natural and social sciences. For instance, in Section 
%TCIMACRO{%
%\TeXButton{CCC 3.3 / 5ET 3.7 / 5E 3.7}{\ifnum\CCC=1 3.3\else 3.7\fi}}%
%BeginExpansion
\ifnum\CCC=1 3.3\else 3.7\fi%
%EndExpansion
\ we will explain why economists often use a polynomial $P(x)$ to represent
the cost of producing $x$ units of a com\nolinebreak mod\nolinebreak ity. In
the following example we use a quadratic function to model the fall of a
ball.

\begin{Example}[4]
%TECHARTS_DUMMY_ITEM_TAG-(4)
A ball is dropped from the upper observation deck of the CN Tower, 450 m
above the ground, and its height $h$ above the ground is recorded at
1-second \pagebreak \linebreak intervals in Table 2. 
\marginpar{\hspace*{\fill}\QTR{FigureNumber}{TABLE 2}\vspace{-6pt}\hspace*{%
\fill}\newline
\hspace*{\fill}%
\begin{tabular}[t]{|c|c|}
\hline
Time & Height \\ 
(seconds) & (meters) \\ \hline
0 & \multicolumn{1}{|r|}{450\hspace*{9pt}} \\ 
1 & \multicolumn{1}{|r|}{445\hspace*{9pt}} \\ 
2 & \multicolumn{1}{|r|}{431\hspace*{9pt}} \\ 
3 & \multicolumn{1}{|r|}{408\hspace*{9pt}} \\ 
4 & \multicolumn{1}{|r|}{375\hspace*{9pt}} \\ 
5 & \multicolumn{1}{|r|}{332\hspace*{9pt}} \\ 
6 & \multicolumn{1}{|r|}{279\hspace*{9pt}} \\ 
7 & \multicolumn{1}{|r|}{216\hspace*{9pt}} \\ 
8 & \multicolumn{1}{|r|}{143\hspace*{9pt}} \\ 
9 & \multicolumn{1}{|r|}{61\hspace*{9pt}} \\ \hline
\end{tabular}%
\hspace*{\fill}}Find a model to fit the data and use the model to predict
the time at which the ball hits the ground.
\end{Example}

\begin{Solution}
We draw a scatter plot of the data in Figure 9 and observe that a linear
model is inappropriate. But it looks as if the data points might lie on a
parabola, so we try a quadratic model instead. Using a graphing calculator
or computer algebra system (which uses the least squares method), we obtain
the following quadratic model:\vspace{-15pt}
\end{Solution}

\QTR{BOXHEAD}{(3)}$\hspace{\stretch{2}}h=449.36+0.96t-4.90t^{2}\hspace{%
\stretch{2}}$\vspace*{6pt}

\noindent \hspace*{\fill}\FRAME{itbpFU}{2.3286in}{1.7059in}{0.0094in}{\Qcb{%
\protect\begin{tabular}{l}
\QTR{FigureNumber}{FIGURE 9}\protect \\ 
Scatter plot for a falling ball%
\protect\end{tabular}
}}{}{6et010209_00061.ai}{\special{language "Scientific Word";type
"GRAPHIC";maintain-aspect-ratio TRUE;display "USEDEF";valid_file "F";width
2.3286in;height 1.7059in;depth 0.0094in;original-width
2.3003in;original-height 1.6785in;cropleft "0";croptop "1";cropright
"1";cropbottom "0";filename 'graphics/6et010209_00061.ai';file-properties
"XNPEU";}}$\hspace*{24pt}$\FRAME{itbpFU}{1.9864in}{1.7068in}{0in}{\Qcb{%
\protect\begin{tabular}{l}
\QTR{FigureNumber}{FIGURE 10}\protect \\ 
Quadratic model for a falling ball%
\protect\end{tabular}
}}{}{6et010210_00062.eps}{\special{language "Scientific Word";type
"GRAPHIC";maintain-aspect-ratio TRUE;display "USEDEF";valid_file "F";width
1.9864in;height 1.7068in;depth 0in;original-width 1.9589in;original-height
1.6793in;cropleft "0";croptop "1";cropright "1";cropbottom "0";filename
'graphics/6et010210_00062.eps';file-properties "XNPEU";}}\hspace*{\fill}%
\vspace*{6pt}

In Figure 10 we plot the graph of Equation 3 together with the data points
and see that the quadratic model gives a very good fit.%
%TCIMACRO{\TeXButton{longpage}{\enlargethispage{\baselineskip}}}%
%BeginExpansion
\enlargethispage{\baselineskip}%
%EndExpansion

The ball hits the ground when $h=0$, so we solve the quadratic equation\\[6pt%
]
\hspace*{\fill}$-4.90t^{2}+0.96t+449.36=0$\hspace*{\fill}\\[6pt]
The quadratic formula gives\\[6pt]
$\hspace*{\fill}t=\dfrac{-0.96\pm \sqrt{(0.96)^{2}-4(-4.90)(449.36)}}{%
2(-4.90)}\hfill $\\[6pt]
The positive root is $t\approx 9.67$, so we predict that the ball will hit
the ground after about 9.7 seconds.$\blacksquare $

\subsection{POWER FUNCTIONS$\protect\vspace{-6pt}$}

A function of the form $f(x)=x^{a}$, where $a$ is a constant, is called a 
\textbf{power} \textbf{function}. We consider several cases.\medskip

\noindent \QTR{CaseHead}{(i) }$%
%TCIMACRO{\TeXButton{a}{\mbox{\boldmath $a$}}}%
%BeginExpansion
\mbox{\boldmath $a$}%
%EndExpansion
=%
%TCIMACRO{\TeXButton{n}{\mbox{\boldmath $n$}}}%
%BeginExpansion
\mbox{\boldmath $n$}%
%EndExpansion
$\QTR{CaseTitle}{, where }$%
%TCIMACRO{\TeXButton{n}{\mbox{\boldmath $n$}}}%
%BeginExpansion
\mbox{\boldmath $n$}%
%EndExpansion
$ \QTR{CaseTitle}{is a positive integer}\\[3pt]
The graphs of $f(x)=x^{n}$ for $n=1$, $2$, $3$, $4$, and $5$ are shown in
Figure 11. (These are polynomials with only one term.) We already know the
shape of the graphs of $y=x$ (a line through the origin with slope 1) and $%
y=x^{2}$ [a parabola, see Example 2(b) in Section 1.1].

%TCIMACRO{\TeXButton{graphicS}{\vspace{12pt}\hskip-150pt\hfil}}%
%BeginExpansion
\vspace{12pt}\hskip-150pt\hfil%
%EndExpansion
\FRAME{itbpFU}{6.7931in}{1.5134in}{0in}{\Qcb{\QTR{FigureNumber}{FIGURE 11}%
\quad Graphs of $f\!\left( x\right) =x^{n}$ for $n=1$, $2$, $3$, $4$, $5$}}{%
}{3c010211.wmf}{\special{language "Scientific Word";type
"GRAPHIC";maintain-aspect-ratio TRUE;display "USEDEF";valid_file "F";width
6.7931in;height 1.5134in;depth 0in;original-width 474.875pt;original-height
113.5pt;cropleft "0";croptop "1";cropright "1";cropbottom "0";filename
'graphics/3c010211.wmf';file-properties "XNPEU";}}%
%TCIMACRO{\TeXButton{graphicE}{\vspace{12pt}\hfil}}%
%BeginExpansion
\vspace{12pt}\hfil%
%EndExpansion
\pagebreak

The general shape of the graph of $f(x)=x^{n}$ depends on whether $n$ is
even or odd. If $n$ is even, then $f(x)=x^{n}$ is an even function and its
graph is similar to the parabola $y=x^{2}$. If $n$ is odd, then $f(x)=x^{n}$
is an odd function and its graph is similar to that of $y=x^{3}$. Notice
from Figure 12, however, that as $n$ increases, the graph of $y=x^{n}$
becomes flatter near 0 and steeper when $\left\vert x\right\vert \geq 1$.
(If $x$ is small, then $x^{2}$ is smaller, $x^{3}$ is even smaller, $x^{4}$
is smaller still, and so on.)\\[4pt]
\marginpar{\vspace{-24pt}\hspace*{\fill}%
\begin{tabular}[b]{r}
\QTR{FigureNumber}{FIGURE 12} \\ 
Families of power functions%
\end{tabular}%
}\hspace*{\fill}\FRAME{itbpFU}{4.682in}{2.0318in}{0in}{\Qcb{}}{}{%
6et010212_00064.ai}{\special{language "Scientific Word";type
"GRAPHIC";maintain-aspect-ratio TRUE;display "USEDEF";valid_file "F";width
4.682in;height 2.0318in;depth 0in;original-width 4.6546in;original-height
2.0044in;cropleft "0";croptop "1";cropright "1";cropbottom "0";filename
'graphics/6et010212_00064.ai';file-properties "XNPEU";}}\hspace*{\fill}%
\vspace{8pt}

\noindent \QTR{CaseHead}{(ii) }$%
%TCIMACRO{\TeXButton{a}{\mbox{\boldmath $a$}}}%
%BeginExpansion
\mbox{\boldmath $a$}%
%EndExpansion
=\QTR{CaseTitle}{1}%
%TCIMACRO{\TeXButton{/n}{\mbox{\boldmath $/n$}}}%
%BeginExpansion
\mbox{\boldmath $/n$}%
%EndExpansion
$\QTR{CaseTitle}{, where }$%
%TCIMACRO{\TeXButton{n}{\mbox{\boldmath $n$}}}%
%BeginExpansion
\mbox{\boldmath $n$}%
%EndExpansion
$\QTR{CaseTitle}{\ is a positive integer}\\[2pt]
The function $f(x)=x^{1/n}=\sqrt[n]{x}$ is a \textbf{root function}. For $%
n=2 $ it is the square root function $f(x)=\sqrt{x}$, whose domain is $%
[0,\infty )$ and whose graph is the upper half of the parabola $x=y^{2}$.
[See Figure 13(a).] For other even values of $n$, the graph of $y=\sqrt[n]{x}
$ is similar to that of $y=\sqrt{x}$. For $n=3$ we have the cube root
function $f(x)=\sqrt[3]{x}$ whose domain is $\mathbb{R}$ (recall that every
real number has a cube root) and whose graph is shown in Figure 13(b). The
graph of $y=\sqrt[n]{x}$ for $n$ odd $(n>3)$ is similar to that of $y=\sqrt[3%
]{x}$.\\[4pt]
\marginpar{\vspace{-24pt}\hspace*{\fill}%
\begin{tabular}[b]{r}
\QTR{FigureNumber}{FIGURE 13} \\ 
Graphs of root functions%
\end{tabular}%
}\hspace*{\fill}\FRAME{itbpFU}{3.9058in}{1.2899in}{0in}{\Qcb{}}{}{%
3c010213.wmf}{\special{language "Scientific Word";type
"GRAPHIC";maintain-aspect-ratio TRUE;display "USEDEF";valid_file "F";width
3.9058in;height 1.2899in;depth 0in;original-width 3.9444in;original-height
1.2566in;cropleft "0";croptop "1";cropright "1";cropbottom "0";filename
'graphics/3c010213.wmf';file-properties "XNPEU";}}\hspace*{\fill}\vspace{8pt}

\noindent \QTR{CaseHead}{(iii) }$%
%TCIMACRO{\TeXButton{a}{\mbox{\boldmath $a$}}}%
%BeginExpansion
\mbox{\boldmath $a$}%
%EndExpansion
=\mathbf{-1}$\\[2pt]
\marginpar{\vspace{-15pt}\FRAME{dtbpFU}{1.3889in}{1.3889in}{0pt}{\Qcb{%
\QTR{FigureNumber}{FIGURE 14\quad }The reciprocal function}}{}{3c010214.wmf}{%
\special{language "Scientific Word";type "GRAPHIC";maintain-aspect-ratio
TRUE;display "USEDEF";valid_file "F";width 1.3889in;height 1.3889in;depth
0pt;original-width 104.375pt;original-height 106.375pt;cropleft "0";croptop
"1";cropright "1";cropbottom "0";filename
'graphics/3c010214.wmf';file-properties "XNPEU";}}}The graph of the \textbf{%
reciprocal function} $f(x)=x^{-1}=1/x$ is shown in Figure 14. Its graph has
the equation $y=1/x$, or $xy=1$, and is a hyperbola with the coordinate axes
as its asymptotes. This function arises in physics and chemistry in
connection with Boyle's Law, which says that, when the temperature is
constant, the volume $V$ of a gas is inversely proportional to the pressure $%
P$: \\[6pt]
\hspace*{\fill}$V=\dfrac{C}{P}$\hspace*{\fill}\\[6pt]
where $C$ is a constant. Thus the graph of $V$ as a function of $P$ (see
Figure 15) has the same general shape as the right half of Figure 14. 
%TCIMACRO{\TeXButton{longpage}{\enlargethispage{12pt}}}%
%BeginExpansion
\enlargethispage{12pt}%
%EndExpansion
\\[6pt]
\marginpar{\vspace{-36pt}\hspace*{\fill}%
\begin{tabular}[b]{r}
\QTR{FigureNumber}{FIGURE 15} \\ 
Volume as a function of \\ 
pressure at constant temperature%
\end{tabular}%
}\hspace*{\fill}\FRAME{itbpFU}{1.3062in}{1.3062in}{0in}{\Qcb{}}{}{%
3c010215.wmf}{\special{language "Scientific Word";type
"GRAPHIC";maintain-aspect-ratio TRUE;display "USEDEF";valid_file "F";width
1.3062in;height 1.3062in;depth 0in;original-width 1.6527in;original-height
1.6388in;cropleft "0";croptop "1";cropright "1";cropbottom "0";filename
'graphics/3c010215.wmf';file-properties "XNPEU";}}\hspace*{\fill}\pagebreak

Another instance in which a power function is used to model a physical
phenomenon is discussed in Exercise 26.\vspace*{9pt}

\subsection{RATIONAL FUNCTIONS\protect\vspace{-6pt}}

\marginpar{\hspace*{\fill}\FRAME{itbpFU}{94.3125pt}{97.25pt}{102.75pt}{\Qcb{%
\hspace*{\fill}%
\protect\begin{tabular}[t]{l}
\QTR{FigureNumber}{FIGURE 16}\quad \protect \\ 
$f(x)=\dfrac{2x^{4}-x^{2}+1}{x^{2}-4}$%
\protect\end{tabular}%
\hspace*{\fill}}}{}{3c010216.wmf}{\special{language "Scientific Word";type
"GRAPHIC";maintain-aspect-ratio TRUE;display "USEDEF";valid_file "F";width
94.3125pt;height 97.25pt;depth 102.75pt;original-width
1.3059in;original-height 1.3474in;cropleft "0";croptop "0.9984";cropright
"0.9989";cropbottom "0";filename 'graphics/3c010216.wmf';file-properties
"XNPEU";}}\hspace*{\fill}}A \textbf{rational function} $f$ is a ratio of two
polynomials:\\[6pt]
\hspace*{\fill}$f(x)=\dfrac{P(x)}{Q(x)}$\hspace*{\fill}\\[6pt]
where $P$ and $Q$ are polynomials. The domain consists of all values of $x$
such that $Q(x)\neq 0$. A simple example of a rational function is the
function $f(x)=1/x$, whose domain is $\{x\mid x\neq 0\}$; this is the
reciprocal function graphed in Figure 14. The function\\[6pt]
\hspace*{\fill}$f(x)=\dfrac{2x^{4}-x^{2}+1}{x^{2}-4}$\hspace*{\fill}\\[8pt]
is a rational function with domain $\{x\mid x\neq \pm 2\}$. Its graph is
shown in Figure 16.\vspace*{9pt}

\subsection{ALGEBRAIC FUNCTIONS\protect\vspace{-6pt}}

A function $f$ is called an \textbf{algebraic function} if it can be
constructed using algebraic operations (such as addition, subtraction,
multiplication, division, and taking roots) starting with polynomials. Any
rational function is automatically an algebraic function. Here are two more
examples:\\[6pt]
\hspace*{\fill}$f(x)=\sqrt{x^{2}+1}\qquad g(x)=\dfrac{x^{4}-16x^{2}}{x+\sqrt{%
x}}+(x-2)\,\sqrt[3]{x+1}$\hspace*{\fill}\\[6pt]
When we sketch algebraic functions in Chapter 4, we will see that their
graphs can assume a variety of shapes. Figure 17 illustrates some of the
possibilities.\vspace*{6pt}

\FRAME{itbpFU}{4.712in}{1.4658in}{0in}{\Qcb{\QTR{FigureNumber}{FIGURE 17}}}{%
}{6et010217_00069.ai}{\special{language "Scientific Word";type
"GRAPHIC";maintain-aspect-ratio TRUE;display "USEDEF";valid_file "F";width
4.712in;height 1.4658in;depth 0in;original-width 4.6837in;original-height
1.4383in;cropleft "0";croptop "1";cropright "1";cropbottom "0";filename
'graphics/6et010217_00069.ai';file-properties "XNPEU";}}\vspace*{6pt}

An example of an algebraic function occurs in the theory of relativity. The
mass of a particle with velocity $v$ is\\[6pt]
\hspace*{\fill}$m=f(v)=\dfrac{m_{0}}{\sqrt{1-v^{2}/c^{2}}}$\hspace*{\fill}\\[%
6pt]
where $m_{0}$ is the rest mass of the particle and $c=3.0\times 10^{5}$ km$/$%
s is the speed of light in a vacuum.%
%TCIMACRO{\TeXButton{longpage}{\enlargethispage{\baselineskip}}}%
%BeginExpansion
\enlargethispage{\baselineskip}%
%EndExpansion
\vspace*{9pt}

\subsection{TRIGONOMETRIC FUNCTIONS\protect\vspace{-6pt}}

\marginpar{\vspace{6pt}
\par
The Reference Pages are located at the front and back of the book.}%
Trigonometry and the trigonometric functions are reviewed on Reference Page
2 and also in Appendix D. In calculus the convention is that radian measure
is always \linebreak used (except when otherwise indicated). For example,
when we use the function \linebreak $f(x)=\sin \,x$, it is understood that $%
\sin \,x$ means the sine of the angle whose ra\nolinebreak dian measure is $%
x $. Thus the graphs of the sine and cosine functions are as shown in
Figure~18.\vspace{-6pt}

%TCIMACRO{\TeXButton{graphicS}{\vspace{12pt}\hskip-180pt\hfil}}%
%BeginExpansion
\vspace{12pt}\hskip-180pt\hfil%
%EndExpansion
\FRAME{itbpFU}{6.8208in}{1.2929in}{0in}{\Qcb{\QTR{FigureNumber}{FIGURE 18}}}{%
}{3c010218.wmf}{\special{language "Scientific Word";type
"GRAPHIC";maintain-aspect-ratio TRUE;display "USEDEF";valid_file "F";width
6.8208in;height 1.2929in;depth 0in;original-width 6.8208in;original-height
1.2929in;cropleft "0";croptop "1";cropright "1";cropbottom "0";filename
'graphics/3c010218.wmf';file-properties "XNPEU";}}%
%TCIMACRO{\TeXButton{graphicE}{\vspace{12pt}\hfil}}%
%BeginExpansion
\vspace{12pt}\hfil%
%EndExpansion

Notice that for both the sine and cosine functions the domain is $(-\infty
,\infty )$ and the range is the closed interval $[-1,1]$. Thus, for all
values of $x$, we have\\[9pt]
\hspace*{\fill}$\fbox{$\hspace{9pt}\rule[-7pt]{0pt}{20pt}-1\leq \sin \,x\leq
1$\qquad $-1\leq \cos \,x\leq 1\hspace{9pt}$}$\hspace*{\fill}\\[9pt]
or, in terms of absolute values,\\[6pt]
\hspace*{\fill}$\left\vert \sin x\right\vert \leq 1\qquad \qquad \left\vert
\cos x\right\vert \leq 1$\hspace*{\fill}\\[6pt]
Also, the zeros of the sine function occur at the integer multiples of $\pi $%
; that is,\\[6pt]
\hspace*{\fill}$\sin \,x=0$ \qquad when\qquad $x=n\pi \quad $ $n$ an integer%
\hspace*{\fill}\vspace*{6pt}

An important property of the sine and cosine functions is that they are
periodic functions and have period $2\pi $. This means that, for all values
of $x$,\\[9pt]
\hspace*{\fill}$\fbox{$\hspace{9pt}\rule[-7pt]{0pt}{20pt}\sin (x+2\pi )=\sin
\,x$\qquad $\cos (x+2\pi )=\cos \,x\hspace{9pt}$}$\hspace*{\fill}\\[9pt]
The periodic nature of these functions makes them suitable for modeling
repetitive phenomena such as tides, vibrating springs, and sound waves. For
instance, in Example 4 in Section 1.3 we will see that a reasonable model
for the number of hours of daylight in Philadelphia $t$ days after January 1
is given by the function\\[6pt]
\hspace*{\fill}$L(t)=12+2.8\sin \left[ \dfrac{2\pi }{365}(t-80)\right] $%
\hspace*{\fill}\vspace*{6pt}

The tangent function is related to the sine and cosine functions by the
equation\\[6pt]
\hspace*{\fill}$\tan \,x=\dfrac{\sin \,x}{\cos \,x}$\hspace*{\fill}\\[6pt]
\marginpar{\hspace*{\fill}\FRAME{itbpFU}{1.9726in}{1.5575in}{0in}{\Qcb{%
\QTR{FigureNumber}{FIGURE 19}\quad $y=\tan \,x$}}{}{3c010219.wmf}{\special%
{language "Scientific Word";type "GRAPHIC";maintain-aspect-ratio
TRUE;display "USEDEF";valid_file "F";width 1.9726in;height 1.5575in;depth
0in;original-width 363.0625pt;original-height 345.25pt;cropleft "0";croptop
"1";cropright "1";cropbottom "0";filename
'graphics/3c010219.wmf';file-properties "XNPEU";}}\hspace*{\fill}}and its
graph is shown in Figure 19. It is undefined whenever $\cos \,x=0$, that is,
when $x=\pm \pi /2$, $\pm 3\pi /2,\ldots $. Its range is $(-\infty ,\infty )$%
. Notice that the tangent function has period $\pi $:\\[6pt]
\hspace*{\fill}$\tan (x+\pi )=\tan \,x\qquad $for all $x$\hspace*{\fill}%
\vspace*{6pt}

The remaining three trigonometric functions (cosecant, secant, and
cotangent) are the reciprocals of the sine, cosine, and tangent functions.
Their graphs are shown in Appendix D.\vspace*{9pt}

\subsection{EXPONENTIAL FUNCTIONS\protect\vspace*{-8pt}}

The \textbf{exponential functions} are the functions of the form $f(x)=a^{x}$%
, where the base $a$ is a positive constant. The graphs of $y=2^{x}$ and $%
y=(0.5)^{x}$ are shown in Figure 20. In both cases the domain is $(-\infty
,\infty )$ and the range is $(0,\infty )$.\\[6pt]
\hspace*{\fill}\FRAME{itbpFU}{3.2077in}{1.5378in}{0in}{\Qcb{%
\QTR{FigureNumber}{FIGURE 20}}}{}{3c010220.wmf}{\special{language
"Scientific Word";type "GRAPHIC";maintain-aspect-ratio TRUE;display
"USEDEF";valid_file "F";width 3.2077in;height 1.5378in;depth
0in;original-width 178.6875pt;original-height 113.4375pt;cropleft
"0";croptop "1";cropright "1";cropbottom "0";filename
'graphics/3c010220.wmf';file-properties "XNPEU";}}\hspace*{\fill}\vspace*{6pt%
}

Exponential functions will be studied in detail in Section 1.5,\ and we will
see that they are useful for modeling many natural phenomena, such as
population growth (if $a>1$) and radioactive decay (if $a<1$).\vspace*{9pt}

\subsection{LOGARITHMIC FUNCTIONS\protect\vspace*{-8pt}}

The \textbf{logarithmic functions} $f(x)=\log _{a}\,x$, where the base $a$
is a positive constant, are the inverse functions of the exponential
functions. They will be studied in Section 1.6. Figure 21 shows the graphs
of four logarithmic functions with various bases. In each case the domain is 
$(0,\infty )$, the range is $(-\infty ,\infty )$, and the function increases
slowly when $x>1$.\\[4pt]
\hspace*{\fill}\FRAME{itbpFU}{1.9649in}{1.7462in}{0in}{\Qcb{%
\QTR{FigureNumber}{FIGURE 21}}}{}{3c010221.wmf}{\special{language
"Scientific Word";type "GRAPHIC";maintain-aspect-ratio TRUE;display
"USEDEF";valid_file "F";width 1.9649in;height 1.7462in;depth
0in;original-width 2.514in;original-height 1.7201in;cropleft "0";croptop
"1";cropright "1";cropbottom "0";filename
'graphics/3c010221.wmf';file-properties "XNPEU";}}\hspace*{\fill}\vspace*{8pt%
}

\subsection{TRANSCENDENTAL FUNCTIONS\protect\vspace*{-8pt}}

These are functions that are not algebraic. The set of transcendental
functions includes the trigonometric, inverse trigonometric, exponential,
and logarithmic functions, but it also includes a vast number of other
functions that have never been named. In Chapter~%
%TCIMACRO{%
%\TeXButton{CCC 8 / 5ET 11 / 5E 12}{\ifnum\CCC=1 8\else \ifnum\ET=1 11\else 12\fi \fi}}%
%BeginExpansion
\ifnum\CCC=1 8\else \ifnum\ET=1 11\else 12\fi \fi%
%EndExpansion
~we will study transcendental functions that are defined as sums of infinite
series.

\begin{Example}[5]
%TECHARTS_DUMMY_ITEM_TAG-(5)
Classify the following functions as one of the types of functions that we
have discussed.\vspace{-3pt}

\begin{enumerate}
\item[(a)]
$f(x)=5^{x}\hspace{43pt}$\textbf{(b)\hspace{7pt}}%
$g(x)=x^{5}$
\item[(c)]
$h(x)=\dfrac{1+x}{1-\sqrt{x}}\qquad $\textbf{(d)\hspace{7pt}}$%
u(t)=1-t+5t^{4}$%
%TCIMACRO{\TeXButton{longpage}{\enlargethispage{12pt}}}%
%BeginExpansion
\enlargethispage{12pt}%
%EndExpansion
\end{enumerate}
\end{Example}

\begin{Solution}
$\,\,\vspace{-4pt}$

\begin{enumerate}
\item[(a)] $f(x)=$ $5^{x}$ is an exponential function. (The $x$ is the
exponent.)

\item[(b)] $g(x)=x^{5}$ is a power function. (The $x$ is the base.) We could
also consider it to be a polynomial of degree 5.

\item[(c)] $h(x)=\dfrac{1+x}{1-\sqrt{x}}$ is an algebraic function.

\item[(d)] $u(t)=1-t+5t^{4}$ is a polynomial of degree 4.$\vspace{-12pt}%
\blacksquare $
\end{enumerate}
\end{Solution}

\QTP{MultColDiv}
Exercises 1.2\vspace{-6pt}

%TCIMACRO{%
%\TeXButton{s2col}{\setlength{\columnsep}{24pt}
%\advance \leftskip by -165pt
%\advance\hsize by 165pt
%\advance\linewidth by 165pt
%\begin{multicols}{2}}}%
%BeginExpansion
\setlength{\columnsep}{24pt}
\advance \leftskip by -165pt
\advance\hsize by 165pt
\advance\linewidth by 165pt
\begin{multicols}{2}%
%EndExpansion

\begin{instructions}
\hspace{9pt}\QTR{SpanExer}{1--2}{\small 
%TCIMACRO{\TeXButton{SQR}{\hskip .5em\rule{4pt}{4pt}\hskip .5em}}%
%BeginExpansion
\hskip .5em\rule{4pt}{4pt}\hskip .5em%
%EndExpansion
Classify each function as a power function, root function, }\newline
{\small polynomial (state its degree), rational function, algebraic
function, }\newline
{\small trigonometric function, exponential function, or logarithmic }%
\newline
{\small function.}
\end{instructions}

\begin{ExerciseList}
\item[$\hfill $1.] 

\begin{ExerciseList}
\item[(a)] $f(x)=\sqrt[5]{x}$\qquad 
%TCIMACRO{%
%\hyperref{\fbox{\textbf{master 00067}}}{}{\fbox{\textbf{master 00067}}}{}}%
%BeginExpansion
\msihyperref{\fbox{\textbf{master 00067}}}{}{\fbox{\textbf{master 00067}}}{}%
%EndExpansion

\item[(b)] $g(x)=\sqrt{1-x^{2}}$\qquad 
%TCIMACRO{%
%\hyperref{\fbox{\textbf{master 00068}}}{}{\fbox{\textbf{master 00068}}}{}}%
%BeginExpansion
\msihyperref{\fbox{\textbf{master 00068}}}{}{\fbox{\textbf{master 00068}}}{}%
%EndExpansion

\item[(c)] $h(x)=x^{9}+x^{4}$\qquad 
%TCIMACRO{%
%\hyperref{\fbox{\textbf{master 00069}}}{}{\fbox{\textbf{master 00069}}}{}}%
%BeginExpansion
\msihyperref{\fbox{\textbf{master 00069}}}{}{\fbox{\textbf{master 00069}}}{}%
%EndExpansion

\item[(d)] $r(x)=\dfrac{x^{2}+1}{x^{3}+x}$\qquad 
%TCIMACRO{%
%\hyperref{\fbox{\textbf{master 00070}}}{}{\fbox{\textbf{master 00070}}}{}}%
%BeginExpansion
\msihyperref{\fbox{\textbf{master 00070}}}{}{\fbox{\textbf{master 00070}}}{}%
%EndExpansion

\item[(e)] $s(x)=\tan 2x$\qquad 
%TCIMACRO{%
%\hyperref{\fbox{\textbf{master 00071}}}{}{\fbox{\textbf{master 00071}}}{}}%
%BeginExpansion
\msihyperref{\fbox{\textbf{master 00071}}}{}{\fbox{\textbf{master 00071}}}{}%
%EndExpansion

\item[(f)] $t(x)=\log _{10}x\vspace{3pt}$\qquad 
%TCIMACRO{%
%\hyperref{\fbox{\textbf{master 00072}}}{}{\fbox{\textbf{master 00072}}}{}}%
%BeginExpansion
\msihyperref{\fbox{\textbf{master 00072}}}{}{\fbox{\textbf{master 00072}}}{}%
%EndExpansion
\end{ExerciseList}

%TCIMACRO{%
%\hyperref{ANSWER}{}{\textbf{ANSWER:}\ (a)~Root\quad (b)~Algebraic\quad (c)~Polynomial (degree 9)\quad (d)~Rational\quad (e) ~Trigonometric\quad (f)~Logarithmic}{}}%
%BeginExpansion
\msihyperref{ANSWER}{}{\textbf{ANSWER:}\ (a)~Root\quad (b)~Algebraic\quad (c)~Polynomial (degree 9)\quad (d)~Rational\quad (e) ~Trigonometric\quad (f)~Logarithmic}{}%
%EndExpansion

\item[$\hfill $2.] 

\begin{ExerciseList}
\item[(a)] $y=\dfrac{x-6}{x+6}$\qquad 
%TCIMACRO{%
%\hyperref{\fbox{\textbf{master 00073}}}{}{\fbox{\textbf{master 00073}}}{}}%
%BeginExpansion
\msihyperref{\fbox{\textbf{master 00073}}}{}{\fbox{\textbf{master 00073}}}{}%
%EndExpansion

\item[(b)] $y=x+\dfrac{x^{2}}{\sqrt{x-1}}$\qquad 
%TCIMACRO{%
%\hyperref{\fbox{\textbf{master 00074}}}{}{\fbox{\textbf{master 00074}}}{}}%
%BeginExpansion
\msihyperref{\fbox{\textbf{master 00074}}}{}{\fbox{\textbf{master 00074}}}{}%
%EndExpansion

\item[(c)] $y=10^{x}$\qquad 
%TCIMACRO{%
%\hyperref{\fbox{\textbf{master 00075}}}{}{\fbox{\textbf{master 00075}}}{}}%
%BeginExpansion
\msihyperref{\fbox{\textbf{master 00075}}}{}{\fbox{\textbf{master 00075}}}{}%
%EndExpansion

\item[(d)] $y=x^{10}$\qquad 
%TCIMACRO{%
%\hyperref{\fbox{\textbf{master 00076}}}{}{\fbox{\textbf{master 00076}}}{}}%
%BeginExpansion
\msihyperref{\fbox{\textbf{master 00076}}}{}{\fbox{\textbf{master 00076}}}{}%
%EndExpansion

\item[(e)] $y=2t^{6}+t^{4}-\pi $\qquad 
%TCIMACRO{%
%\hyperref{\fbox{\textbf{master 00077}}}{}{\fbox{\textbf{master 00077}}}{}}%
%BeginExpansion
\msihyperref{\fbox{\textbf{master 00077}}}{}{\fbox{\textbf{master 00077}}}{}%
%EndExpansion

\item[(f)] $y=\cos \,\theta +\sin \,\theta $\qquad 
%TCIMACRO{%
%\hyperref{\fbox{\textbf{master 00078}}}{}{\fbox{\textbf{master 00078}}}{}}%
%BeginExpansion
\msihyperref{\fbox{\textbf{master 00078}}}{}{\fbox{\textbf{master 00078}}}{}%
%EndExpansion
\vspace{-6pt}
\end{ExerciseList}
\end{ExerciseList}

\begin{instructions}
\FRAME{itbpF}{235.6875pt}{5.5pt}{0pt}{}{}{dots.wmf}{\special{language
"Scientific Word";type "GRAPHIC";maintain-aspect-ratio TRUE;display
"USEDEF";valid_file "F";width 235.6875pt;height 5.5pt;depth
0pt;original-width 3.9167in;original-height 0.0735in;cropleft "0";croptop
"0.9772";cropright "0.8327";cropbottom "0.0226";filename
'graphics/dots.wmf';file-properties "XNPEU";}}\vspace{-6pt}
\end{instructions}

\begin{instructions}
\QTR{SpanExer}{3--4}{\small 
%TCIMACRO{\TeXButton{SQR}{\hskip .5em\rule{4pt}{4pt}\hskip .5em}}%
%BeginExpansion
\hskip .5em\rule{4pt}{4pt}\hskip .5em%
%EndExpansion
Match each equation with its graph. Explain your choices. (Don't use a
computer or graphing calculator.)}
\end{instructions}

\begin{ExerciseList}
\item[{\hfill \protect\fbox{3.\hspace{-2pt}}}] 

\begin{ExerciseList}
\item[(a)] $y=x^{2}$\hspace{36pt}\textbf{(b)}\hspace{5pt}$y=x^{5}$\hspace{%
36pt}\textbf{(c)}\hspace{5pt}$y=x^{8}$\vspace*{6pt}
\end{ExerciseList}

%TCIMACRO{%
%\hyperref{ANSWER}{}{\textbf{ANSWER:}\ (a)~$h$\quad (b)~$f$\quad (c)~$g$}{}}%
%BeginExpansion
\msihyperref{ANSWER}{}{\textbf{ANSWER:}\ (a)~$h$\quad (b)~$f$\quad (c)~$g$}{}%
%EndExpansion

%TCIMACRO{%
%\hyperref{\fbox{\textbf{master 00079}}}{}{\fbox{\textbf{master 00079}}}{}}%
%BeginExpansion
\msihyperref{\fbox{\textbf{master 00079}}}{}{\fbox{\textbf{master 00079}}}{}%
%EndExpansion

\hspace*{\fill}\FRAME{itbpF}{1.9726in}{1.8054in}{0in}{}{}{3c0102x03.wmf}{%
\special{language "Scientific Word";type "GRAPHIC";maintain-aspect-ratio
TRUE;display "USEDEF";valid_file "F";width 1.9726in;height 1.8054in;depth
0in;original-width 1.9441in;original-height 1.7772in;cropleft "0";croptop
"1";cropright "1";cropbottom "0";filename
'graphics/3c0102x03.wmf';file-properties "XNPEU";}}\hspace*{\fill}

\item[\hfill 4.] 

\begin{ExerciseList}
\item[(a)] $y=3x\hspace{15pt}$\textbf{(b)}\hspace{5pt}$y=3^{x}\hspace{15pt}$%
\textbf{(c)}\hspace{5pt}$y=x^{3}\hspace{15pt}$\textbf{(d)}\hspace{5pt}$y=%
\sqrt[3]{x}$\vspace*{6pt}
\end{ExerciseList}

\hspace*{\fill}\FRAME{itbpF}{1.9581in}{1.7771in}{0in}{}{}{3c0102x04.wmf}{%
\special{language "Scientific Word";type "GRAPHIC";maintain-aspect-ratio
TRUE;display "USEDEF";valid_file "F";width 1.9581in;height 1.7771in;depth
0in;original-width 1.9579in;original-height 1.7772in;cropleft "0";croptop
"1";cropright "1";cropbottom "0";filename
'graphics/3c0102x04.wmf';file-properties "XNPEU";}}\hspace*{\fill}

%TCIMACRO{%
%\hyperref{\fbox{\textbf{master 00080a}}}{}{\fbox{\textbf{master 00080a}}}{}}%
%BeginExpansion
\msihyperref{\fbox{\textbf{master 00080a}}}{}{\fbox{\textbf{master 00080a}}}{}%
%EndExpansion
\quad 
%TCIMACRO{%
%\hyperref{\fbox{\textbf{master 00080a}}}{}{\fbox{\textbf{master 00080a}}}{}}%
%BeginExpansion
\msihyperref{\fbox{\textbf{master 00080a}}}{}{\fbox{\textbf{master 00080a}}}{}%
%EndExpansion

%TCIMACRO{%
%\hyperref{\fbox{\textbf{master 00080c}}}{}{\fbox{\textbf{master 00080c}}}{}}%
%BeginExpansion
\msihyperref{\fbox{\textbf{master 00080c}}}{}{\fbox{\textbf{master 00080c}}}{}%
%EndExpansion
\quad 
%TCIMACRO{%
%\hyperref{\fbox{\textbf{master 00080d}}}{}{\fbox{\textbf{master 00080d}}}{}}%
%BeginExpansion
\msihyperref{\fbox{\textbf{master 00080d}}}{}{\fbox{\textbf{master 00080d}}}{}%
%EndExpansion
\vspace{-24pt}
\end{ExerciseList}

\begin{instructions}
\FRAME{itbpF}{235.6875pt}{5.5pt}{0pt}{}{}{dots.wmf}{\special{language
"Scientific Word";type "GRAPHIC";maintain-aspect-ratio TRUE;display
"USEDEF";valid_file "F";width 235.6875pt;height 5.5pt;depth
0pt;original-width 3.9167in;original-height 0.0735in;cropleft "0";croptop
"0.9772";cropright "0.8327";cropbottom "0.0226";filename
'graphics/dots.wmf';file-properties "XNPEU";}}
\end{instructions}

\begin{ExerciseList}
\item[{\hfill \protect\fbox{5.\hspace{-2pt}}}] 

\begin{ExerciseList}
\item[(a)] Find an equation for the family of linear functions with slope 2
and sketch several members of the family.

%TCIMACRO{%
%\hyperref{ANSWER}{}{\textbf{ANSWER:}\ (a)~$y=2x+b$, where $b$ is the \textit{y}-intercept\newline
%\FRAME{itbpF}{1.2496in}{1.3197in}{1.0032in}{}{}{4a010205a.wmf}{\special{language "Scientific Word";type "GRAPHIC";maintain-aspect-ratio TRUE;display "USEDEF";valid_file "F";width 1.2496in;height 1.3197in;depth 1.0032in;original-width 6.1229in;original-height 6.4627in;cropleft "0";croptop "1";cropright "1";cropbottom "0";filename '../../../4c3/4c-SW-ch01/graphics/4a010205a.wmf';file-properties "XNPEU";}}\newline
%}{}}%
%BeginExpansion
\msihyperref{ANSWER}{}{\textbf{ANSWER:}\ (a)~$y=2x+b$, where $b$ is the \textit{y}-intercept\newline
\FRAME{itbpF}{1.2496in}{1.3197in}{1.0032in}{}{}{4a010205a.wmf}{\special{language "Scientific Word";type "GRAPHIC";maintain-aspect-ratio TRUE;display "USEDEF";valid_file "F";width 1.2496in;height 1.3197in;depth 1.0032in;original-width 6.1229in;original-height 6.4627in;cropleft "0";croptop "1";cropright "1";cropbottom "0";filename '../../../4c3/4c-SW-ch01/graphics/4a010205a.wmf';file-properties "XNPEU";}}\newline
}{}%
%EndExpansion

%TCIMACRO{%
%\hyperref{\fbox{\textbf{master 00081a}}}{}{\fbox{\textbf{master 00081a}}}{}}%
%BeginExpansion
\msihyperref{\fbox{\textbf{master 00081a}}}{}{\fbox{\textbf{master 00081a}}}{}%
%EndExpansion

\item[(b)] Find an equation for the family of linear functions such that $%
f(2)=1$ and sketch several members of the family.

%TCIMACRO{%
%\hyperref{ANSWER}{}{\textbf{ANSWER:}\ \FRAME{itbpF}{1.2496in}{1.3197in}{1.0032in}{}{}{4a010205a.wmf}{\special{language "Scientific Word";type "GRAPHIC";maintain-aspect-ratio TRUE;display "USEDEF";valid_file "F";width 1.2496in;height 1.3197in;depth 1.0032in;original-width 6.1229in;original-height 6.4627in;cropleft "0";croptop "1";cropright "1";cropbottom "0";filename '../../../4c3/4c-SW-ch01/graphics/4a010205a.wmf';file-properties "XNPEU";}}\newline
%(b)~$y=mx+1-2m$, where $m$ is the slope\newline
%\FRAME{itbpF}{1.2081in}{1.2081in}{1.0032in}{}{}{4a010205b.wmf}{\special{language "Scientific Word";type "GRAPHIC";maintain-aspect-ratio TRUE;display "USEDEF";valid_file "F";width 1.2081in;height 1.2081in;depth 1.0032in;original-width 6.4627in;original-height 6.4627in;cropleft "0";croptop "1";cropright "1";cropbottom "0";filename '../../../4c3/4c-SW-ch01/graphics/4a010205b.wmf';file-properties "XNPEU";}}\newline
%}{}}%
%BeginExpansion
\msihyperref{ANSWER}{}{\textbf{ANSWER:}\ \FRAME{itbpF}{1.2496in}{1.3197in}{1.0032in}{}{}{4a010205a.wmf}{\special{language "Scientific Word";type "GRAPHIC";maintain-aspect-ratio TRUE;display "USEDEF";valid_file "F";width 1.2496in;height 1.3197in;depth 1.0032in;original-width 6.1229in;original-height 6.4627in;cropleft "0";croptop "1";cropright "1";cropbottom "0";filename '../../../4c3/4c-SW-ch01/graphics/4a010205a.wmf';file-properties "XNPEU";}}\newline
(b)~$y=mx+1-2m$, where $m$ is the slope\newline
\FRAME{itbpF}{1.2081in}{1.2081in}{1.0032in}{}{}{4a010205b.wmf}{\special{language "Scientific Word";type "GRAPHIC";maintain-aspect-ratio TRUE;display "USEDEF";valid_file "F";width 1.2081in;height 1.2081in;depth 1.0032in;original-width 6.4627in;original-height 6.4627in;cropleft "0";croptop "1";cropright "1";cropbottom "0";filename '../../../4c3/4c-SW-ch01/graphics/4a010205b.wmf';file-properties "XNPEU";}}\newline
}{}%
%EndExpansion

%TCIMACRO{%
%\hyperref{\fbox{\textbf{master 00081b}}}{}{\fbox{\textbf{master 00081b}}}{}}%
%BeginExpansion
\msihyperref{\fbox{\textbf{master 00081b}}}{}{\fbox{\textbf{master 00081b}}}{}%
%EndExpansion

\item[(c)] Which function belongs to both families?

%TCIMACRO{\hyperref{ANSWER}{}{\textbf{ANSWER:}\ (c)~$y=2x-3$}{}}%
%BeginExpansion
\msihyperref{ANSWER}{}{\textbf{ANSWER:}\ (c)~$y=2x-3$}{}%
%EndExpansion

%TCIMACRO{%
%\hyperref{\fbox{\textbf{master 00081c}}}{}{\fbox{\textbf{master 00081c}}}{}}%
%BeginExpansion
\msihyperref{\fbox{\textbf{master 00081c}}}{}{\fbox{\textbf{master 00081c}}}{}%
%EndExpansion
\end{ExerciseList}

\item[\hfill 6.] What do all members of the family of linear functions 
\newline
$f(x)=1+m(x+3)$ have in common? Sketch several members of the family.

%TCIMACRO{%
%\hyperref{\fbox{\textbf{master 00082}}}{}{\fbox{\textbf{master 00082}}}{}}%
%BeginExpansion
\msihyperref{\fbox{\textbf{master 00082}}}{}{\fbox{\textbf{master 00082}}}{}%
%EndExpansion

\item[\hfill 7.] What do all members of the family of linear functions 
\newline
$f(x)=c-x$ have in common? Sketch several members of the family.

%TCIMACRO{%
%\hyperref{ANSWER}{}{\textbf{ANSWER:} Their graphs have slope $-1$.\newline
%\FRAME{itbpF}{1.5143in}{1.2488in}{0in}{}{}{5et010207.wmf}{\special{language "Scientific Word";type "GRAPHIC";maintain-aspect-ratio TRUE;display "USEDEF";valid_file "F";width 1.5143in;height 1.2488in;depth 0in;original-width 1.5143in;original-height 1.2488in;cropleft "0";croptop "1";cropright "1";cropbottom "0";filename 'graphics/5et010207.wmf';file-properties "XNPEU";}}}{}}%
%BeginExpansion
\msihyperref{ANSWER}{}{\textbf{ANSWER:} Their graphs have slope $-1$.\newline
\FRAME{itbpF}{1.5143in}{1.2488in}{0in}{}{}{5et010207.wmf}{\special{language "Scientific Word";type "GRAPHIC";maintain-aspect-ratio TRUE;display "USEDEF";valid_file "F";width 1.5143in;height 1.2488in;depth 0in;original-width 1.5143in;original-height 1.2488in;cropleft "0";croptop "1";cropright "1";cropbottom "0";filename 'graphics/5et010207.wmf';file-properties "XNPEU";}}}{}%
%EndExpansion

%TCIMACRO{%
%\hyperref{\fbox{\textbf{master 00083}}}{}{\fbox{\textbf{master 00083}}}{}}%
%BeginExpansion
\msihyperref{\fbox{\textbf{master 00083}}}{}{\fbox{\textbf{master 00083}}}{}%
%EndExpansion

\item[\hfill 8.] Find expressions for the quadratic functions whose graphs
are shown.

\FRAME{dtbpF}{2.7492in}{0.9816in}{0pt}{}{}{6et0102x08_00076.ai}{\special%
{language "Scientific Word";type "GRAPHIC";maintain-aspect-ratio
TRUE;display "USEDEF";valid_file "F";width 2.7492in;height 0.9816in;depth
0pt;original-width 2.7224in;original-height 0.9539in;cropleft "0";croptop
"1";cropright "1";cropbottom "0";filename
'graphics/6et0102x08_00076.ai';file-properties "XNPEU";}}

%TCIMACRO{%
%\hyperref{\fbox{\textbf{master 30016}}}{}{\fbox{\textbf{master 30016}}}{}}%
%BeginExpansion
\msihyperref{\fbox{\textbf{master 30016}}}{}{\fbox{\textbf{master 30016}}}{}%
%EndExpansion

\item[9.] Find an expression for a cubic function $f$ if $f(1)=6$ and

$f(-1)=f(0)=f(2)=0$.

%TCIMACRO{\hyperref{ANSWER}{}{\textbf{ANSWER:} $f(x)=-3x(x+1)(x-2)$}{}}%
%BeginExpansion
\msihyperref{ANSWER}{}{\textbf{ANSWER:} $f(x)=-3x(x+1)(x-2)$}{}%
%EndExpansion

%TCIMACRO{%
%\hyperref{\fbox{\textbf{master 30017}}}{}{\fbox{\textbf{master 30017}}}{}}%
%BeginExpansion
\msihyperref{\fbox{\textbf{master 30017}}}{}{\fbox{\textbf{master 30017}}}{}%
%EndExpansion

\item[10.] Recent studies indicate that the average surface temperature of
the earth has been rising steadily. Some scientists have modeled the
temperature by the linear function $T=0.02t+8.50$, where $T$ is temperature
in $^{\circ }\mathrm{C}$ and $t$ represents years since 1900.

\begin{ExerciseList}
\item[(a)] What do the slope and $T$-intercept represent?
\end{ExerciseList}

%TCIMACRO{%
%\hyperref{\fbox{\textbf{master 30018a}}}{}{\fbox{\textbf{master 30018a}}}{}}%
%BeginExpansion
\msihyperref{\fbox{\textbf{master 30018a}}}{}{\fbox{\textbf{master 30018a}}}{}%
%EndExpansion

\begin{ExerciseList}
\item[(b)] Use the equation to predict the average global surface
temperature in 2100.
\end{ExerciseList}

%TCIMACRO{%
%\hyperref{\fbox{\textbf{master 30018b}}}{}{\fbox{\textbf{master 30018b}}}{}}%
%BeginExpansion
\msihyperref{\fbox{\textbf{master 30018b}}}{}{\fbox{\textbf{master 30018b}}}{}%
%EndExpansion

\item[11.] If the recommended adult dosage for a drug is $D$ (in mg), then
to determine the appropriate dosage $c$ for a child of age $a$, pharmacists
use the equation $c=0.0417D(a+1)$. Suppose the dosage for an adult is 200 mg.

\begin{ExerciseList}
\item[(a)] Find the slope of the graph of $c$. What does it represent?
\end{ExerciseList}

%TCIMACRO{%
%\hyperref{\fbox{\textbf{master 30019a}}}{}{\fbox{\textbf{master 30019a}}}{}}%
%BeginExpansion
\msihyperref{\fbox{\textbf{master 30019a}}}{}{\fbox{\textbf{master 30019a}}}{}%
%EndExpansion

\begin{ExerciseList}
\item[(b)] What is the dosage for a newborn?
\end{ExerciseList}

%TCIMACRO{%
%\hyperref{\fbox{\textbf{master 30019b}}}{}{\fbox{\textbf{master 30019b}}}{}}%
%BeginExpansion
\msihyperref{\fbox{\textbf{master 30019b}}}{}{\fbox{\textbf{master 30019b}}}{}%
%EndExpansion

%TCIMACRO{%
%\hyperref{ANSWER}{}{\textbf{ANSWER:} (a)~8.34, change in mg for every 1 year change\quad (b)~8.34 mg}{}}%
%BeginExpansion
\msihyperref{ANSWER}{}{\textbf{ANSWER:} (a)~8.34, change in mg for every 1 year change\quad (b)~8.34 mg}{}%
%EndExpansion

\item[12.] The manager of a weekend flea market knows from past experience
that if he charges $x$ dollars for a rental space at the flea market, then
the number $y$ of spaces he can rent is given by the equation $y=200-4x$.%
\vspace{-2pt}

\begin{ExerciseList}
\item[(a)] Sketch a graph of this linear function. (Remember that the rental
charge per space and the number of spaces rented can't be negative
quantities.)

%TCIMACRO{%
%\hyperref{\fbox{\textbf{master 00084a}}}{}{\fbox{\textbf{master 00084a}}}{}}%
%BeginExpansion
\msihyperref{\fbox{\textbf{master 00084a}}}{}{\fbox{\textbf{master 00084a}}}{}%
%EndExpansion

\item[(b)] What do the slope, the $y$-intercept, and the $x$-intercept of
the graph represent?

%TCIMACRO{%
%\hyperref{\fbox{\textbf{master 00084b}}}{}{\fbox{\textbf{master 00084b}}}{}}%
%BeginExpansion
\msihyperref{\fbox{\textbf{master 00084b}}}{}{\fbox{\textbf{master 00084b}}}{}%
%EndExpansion
\end{ExerciseList}

\item[13.] The relationship between the Fahrenheit ($F$) and Celsius ($C$)
temperature scales is given by the linear function \newline
$F=\frac{9}{5}C+32$.\vspace{-2pt}

\begin{ExerciseList}
\item[(a)] Sketch a graph of this function.

%TCIMACRO{%
%\hyperref{\fbox{\textbf{master 00085a}}}{}{\fbox{\textbf{master 00085a}}}{}}%
%BeginExpansion
\msihyperref{\fbox{\textbf{master 00085a}}}{}{\fbox{\textbf{master 00085a}}}{}%
%EndExpansion

\item[(b)] What is the slope of the graph and what does it represent? What
is the $F$-intercept and what does it represent?

%TCIMACRO{%
%\hyperref{\fbox{\textbf{master 00085b}}}{}{\fbox{\textbf{master 00085b}}}{}}%
%BeginExpansion
\msihyperref{\fbox{\textbf{master 00085b}}}{}{\fbox{\textbf{master 00085b}}}{}%
%EndExpansion
\end{ExerciseList}

%TCIMACRO{%
%\hyperref{ANSWER}{}{\textbf{ANSWER:}\ (a)~\FRAME{itbpF}{1.4382in}{1.2981in}{1.254in}{}{}{4a010207a.wmf}{\special{language "Scientific Word";type "GRAPHIC";maintain-aspect-ratio TRUE;display "USEDEF";valid_file "F";width 1.4382in;height 1.2981in;depth 1.254in;original-width 6.4627in;original-height 5.8237in;cropleft "0";croptop "1";cropright "1";cropbottom "0";filename 'graphics/4a010207a.wmf';file-properties "XNPEU";}}\newline
%(b)~$\tfrac{9}{5}$, change in $^{\circ }$F for every 1$^{\circ }$C change; 32, Fahrenheit temperature corresponding to 0$^{\circ }$C}{}}%
%BeginExpansion
\msihyperref{ANSWER}{}{\textbf{ANSWER:}\ (a)~\FRAME{itbpF}{1.4382in}{1.2981in}{1.254in}{}{}{4a010207a.wmf}{\special{language "Scientific Word";type "GRAPHIC";maintain-aspect-ratio TRUE;display "USEDEF";valid_file "F";width 1.4382in;height 1.2981in;depth 1.254in;original-width 6.4627in;original-height 5.8237in;cropleft "0";croptop "1";cropright "1";cropbottom "0";filename 'graphics/4a010207a.wmf';file-properties "XNPEU";}}\newline
(b)~$\tfrac{9}{5}$, change in $^{\circ }$F for every 1$^{\circ }$C change; 32, Fahrenheit temperature corresponding to 0$^{\circ }$C}{}%
%EndExpansion

\item[14.] Jason leaves Detroit at 2:00 \textsc{pm} and drives at a constant
speed west along I-96. He passes Ann Arbor, 40 mi from Detroit, at 2:50 
\textsc{pm}.\textsc{\vspace{-2pt}}

\begin{ExerciseList}
\item[(a)] Express the distance traveled in terms of the time elapsed.

%TCIMACRO{%
%\hyperref{\fbox{\textbf{master 00086a}}}{}{\fbox{\textbf{master 00086a}}}{}}%
%BeginExpansion
\msihyperref{\fbox{\textbf{master 00086a}}}{}{\fbox{\textbf{master 00086a}}}{}%
%EndExpansion

\item[(b)] Draw the graph of the equation in part (a).

%TCIMACRO{%
%\hyperref{\fbox{\textbf{master 00086b}}}{}{\fbox{\textbf{master 00086b}}}{}}%
%BeginExpansion
\msihyperref{\fbox{\textbf{master 00086b}}}{}{\fbox{\textbf{master 00086b}}}{}%
%EndExpansion

\item[(c)] What is the slope of this line? What does it represent?

%TCIMACRO{%
%\hyperref{\fbox{\textbf{master 00086c}}}{}{\fbox{\textbf{master 00086c}}}{}}%
%BeginExpansion
\msihyperref{\fbox{\textbf{master 00086c}}}{}{\fbox{\textbf{master 00086c}}}{}%
%EndExpansion
\end{ExerciseList}

\item[{\hfill \protect\fbox{\hspace{-2pt}15.\hspace{-2pt}}}] Biologists have
noticed that the chirping rate of crickets of a certain species is related
to temperature, and the relationship appears to be very nearly linear. A
cricket produces 113 chirps per minute at $70^{\circ }$F and 173 chirps per
minute at $80^{\circ }$F.\vspace{-2pt}

\begin{ExerciseList}
\item[(a)] Find a linear equation that models the temperature $T$ as a
function of the number of chirps per minute $N$.

%TCIMACRO{%
%\hyperref{\fbox{\textbf{master 00087a}}}{}{\fbox{\textbf{master 00087a}}}{}}%
%BeginExpansion
\msihyperref{\fbox{\textbf{master 00087a}}}{}{\fbox{\textbf{master 00087a}}}{}%
%EndExpansion

\item[(b)] What is the slope of the graph? What does it represent?

%TCIMACRO{%
%\hyperref{\fbox{\textbf{master 00087b}}}{}{\fbox{\textbf{master 00087b}}}{}}%
%BeginExpansion
\msihyperref{\fbox{\textbf{master 00087b}}}{}{\fbox{\textbf{master 00087b}}}{}%
%EndExpansion

\item[(c)] If the crickets are chirping at 150 chirps per minute, estimate
the temperature.

%TCIMACRO{%
%\hyperref{\fbox{\textbf{master 00087c}}}{}{\fbox{\textbf{master 00087c}}}{}}%
%BeginExpansion
\msihyperref{\fbox{\textbf{master 00087c}}}{}{\fbox{\textbf{master 00087c}}}{}%
%EndExpansion
\end{ExerciseList}

%TCIMACRO{%
%\hyperref{ANSWER}{}{\textbf{ANSWER:}\ (a)~$T=\tfrac{1}{6}N+\tfrac{307}{6}$\quad (b)~$\tfrac{1}{6}$, change in $^{\circ }$F for every chirp per minute change\quad (c)~76$^{\circ }$F}{}}%
%BeginExpansion
\msihyperref{ANSWER}{}{\textbf{ANSWER:}\ (a)~$T=\tfrac{1}{6}N+\tfrac{307}{6}$\quad (b)~$\tfrac{1}{6}$, change in $^{\circ }$F for every chirp per minute change\quad (c)~76$^{\circ }$F}{}%
%EndExpansion

\item[16.] The manager of a furniture factory finds that it costs \$2200 to
manufacture 100 chairs in one day and \$4800 to produce 300 chairs in one
day.\vspace{-2pt}

\begin{ExerciseList}
\item[(a)] Express the cost as a function of the number of chairs produced,
assuming that it is linear. Then sketch the graph.

%TCIMACRO{%
%\hyperref{\fbox{\textbf{master 00088a}}}{}{\fbox{\textbf{master 00088a}}}{}}%
%BeginExpansion
\msihyperref{\fbox{\textbf{master 00088a}}}{}{\fbox{\textbf{master 00088a}}}{}%
%EndExpansion

\item[(b)] What is the slope of the graph and what does it represent?

%TCIMACRO{%
%\hyperref{\fbox{\textbf{master 00088b}}}{}{\fbox{\textbf{master 00088b}}}{}}%
%BeginExpansion
\msihyperref{\fbox{\textbf{master 00088b}}}{}{\fbox{\textbf{master 00088b}}}{}%
%EndExpansion

\item[(c)] What is the $y$-intercept of the graph and what does it \newline
represent?

%TCIMACRO{%
%\hyperref{\fbox{\textbf{master 00088c}}}{}{\fbox{\textbf{master 00088c}}}{}}%
%BeginExpansion
\msihyperref{\fbox{\textbf{master 00088c}}}{}{\fbox{\textbf{master 00088c}}}{}%
%EndExpansion
\end{ExerciseList}

\item[{\hfill \protect\fbox{\hspace{-2pt}17.\hspace{-2pt}}}] At the surface
of the ocean, the water pressure is the same as the air pressure above the
water, 15 lb$/$in$^{2}$. Below the surface, the water pressure increases by
4.34 lb$/$in$^{2}$ for every 10 ft of descent.\vspace{-2pt}

\begin{ExerciseList}
\item[(a)] Express the water pressure as a function of the depth below the
ocean surface.

%TCIMACRO{%
%\hyperref{\fbox{\textbf{master 00089a}}}{}{\fbox{\textbf{master 00089a}}}{}}%
%BeginExpansion
\msihyperref{\fbox{\textbf{master 00089a}}}{}{\fbox{\textbf{master 00089a}}}{}%
%EndExpansion

\item[(b)] At what depth is the pressure 100 lb$/$in$^{2}$?

%TCIMACRO{%
%\hyperref{\fbox{\textbf{master 00089b}}}{}{\fbox{\textbf{master 00089b}}}{}}%
%BeginExpansion
\msihyperref{\fbox{\textbf{master 00089b}}}{}{\fbox{\textbf{master 00089b}}}{}%
%EndExpansion
\end{ExerciseList}

%TCIMACRO{%
%\hyperref{ANSWER}{}{\textbf{ANSWER:}\ (a)~$P=0.434d+15$\quad (b)~196~ft}{}}%
%BeginExpansion
\msihyperref{ANSWER}{}{\textbf{ANSWER:}\ (a)~$P=0.434d+15$\quad (b)~196~ft}{}%
%EndExpansion

\item[18.] The monthly cost of driving a car depends on the number of miles
driven. Lynn found that in May it cost her \$380 to drive 480 mi and in June
it cost her \$460 to drive 800 mi.\vspace{-2pt}

\begin{ExerciseList}
\item[(a)] Express the monthly cost $C$ as a function of the distance driven 
$d$, assuming that a linear relationship gives a suitable model.

%TCIMACRO{%
%\hyperref{\fbox{\textbf{master 00090a}}}{}{\fbox{\textbf{master 00090a}}}{}}%
%BeginExpansion
\msihyperref{\fbox{\textbf{master 00090a}}}{}{\fbox{\textbf{master 00090a}}}{}%
%EndExpansion

\item[(b)] Use part (a) to predict the cost of driving 1500 miles per month.

%TCIMACRO{%
%\hyperref{\fbox{\textbf{master 00090b}}}{}{\fbox{\textbf{master 00090b}}}{}}%
%BeginExpansion
\msihyperref{\fbox{\textbf{master 00090b}}}{}{\fbox{\textbf{master 00090b}}}{}%
%EndExpansion

\item[(c)] Draw the graph of the linear function. What does the slope
represent?

%TCIMACRO{%
%\hyperref{\fbox{\textbf{master 00090c}}}{}{\fbox{\textbf{master 00090c}}}{}}%
%BeginExpansion
\msihyperref{\fbox{\textbf{master 00090c}}}{}{\fbox{\textbf{master 00090c}}}{}%
%EndExpansion

\item[(d)] What does the $y$-intercept represent?

%TCIMACRO{%
%\hyperref{\fbox{\textbf{master 00090d}}}{}{\fbox{\textbf{master 00090d}}}{}}%
%BeginExpansion
\msihyperref{\fbox{\textbf{master 00090d}}}{}{\fbox{\textbf{master 00090d}}}{}%
%EndExpansion

\item[(e)] Why does a linear function give a suitable model in this
situation?

%TCIMACRO{%
%\hyperref{\fbox{\textbf{master 00090e}}}{}{\fbox{\textbf{master 00090e}}}{}}%
%BeginExpansion
\msihyperref{\fbox{\textbf{master 00090e}}}{}{\fbox{\textbf{master 00090e}}}{}%
%EndExpansion
\end{ExerciseList}
\end{ExerciseList}

\begin{instructions}
\QTR{SpanExer}{19--20}{\small 
%TCIMACRO{\TeXButton{SQR}{\hskip .5em\rule{4pt}{4pt}\hskip .5em}}%
%BeginExpansion
\hskip .5em\rule{4pt}{4pt}\hskip .5em%
%EndExpansion
For each scatter plot, decide what type of function you might choose as a
model for the data. Explain your choices.}
\end{instructions}

\begin{ExerciseList}
\item[19.] 

\begin{ExerciseList}
\item[(a)] \ \FRAME{itbpF}{1.4304in}{1.4304in}{1.305in}{}{}{4e0102x13a.wmf}{%
\special{language "Scientific Word";type "GRAPHIC";maintain-aspect-ratio
TRUE;display "USEDEF";valid_file "F";width 1.4304in;height 1.4304in;depth
1.305in;original-width 1.4304in;original-height 1.4304in;cropleft
"0";croptop "1";cropright "1";cropbottom "0";filename
'graphics/4e0102x13a.wmf';file-properties "XNPEU";}}\quad 
%TCIMACRO{%
%\hyperref{\fbox{\textbf{master 00091}}}{}{\fbox{\textbf{master 00091}}}{}}%
%BeginExpansion
\msihyperref{\fbox{\textbf{master 00091}}}{}{\fbox{\textbf{master 00091}}}{}%
%EndExpansion
\vspace{6pt}

%TCIMACRO{\hyperref{ANSWER}{}{\textbf{ANSWER:}\ (a)~Cosine}{}}%
%BeginExpansion
\msihyperref{ANSWER}{}{\textbf{ANSWER:}\ (a)~Cosine}{}%
%EndExpansion

\item[(b)] \QTR{PartLetter}{\ }\FRAME{itbpF}{1.4304in}{1.4304in}{1.305in}{}{%
}{4e0102x13b.wmf}{\special{language "Scientific Word";type
"GRAPHIC";maintain-aspect-ratio TRUE;display "USEDEF";valid_file "F";width
1.4304in;height 1.4304in;depth 1.305in;original-width
1.4304in;original-height 1.4304in;cropleft "0";croptop "1";cropright
"1";cropbottom "0";filename 'graphics/4e0102x13b.wmf';file-properties
"XNPEU";}}\quad 
%TCIMACRO{%
%\hyperref{\fbox{\textbf{master 00092}}}{}{\fbox{\textbf{master 00092}}}{}}%
%BeginExpansion
\msihyperref{\fbox{\textbf{master 00092}}}{}{\fbox{\textbf{master 00092}}}{}%
%EndExpansion

%TCIMACRO{\hyperref{ANSWER}{}{\textbf{ANSWER:}\ (b)~Linear}{}}%
%BeginExpansion
\msihyperref{ANSWER}{}{\textbf{ANSWER:}\ (b)~Linear}{}%
%EndExpansion
\end{ExerciseList}

\item[20.] 

\begin{ExerciseList}
\item[(a)] \ \FRAME{itbpF}{1.2635in}{1.2782in}{1.305in}{}{}{4e0102x14a.wmf}{%
\special{language "Scientific Word";type "GRAPHIC";maintain-aspect-ratio
TRUE;display "USEDEF";valid_file "F";width 1.2635in;height 1.2782in;depth
1.305in;original-width 1.2635in;original-height 1.2782in;cropleft
"0";croptop "1";cropright "1";cropbottom "0";filename
'graphics/4e0102x14a.wmf';file-properties "XNPEU";}}\quad 
%TCIMACRO{%
%\hyperref{\fbox{\textbf{master 00093}}}{}{\fbox{\textbf{master 00093}}}{}}%
%BeginExpansion
\msihyperref{\fbox{\textbf{master 00093}}}{}{\fbox{\textbf{master 00093}}}{}%
%EndExpansion
\vspace{6pt}

\item[(b)] \QTR{PartLetter}{\ }\FRAME{itbpF}{1.299in}{1.3128in}{1.305in}{}{}{%
4e0102x14b.wmf}{\special{language "Scientific Word";type
"GRAPHIC";maintain-aspect-ratio TRUE;display "USEDEF";valid_file "F";width
1.299in;height 1.3128in;depth 1.305in;original-width
1.2635in;original-height 1.2782in;cropleft "0";croptop "1";cropright
"1";cropbottom "0";filename 'graphics/4e0102x14b.wmf';file-properties
"XNPEU";}}\quad 
%TCIMACRO{%
%\hyperref{\fbox{\textbf{master 00094}}}{}{\fbox{\textbf{master 00094}}}{}}%
%BeginExpansion
\msihyperref{\fbox{\textbf{master 00094}}}{}{\fbox{\textbf{master 00094}}}{}%
%EndExpansion
\vspace{6pt}
\end{ExerciseList}
\end{ExerciseList}

\begin{instructions}
\FRAME{itbpF}{235.6875pt}{5.5pt}{0pt}{}{}{dots.wmf}{\special{language
"Scientific Word";type "GRAPHIC";maintain-aspect-ratio TRUE;display
"USEDEF";valid_file "F";width 235.6875pt;height 5.5pt;depth
0pt;original-width 3.9167in;original-height 0.0735in;cropleft "0";croptop
"0.9772";cropright "0.8327";cropbottom "0.0226";filename
'graphics/dots.wmf';file-properties "XNPEU";}}
\end{instructions}

\begin{ExerciseList}
\item[21.] 
%TCIMACRO{\TeXButton{GCALCX}{\GCALCX}}%
%BeginExpansion
\GCALCX%
%EndExpansion
The table shows (lifetime) peptic ulcer rates (per 100 population) for
various family incomes as reported by the National Health Interview
Survey.\bigskip

$\hfill ${\small $%
\begin{tabular}{|c|c|}
\hline
& Ulcer rate \\ 
Income & (per 100 population) \\ \hline
\$\QTR{white}{\hspace*{4pt}}4,000 & 14.1 \\ 
\$\QTR{white}{\hspace*{4pt}}6,000 & 13.0 \\ 
\$\QTR{white}{\hspace*{4pt}}8,000 & 13.4 \\ 
\$12,000 & 12.5 \\ 
\$16,000 & 12.0 \\ 
\$20,000 & 12.4 \\ 
\$30,000 & 10.5 \\ 
\$45,000 & \QTR{white}{\hspace*{4pt}}9.4 \\ 
\$60,000 & \QTR{white}{\hspace*{4pt}}8.2 \\ \hline
\end{tabular}%
\ $}$\hfill $\bigskip

\begin{ExerciseList}
\item[(a)] Make a scatter plot of these data and decide whether a linear
model is appropriate.

%TCIMACRO{%
%\hyperref{\fbox{\textbf{master 00095a}}}{}{\fbox{\textbf{master 00095a}}}{}}%
%BeginExpansion
\msihyperref{\fbox{\textbf{master 00095a}}}{}{\fbox{\textbf{master 00095a}}}{}%
%EndExpansion

\item[(b)] Find and graph a linear model using the first and last data
points.

%TCIMACRO{%
%\hyperref{\fbox{\textbf{master 00095b}}}{}{\fbox{\textbf{master 00095b}}}{}}%
%BeginExpansion
\msihyperref{\fbox{\textbf{master 00095b}}}{}{\fbox{\textbf{master 00095b}}}{}%
%EndExpansion

\item[(c)] Find and graph the least squares regression line.

%TCIMACRO{%
%\hyperref{\fbox{\textbf{master 00095c}}}{}{\fbox{\textbf{master 00095c}}}{}}%
%BeginExpansion
\msihyperref{\fbox{\textbf{master 00095c}}}{}{\fbox{\textbf{master 00095c}}}{}%
%EndExpansion

\item[(d)] Use the linear model in part (c) to estimate the ulcer rate for
an income of \$25,000.

%TCIMACRO{%
%\hyperref{\fbox{\textbf{master 00095d}}}{}{\fbox{\textbf{master 00095d}}}{}}%
%BeginExpansion
\msihyperref{\fbox{\textbf{master 00095d}}}{}{\fbox{\textbf{master 00095d}}}{}%
%EndExpansion

\item[(e)] According to the model, how likely is someone with an income of
\$80,000 to suffer from peptic ulcers?

%TCIMACRO{%
%\hyperref{\fbox{\textbf{master 00095e}}}{}{\fbox{\textbf{master 00095e}}}{}}%
%BeginExpansion
\msihyperref{\fbox{\textbf{master 00095e}}}{}{\fbox{\textbf{master 00095e}}}{}%
%EndExpansion

\item[(f)] Do you think it would be reasonable to apply the model to someone
with an income of \$200,000?

%TCIMACRO{%
%\hyperref{\fbox{\textbf{master 00095f}}}{}{\fbox{\textbf{master 00095f}}}{}}%
%BeginExpansion
\msihyperref{\fbox{\textbf{master 00095f}}}{}{\fbox{\textbf{master 00095f}}}{}%
%EndExpansion
\end{ExerciseList}

%TCIMACRO{%
%\hyperref{ANSWER}{}{\textbf{ANSWER:}\ (a)~\FRAME{itbpF}{1.5696in}{0.9842in}{0.8786in}{}{}{4a010215a.wmf}{\special{language "Scientific Word";type "GRAPHIC";maintain-aspect-ratio TRUE;display "USEDEF";valid_file "F";width 1.5696in;height 0.9842in;depth 0.8786in;original-width 0pt;original-height 0pt;cropleft "0";croptop "1";cropright "1";cropbottom "0";filename 'graphics/4a010215a.wmf';file-properties "XNPEU";}}\newline
%Linear model is appropriate.\newline
%(b)~$y=-0.000105x+14.521$\newline
%\FRAME{itbpF}{1.5696in}{0.9842in}{0.8536in}{}{}{4a010215b.wmf}{\special{language "Scientific Word";type "GRAPHIC";maintain-aspect-ratio TRUE;display "USEDEF";valid_file "F";width 1.5696in;height 0.9842in;depth 0.8536in;original-width 0pt;original-height 0pt;cropleft "0";croptop "1";cropright "1";cropbottom "0";filename 'graphics/4a010215b.wmf';file-properties "XNPEU";}}\newline
%(c)~$y=-0.00009979x+13.951$ [See graph in (b).]\quad \newline
%(d)~About 11.5 per 100 population\quad (e)~About 6\%\quad (f)~No}{}}%
%BeginExpansion
\msihyperref{ANSWER}{}{\textbf{ANSWER:}\ (a)~\FRAME{itbpF}{1.5696in}{0.9842in}{0.8786in}{}{}{4a010215a.wmf}{\special{language "Scientific Word";type "GRAPHIC";maintain-aspect-ratio TRUE;display "USEDEF";valid_file "F";width 1.5696in;height 0.9842in;depth 0.8786in;original-width 0pt;original-height 0pt;cropleft "0";croptop "1";cropright "1";cropbottom "0";filename 'graphics/4a010215a.wmf';file-properties "XNPEU";}}\newline
Linear model is appropriate.\newline
(b)~$y=-0.000105x+14.521$\newline
\FRAME{itbpF}{1.5696in}{0.9842in}{0.8536in}{}{}{4a010215b.wmf}{\special{language "Scientific Word";type "GRAPHIC";maintain-aspect-ratio TRUE;display "USEDEF";valid_file "F";width 1.5696in;height 0.9842in;depth 0.8536in;original-width 0pt;original-height 0pt;cropleft "0";croptop "1";cropright "1";cropbottom "0";filename 'graphics/4a010215b.wmf';file-properties "XNPEU";}}\newline
(c)~$y=-0.00009979x+13.951$ [See graph in (b).]\quad \newline
(d)~About 11.5 per 100 population\quad (e)~About 6\%\quad (f)~No}{}%
%EndExpansion

\item[22.] 
%TCIMACRO{\TeXButton{GCALCX}{\GCALCX}}%
%BeginExpansion
\GCALCX%
%EndExpansion
Biologists have observed that the chirping rate of crickets of a certain
species appears to be related to temperature. The table shows the chirping
rates for various temperatures.\bigskip

$\hfill ${\small $%
\begin{tabular}{|c|c||c|c|}
\hline
\hspace*{-7pt}%
\begin{tabular}[t]{c}
Temperature \\ 
($^{\circ }$F$)$%
\end{tabular}%
\hspace*{-7pt} & \hspace*{-7pt}%
\begin{tabular}[t]{c}
Chirping rate \\ 
(chirps/min$)$%
\end{tabular}%
\hspace*{-7pt} & \hspace*{-7pt}%
\begin{tabular}[t]{c}
Temperature \\ 
($^{\circ }$F$)$%
\end{tabular}%
\hspace*{-7pt} & \hspace*{-7pt}%
\begin{tabular}[t]{c}
Chirping rate \\ 
(chirps/min$)$%
\end{tabular}%
\hspace*{-7pt} \\ \hline
50 & 20 & 75 & 140 \\ 
55 & 46 & 80 & 173 \\ 
60 & 79 & 85 & 198 \\ 
65 & 91 & 90 & 211 \\ 
70 & 113 &  &  \\ \hline
\end{tabular}%
\ \ $}$\hfill $\bigskip

\begin{ExerciseList}
\item[(a)] Make a scatter plot of the data.

%TCIMACRO{%
%\hyperref{\fbox{\textbf{master 00096a}}}{}{\fbox{\textbf{master 00096a}}}{}}%
%BeginExpansion
\msihyperref{\fbox{\textbf{master 00096a}}}{}{\fbox{\textbf{master 00096a}}}{}%
%EndExpansion

\item[(b)] Find and graph the regression line.

%TCIMACRO{%
%\hyperref{\fbox{\textbf{master 00096b}}}{}{\fbox{\textbf{master 00096b}}}{}}%
%BeginExpansion
\msihyperref{\fbox{\textbf{master 00096b}}}{}{\fbox{\textbf{master 00096b}}}{}%
%EndExpansion

\item[(c)] Use the linear model in part (b) to estimate the chirping rate at 
$100^{\circ }$F.

%TCIMACRO{%
%\hyperref{\fbox{\textbf{master 00096c}}}{}{\fbox{\textbf{master 00096c}}}{}}%
%BeginExpansion
\msihyperref{\fbox{\textbf{master 00096c}}}{}{\fbox{\textbf{master 00096c}}}{}%
%EndExpansion
\end{ExerciseList}

\item[23.] 
%TCIMACRO{\TeXButton{GCALCX}{\GCALCX}}%
%BeginExpansion
\GCALCX%
%EndExpansion
The table gives the winning heights for the Olympic pole vault competitions
in the 20th century.\bigskip

$\hfill ${\small $%
\begin{tabular}{|c|c||c|c|}
\hline
Year & Height (ft) & Year & Height (ft) \\ \hline
1900 & 10.83 & 1956 & 14.96 \\ 
1904 & 11.48 & 1960 & 15.42 \\ 
1908 & 12.17 & 1964 & 16.73 \\ 
1912 & 12.96 & 1968 & 17.71 \\ 
1920 & 13.42 & 1972 & 18.04 \\ 
1924 & 12.96 & 1976 & 18.04 \\ 
1928 & 13.77 & 1980 & 18.96 \\ 
1932 & 14.15 & 1984 & 18.85 \\ 
1936 & 14.27 & 1988 & 19.77 \\ 
1948 & 14.10 & 1992 & 19.02 \\ 
1952 & 14.92 & 1996 & 19.42 \\ \hline
\end{tabular}%
$}$\hfill $\bigskip

\begin{ExerciseList}
\item[(a)] Make a scatter plot and decide whether a linear model is
appropriate.

%TCIMACRO{%
%\hyperref{\fbox{\textbf{master 00097a}}}{}{\fbox{\textbf{master 00097a}}}{}}%
%BeginExpansion
\msihyperref{\fbox{\textbf{master 00097a}}}{}{\fbox{\textbf{master 00097a}}}{}%
%EndExpansion

\item[(b)] Find and graph the regression line.

%TCIMACRO{%
%\hyperref{\fbox{\textbf{master 00097b}}}{}{\fbox{\textbf{master 00097b}}}{}}%
%BeginExpansion
\msihyperref{\fbox{\textbf{master 00097b}}}{}{\fbox{\textbf{master 00097b}}}{}%
%EndExpansion

\item[(c)] Use the linear model to predict the height of the winning pole
vault at the 2000 Olympics and compare with the actual winning height of
19.36 feet.

%TCIMACRO{%
%\hyperref{\fbox{\textbf{master 00097c}}}{}{\fbox{\textbf{master 00097c}}}{}}%
%BeginExpansion
\msihyperref{\fbox{\textbf{master 00097c}}}{}{\fbox{\textbf{master 00097c}}}{}%
%EndExpansion

\item[(d)] Is it reasonable to use the model to predict the winning height
at the 2100 Olympics?

%TCIMACRO{%
%\hyperref{\fbox{\textbf{master 00097d}}}{}{\fbox{\textbf{master 00097d}}}{}}%
%BeginExpansion
\msihyperref{\fbox{\textbf{master 00097d}}}{}{\fbox{\textbf{master 00097d}}}{}%
%EndExpansion
\end{ExerciseList}

%TCIMACRO{%
%\hyperref{ANSWER}{}{\textbf{ANSWER:}\ (a)~\FRAME{itbpF}{1.8749in}{1.011in}{1.0032in}{}{}{4a010217a.wmf}{\special{language "Scientific Word";type "GRAPHIC";maintain-aspect-ratio TRUE;display "USEDEF";valid_file "F";width 1.8749in;height 1.011in;depth 1.0032in;original-width 0pt;original-height 0pt;cropleft "0";croptop "1";cropright "1";cropbottom "0";filename 'graphics/4a010217a.wmf';file-properties "XNPEU";}}\newline
%Linear model is appropriate.\newline
%(b)~$y=0.08912x-158.24$\newline
%\FRAME{itbpF}{1.8749in}{1.011in}{1.0032in}{}{}{4a010217b.wmf}{\special{language "Scientific Word";type "GRAPHIC";maintain-aspect-ratio TRUE;display "USEDEF";valid_file "F";width 1.8749in;height 1.011in;depth 1.0032in;original-width 0pt;original-height 0pt;cropleft "0";croptop "1";cropright "1";cropbottom "0";filename 'graphics/4a010217b.wmf';file-properties "XNPEU";}}\newline
%(c)~20~ft\quad (d)~No}{}}%
%BeginExpansion
\msihyperref{ANSWER}{}{\textbf{ANSWER:}\ (a)~\FRAME{itbpF}{1.8749in}{1.011in}{1.0032in}{}{}{4a010217a.wmf}{\special{language "Scientific Word";type "GRAPHIC";maintain-aspect-ratio TRUE;display "USEDEF";valid_file "F";width 1.8749in;height 1.011in;depth 1.0032in;original-width 0pt;original-height 0pt;cropleft "0";croptop "1";cropright "1";cropbottom "0";filename 'graphics/4a010217a.wmf';file-properties "XNPEU";}}\newline
Linear model is appropriate.\newline
(b)~$y=0.08912x-158.24$\newline
\FRAME{itbpF}{1.8749in}{1.011in}{1.0032in}{}{}{4a010217b.wmf}{\special{language "Scientific Word";type "GRAPHIC";maintain-aspect-ratio TRUE;display "USEDEF";valid_file "F";width 1.8749in;height 1.011in;depth 1.0032in;original-width 0pt;original-height 0pt;cropleft "0";croptop "1";cropright "1";cropbottom "0";filename 'graphics/4a010217b.wmf';file-properties "XNPEU";}}\newline
(c)~20~ft\quad (d)~No}{}%
%EndExpansion

\item[24.] 
%TCIMACRO{\TeXButton{GCALCX}{\GCALCX}}%
%BeginExpansion
\GCALCX%
%EndExpansion
A study by the US Office of Science and Technology in 1972 estimated the
cost (in 1972 dollars) to reduce automobile emissions by certain
percentages:\bigskip

$\hfill ${\small $%
\begin{tabular}{|c|c||c|c|}
\hline
\hspace*{-7pt}%
\begin{tabular}[t]{c}
Reduction in \\ 
emissions (\%)%
\end{tabular}%
\hspace*{-7pt} & \hspace*{-7pt}%
\begin{tabular}[t]{c}
Cost per \\ 
car (in \$)%
\end{tabular}%
\hspace*{-7pt} & \hspace*{-7pt}%
\begin{tabular}[t]{c}
Reduction in \\ 
emissions (\%)%
\end{tabular}%
\hspace*{-7pt} & \hspace*{-7pt}%
\begin{tabular}[t]{c}
Cost per \\ 
car (in \$)%
\end{tabular}%
\hspace*{-7pt} \\ \hline
50 & 45 & 75 & \hspace{3pt}90 \\ 
55 & 55 & 80 & 100 \\ 
60 & 62 & 85 & 200 \\ 
65 & 70 & 90 & 375 \\ 
70 & 80 & 95 & 600 \\ \hline
\end{tabular}%
\ $}$\hfill $\bigskip

Find a model that captures the \textquotedblleft diminishing
returns\textquotedblright\ trend of these data.

%TCIMACRO{%
%\hyperref{\fbox{\textbf{master 00098}}}{}{\fbox{\textbf{master 00098}}}{}}%
%BeginExpansion
\msihyperref{\fbox{\textbf{master 00098}}}{}{\fbox{\textbf{master 00098}}}{}%
%EndExpansion

\item[25.] 
%TCIMACRO{\TeXButton{GCALCX}{\GCALCX}}%
%BeginExpansion
\GCALCX%
%EndExpansion
Use the data in the table to model the population of the world in the 20th
century by a cubic function. Then use your model to estimate the population
in the year 1925.\bigskip

$\hfill ${\small $%
\begin{tabular}{|c|c||c|c|}
\hline
\begin{tabular}{l}
\\ 
Year%
\end{tabular}
& 
\begin{tabular}{l}
Population \\ 
(millions)%
\end{tabular}
& 
\begin{tabular}{l}
\\ 
Year%
\end{tabular}
& 
\begin{tabular}{l}
Population \\ 
(millions)%
\end{tabular}
\\ \hline
1900 & 1650 & 1960 & 3040 \\ 
1910 & 1750 & 1970 & 3710 \\ 
1920 & 1860 & 1980 & 4450 \\ 
1930 & 2070 & 1990 & 5280 \\ 
1940 & 2300 & 2000 & 6080 \\ 
1950 & 2560 &  &  \\ \hline
\end{tabular}%
\ \ \ $}$\hfill $\vspace{3pt}

%TCIMACRO{%
%\hyperref{ANSWER}{}{\textbf{ANSWER:}\newline
%\ $y=0.0012937x^{3}-7.06142x^{2}+12$,$823x-7$,$743$,$770$; \newline
%\textbf{Answer:}\ 1914 million}{}}%
%BeginExpansion
\msihyperref{ANSWER}{}{\textbf{ANSWER:}\newline
\ $y=0.0012937x^{3}-7.06142x^{2}+12$,$823x-7$,$743$,$770$; \newline
\textbf{Answer:}\ 1914 million}{}%
%EndExpansion

%TCIMACRO{%
%\hyperref{\fbox{\textbf{master 00099}}}{}{\fbox{\textbf{master 00099}}}{}}%
%BeginExpansion
\msihyperref{\fbox{\textbf{master 00099}}}{}{\fbox{\textbf{master 00099}}}{}%
%EndExpansion

\item[26.] 
%TCIMACRO{\TeXButton{GCALCX}{\GCALCX}}%
%BeginExpansion
\GCALCX%
%EndExpansion
The table shows the mean (average) distances $d$ of the planets from the sun
(taking the unit of measurement to be the distance from earth to the sun)
and their periods $T$ (time of revolution in years).\bigskip

$\hfill ${\small $%
\begin{tabular}{|c|c|c|}
\hline
Planet & $d$ & $T$ \\ \hline
\multicolumn{1}{|l|}{Mercury} & \multicolumn{1}{|r|}{0.387} & 
\multicolumn{1}{|r|}{0.241} \\ 
\multicolumn{1}{|l|}{Venus} & \multicolumn{1}{|r|}{0.723} & 
\multicolumn{1}{|r|}{0.615} \\ 
\multicolumn{1}{|l|}{Earth} & \multicolumn{1}{|r|}{1.000} & 
\multicolumn{1}{|r|}{1.000} \\ 
\multicolumn{1}{|l|}{Mars} & \multicolumn{1}{|r|}{1.523} & 
\multicolumn{1}{|r|}{1.881} \\ 
\multicolumn{1}{|l|}{Jupiter} & \multicolumn{1}{|r|}{5.203} & 
\multicolumn{1}{|r|}{11.861} \\ 
\multicolumn{1}{|l|}{Saturn} & \multicolumn{1}{|r|}{9.541} & 
\multicolumn{1}{|r|}{29.457} \\ 
\multicolumn{1}{|l|}{Uranus} & \multicolumn{1}{|r|}{19.190} & 
\multicolumn{1}{|r|}{84.008} \\ 
\multicolumn{1}{|l|}{Neptune} & \multicolumn{1}{|r|}{30.086} & 
\multicolumn{1}{|r|}{164.784} \\ \hline
\end{tabular}%
\ \ $}$\hfill $\bigskip

\begin{ExerciseList}
\item[(a)] Fit a power model to the data.

%TCIMACRO{%
%\hyperref{\fbox{\textbf{master 81407}}}{}{\fbox{\textbf{master 81407}}}{}}%
%BeginExpansion
\msihyperref{\fbox{\textbf{master 81407}}}{}{\fbox{\textbf{master 81407}}}{}%
%EndExpansion

\item[(b)] Kepler's Third Law of Planetary Motion states that \\[3pt]
\hspace*{8pt}%
\begin{tabular}[t]{p{175pt}}
%TCIMACRO{\TeXButton{rr}{\raggedright}}%
%BeginExpansion
\raggedright%
%EndExpansion
\textquotedblleft The square of the period of revolution of a planet is
proportional to the cube of its mean distance from the sun.\textquotedblright%
\end{tabular}%
\\[3pt]
Does your model corroborate Kepler's Third Law?

%TCIMACRO{%
%\hyperref{\fbox{\textbf{master 00100b}}}{}{\fbox{\textbf{master 00100b}}}{}}%
%BeginExpansion
\msihyperref{\fbox{\textbf{master 00100b}}}{}{\fbox{\textbf{master 00100b}}}{}%
%EndExpansion
\end{ExerciseList}
\end{ExerciseList}

%TCIMACRO{%
%\TeXButton{e2col}{\end{multicols}
%\advance \leftskip by 165pt
%\advance\hsize by -165pt
%\advance\linewidth by -165pt
%}}%
%BeginExpansion
\end{multicols}
\advance \leftskip by 165pt
\advance\hsize by -165pt
\advance\linewidth by -165pt
%
%EndExpansion

\end{document}
