
\documentclass{sebase}
%%%%%%%%%%%%%%%%%%%%%%%%%%%%%%%%%%%%%%%%%%%%%%%%%%%%%%%%%%%%%%%%%%%%%%%%%%%%%%%%%%%%%%%%%%%%%%%%%%%%%%%%%%%%%%%%%%%%%%%%%%%%%%%%%%%%%%%%%%%%%%%%%%%%%%%%%%%%%%%%%%%%%%%%%%%%%%%%%%%%%%%%%%%%%%%%%%%%%%%%%%%%%%%%%%%%%%%%%%%%%%%%%%%%%%%%%%%%%%%%%%%%%%%%%%%%
\usepackage{amssymb}
\usepackage{amsmath}
\usepackage{SECALCUL}
\usepackage{lie}

\setcounter{MaxMatrixCols}{10}
%TCIDATA{OutputFilter=LATEX.DLL}
%TCIDATA{Version=5.50.0.2953}
%TCIDATA{<META NAME="SaveForMode" CONTENT="1">}
%TCIDATA{BibliographyScheme=Manual}
%TCIDATA{Created=Mon Dec 15 16:20:00 1997}
%TCIDATA{LastRevised=Thursday, July 02, 2009 15:57:20}
%TCIDATA{<META NAME="ViewSettings" CONTENT="0">}
%TCIDATA{<META NAME="GraphicsSave" CONTENT="32">}
%TCIDATA{Language=American English}
%TCIDATA{CSTFile=SECALCUL.cst}

\input tcilatex
\newenvironment{instructions}{\STARTINSTR}{\ENDINSTR}
\begin{document}


\chapter[1\quad Functions and Models]{}

\section{1.1\quad Four Ways to Represent a Function}

%TCIMACRO{\TeXButton{noCCC}{\noCCC}}%
%BeginExpansion
\noCCC%
%EndExpansion
\setET%
%TCIMACRO{\TeXButton{noRM}{\renewcommand{\RM}{0}}}%
%BeginExpansion
\renewcommand{\RM}{0}%
%EndExpansion
%TCIMACRO{\TeXButton{setRM}{\renewcommand{\RM}{1}}}%
%BeginExpansion
\renewcommand{\RM}{1}%
%EndExpansion
%TCIMACRO{%
%\TeXButton{set page 2 / 2}{\ifnum\RM=1 \setcounter{page}{2}\else \setcounter{page}{2}\fi}}%
%BeginExpansion
\ifnum\RM=1 \setcounter{page}{2}\else \setcounter{page}{2}\fi%
%EndExpansion

Functions arise whenever one quantity depends on another. Consider the
following four situations.

\begin{enumerate}
\item[A.] The area $A$ of a circle depends on the radius $r$ of the circle.
The rule that connects $r$ and $A$ is given by the equation $A=\pi r^{2}$.
With each positive number $r$ there is associated one value of $A$, and we
say that $A$ is a \textit{function} of $r$.%
\marginpar{\vspace{12pt}
\par
\begin{tabular}{|l|c|}
\hline
Year & Population \\ 
& (millions) \\ \hline
1900 & 1650 \\ 
1910 & 1750 \\ 
1920 & 1860 \\ 
1930 & 2070 \\ 
1940 & 2300 \\ 
1950 & 2560 \\ 
1960 & 3040 \\ 
1970 & 3710 \\ 
1980 & 4450 \\ 
1990 & 5280 \\ 
2000 & 6080 \\ \hline
\end{tabular}%
}

\item[B.] The human population of the world $P$ depends on the time $t$. The
table gives estimates of the world population $P(t)$ at time $t$, for
certain years. For instance,\\[6pt]
\hspace*{\fill}$P(1950)\approx 2$,$560$,$000$,$000$\hspace*{\fill}\\[6pt]
But for each value of the time $t$ there is a corresponding value of $P$,
and we say that $P$ is a function of $t$.

\item[C.] The cost $C$ of mailing a first-class letter depends on the weight 
$w$ of the letter. Although there is no simple formula that connects $w$ and 
$C$, the post office has a rule for determining $C$ when $w$ is known.

\item[D.] The vertical acceleration $a$ of the ground as measured by a
seismograph during an earthquake is a function of the elapsed time $t$.
Figure 1 shows a graph generated by seismic activity during the Northridge
earthquake that shook Los Angeles in 1994. For a given value of $t$, the
graph provides a corresponding value of $a$.\FRAME{dtbpFU}{3.5544in}{2.002in%
}{0pt}{\Qcb{\QTR{FigureNumber}{FIGURE 1}\quad Vertical ground acceleration
during the Northridge earthquake}}{}{r010101.wmf}{\special{language
"Scientific Word";type "GRAPHIC";maintain-aspect-ratio TRUE;display
"USEDEF";valid_file "F";width 3.5544in;height 2.002in;depth
0pt;original-width 3.5544in;original-height 2.002in;cropleft "0";croptop
"1";cropright "1";cropbottom "0";filename
'graphics/r010101.wmf';file-properties "XNPEU";}}
\end{enumerate}

Each of these examples describes a rule whereby, given a number ($r$, $t$, $%
w $, or $t$), another number ($A$, $P$, $C$, or $a$) is assigned. In each
case we say that the second number is a function of the first number.

%TCIMACRO{\TeXButton{BOXSTART}{\STARTBOX}}%
%BeginExpansion
\STARTBOX%
%EndExpansion

A \textbf{function} $f$ is a rule that assigns to each element $x$ in a set $%
D$ exactly one element, called $f\!\left( x\right) $, in a set $E$.

%TCIMACRO{\TeXButton{BOXEND}{\ENDBOX}}%
%BeginExpansion
\ENDBOX%
%EndExpansion

We usually consider functions for which the sets $D$ and $E$ are sets of
real numbers. The set $D$ is called the \textbf{domain} of the function. The
number $f(x)$ is the \textbf{value of 
%TCIMACRO{\TeXButton{f}{\mbox{\boldmath $f$}} }%
%BeginExpansion
\mbox{\boldmath $f$}
%EndExpansion
\ at 
%TCIMACRO{\TeXButton{x}{\mbox{\boldmath $x$}} }%
%BeginExpansion
\mbox{\boldmath $x$}
%EndExpansion
\ }and is read ``$f$ of $x$.'' The \textbf{range} of $f$ is the set of all
possible values of $f(x)$ as $x$ varies throughout the domain. A symbol that
represents an arbitrary number in the \textit{domain} of a function $f$ is
called an \textbf{independent variable}. A symbol that represents a number
in the\textit{\ range} of $f$ is called a\textbf{\ dependent} \textbf{%
variable}. In Example A, for instance, $r$ is the independent variable and $%
A $ is the dependent variable.

It's helpful to think of a function as a \textbf{machine} (see Figure 2). If 
$x$ is in the domain of the function $f$, then when $x$ enters the machine,
it's accepted as an input and the machine produces an output $f(x)$
according to the rule of the function. Thus we can \pagebreak \linebreak
think of the domain as the set of all possible inputs and the range as the
set of all possible outputs.\\[6pt]
\hspace*{\fill}\FRAME{itbpFU}{1.8869in}{0.4134in}{0in}{\Qcb{%
\QTR{FigureNumber}{FIGURE 2}\quad Machine diagram for a function $f$}}{}{%
4e010102.wmf}{\special{language "Scientific Word";type
"GRAPHIC";maintain-aspect-ratio TRUE;display "USEDEF";valid_file "F";width
1.8869in;height 0.4134in;depth 0in;original-width 1.8887in;original-height
0.3969in;cropleft "0";croptop "1";cropright "1";cropbottom "0";filename
'graphics/4e010102.wmf';file-properties "XNPEU";}}\hspace*{\fill}\vspace*{6pt%
}

The preprogrammed functions in a calculator are good examples of a function
as a machine. For example, the square root key on your calculator computes
such a function. You press the key labeled $\sqrt{\quad }\ $(or $\sqrt{x}$)
and enter the input $x$. If $x<0$, then $x$ is not in the domain of this
function; that is, $x$ is not an acceptable input, and the calculator will
indicate an error. If $x\geq 0$, then an \textit{approximation} to $\sqrt{x}$
will appear in the display. Thus the $\sqrt{x}$ key on your calculator is
not quite the same as the exact mathematical function $f$ defined by $f(x)=%
\sqrt{x}$.

\marginpar{\hspace*{\fill}\FRAME{itbpFU}{124.75pt}{91.5pt}{88.625pt}{\Qcb{%
\QTR{FigureNumber}{FIGURE 3}\quad Arrow diagram for $f$}}{}{%
6et010103_00015.ai}{\special{language "Scientific Word";type
"GRAPHIC";maintain-aspect-ratio TRUE;display "USEDEF";valid_file "F";width
124.75pt;height 91.5pt;depth 88.625pt;original-width
1.6942in;original-height 1.1934in;cropleft "0";croptop "1.0028";cropright
"1.0027";cropbottom "-0.0350";filename
'graphics/6et010103_00015.ai';file-properties "XNPEU";}}\hspace*{\fill}}%
Another way to picture a function is by an \textbf{arrow diagram} as in
Figure 3. Each arrow connects an element of $D$ to an element of $E$. The
arrow indicates that $f(x)$ is associated with $x$, $f(a)$ is associated
with $a$, and so on.

The most common method for visualizing a function is its graph. If $f$ is a
function with domain $D$, then its \textbf{graph} is the set of ordered pairs%
\\[6pt]
\hspace*{\fill}$\left\{ (x,f(x))\mid x\in D\right\} $\hspace*{\fill}\\[6pt]
(Notice that these are input-output pairs.) In other words, the graph of $f$
consists of all points $(x,y)$ in the coordinate plane such that $y=f(x)$
and $x$ is in the domain of~$f$.

The graph of a function $f$ gives us a useful picture of the behavior or
\textquotedblleft life history\textquotedblright\ of a function. Since the $%
y $-coordinate of any point $(x,y)$ on the graph is $y=f(x)$, we can read
the value of $f(x)$ from the graph as being the height of the graph above
the point $x$ (see Figure 4). The graph of $f$ also allows us to picture the
domain of $f$ on the $x$-axis and its range on the $y$-axis as in Figure 5.\\%
[6pt]
$\hfill $\FRAME{itbpFU}{1.8196in}{1.3629in}{0in}{\Qcb{\QTR{FigureNumber}{%
FIGURE 4}}}{}{3c010104.wmf}{\special{language "Scientific Word";type
"GRAPHIC";maintain-aspect-ratio TRUE;display "USEDEF";valid_file "F";width
1.8196in;height 1.3629in;depth 0in;original-width 1.8196in;original-height
1.3629in;cropleft "0";croptop "1";cropright "1";cropbottom "0";filename
'graphics/3c010104.wmf';file-properties "XNPEU";}}\hspace*{36pt}\FRAME{itbpFU%
}{1.9995in}{1.4148in}{0in}{\Qcb{\QTR{FigureNumber}{FIGURE 5}}}{}{3c010105.wmf%
}{\special{language "Scientific Word";type "GRAPHIC";maintain-aspect-ratio
TRUE;display "USEDEF";valid_file "F";width 1.9995in;height 1.4148in;depth
0in;original-width 1.9726in;original-height 1.3872in;cropleft "0";croptop
"1";cropright "1";cropbottom "0";filename
'graphics/3c010105.wmf';file-properties "XNPEU";}}$\hfill \vspace{6pt}$

\begin{Example}[1]
%TECHARTS_DUMMY_ITEM_TAG-(1)
\marginpar{\vspace{-18pt}\FRAME{dtbpFU}{1.7071in}{1.375in}{0pt}{\Qcb{%
\QTR{FigureNumber}{FIGURE 6}}}{}{3c010106.wmf}{\special{language "Scientific
Word";type "GRAPHIC";maintain-aspect-ratio TRUE;display "USEDEF";valid_file
"F";width 1.7071in;height 1.375in;depth 0pt;original-width
6.4627in;original-height 5.8228in;cropleft "0";croptop "1";cropright
"1";cropbottom "0";filename 'graphics/3c010106.wmf';file-properties "XNPEU";}%
}}The graph of a function \textit{f} is shown in Figure 6.

\begin{enumerate}
\item[(a)] 
%TCIMACRO{%
%\hyperref{\fbox{{\tiny ITM 00001}}\quad }{}{\fbox{{\tiny ITM 00001}}\quad }{}}%
%BeginExpansion
\msihyperref{\fbox{{\tiny ITM 00001}}\quad }{}{\fbox{{\tiny ITM 00001}}\quad }{}%
%EndExpansion
Find the values of $f(1)$ and $f(5)$.

\item[(b)] 
%TCIMACRO{%
%\hyperref{\fbox{{\tiny ITM 00002}}\quad }{}{\fbox{{\tiny ITM 00002}}\quad }{}}%
%BeginExpansion
\msihyperref{\fbox{{\tiny ITM 00002}}\quad }{}{\fbox{{\tiny ITM 00002}}\quad }{}%
%EndExpansion
What are the domain and range of \textit{f\thinspace }?\vspace{-6pt}
\end{enumerate}
\end{Example}

\begin{Solution}
\thinspace

\begin{enumerate}
\item[(a)] We see from Figure 6 that the point $(1,3)$ lies on the graph of 
\textit{f}, so the value of \textit{f} at 1 is $f(1)=3$. (In other words,
the point on the graph that lies above $x=1$ is $3$ units above the \textit{x%
}-axis.)

\quad When $x=5$, the graph lies about 0.7 unit below the \textit{x}-axis,
so we estimate that $f(5)\approx -0.7$.

\item[(b)] 
\marginpar{\vspace{3pt}The notation for intervals is given in Appendix A.}We
see that $f\left( x\right) $ is defined when $0\leq x\leq 7$, so the domain
of $f$ is the closed interval $\left[ 0,7\right] $. Notice that $f$ takes on
all values from $-2$ to $4$, so the range of $f$ is%
%TCIMACRO{\TeXButton{longpage}{\enlargethispage{12pt}}}%
%BeginExpansion
\enlargethispage{12pt}%
%EndExpansion
\\[4pt]
$\hspace*{\stretch{2}}\left\{ y\mid -2\leq y\leq 4\right\} =\left[ -2,4%
\right] \hfill \blacksquare $\pagebreak
\end{enumerate}
\end{Solution}

\begin{Example}[2]
%TECHARTS_DUMMY_ITEM_TAG-(2)
Sketch the graph and find the domain and range of each function.

\begin{enumerate}
\item[(a)] 
%TCIMACRO{%
%\hyperref{\fbox{{\tiny ITM 00003}}\quad }{}{\fbox{{\tiny ITM 00003}}\quad }{}}%
%BeginExpansion
\msihyperref{\fbox{{\tiny ITM 00003}}\quad }{}{\fbox{{\tiny ITM 00003}}\quad }{}%
%EndExpansion
$f(x)=2x-1$

\item[(b)] 
%TCIMACRO{%
%\hyperref{\fbox{{\tiny ITM 00004}}\quad }{}{\fbox{{\tiny ITM 00004}}\quad }{}}%
%BeginExpansion
\msihyperref{\fbox{{\tiny ITM 00004}}\quad }{}{\fbox{{\tiny ITM 00004}}\quad }{}%
%EndExpansion
$g(x)=x^{2}$
\end{enumerate}
\end{Example}

\begin{Solution}
$\,\,$

\begin{enumerate}
\item[(a)] The equation of the graph is $y=2x-1$, and we recognize this as
being the equation of a line with slope 2 and $y$-intercept $-1$. (Recall
the slope-intercept form of the equation of a line: $y=mx+b$. See Appendix
B.) This enables us to sketch a portion of the graph of $f$ in Figure 7. The
expression $2x-1$ is defined for all real numbers, so the domain of $f$ is
the set of all real numbers, which we denote by $\mathbb{R}$. The graph
shows that the range is also $\mathbb{R}$.

\item[(b)] Since $g(2)=2^{2}=4$ and $g(-1)=(-1)^{2}=1$, we could plot the
points $(2,4)$ and $(-1,1)$, together with a few other points on the graph,
and join them to produce the graph (Figure 8). The equation of the graph is $%
y=x^{2}$, which represents a parabola (see Appendix C). The domain of $g$ is 
$\mathbb{R}$. The range of $g$ consists of all values of $g(x)$, that is,
all numbers of the form $x^{2}$. But $x^{2}\geq 0$ for all numbers $x$ and
any positive number $y$ is a square. So the range of $g$ is $\{y\mid y\geq
0\}=[0,\infty )$. This can also be seen from Figure 8.\medskip

$\hfill 
\begin{array}{l}
\text{\FRAME{itbpFU}{118.4375pt}{115.625pt}{0.75pt}{\Qcb{\QTR{FigureNumber}{%
FIGURE 7}}}{}{r010107.wmf}{\special{language "Scientific Word";type
"GRAPHIC";maintain-aspect-ratio TRUE;display "USEDEF";valid_file "F";width
118.4375pt;height 115.625pt;depth 0.75pt;original-width
6.4627in;original-height 5.3143in;cropleft "0";croptop "1";cropright
"1";cropbottom "-0.1860";filename 'graphics/r010107.wmf';file-properties
"XNPEU";}}}%
\end{array}%
$\hspace{36pt}$%
\begin{array}{l}
\text{\FRAME{itbpFU}{1.6371in}{1.6215in}{0in}{\Qcb{\QTR{FigureNumber}{FIGURE
8}}}{}{3c010108.wmf}{\special{language "Scientific Word";type
"GRAPHIC";maintain-aspect-ratio TRUE;display "USEDEF";valid_file "F";width
1.6371in;height 1.6215in;depth 0in;original-width 6.4627in;original-height
5.4794in;cropleft "0";croptop "1";cropright "1";cropbottom
"-0.1037";filename 'graphics/3c010108.wmf';file-properties "XNPEU";}}}%
\end{array}%
\hfill $%
\begin{tabular}{l}
\\ 
\\ 
\\ 
\\ 
\\ 
\\ 
\\ 
\\ 
\\ 
\\ 
\\ 
$\blacksquare $%
\end{tabular}
$\vspace{-15pt}$
\end{enumerate}
\end{Solution}

\begin{Example}[3]
%TECHARTS_DUMMY_ITEM_TAG-(3)
%TCIMACRO{%
%\hyperref{\fbox{{\tiny ITM 00005}}\quad }{}{\fbox{{\tiny ITM 00005}}\quad }{}}%
%BeginExpansion
\msihyperref{\fbox{{\tiny ITM 00005}}\quad }{}{\fbox{{\tiny ITM 00005}}\quad }{}%
%EndExpansion
If $f(x)=2x^{2}-5x+1$ and $h\neq 0$, evaluate $\dfrac{f(a+h)-f(a)}{h}$.
\end{Example}

\begin{Solution}
\marginpar{\vspace{24pt}The expression%
\begin{equation*}
\frac{f(a+h)-f(a)}{h}
\end{equation*}%
in Example 3 is called a \textbf{difference quotient} and occurs frequently
in calculus. As we will see in Chapter 2, it represents the average rate of
change of $f(x)$ between $x=a$ and $x=a+h$.}We first evaluate $f(a+h)$ by
replacing $x$ by $a+h$ in the expression for~$f(x)$:%
\begin{eqnarray*}
f(a+h) &=&2(a+h)^{2}-5(a+h)+1\vspace{2pt} \\
&=&2(a^{2}+2ah+h^{2})-5(a+h)+1\vspace{2pt} \\
&=&2a^{2}+4ah+2h^{2}-5a-5h+1
\end{eqnarray*}

Then we substitute into the given expression and simplify:\vspace{12pt}

$\hfill $%
\begin{tabular}{r}
$\dfrac{f(a+h)-f(a)}{h}=\dfrac{(2a^{2}+4ah+2h^{2}-5a-5h+1)-(2a^{2}-5a+1)}{h}%
\vspace{2pt}$ \\ 
\multicolumn{1}{l}{$\hspace{73.5pt}=\dfrac{%
2a^{2}+4ah+2h^{2}-5a-5h+1-2a^{2}+5a-1}{h}\vspace{2pt}$} \\ 
\multicolumn{1}{l}{$\hspace{73.5pt}=\dfrac{4ah+2h^{2}-5h}{h}=4a+2h-5$}%
\end{tabular}%
$\hfill $%
\begin{tabular}{l}
\\ 
\\ 
\\ 
\vspace{10pt} \\ 
$\blacksquare $%
\end{tabular}
\end{Solution}

\subsection{REPRESENTATIONS OF FUNCTIONS}

There are four possible ways to represent a function:\medskip

\begin{tabular}{ll}
$\bullet \quad $verbally & (by a description in words) \\ 
$\bullet \quad $numerically & (by a table of values) \\ 
$\bullet \quad $visually & (by a graph) \\ 
$\bullet \quad $algebraically & (by an explicit formula)%
\end{tabular}%
\medskip

If a single function can be represented in all four ways, it's often useful
to go from one representation to another to gain additional insight into the
function. (In Example 2, for instance, we started with algebraic formulas
and then obtained the graphs.) But certain functions are described more
naturally by one method than by another. With this in mind, let's reexamine
the four situations that we considered at the beginning of this section.

\begin{enumerate}
\item[A.] The most useful representation of the area of a circle as a
function of its radius is probably the algebraic formula $A(r)=\pi r^{2}$,
though it is possible to compile a table of values or to sketch a graph
(half a parabola). Because a circle has to have a positive radius, the
domain is $\{r\mid r>0\}=(0,\infty )$, and the range is also $(0,\infty )$.

\item[B.] 
\marginpar{\vspace{6pt}
\par
\hspace*{\fill}%
\begin{tabular}{|l|c|}
\hline
Year & Population \\ 
& (millions) \\ \hline
1900 & 1650 \\ 
1910 & 1750 \\ 
1920 & 1860 \\ 
1930 & 2070 \\ 
1940 & 2300 \\ 
1950 & 2560 \\ 
1960 & 3040 \\ 
1970 & 3710 \\ 
1980 & 4450 \\ 
1990 & 5280 \\ 
2000 & 6080 \\ \hline
\end{tabular}%
\hspace*{\fill}}We are given a description of the function in words: $P(t)$
is the human population of the world at time $t$. The table of values of
world population provides a convenient representation of this function. If
we plot these values, we get the graph (called a \textit{scatter plot}) in
Figure 9. It too is a useful representation; the graph allows us to absorb
all the data at once. What about a formula? Of course, it's impossible to
devise an explicit formula that gives the exact human population $P(t)$ at
any time $t$. But it is possible to find an expression for a function that 
\textit{approximates} $P(t)$. In fact, using methods explained in Section~%
%TCIMACRO{\TeXButton{1.5 / 1.2}{\ifnum\CCC=1 1.5\else 1.2\fi}}%
%BeginExpansion
\ifnum\CCC=1 1.5\else 1.2\fi%
%EndExpansion
, we obtain the approximation\\[6pt]
\hspace*{\fill}$P(t)\approx f(t)=(0.008079266)\cdot (1.013731)^{t}$\hspace*{%
\fill}\\[6pt]
and Figure 10 shows that it is a reasonably good \textquotedblleft
fit.\textquotedblright\ The function $f$ is called a \textit{mathematical
model} for population growth. In other words, it is a function with an
explicit formula that approximates the behavior of our given function. We
will see, however, that the ideas of calculus can be applied to a table of
values; an explicit formula is not necessary.

%TCIMACRO{\TeXButton{graphicS}{\vspace{12pt}\hskip-180pt\hfil}}%
%BeginExpansion
\vspace{12pt}\hskip-180pt\hfil%
%EndExpansion
\FRAME{itbpFU}{3.2643in}{1.8834in}{0in}{\Qcb{\QTR{FigureNumber}{FIGURE\ 9}}}{%
}{6et010109_00021.ai}{\special{language "Scientific Word";type
"GRAPHIC";maintain-aspect-ratio TRUE;display "USEDEF";valid_file "F";width
3.2643in;height 1.8834in;depth 0in;original-width 3.2368in;original-height
1.8568in;cropleft "0";croptop "1";cropright "1";cropbottom "0";filename
'graphics/6et010109_00021.ai';file-properties "XNPEU";}}$\hspace*{24pt}%
\FRAME{itbpFU}{3.2643in}{1.8834in}{0in}{\Qcb{\QTR{FigureNumber}{FIGURE\ 10}}%
}{}{6et010110_00022.ai}{\special{language "Scientific Word";type
"GRAPHIC";maintain-aspect-ratio TRUE;display "USEDEF";valid_file "F";width
3.2643in;height 1.8834in;depth 0in;original-width 3.2368in;original-height
1.8568in;cropleft "0";croptop "1";cropright "1";cropbottom "0";filename
'graphics/6et010110_00022.ai';file-properties "XNPEU";}}$%
%TCIMACRO{\TeXButton{graphicE}{\vspace{12pt}\hfil}}%
%BeginExpansion
\vspace{12pt}\hfil%
%EndExpansion

\marginpar{\vspace{25pt}
\par
A function defined by a table of values is called a \textit{tabular}
function.}\quad The function $P$ is typical of the functions that arise
whenever we attempt to apply calculus to the real world. We start with a
verbal description of a function. Then we may be able to construct a table
of values of the function, perhaps from instrument readings in a scientific
experiment. Even though we don't have complete knowledge of the values of
the function, we will see throughout the book that it is still possible to
perform the operations of calculus on such a function.%
%TCIMACRO{\TeXButton{longpage}{\enlargethispage{\baselineskip}}}%
%BeginExpansion
\enlargethispage{\baselineskip}%
%EndExpansion

\item[C.] Again the function is described in words: $C(w)$ is the cost of
mailing a first-class letter with weight $w$. The rule that the US Postal
Service used as of 2007 is as follows: The cost is 39 cents for up to one
ounce, plus 24 cents for each successive ounce up to 13 ounces. The table of
values shown in the margin is the most convenient representation for this
function, though it is possible to sketch a graph (see Example 10).\\[6pt]
\hspace*{\fill}$%
\begin{tabular}{|c|c|}
\hline
$\quad w$ (ounces)$\quad $ & $\quad C(w)$ (dollars)$\quad $ \\ \hline
$0<w\leq 1$ & $0.39$ \\ 
$1<w\leq 2$ & $0.63$ \\ 
$2<w\leq 3$ & $0.87$ \\ 
$3<w\leq 4$ & $1.11$ \\ 
$4<w\leq 5$ & $1.35$ \\ 
$\vdots $ & $\vdots $ \\ 
$12<w\leq 13$ & $3.27$ \\ \hline
\end{tabular}%
$\hspace*{\fill}\bigskip

\item[D.] The graph shown in Figure 1 is the most natural representation of
the vertical acceleration function $a(t)$. It's true that a table of values
could be compiled, and it is even possible to devise an approximate formula.
But everything a geologist needs to know---amplitudes and patterns---can be
seen easily from the graph. (The same is true for the patterns seen in
electrocardiograms of heart patients and polygraphs for lie-detection.)
\end{enumerate}

In the next example we sketch the graph of a function that is defined
verbally.\vspace{-6pt}

\begin{Example}[4]
%TECHARTS_DUMMY_ITEM_TAG-(4)
%TCIMACRO{%
%\hyperref{\fbox{{\tiny ITM 00006}}\quad }{}{\fbox{{\tiny ITM 00006}}\quad }{}}%
%BeginExpansion
\msihyperref{\fbox{{\tiny ITM 00006}}\quad }{}{\fbox{{\tiny ITM 00006}}\quad }{}%
%EndExpansion
When you turn on a hot-water faucet, the temperature $T$ of the water
depends on how long the water has been running. Draw a rough graph of $T$ as
a function of the time $t$ that has elapsed since the faucet was turned on.
\end{Example}

\begin{Solution}
\marginpar{\FRAME{dtbpFU}{1.9865in}{1.0464in}{0pt}{\Qcb{\QTR{FigureNumber}{%
FIGURE\ 11}}}{}{r010113.wmf}{\special{language "Scientific Word";type
"GRAPHIC";maintain-aspect-ratio TRUE;display "USEDEF";valid_file "F";width
1.9865in;height 1.0464in;depth 0pt;original-width 1.9865in;original-height
1.0464in;cropleft "0";croptop "1";cropright "1";cropbottom "0";filename
'graphics/r010113.wmf';file-properties "XNPEU";}}}The initial temperature of
the running water is close to room temperature because the water has been
sitting in the pipes. When the water from the hot-water tank starts flowing
from the faucet, $T$ increases quickly. In the next phase, $T$ is constant
at the temperature of the heated water in the tank. When the tank is
drained, $T$ decreases to the temperature of the water supply. This enables
us to make the rough sketch of $T$ as a function of $t$ in Figure 11.\vspace{%
-12pt}$\blacksquare $
\end{Solution}

In the following example we start with a verbal description of a function in
a physical situation and obtain an explicit algebraic formula. The ability
to do this is a useful skill in solving calculus problems that ask for the
maximum or minimum values of quantities.\vspace{-6pt}

\begin{Example}[5]
%TECHARTS_DUMMY_ITEM_TAG-(5)
%TCIMACRO{\TeXButton{VIDEO}{\VIDEO}}%
%BeginExpansion
\VIDEO%
%EndExpansion
%TCIMACRO{%
%\hyperref{\fbox{{\tiny ITM 00007}}\quad }{}{\fbox{{\tiny ITM 00007}}\quad }{}}%
%BeginExpansion
\msihyperref{\fbox{{\tiny ITM 00007}}\quad }{}{\fbox{{\tiny ITM 00007}}\quad }{}%
%EndExpansion
%TCIMACRO{%
%\hyperref{\frame{{\footnotesize video 5et 010105 / 001}}\quad }{}{\frame{{\footnotesize video 5et 010105 / 001}}\quad }{}}%
%BeginExpansion
\msihyperref{\frame{{\footnotesize video 5et 010105 / 001}}\quad }{}{\frame{{\footnotesize video 5et 010105 / 001}}\quad }{}%
%EndExpansion
A rectangular storage container with an open top has a volume of 10~m$^{3}$.
The length of its base is twice its width. Material for the base costs \$10
per square meter; material for the sides costs \$6 per square meter. Express
the cost of materials as a function of the width of the base.
\end{Example}

\begin{Solution}
\marginpar{\vspace{-6pt}\FRAME{dtbpFU}{1.4719in}{1.1519in}{0pt}{\Qcb{%
\QTR{FigureNumber}{FIGURE 12}}}{}{c010116.wmf}{\special{language "Scientific
Word";type "GRAPHIC";maintain-aspect-ratio TRUE;display "USEDEF";valid_file
"F";width 1.4719in;height 1.1519in;depth 0pt;original-width
1.4719in;original-height 1.1519in;cropleft "0";croptop "1";cropright
"1";cropbottom "0";filename 'graphics/c010116.wmf';file-properties "XNPEU";}}%
}We draw a diagram as in Figure 12 and introduce notation by letting $w$ and 
$2w$ be the width and length of the base, respectively, and $h$ be the
height.

The area of the base is $(2w)w=2w^{2}$, so the cost, in dollars, of the
material for the base is $10(2w^{2})$. Two of the sides have area $wh$ and
the other two have area $2wh$, so the cost of the material for the sides is $%
6[2(wh)+2(2wh)]$. The total cost is therefore\\[6pt]
\hspace*{\fill}$C=10(2w^{2})+6[2(wh)+2(2wh)]=20w^{2}+36wh$\hspace*{\fill}\\[%
6pt]
To express $C$ as a function of $w$ alone, we need to eliminate $h$ and we
do so by using the fact that the volume is 10 m$^{3}$. Thus\\[4pt]
\hspace*{\fill}$w(2w)h=10$\hspace*{\fill}\\[6pt]
which gives \\[3pt]
\hspace*{\fill}$h=\dfrac{10}{2w^{2}}=\dfrac{5}{w^{2}}$\hspace*{\fill}\\[6pt]
\marginpar{
In setting up applied functions as in Example 5, it may be useful to review
the principles of problem solving as discussed on page~78,%
%TCIMACRO{%
%\hyperref{{\tiny [edit for pages]}}{}{{\tiny [edit for pages]}}{} }%
%BeginExpansion
\msihyperref{{\tiny [edit for pages]}}{}{{\tiny [edit for pages]}}{}
%EndExpansion
particularly \textit{Step~1: Understand the Problem}.}Substituting this into
the expression for $C$, we have\\[4pt]
\hspace*{\fill}$C=20w^{2}+36w\left( \dfrac{5}{w^{2}}\right) =20w^{2}+\dfrac{%
180}{w}$\hspace*{\fill}\\[6pt]
Therefore, the equation \\[4pt]
\hspace*{\fill}$C(w)=20w^{2}+\dfrac{180}{w}\qquad w>0$\hspace*{\fill}\\[6pt]
expresses $C$ as a function of $w$.$\vspace{-12pt}\blacksquare $
\end{Solution}

\begin{Example}[6]
%TECHARTS_DUMMY_ITEM_TAG-(6)
Find the domain of each function.\vspace{-6pt}
\end{Example} %TECHARTS_DISABLED_TAG

\begin{enumerate}
\item[(a)]
%TCIMACRO{%
%\hyperref{\fbox{{\tiny ITM 00008}}\quad }{}{\fbox{{\tiny ITM 00008}}\quad }{}}%
%BeginExpansion
\msihyperref{\fbox{{\tiny ITM 00008}}\quad }{}{\fbox{{\tiny ITM 00008}}\quad }{}%
%EndExpansion
$f(x)=\sqrt{x+2}\hspace{36pt}$(b)\hspace{9pt}%
%TECHARTS_DUMMY_ITEM_TAG-(b)
%TCIMACRO{%
%\hyperref{\fbox{{\tiny ITM 00009}}\quad }{}{\fbox{{\tiny ITM 00009}}\quad }{}}%
%BeginExpansion
\msihyperref{\fbox{{\tiny ITM 00009}}\quad }{}{\fbox{{\tiny ITM 00009}}\quad }{}%
%EndExpansion
$g(x)=\dfrac{1}{x^{2}-x}$
\end{enumerate}
%TECHARTS_DUMMY_END_TAG
\begin{Solution}
$\,\,\vspace{-24pt}$
\end{Solution}

\begin{enumerate}
\item[(a)] 
\marginpar{\vspace{9pt}
\par
If a function is given by a formula and the domain is not stated explicitly,
the convention is that the domain is the set of all numbers for which the
formula makes sense and defines a real number.}Because the square root of a
negative number is not defined (as a real number), the domain of $f$
consists of all values of $x$ such that $x+2\geq 0$. This is equivalent to $%
x\geq -2$, so the domain is the interval $[-2,\infty )$.

\item[(b)] Since \\[3pt]
\hspace*{\fill}$g(x)=\dfrac{1}{x^{2}-x}=\dfrac{1}{x(x-1)}$\hspace*{\fill}\\[%
6pt]
and division by $0$ is not allowed, we see that $g(x)$ is not defined when $%
x=0$ or $x=1$. Thus the domain of $g$ is $\{x\mid x\neq 0$, $x\neq 1\}$,
which could also be written in interval notation as $(-\infty ,0)\cup
(0,1)\cup (1,\infty )$.$\blacksquare $
\end{enumerate}

The graph of a function is a curve in the $xy$-plane. But the question
arises: Which curves in the $xy$-plane are graphs of functions? This is
answered by the following test.

%TCIMACRO{\TeXButton{BOXSTART}{\STARTBOX}}%
%BeginExpansion
\STARTBOX%
%EndExpansion

\QTR{BOXHEAD}{THE VERTICAL LINE TEST}\quad A curve in the $xy$-plane is the
graph of a function of $x$ if and only if no vertical line intersects the
curve more than once.

%TCIMACRO{\TeXButton{BOXEND}{\ENDBOX}}%
%BeginExpansion
\ENDBOX%
%EndExpansion

The reason for the truth of the Vertical Line Test can be seen in Figure 13.
If each vertical line $x=a$ intersects a curve only once, at $(a,b)$, then
exactly one functional value is defined by $f(a)=b$. But if a line $x=a$
intersects the curve twice, at $(a,b)$ and $(a,c)$, then the curve can't
represent a function because a function can't assign two different values to 
$a$.\\[6pt]
\hspace*{\fill}\FRAME{itbpFU}{4.2497in}{1.1964in}{0in}{\Qcb{%
\QTR{FigureNumber}{FIGURE 13}}}{}{4e010117.wmf}{\special{language
"Scientific Word";type "GRAPHIC";maintain-aspect-ratio TRUE;display
"USEDEF";valid_file "F";width 4.2497in;height 1.1964in;depth
0in;original-width 4.2497in;original-height 1.1969in;cropleft "0";croptop
"1";cropright "1";cropbottom "0";filename
'graphics/4e010117.wmf';file-properties "XNPEU";}}\hspace*{\fill}\vspace*{8pt%
}

For example, the parabola $x=y^{2}-2$ shown in Figure 14(a) is not the graph
of a function of $x$ because, as you can see, there are vertical lines that
intersect the parabola twice. The parabola, however, does contain the graphs
of \textit{two} functions of $x$. Notice that the equation $x=y^{2}-2$
implies $y^{2}=x+2$, so $y=\pm \sqrt{x+2}$. Thus the upper and lower halves
of the parabola are the graphs of the functions $f(x)=$ $\sqrt{x+2}$ [from
Example 6(a)] and $g(x)=$ $-\sqrt{x+2}$. [See Figures 14(b) and (c).] We
observe that if we reverse the roles of $x$ and $y$, then the equation $%
x=h(y)=y^{2}-2$ \textit{does} define $x$ as a function of $y$ (with $y$ as
the independent variable and $x$ as the dependent variable) and the parabola
now appears as the graph of the function $h$.

%TCIMACRO{\TeXButton{graphicS}{\vspace{12pt}\hskip-40pt\hfil}}%
%BeginExpansion
\vspace{12pt}\hskip-40pt\hfil%
%EndExpansion
\FRAME{itbpFU}{4.7634in}{1.5575in}{0in}{\Qcb{\QTR{FigureNumber}{FIGURE 14}}}{%
}{3c010118.wmf}{\special{language "Scientific Word";type
"GRAPHIC";maintain-aspect-ratio TRUE;display "USEDEF";valid_file "F";width
4.7634in;height 1.5575in;depth 0in;original-width 4.7634in;original-height
1.5575in;cropleft "0";croptop "1";cropright "1";cropbottom "0";filename
'graphics/3c010118.wmf';file-properties "XNPEU";}}%
%TCIMACRO{\TeXButton{graphicE}{\vspace{12pt}\hfil}}%
%BeginExpansion
\vspace{12pt}\hfil%
%EndExpansion

\subsection{PIECEWISE DEFINED FUNCTIONS}

The functions in the following four examples are defined by different
formulas in different parts of their domains.

\begin{Example}[7]
%TECHARTS_DUMMY_ITEM_TAG-(7)
%TCIMACRO{\TeXButton{VIDEO}{\VIDEO}}%
%BeginExpansion
\VIDEO%
%EndExpansion
%TCIMACRO{%
%\hyperref{\fbox{{\tiny ITM 00010}}\quad }{}{\fbox{{\tiny ITM 00010}}\quad }{}}%
%BeginExpansion
\msihyperref{\fbox{{\tiny ITM 00010}}\quad }{}{\fbox{{\tiny ITM 00010}}\quad }{}%
%EndExpansion
%TCIMACRO{%
%\hyperref{\frame{{\footnotesize video 5et 010107 / 002}}\quad }{}{\frame{{\footnotesize video 5et 010107 / 002}}\quad }{}}%
%BeginExpansion
\msihyperref{\frame{{\footnotesize video 5et 010107 / 002}}\quad }{}{\frame{{\footnotesize video 5et 010107 / 002}}\quad }{}%
%EndExpansion
A function $f$ is defined by \\[6pt]
\hspace*{\fill}$f(x)=\left\{ 
\begin{array}{ll}
1-x\quad  & \text{if}%
%TCIMACRO{\TeXButton{en}{\enskip}}%
%BeginExpansion
\enskip%
%EndExpansion
x\leq 1 \\ 
x^{2} & \text{if}%
%TCIMACRO{\TeXButton{en}{\enskip}}%
%BeginExpansion
\enskip%
%EndExpansion
x>1%
\end{array}%
\right. $\hspace*{\fill}\\[6pt]
Evaluate $f(0)$, $f(1)$, and $f(2)$ and sketch the graph.
\end{Example}

\begin{Solution}
Remember that a function is a rule. For this particular function the rule is
the following: First look at the value of the input $x$. If it happens that $%
x$ $\leq 1$, then the value of $f(x)$ is $1-x$. On the other hand, if $x>1$,
then the value of $f(x)$ is $x^{2}$.\\[6pt]
\hspace*{\fill}$%
\begin{array}{l}
\vspace*{6pt}\text{Since }0\leq 1\text{, we have }f(0)=1-0=1\text{.} \\ 
\vspace*{6pt}\text{Since }1\leq 1\text{, we have }f(1)=1-1=0\text{.} \\ 
\vspace{6pt}\text{Since }2>1\text{, we have }f(2)=2^{2}=4\text{.}%
\end{array}%
$\hspace*{\fill}

\marginpar{\vspace{-27pt}\FRAME{dtbpFU}{1.3889in}{1.2644in}{0pt}{\Qcb{%
\QTR{FigureNumber}{FIGURE 15}}}{}{4e010119.wmf}{\special{language
"Scientific Word";type "GRAPHIC";maintain-aspect-ratio TRUE;display
"USEDEF";valid_file "F";width 1.3889in;height 1.2644in;depth
0pt;original-width 101.375pt;original-height 97.375pt;cropleft "0";croptop
"1";cropright "1";cropbottom "0";filename
'graphics/4e010119.wmf';file-properties "XNPEU";}}}How do we draw the graph
of $f$? We observe that if $x\leq 1$, then $f(x)=1-x$, so the part of the
graph of $f$ that lies to the left of the vertical line $x=1$ must coincide
with the line $y=1-x$, which has slope $-1$ and $y$-intercept 1. If $x>1$,
then $f(x)=x^{2}$, so the part of the graph of $f$ that lies to the right of
the line $x=1$ must coincide with the graph of $y=x^{2}$, which is a
parabola. This enables us to sketch the graph in Figure 15. The solid dot
indicates that the point $\left( 1,0\right) $ is included on the graph; the
open dot indicates that the point $\left( 1,1\right) $ is excluded from the
graph.$\vspace{-6pt}\blacksquare $
\end{Solution}

\marginpar{\vspace{45pt}
\par
For a more extensive review of absolute values, see Appendix A.}The next
example of a piecewise defined function is the absolute value function.
Recall that the \textbf{absolute value} of a number $a$, denoted by $%
\left\vert a\right\vert $, is the distance from $a$ to $0$ on the real
number line. Distances are always positive or $0$, so we have\\[6pt]
\hspace*{\fill}$\left\vert a\right\vert \geq 0\qquad $for every number $a$%
\hspace*{\fill}\\[6pt]
For example, \\[6pt]
\hspace*{\fill}$\left\vert 3\right\vert =3\qquad \left\vert -3\right\vert
=3\qquad \left\vert 0\right\vert =0\qquad \left\vert \sqrt{2}-1\right\vert =%
\sqrt{2}-1\qquad \quad \left\vert 3-\pi \right\vert =\pi -3$\hspace*{\fill}\\%
[6pt]
In general, we have \\[6pt]
\hspace*{\fill}$\fbox{$\qquad 
\begin{array}{lc}
\left\vert a\right\vert =a & \text{\quad if }a\geq 0\vspace{6pt} \\ 
\left\vert a\right\vert =-a & \text{\quad if }a<0\vspace{2pt}%
\end{array}%
\qquad $}$\hspace*{\fill}\\[6pt]
\marginpar{\FRAME{dtbpFU}{1.4027in}{1.0992in}{0pt}{\Qcb{\QTR{FigureNumber}{%
FIGURE 16}}}{}{r010120.wmf}{\special{language "Scientific Word";type
"GRAPHIC";maintain-aspect-ratio TRUE;display "USEDEF";valid_file "F";width
1.4027in;height 1.0992in;depth 0pt;original-width 147.5625pt;original-height
102.375pt;cropleft "0";croptop "1";cropright "1";cropbottom "0";filename
'graphics/r010120.wmf';file-properties "XNPEU";}}}(Remember that if $a$ is
negative, then $-a$ is positive.)\vspace*{6pt}

\begin{Example}[8]
%TECHARTS_DUMMY_ITEM_TAG-(8)
%TCIMACRO{%
%\hyperref{\fbox{{\tiny ITM 00011}}\quad }{}{\fbox{{\tiny ITM 00011}}\quad }{}}%
%BeginExpansion
\msihyperref{\fbox{{\tiny ITM 00011}}\quad }{}{\fbox{{\tiny ITM 00011}}\quad }{}%
%EndExpansion
Sketch the graph of the absolute value function $f(x)=\left| x \right| $.
\end{Example}

\begin{Solution}
From the preceding discussion we know that \\[6pt]
\hspace*{\fill}$\left\vert x\right\vert =\left\{ 
\begin{array}{ll}
x & \text{if }x\geq 0 \\ 
-x\quad & \text{if }x<0%
\end{array}%
\right. $\hspace*{\fill}\\[6pt]
Using the same method as in Example 7, we see that the graph of $f$
coincides with the line $y=x$ to the right of the $y$-axis and coincides
with the line $y=-x$ to the left of the $y$-axis (see Figure 16).$%
\blacksquare $\vspace{-12pt}
\end{Solution}

\begin{Example}[9]
%TECHARTS_DUMMY_ITEM_TAG-(9)
%TCIMACRO{%
%\hyperref{\fbox{{\tiny ITM 00012}}\quad }{}{\fbox{{\tiny ITM 00012}}\quad }{}}%
%BeginExpansion
\msihyperref{\fbox{{\tiny ITM 00012}}\quad }{}{\fbox{{\tiny ITM 00012}}\quad }{}%
%EndExpansion
\marginpar{\hspace*{\fill}\FRAME{itbpFU}{1.529in}{1.0118in}{0.5016in}{\Qcb{%
\QTR{FigureNumber}{FIGURE 17}}}{}{3c010121.wmf}{\special{language
"Scientific Word";type "GRAPHIC";maintain-aspect-ratio TRUE;display
"USEDEF";valid_file "F";width 1.529in;height 1.0118in;depth
0.5016in;original-width 1.5281in;original-height 1.0127in;cropleft
"0";croptop "1";cropright "1";cropbottom "0";filename
'graphics/3c010121.wmf';file-properties "XNPEU";}}\hspace*{\fill}}Find a
formula for the function $f$ graphed in Figure 17.
\end{Example}

\begin{Solution}
The line through $(0,0)$ and $(1,1)$ has slope $m=1$ and $y$-intercept $b=0$%
, so its equation is $y=x$. Thus, for the part of the graph of $f$ that
joins $(0,0)$ to $(1,1)$, we have\\[6pt]
\hspace*{\fill}$f(x)=x$\qquad if$%
%TCIMACRO{\TeXButton{en}{\enskip}}%
%BeginExpansion
\enskip%
%EndExpansion
0\leq x\leq 1$\hspace*{\fill}\\[6pt]
\marginpar{\vspace{6pt}Point-slope form of the equation of a line:\\[2pt]
\hspace*{\fill}$y-y_{1}=m\left( x-x_{1}\right) $\hspace*{\fill}\\[2pt]
See Appendix B.}The line through $(1,1)$ and $(2,0)$ has slope $m=-1$, so
its point-slope form is\\[6pt]
\hspace*{\fill}$y-0=(-1)(x-2)\qquad $or$\qquad y=2-x$\hspace*{\fill}\\[6pt]
So we have\hspace*{\fill}$f(x)=2-x\qquad $if$%
%TCIMACRO{\TeXButton{en}{\enskip}}%
%BeginExpansion
\enskip%
%EndExpansion
1<x\leq 2$\hspace*{\fill}\\[6pt]
We also see that the graph of $f$ coincides with the $x$-axis for $x>2$.
Putting this information together, we have the following three-piece formula
for $f$:\\[8pt]
$\hspace*{\fill}f(x)=\left\{ 
\begin{array}{ll}
x & \text{if\ \ }0\leq x\leq 1\text{ } \\ 
2-x\quad & \text{if\ \ }1<x\leq 2 \\ 
0 & \text{if\ \ }x>2%
\end{array}%
\right. \hfill \vspace{-12pt}$%
\begin{tabular}{l}
\\ 
\\ 
$\blacksquare $%
\end{tabular}
\end{Solution}

\begin{Example}[10]
%TECHARTS_DUMMY_ITEM_TAG-(10)
%TCIMACRO{%
%\hyperref{\fbox{{\tiny ITM 00014}}\quad }{}{\fbox{{\tiny ITM 00014}}\quad }{}}%
%BeginExpansion
\msihyperref{\fbox{{\tiny ITM 00014}}\quad }{}{\fbox{{\tiny ITM 00014}}\quad }{}%
%EndExpansion
In Example C at the beginning of this section we considered the cost $C(w)$
of mailing a first-class letter with weight $w$. In effect, this is a
piecewise defined function because, from the table of values, we have\\[8pt]
\hspace*{\fill}$C(w)=\left\{ 
\begin{tabular}{ll}
$0.39\quad $ & if$\text{\ \ }0<w\leq 1$ \\ 
$0.63$ & if$\text{\ \ }1<w\leq 2$ \\ 
$0.87$ & if$\text{\ \ }2<w\leq 3$ \\ 
$1.11$ & if$\text{\ \ }3<w\leq 4$ \\ 
$\vdots $ & 
\end{tabular}%
\ \ \right. $\hspace*{\fill}\\[6pt]
\marginpar{\vspace{-3cm}\FRAME{dtbpFU}{1.8334in}{1.3604in}{0pt}{\Qcb{%
\QTR{FigureNumber}{FIGURE 18}}}{}{6et010118.wmf}{\special{language
"Scientific Word";type "GRAPHIC";maintain-aspect-ratio TRUE;display
"USEDEF";valid_file "F";width 1.8334in;height 1.3604in;depth
0pt;original-width 1.8057in;original-height 1.3327in;cropleft "0";croptop
"1";cropright "1";cropbottom "0";filename
'graphics/6et010118.wmf';file-properties "XNPEU";}}}The graph is shown in
Figure 18. You can see why functions similar to this one are called \textbf{%
step functions}---they jump from one value to the next. Such functions will
be studied in Chapter 2.\bigskip $\blacksquare $
\end{Example}

\subsection{SYMMETRY}

If a function $f$ satisfies $f(-x)=f(x)$ for every number $x$ in its domain,
then $f$ is called an \textbf{even function}. For instance, the function $%
f(x)=x^{2}$ is even because 
\marginpar{\vspace{-24pt}\FRAME{dtbpFU}{1.6942in}{1.3223in}{0pt}{\Qcb{%
\QTR{FigureNumber}{FIGURE 19}\quad An even function}}{}{4e010123.wmf}{%
\special{language "Scientific Word";type "GRAPHIC";maintain-aspect-ratio
TRUE;display "USEDEF";valid_file "F";width 1.6942in;height 1.3223in;depth
0pt;original-width 124.4375pt;original-height 100.375pt;cropleft "0";croptop
"1";cropright "1";cropbottom "0";filename
'graphics/4e010123.wmf';file-properties "XNPEU";}}\FRAME{dtbpFU}{1.7919in}{%
1.3188in}{0pt}{\Qcb{\QTR{FigureNumber}{FIGURE 20\quad }An odd function}}{}{%
3c010124.wmf}{\special{language "Scientific Word";type
"GRAPHIC";maintain-aspect-ratio TRUE;display "USEDEF";valid_file "F";width
1.7919in;height 1.3188in;depth 0pt;original-width 1.7919in;original-height
1.3188in;cropleft "0";croptop "1";cropright "1";cropbottom "0";filename
'graphics/3c010124.wmf';file-properties "XNPEU";}}}\\[6pt]
\hspace*{\fill}$f(-x)=(-x)^{2}=x^{2}=f(x)$\hspace*{\fill}\\[6pt]
The geometric significance of an even function is that its graph is
symmetric with respect to the $y$-axis (see Figure 19). This means that if
we have plotted the graph of $f$ for $x\geq 0$, we obtain the entire graph
simply by reflecting this portion about the $y$-axis.

If $f$ satisfies $f(-x)=-f(x)$ for every number $x$ in its domain, then $f$
is called an \textbf{odd function}. For example, the function $f(x)=x^{3}$
is odd because\\[6pt]
\hspace*{\fill}$f(-x)=(-x)^{3}=-x^{3}=-f(x)$\hspace*{\fill}\\[6pt]
The graph of an odd function is symmetric about the origin (see Figure 20).
If we already have the graph of $f$ for $x\geq 0$, we can obtain the entire
graph by rotating this portion through $180^{\circ }$ about the origin.

\begin{Example}[11]
%TECHARTS_DUMMY_ITEM_TAG-(11)
%TCIMACRO{\TeXButton{VIDEO}{\VIDEO}}%
%BeginExpansion
\VIDEO%
%EndExpansion
%TCIMACRO{%
%\hyperref{\frame{{\footnotesize video 5et 010111 / 003}}\quad }{}{\frame{{\footnotesize video 5et 010111 / 003}}\quad }{}}%
%BeginExpansion
\msihyperref{\frame{{\footnotesize video 5et 010111 / 003}}\quad }{}{\frame{{\footnotesize video 5et 010111 / 003}}\quad }{}%
%EndExpansion
Determine whether each of the following functions is even, odd, or neither
even nor odd.

\begin{enumerate}
\item[(a)]
%TCIMACRO{%
%\hyperref{\fbox{{\tiny ITM 00015}}\quad }{}{\fbox{{\tiny ITM 00015}}\quad }{}}%
%BeginExpansion
\msihyperref{\fbox{{\tiny ITM 00015}}\quad }{}{\fbox{{\tiny ITM 00015}}\quad }{}%
%EndExpansion
$f(x)=x^{5}+x\qquad $\textbf{(b)}~%
%TCIMACRO{%
%\hyperref{\fbox{{\tiny ITM 00016}}\quad }{}{\fbox{{\tiny ITM 00016}}\quad }{}}%
%BeginExpansion
\msihyperref{\fbox{{\tiny ITM 00016}}\quad }{}{\fbox{{\tiny ITM 00016}}\quad }{}%
%EndExpansion
$g(x)=1-x^{4}\qquad $\textbf{(c)}~%
%TCIMACRO{%
%\hyperref{\fbox{{\tiny ITM 00017}}\quad }{}{\fbox{{\tiny ITM 00017}}\quad }{}}%
%BeginExpansion
\msihyperref{\fbox{{\tiny ITM 00017}}\quad }{}{\fbox{{\tiny ITM 00017}}\quad }{}%
%EndExpansion
$h(x)=2x-x^{2}$
\end{enumerate}
\end{Example}

\begin{Solution}
\vspace{-9pt}

\begin{enumerate}
\item[(a)] $\hfill f\left( -x\right) =\left( -x\right) ^{5}+\left( -x\right)
=\left( -1\right) ^{5}x^{5}+\left( -x\right) \hspace{8pt}\hfill $\vspace*{6pt%
}

$\hfill =-x^{5}-x=-\left( x^{5}+x\right) \vspace*{6pt}\hspace{23pt}\hfill $

$\hfill =-f\left( x\right) $\vspace*{8pt}$\hspace{90pt}\hfill $\newline
Therefore, $f$ is an odd function.\vspace{3pt}

\item[(b)] $\hfill g\left( -x\right) =1-\left( -x\right)
^{4}=1-x^{4}=g\left( x\right) \hfill $\\[6pt]
So $g$ is even.\vspace{3pt}

\item[(c)] $\hfill h\left( -x\right) =2\left( -x\right) -\left( -x\right)
^{2}=-2x-x^{2}\hfill $\\[6pt]
Since $h\left( -x\right) \neq h\left( x\right) $ and $h\left( -x\right) \neq
-h\left( x\right) $, we conclude that $h$ is neither even nor odd.$%
\blacksquare $
\end{enumerate}
\end{Solution}

The graphs of the functions in Example 11 are shown in Figure 21. Notice
that the graph of $h$ is symmetric neither about the $y$-axis nor about the
origin.\\[6pt]
\hspace*{\fill}\FRAME{itbpFU}{4.7635in}{1.1141in}{0in}{\Qcb{%
\QTR{FigureNumber}{FIGURE 21}}}{}{3c010125.wmf}{\special{language
"Scientific Word";type "GRAPHIC";maintain-aspect-ratio TRUE;display
"USEDEF";valid_file "F";width 4.7635in;height 1.1141in;depth
0in;original-width 296.1875pt;original-height 106.375pt;cropleft "0";croptop
"1";cropright "1";cropbottom "0.1863";filename
'graphics/3c010125.wmf';file-properties "XNPEU";}}\hspace*{\fill}\vspace*{6pt%
}

\subsection{INCREASING AND DECREASING FUNCTIONS}

The graph shown in Figure 22 rises from $A$ to $B$, falls from $B$ to $C$,
and rises again from $C$ to $D$. The function $f$ is said to be increasing
on the interval $[a,b]$, decreasing on $[b,c]$, and increasing again on $%
[c,d]$. Notice that if $x_{1}$and $x_{2}$ are any two numbers between $a$
and $b$ with $x_{1}<x_{2}$, then $f(x_{1})<f(x_{2})$. We use this as the
defining property of an increasing function.\\[6pt]
\hspace*{\fill}\FRAME{itbpFU}{4.736in}{1.627in}{0in}{\Qcb{\QTR{FigureNumber}{%
FIGURE 22}}}{}{4e010126.wmf}{\special{language "Scientific Word";type
"GRAPHIC";maintain-aspect-ratio TRUE;display "USEDEF";valid_file "F";width
4.736in;height 1.627in;depth 0in;original-width 298.5pt;original-height
119pt;cropleft "0";croptop "1";cropright "1";cropbottom "0";filename
'graphics/4e010126.wmf';file-properties "XNPEU";}}\hspace*{\fill}\vspace{3pt}

\marginpar{\vspace{49pt}\FRAME{dtbpFU}{1.6388in}{1.3906in}{0pt}{\Qcb{%
\QTR{FigureNumber}{FIGURE 23}}}{}{3c010127.wmf}{\special{language
"Scientific Word";type "GRAPHIC";maintain-aspect-ratio TRUE;display
"USEDEF";valid_file "F";width 1.6388in;height 1.3906in;depth
0pt;original-width 1.6388in;original-height 1.3906in;cropleft "0";croptop
"1";cropright "1";cropbottom "0";filename
'graphics/3c010127.wmf';file-properties "XNPEU";}}}%
%TCIMACRO{\TeXButton{BOXSTART}{\STARTBOX}}%
%BeginExpansion
\STARTBOX%
%EndExpansion
\vspace{3pt}A function $f$ is called \textbf{increasing} on an interval $I$
if\\[6pt]
\hspace*{\fill}$f(x_{1})<f(x_{2})\qquad $whenever $x_{1}<x_{2}$ in $I$%
\hspace*{\fill}\\[6pt]
It is called \textbf{decreasing} on $I$ if \\[6pt]
\hspace*{\fill}$f(x_{1})>f(x_{2})\qquad $whenever $x_{1}<x_{2}$ in $I$%
\hspace*{\fill}\vspace*{6pt}%
%TCIMACRO{\TeXButton{BOXEND}{\ENDBOX}}%
%BeginExpansion
\ENDBOX%
%EndExpansion

In the definition of an increasing function it is important to realize that
the inequality $f(x_{1})<f(x_{2})$ must be satisfied for\textit{\ every}
pair of numbers $x_{1}$ and $x_{2}$ in $I$ with $x_{1}\nolinebreak
<\nolinebreak x_{2}$.

You can see from Figure 23 that the function $f(x)=x^{2}$ is decreasing on
the interval $(-\infty ,0]$ and increasing on the interval $[0,\infty )$.

\QTP{MultColDiv}
Exercises 1.1

%TCIMACRO{%
%\TeXButton{s2col}{\setlength{\columnsep}{24pt}
%\advance \leftskip by -165pt
%\advance\hsize by 165pt
%\advance\linewidth by 165pt
%\begin{multicols}{2}
%}}%
%BeginExpansion
\setlength{\columnsep}{24pt}
\advance \leftskip by -165pt
\advance\hsize by 165pt
\advance\linewidth by 165pt
\begin{multicols}{2}
%
%EndExpansion

\begin{ExerciseList}
\item[$\hfill $1.] The graph of a function $f$ is given.\vspace{-3pt}

\begin{ExerciseList}
\item[(a)] State the value of $f(-1)$.

%TCIMACRO{%
%\hyperref{\fbox{\textbf{master 00001a}}}{}{\fbox{\textbf{master 00001a}}}{}}%
%BeginExpansion
\msihyperref{\fbox{\textbf{master 00001a}}}{}{\fbox{\textbf{master 00001a}}}{}%
%EndExpansion

\item[(b)] Estimate the value of $f(2)$.

%TCIMACRO{%
%\hyperref{\fbox{\textbf{master 00001b}}}{}{\fbox{\textbf{master 00001b}}}{}}%
%BeginExpansion
\msihyperref{\fbox{\textbf{master 00001b}}}{}{\fbox{\textbf{master 00001b}}}{}%
%EndExpansion

\item[(c)] For what values of $x$ is $f(x)=2$?

%TCIMACRO{%
%\hyperref{\fbox{\textbf{master 00001c}}}{}{\fbox{\textbf{master 00001c}}}{}}%
%BeginExpansion
\msihyperref{\fbox{\textbf{master 00001c}}}{}{\fbox{\textbf{master 00001c}}}{}%
%EndExpansion

\item[(d)] Estimate the values of $x$ such that $f(x)=0$.

%TCIMACRO{%
%\hyperref{\fbox{\textbf{master 00001d}}}{}{\fbox{\textbf{master 00001d}}}{}}%
%BeginExpansion
\msihyperref{\fbox{\textbf{master 00001d}}}{}{\fbox{\textbf{master 00001d}}}{}%
%EndExpansion

\item[(e)] State the domain and range of $f$.

%TCIMACRO{%
%\hyperref{\fbox{\textbf{master 00001e}}}{}{\fbox{\textbf{master 00001e}}}{}}%
%BeginExpansion
\msihyperref{\fbox{\textbf{master 00001e}}}{}{\fbox{\textbf{master 00001e}}}{}%
%EndExpansion

\item[(f)] On what interval is $f$ increasing?

%TCIMACRO{%
%\hyperref{\fbox{\textbf{master 00001f}}}{}{\fbox{\textbf{master 00001f}}}{}}%
%BeginExpansion
\msihyperref{\fbox{\textbf{master 00001f}}}{}{\fbox{\textbf{master 00001f}}}{}%
%EndExpansion
\end{ExerciseList}

%TCIMACRO{%
%\hyperref{ANSWER}{}{\textbf{ANSWER:} (a)~$-2\quad $(b)~2.8$\quad $(c)~$-3$,~1$\quad $(d)~$-2.5$,~0.3$\quad $\linebreak (e)~$\left[ -3,3\right] $,~$\left[ -2,3\right] \quad $(f)~$\left[ -1,3\right] $}{}}%
%BeginExpansion
\msihyperref{ANSWER}{}{\textbf{ANSWER:} (a)~$-2\quad $(b)~2.8$\quad $(c)~$-3$,~1$\quad $(d)~$-2.5$,~0.3$\quad $\linebreak (e)~$\left[ -3,3\right] $,~$\left[ -2,3\right] \quad $(f)~$\left[ -1,3\right] $}{}%
%EndExpansion
\thinspace

\hspace*{\fill}\FRAME{itbpF}{1.3474in}{1.1813in}{1.254in}{}{}{c0101x01.wmf}{%
\special{language "Scientific Word";type "GRAPHIC";maintain-aspect-ratio
TRUE;display "USEDEF";valid_file "F";width 1.3474in;height 1.1813in;depth
1.254in;original-width 1.3474in;original-height 1.1813in;cropleft
"0";croptop "1";cropright "1";cropbottom "0";filename
'graphics/c0101x01.wmf';file-properties "XNPEU";}}\hspace*{\fill}

\item[{$\hfill $\protect\fbox{2.\hspace{-2pt}}}] The graphs of $f$ and $g$
are given.

\begin{ExerciseList}
\item[(a)] State the values of $f(-4)$ and $g(3)$.

%TCIMACRO{%
%\hyperref{\fbox{\textbf{master 00002a}}}{}{\fbox{\textbf{master 00002a}}}{}}%
%BeginExpansion
\msihyperref{\fbox{\textbf{master 00002a}}}{}{\fbox{\textbf{master 00002a}}}{}%
%EndExpansion

\item[(b)] For what values of $x$ is $f(x)=g(x)$?

%TCIMACRO{%
%\hyperref{\fbox{\textbf{master 00002b}}}{}{\fbox{\textbf{master 00002b}}}{}}%
%BeginExpansion
\msihyperref{\fbox{\textbf{master 00002b}}}{}{\fbox{\textbf{master 00002b}}}{}%
%EndExpansion

\item[(c)] Estimate the solution of the equation $f(x)=-1$.

%TCIMACRO{%
%\hyperref{\fbox{\textbf{master 00002c}}}{}{\fbox{\textbf{master 00002c}}}{}}%
%BeginExpansion
\msihyperref{\fbox{\textbf{master 00002c}}}{}{\fbox{\textbf{master 00002c}}}{}%
%EndExpansion

\item[(d)] On what interval is $f$ decreasing?

%TCIMACRO{%
%\hyperref{\fbox{\textbf{master 00002d}}}{}{\fbox{\textbf{master 00002d}}}{}}%
%BeginExpansion
\msihyperref{\fbox{\textbf{master 00002d}}}{}{\fbox{\textbf{master 00002d}}}{}%
%EndExpansion

\item[(e)] State the domain and range of $f$.

%TCIMACRO{%
%\hyperref{\fbox{\textbf{master 00002e}}}{}{\fbox{\textbf{master 00002e}}}{}}%
%BeginExpansion
\msihyperref{\fbox{\textbf{master 00002e}}}{}{\fbox{\textbf{master 00002e}}}{}%
%EndExpansion

\item[(f)] State the domain and range of $g$.

%TCIMACRO{%
%\hyperref{\fbox{\textbf{master 00002f}}}{}{\fbox{\textbf{master 00002f}}}{}}%
%BeginExpansion
\msihyperref{\fbox{\textbf{master 00002f}}}{}{\fbox{\textbf{master 00002f}}}{}%
%EndExpansion
\end{ExerciseList}

\hspace*{\fill}\FRAME{itbpF}{1.6939in}{1.3491in}{0in}{}{}{c0101x02.wmf}{%
\special{language "Scientific Word";type "GRAPHIC";maintain-aspect-ratio
TRUE;display "USEDEF";valid_file "F";width 1.6939in;height 1.3491in;depth
0in;original-width 1.6942in;original-height 1.3491in;cropleft "0";croptop
"1";cropright "1";cropbottom "0";filename
'graphics/c0101x02.wmf';file-properties "XNPEU";}}\hspace*{\fill}

\item[$\hfill $3.] Figure 1 was recorded by an instrument operated by the
California Department of Mines and Geology at the University Hospital of the
University of Southern California in Los Angeles. Use it to estimate the
range of the vertical ground acceleration function at USC during the
Northridge earthquake.

%TCIMACRO{\hyperref{ANSWER}{}{\textbf{ANSWER:} $\left[ -85,115\right] $}{}}%
%BeginExpansion
\msihyperref{ANSWER}{}{\textbf{ANSWER:} $\left[ -85,115\right] $}{}%
%EndExpansion

%TCIMACRO{%
%\hyperref{\fbox{\textbf{master 80000}}}{}{\fbox{\textbf{master 80000}}}{}}%
%BeginExpansion
\msihyperref{\fbox{\textbf{master 80000}}}{}{\fbox{\textbf{master 80000}}}{}%
%EndExpansion
\quad 
%TCIMACRO{%
%\hyperref{\textbf{[Revised 3C3 1.1.3]}}{}{\textbf{[Revised 3C3 1.1.3]}}{}}%
%BeginExpansion
\msihyperref{\textbf{[Revised 3C3 1.1.3]}}{}{\textbf{[Revised 3C3 1.1.3]}}{}%
%EndExpansion

\item[$\hfill $4.] In this section we discussed examples of ordinary,
everyday functions: Population is a function of time, postage cost is a
function of weight, water temperature is a function of time. Give three
other examples of functions from everyday life that are described verbally.
What can you say about the domain and range of each of your functions? If
possible, sketch a rough graph of each function.

%TCIMACRO{%
%\hyperref{\fbox{\textbf{master 00004}}}{}{\fbox{\textbf{master 00004}}}{}}%
%BeginExpansion
\msihyperref{\fbox{\textbf{master 00004}}}{}{\fbox{\textbf{master 00004}}}{}%
%EndExpansion
\end{ExerciseList}

\begin{instructions}
\FRAME{itbpF}{235.6875pt}{5.5pt}{0pt}{}{}{dots.wmf}{\special{language
"Scientific Word";type "GRAPHIC";maintain-aspect-ratio TRUE;display
"USEDEF";valid_file "F";width 235.6875pt;height 5.5pt;depth
0pt;original-width 3.9167in;original-height 0.0735in;cropleft "0";croptop
"0.9772";cropright "0.8327";cropbottom "0.0226";filename
'graphics/dots.wmf';file-properties "XNPEU";}}
\end{instructions}

\begin{instructions}
\QTR{SpanExer}{5--8}{\small 
%TCIMACRO{\TeXButton{SQR}{\hskip .5em\rule{4pt}{4pt}\hskip .5em}}%
%BeginExpansion
\hskip .5em\rule{4pt}{4pt}\hskip .5em%
%EndExpansion
Determine whether the curve is the graph of a function of }$x${\small . If
it is, state the domain and range of the function.}
\end{instructions}

\begin{ExerciseList}
\item[$\hfill $5.] \ \FRAME{itbpF}{1.0274in}{1.0136in}{0.9037in}{}{}{%
5et0101x05.wmf}{\special{language "Scientific Word";type
"GRAPHIC";maintain-aspect-ratio TRUE;display "USEDEF";valid_file "F";width
1.0274in;height 1.0136in;depth 0.9037in;original-width
1.0274in;original-height 1.0136in;cropleft "0";croptop "1";cropright
"1";cropbottom "0";filename 'graphics/5et0101x05.wmf';file-properties
"XNPEU";}}

%TCIMACRO{\hyperref{ANSWER\quad }{}{\textbf{ANSWER:}\ No}{}}%
%BeginExpansion
\msihyperref{ANSWER\quad }{}{\textbf{ANSWER:}\ No}{}%
%EndExpansion

%TCIMACRO{%
%\hyperref{\fbox{\textbf{master 00005}}}{}{\fbox{\textbf{master 00005}}}{}}%
%BeginExpansion
\msihyperref{\fbox{\textbf{master 00005}}}{}{\fbox{\textbf{master 00005}}}{}%
%EndExpansion

\item[$\hfill $6.] \ \FRAME{itbpF}{1.0274in}{1.0136in}{0.9037in}{}{}{%
5et0101x06.wmf}{\special{language "Scientific Word";type
"GRAPHIC";maintain-aspect-ratio TRUE;display "USEDEF";valid_file "F";width
1.0274in;height 1.0136in;depth 0.9037in;original-width
1.0274in;original-height 1.0136in;cropleft "0";croptop "1";cropright
"1";cropbottom "0";filename 'graphics/5et0101x06.wmf';file-properties
"XNPEU";}}

%TCIMACRO{%
%\hyperref{\fbox{\textbf{master 00006}}}{}{\fbox{\textbf{master 00006}}}{}}%
%BeginExpansion
\msihyperref{\fbox{\textbf{master 00006}}}{}{\fbox{\textbf{master 00006}}}{}%
%EndExpansion

\item[$\hfill $7.] \ \FRAME{itbpF}{1.0421in}{1.0421in}{0.9037in}{}{}{%
5et0101x07.wmf}{\special{language "Scientific Word";type
"GRAPHIC";maintain-aspect-ratio TRUE;display "USEDEF";valid_file "F";width
1.0421in;height 1.0421in;depth 0.9037in;original-width
1.0421in;original-height 1.0421in;cropleft "0";croptop "1";cropright
"1";cropbottom "0";filename 'graphics/5et0101x07.wmf';file-properties
"XNPEU";}}

%TCIMACRO{%
%\hyperref{ANSWER}{}{\textbf{ANSWER:}\ Yes, $\left[ -3,2\right] $, $\left[ -3,-2\right) \cup \left( -1,3\right] $}{}}%
%BeginExpansion
\msihyperref{ANSWER}{}{\textbf{ANSWER:}\ Yes, $\left[ -3,2\right] $, $\left[ -3,-2\right) \cup \left( -1,3\right] $}{}%
%EndExpansion

%TCIMACRO{%
%\hyperref{\fbox{\textbf{master 00007}}}{}{\fbox{\textbf{master 00007}}}{}}%
%BeginExpansion
\msihyperref{\fbox{\textbf{master 00007}}}{}{\fbox{\textbf{master 00007}}}{}%
%EndExpansion

\item[$\hfill $8.] \ \FRAME{itbpF}{1.0136in}{1.0274in}{0.9037in}{}{}{%
3c0101x08.wmf}{\special{language "Scientific Word";type
"GRAPHIC";maintain-aspect-ratio TRUE;display "USEDEF";valid_file "F";width
1.0136in;height 1.0274in;depth 0.9037in;original-width
1.0136in;original-height 1.0274in;cropleft "0";croptop "1";cropright
"1";cropbottom "0";filename 'graphics/3c0101x08.wmf';file-properties
"XNPEU";}}

%TCIMACRO{%
%\hyperref{\fbox{\textbf{master 00008}}}{}{\fbox{\textbf{master 00008}}}{}}%
%BeginExpansion
\msihyperref{\fbox{\textbf{master 00008}}}{}{\fbox{\textbf{master 00008}}}{}%
%EndExpansion
\end{ExerciseList}

\begin{instructions}
\FRAME{itbpF}{235.6875pt}{5.5pt}{0pt}{}{}{dots.wmf}{\special{language
"Scientific Word";type "GRAPHIC";maintain-aspect-ratio TRUE;display
"USEDEF";valid_file "F";width 235.6875pt;height 5.5pt;depth
0pt;original-width 3.9167in;original-height 0.0735in;cropleft "0";croptop
"0.9772";cropright "0.8327";cropbottom "0.0226";filename
'graphics/dots.wmf';file-properties "XNPEU";}}\vspace{15pt}
\end{instructions}

\begin{ExerciseList}
\item[{$\hfill $\protect\fbox{9.\hspace{-2pt}}}] The graph shown gives the
weight of a certain person as a function of age. Describe in words how this
person's weight varies over time. What do you think happened when this
person was 30 years old?\\[6pt]
\hspace*{\fill}\FRAME{itbpF}{2.8055in}{1.4538in}{0in}{}{}{6et0101x09_00044.ai%
}{\special{language "Scientific Word";type "GRAPHIC";maintain-aspect-ratio
TRUE;display "USEDEF";valid_file "F";width 2.8055in;height 1.4538in;depth
0in;original-width 2.7778in;original-height 1.4269in;cropleft "0";croptop
"1";cropright "1";cropbottom "0";filename
'graphics/6et0101x09_00044.ai';file-properties "XNPEU";}}\hspace*{\fill}

%TCIMACRO{%
%\hyperref{ANSWER}{}{\textbf{ANSWER:} Diet, exercise, or illness}{}}%
%BeginExpansion
\msihyperref{ANSWER}{}{\textbf{ANSWER:} Diet, exercise, or illness}{}%
%EndExpansion

%TCIMACRO{%
%\hyperref{\fbox{\textbf{master 00009}}}{}{\fbox{\textbf{master 00009}}}{}}%
%BeginExpansion
\msihyperref{\fbox{\textbf{master 00009}}}{}{\fbox{\textbf{master 00009}}}{}%
%EndExpansion

\item[$\hfill $10.] The graph shown gives a salesman's distance from his
home as a function of time on a certain day. Describe in words what the
graph indicates about his travels on this day.

\FRAME{itbpF}{3.2969in}{1.35in}{0in}{}{}{6et0101x10_00045.ai}{\special%
{language "Scientific Word";type "GRAPHIC";maintain-aspect-ratio
TRUE;display "USEDEF";valid_file "F";width 3.2969in;height 1.35in;depth
0in;original-width 3.224in;original-height 1.3045in;cropleft "0";croptop
"1";cropright "1";cropbottom "0";filename
'graphics/6et0101x10_00045.ai';file-properties "XNPEU";}}

%TCIMACRO{%
%\hyperref{\fbox{\textbf{master 00010}}}{}{\fbox{\textbf{master 00010}}}{}}%
%BeginExpansion
\msihyperref{\fbox{\textbf{master 00010}}}{}{\fbox{\textbf{master 00010}}}{}%
%EndExpansion

\item[{$\hfill $\protect\fbox{\hspace{-2pt}11.\hspace{-2pt}}}] You put some
ice cubes in a glass, fill the glass with cold water, and then let the glass
sit on a table. Describe how the temperature of the water changes as time
passes. Then sketch a rough graph of the temperature of the water as a
function of the elapsed time.

%TCIMACRO{%
%\hyperref{ANSWER}{}{\textbf{ANSWER:}\ \FRAME{itbpF}{1.3335in}{0.8328in}{0.7524in}{}{}{4a010111.wmf}{\special{language "Scientific Word";type "GRAPHIC";maintain-aspect-ratio TRUE;display "USEDEF";valid_file "F";width 1.3335in;height 0.8328in;depth 0.7524in;original-width 0pt;original-height 0pt;cropleft "0";croptop "1";cropright "1";cropbottom "0";filename 'graphics/4a010111.wmf';file-properties "XNPEU";}}}{}}%
%BeginExpansion
\msihyperref{ANSWER}{}{\textbf{ANSWER:}\ \FRAME{itbpF}{1.3335in}{0.8328in}{0.7524in}{}{}{4a010111.wmf}{\special{language "Scientific Word";type "GRAPHIC";maintain-aspect-ratio TRUE;display "USEDEF";valid_file "F";width 1.3335in;height 0.8328in;depth 0.7524in;original-width 0pt;original-height 0pt;cropleft "0";croptop "1";cropright "1";cropbottom "0";filename 'graphics/4a010111.wmf';file-properties "XNPEU";}}}{}%
%EndExpansion

%TCIMACRO{%
%\hyperref{\fbox{\textbf{master 00011}}}{}{\fbox{\textbf{master 00011}}}{}}%
%BeginExpansion
\msihyperref{\fbox{\textbf{master 00011}}}{}{\fbox{\textbf{master 00011}}}{}%
%EndExpansion

\item[$\hfill $12.] Sketch a rough graph of the number of hours of daylight
as a function of the time of year.

%TCIMACRO{%
%\hyperref{\fbox{\textbf{master 00012}}}{}{\fbox{\textbf{master 00012}}}{}}%
%BeginExpansion
\msihyperref{\fbox{\textbf{master 00012}}}{}{\fbox{\textbf{master 00012}}}{}%
%EndExpansion

\item[{$\hfill $\protect\fbox{\hspace{-2pt}13.\hspace{-2pt}}}] Sketch a rough
graph of the outdoor temperature as a function of time during a typical
spring day.

%TCIMACRO{%
%\hyperref{ANSWER}{}{\textbf{ANSWER:}\ \FRAME{itbpF}{1.4996in}{0.8034in}{0.7524in}{}{}{4a010113.wmf}{\special{language "Scientific Word";type "GRAPHIC";maintain-aspect-ratio TRUE;display "USEDEF";valid_file "F";width 1.4996in;height 0.8034in;depth 0.7524in;original-width 0pt;original-height 0pt;cropleft "0";croptop "1";cropright "1";cropbottom "0";filename 'graphics/4a010113.wmf';file-properties "XNPEU";}}}{}}%
%BeginExpansion
\msihyperref{ANSWER}{}{\textbf{ANSWER:}\ \FRAME{itbpF}{1.4996in}{0.8034in}{0.7524in}{}{}{4a010113.wmf}{\special{language "Scientific Word";type "GRAPHIC";maintain-aspect-ratio TRUE;display "USEDEF";valid_file "F";width 1.4996in;height 0.8034in;depth 0.7524in;original-width 0pt;original-height 0pt;cropleft "0";croptop "1";cropright "1";cropbottom "0";filename 'graphics/4a010113.wmf';file-properties "XNPEU";}}}{}%
%EndExpansion

%TCIMACRO{%
%\hyperref{\fbox{\textbf{master 00013}}}{}{\fbox{\textbf{master 00013}}}{}}%
%BeginExpansion
\msihyperref{\fbox{\textbf{master 00013}}}{}{\fbox{\textbf{master 00013}}}{}%
%EndExpansion

\item[$\hfill $14.] Sketch a rough graph of the market value of a new car as
a function of time for a period of 20 years. Assume the car is well
maintained.

%TCIMACRO{%
%\hyperref{\fbox{\textbf{master 30001}}}{}{\fbox{\textbf{master 30001}}}{}}%
%BeginExpansion
\msihyperref{\fbox{\textbf{master 30001}}}{}{\fbox{\textbf{master 30001}}}{}%
%EndExpansion

\item[$\hfill $15.] Sketch the graph of the amount of a particular brand of
coffee sold by a store as a function of the price of the coffee.

%TCIMACRO{%
%\hyperref{ANSWER}{}{\textbf{ANSWER:} \FRAME{itbpF}{1.5281in}{1.1381in}{1.0032in}{}{}{3ca010115.wmf}{\special{language "Scientific Word";type "GRAPHIC";maintain-aspect-ratio TRUE;display "USEDEF";valid_file "F";width 1.5281in;height 1.1381in;depth 1.0032in;original-width 1.5281in;original-height 1.1381in;cropleft "0";croptop "1";cropright "1";cropbottom "0";filename 'graphics/3cA010115.wmf';file-properties "XNPEU";}}}{}}%
%BeginExpansion
\msihyperref{ANSWER}{}{\textbf{ANSWER:} \FRAME{itbpF}{1.5281in}{1.1381in}{1.0032in}{}{}{3ca010115.wmf}{\special{language "Scientific Word";type "GRAPHIC";maintain-aspect-ratio TRUE;display "USEDEF";valid_file "F";width 1.5281in;height 1.1381in;depth 1.0032in;original-width 1.5281in;original-height 1.1381in;cropleft "0";croptop "1";cropright "1";cropbottom "0";filename 'graphics/3cA010115.wmf';file-properties "XNPEU";}}}{}%
%EndExpansion

%TCIMACRO{%
%\hyperref{\fbox{\textbf{master 30002}}}{}{\fbox{\textbf{master 30002}}}{}}%
%BeginExpansion
\msihyperref{\fbox{\textbf{master 30002}}}{}{\fbox{\textbf{master 30002}}}{}%
%EndExpansion

\item[$\hfill $16.] You place a frozen pie in an oven and bake it for an
hour. Then you take it out and let it cool before eating it. Describe how
the temperature of the pie changes as time passes. Then sketch a rough graph
of the temperature of the pie as a function of time.

%TCIMACRO{%
%\hyperref{\fbox{\textbf{master 00014}}}{}{\fbox{\textbf{master 00014}}}{}}%
%BeginExpansion
\msihyperref{\fbox{\textbf{master 00014}}}{}{\fbox{\textbf{master 00014}}}{}%
%EndExpansion

\item[$\hfill $17.] A homeowner mows the lawn every Wednesday afternoon.
Sketch a rough graph of the height of the grass as a function of time over
the course of a four-week period.

%TCIMACRO{%
%\hyperref{ANSWER}{}{\textbf{ANSWER:}\ \FRAME{itbpF}{1.7365in}{0.8034in}{0.7524in}{}{}{6eta010117.ai}{\special{language "Scientific Word";type "GRAPHIC";maintain-aspect-ratio TRUE;display "USEDEF";valid_file "F";width 1.7365in;height 0.8034in;depth 0.7524in;original-width 1.708in;original-height 0.7757in;cropleft "0";croptop "1";cropright "1";cropbottom "0";filename 'graphics/6etA010117.ai';file-properties "XNPEU";}}}{}}%
%BeginExpansion
\msihyperref{ANSWER}{}{\textbf{ANSWER:}\ \FRAME{itbpF}{1.7365in}{0.8034in}{0.7524in}{}{}{6eta010117.ai}{\special{language "Scientific Word";type "GRAPHIC";maintain-aspect-ratio TRUE;display "USEDEF";valid_file "F";width 1.7365in;height 0.8034in;depth 0.7524in;original-width 1.708in;original-height 0.7757in;cropleft "0";croptop "1";cropright "1";cropbottom "0";filename 'graphics/6etA010117.ai';file-properties "XNPEU";}}}{}%
%EndExpansion

%TCIMACRO{%
%\hyperref{\fbox{\textbf{master 00015}}}{}{\fbox{\textbf{master 00015}}}{}}%
%BeginExpansion
\msihyperref{\fbox{\textbf{master 00015}}}{}{\fbox{\textbf{master 00015}}}{}%
%EndExpansion

\item[$\hfill $18.] An airplane takes off from an airport and lands an hour
later at another airport, 400 miles away. If $t$ represents the time in
minutes since the plane has left the terminal building, let $x(t)$ be the
horizontal distance traveled and $y(t)$ be the altitude of the plane.

\begin{ExerciseList}
\item[(a)] Sketch a possible graph of $x(t)$.

%TCIMACRO{%
%\hyperref{\fbox{\textbf{master 00016a}}}{}{\fbox{\textbf{master 00016a}}}{}}%
%BeginExpansion
\msihyperref{\fbox{\textbf{master 00016a}}}{}{\fbox{\textbf{master 00016a}}}{}%
%EndExpansion

\item[(b)] Sketch a possible graph of $y(t)$.

%TCIMACRO{%
%\hyperref{\fbox{\textbf{master 00016b}}}{}{\fbox{\textbf{master 00016b}}}{}}%
%BeginExpansion
\msihyperref{\fbox{\textbf{master 00016b}}}{}{\fbox{\textbf{master 00016b}}}{}%
%EndExpansion

\item[(c)] Sketch a possible graph of the ground speed.

%TCIMACRO{%
%\hyperref{\fbox{\textbf{master 00016c}}}{}{\fbox{\textbf{master 00016c}}}{}}%
%BeginExpansion
\msihyperref{\fbox{\textbf{master 00016c}}}{}{\fbox{\textbf{master 00016c}}}{}%
%EndExpansion

\item[(d)] Sketch a possible graph of the vertical velocity.

%TCIMACRO{%
%\hyperref{\fbox{\textbf{master 00016d}}}{}{\fbox{\textbf{master 00016d}}}{}}%
%BeginExpansion
\msihyperref{\fbox{\textbf{master 00016d}}}{}{\fbox{\textbf{master 00016d}}}{}%
%EndExpansion
\end{ExerciseList}

\item[$\hfill $19.] The number $N$ (in millions) of cellular phone
subscribers worldwide is shown in the table. (Midyear estimates are given.)%
\vspace{9pt}

$\hfill 
\begin{tabular}{|c|c|c|c|c|c|c|}
\hline
$t$ & $1990$ & $1992$ & $1994$ & $1996$ & $1998$ & $2000$ \\ \hline
$N$ & $11$ & $26$ & $60$ & $160$ & $340$ & $650$ \\ \hline
\end{tabular}%
\ \vspace{9pt}\hfill $

\begin{ExerciseList}
\item[(a)] Use the data to sketch a rough graph of $N$ as a function of~$t$.

%TCIMACRO{%
%\hyperref{\fbox{\textbf{master 30003a}}}{}{\fbox{\textbf{master 30003a}}}{}}%
%BeginExpansion
\msihyperref{\fbox{\textbf{master 30003a}}}{}{\fbox{\textbf{master 30003a}}}{}%
%EndExpansion

\item[(b)] Use your graph to estimate the number of cell-phone subscribers
at midyear in 1995 and 1999.

%TCIMACRO{%
%\hyperref{\fbox{\textbf{master 30003b}}}{}{\fbox{\textbf{master 30003b}}}{}}%
%BeginExpansion
\msihyperref{\fbox{\textbf{master 30003b}}}{}{\fbox{\textbf{master 30003b}}}{}%
%EndExpansion
\end{ExerciseList}

%TCIMACRO{%
%\hyperref{ANSWER}{}{\textbf{ANSWER:} (a)~\FRAME{itbpF}{1.625in}{1.3188in}{1.254in}{}{}{3ca010119a.wmf}{\special{language "Scientific Word";type "GRAPHIC";maintain-aspect-ratio TRUE;display "USEDEF";valid_file "F";width 1.625in;height 1.3188in;depth 1.254in;original-width 1.625in;original-height 1.3188in;cropleft "0";croptop "1";cropright "1";cropbottom "0";filename 'graphics/3cA010119a.wmf';file-properties "XNPEU";}}\quad \linebreak (b)~In millions: 92; 485}{}}%
%BeginExpansion
\msihyperref{ANSWER}{}{\textbf{ANSWER:} (a)~\FRAME{itbpF}{1.625in}{1.3188in}{1.254in}{}{}{3ca010119a.wmf}{\special{language "Scientific Word";type "GRAPHIC";maintain-aspect-ratio TRUE;display "USEDEF";valid_file "F";width 1.625in;height 1.3188in;depth 1.254in;original-width 1.625in;original-height 1.3188in;cropleft "0";croptop "1";cropright "1";cropbottom "0";filename 'graphics/3cA010119a.wmf';file-properties "XNPEU";}}\quad \linebreak (b)~In millions: 92; 485}{}%
%EndExpansion

\item[$\hfill $20.] Temperature readings $T$ ( in $^{\circ }$F) were
recorded every two hours from midnight to 2:00 \textsc{pm} in Dallas on June
2, 2001. The time $t$ was measured in hours from midnight.\vspace{9pt}

$\hfill 
\begin{tabular}{|c|r|r|r|r|r|r|r|r|}
\hline
$t$ & $0$ & $2$ & $4$ & $6$ & $8$ & $10$ & $12$ & $14$ \\ \hline
$T$ & $73$ & $73$ & $70$ & $69$ & $72$ & $81$ & $88$ & $91$ \\ \hline
\end{tabular}%
\ $\vspace{9pt}$\hfill $

\begin{ExerciseList}
\item[(a)] Use the readings to sketch a rough graph of $T$ as a function of $%
t$.

%TCIMACRO{%
%\hyperref{\fbox{\textbf{master 00018a}}}{}{\fbox{\textbf{master 00018a}}}{}}%
%BeginExpansion
\msihyperref{\fbox{\textbf{master 00018a}}}{}{\fbox{\textbf{master 00018a}}}{}%
%EndExpansion

\item[(b)] Use your graph to estimate the temperature at 11:00 \textsc{am}.

%TCIMACRO{%
%\hyperref{\fbox{\textbf{master 00018b}}}{}{\fbox{\textbf{master 00018b}}}{}}%
%BeginExpansion
\msihyperref{\fbox{\textbf{master 00018b}}}{}{\fbox{\textbf{master 00018b}}}{}%
%EndExpansion
\end{ExerciseList}

\item[$\hfill $21.] If $f(x)=3x^{2}-x+2$, find $f(2)$, $f(-2)$, $f\left(
a\right) $, $f\left( -a\right) $, $f(a+1)$, $2f(a)$, $f(2a)$, $f(a^{2})$, $%
[f(a)]^{2}$, and $f(a+h)$.

%TCIMACRO{%
%\hyperref{ANSWER}{}{\textbf{ANSWER:}\ $12$, $16$, $3a^{2}-a+2$, $3a^{2}+a+2$, $3a^{2}+5a+4$, $6a^{2}-2a+4$, $12a^{2}-2a+2$, $3a^{4}-a^{2}+2$, $9a^{4}-6a^{3}+13a^{2}-4a+4$, $3a^{2}+6ah+3h^{2}-a-h+2$}{}}%
%BeginExpansion
\msihyperref{ANSWER}{}{\textbf{ANSWER:}\ $12$, $16$, $3a^{2}-a+2$, $3a^{2}+a+2$, $3a^{2}+5a+4$, $6a^{2}-2a+4$, $12a^{2}-2a+2$, $3a^{4}-a^{2}+2$, $9a^{4}-6a^{3}+13a^{2}-4a+4$, $3a^{2}+6ah+3h^{2}-a-h+2$}{}%
%EndExpansion

%TCIMACRO{%
%\hyperref{\fbox{\textbf{master 00019}}}{}{\fbox{\textbf{master 00019}}}{}}%
%BeginExpansion
\msihyperref{\fbox{\textbf{master 00019}}}{}{\fbox{\textbf{master 00019}}}{}%
%EndExpansion

\item[$\hfill $22.] A spherical balloon with radius $r$ inches has volume 
\newline
$V(r)=\frac{4}{3}\pi r^{3}$. Find a function that represents the amount of
air required to inflate the balloon from a radius of $r$ inches to a radius
of $r+1$ inches.

%TCIMACRO{%
%\hyperref{\fbox{\textbf{master 00020}}}{}{\fbox{\textbf{master 00020}}}{}}%
%BeginExpansion
\msihyperref{\fbox{\textbf{master 00020}}}{}{\fbox{\textbf{master 00020}}}{}%
%EndExpansion
\end{ExerciseList}

\begin{instructions}
\QTR{SpanExer}{23--26}{\small 
%TCIMACRO{\TeXButton{SQR}{\hskip .5em\rule{4pt}{4pt}\hskip .5em}}%
%BeginExpansion
\hskip .5em\rule{4pt}{4pt}\hskip .5em%
%EndExpansion
Evaluate the difference quotient for the given function. Simplify your
answer.}
\end{instructions}

\begin{ExerciseList}
\item[{$\hfill $\protect\fbox{\hspace{-2pt}23.\hspace{-2pt}}}] $%
f(x)=4+3x-x^{2}$,\qquad $\dfrac{f(3+h)-f(3)}{h}$

%TCIMACRO{\hyperref{ANSWER}{}{\textbf{ANSWER:} $-3-h$}{}}%
%BeginExpansion
\msihyperref{ANSWER}{}{\textbf{ANSWER:} $-3-h$}{}%
%EndExpansion

%TCIMACRO{%
%\hyperref{\fbox{\textbf{master 30004}}}{}{\fbox{\textbf{master 30004}}}{}}%
%BeginExpansion
\msihyperref{\fbox{\textbf{master 30004}}}{}{\fbox{\textbf{master 30004}}}{}%
%EndExpansion

\item[$\hfill $24.] $f(x)=x^{3}$,\qquad $\dfrac{f(a+h)-f(a)}{h}$

%TCIMACRO{%
%\hyperref{\fbox{\textbf{master 30005}}}{}{\fbox{\textbf{master 30005}}}{}}%
%BeginExpansion
\msihyperref{\fbox{\textbf{master 30005}}}{}{\fbox{\textbf{master 30005}}}{}%
%EndExpansion

\item[$\hfill $25.] $f(x)=\dfrac{1}{x}$,\qquad $\dfrac{f(x)-f(a)}{x-a}$

%TCIMACRO{\hyperref{ANSWER}{}{\textbf{ANSWER:} $-1/(ax)$}{}}%
%BeginExpansion
\msihyperref{ANSWER}{}{\textbf{ANSWER:} $-1/(ax)$}{}%
%EndExpansion

%TCIMACRO{%
%\hyperref{\fbox{\textbf{master 30006}}}{}{\fbox{\textbf{master 30006}}}{}}%
%BeginExpansion
\msihyperref{\fbox{\textbf{master 30006}}}{}{\fbox{\textbf{master 30006}}}{}%
%EndExpansion

\item[$\hfill $26.] $f(x)=\dfrac{x+3}{x+1}$,$\qquad \dfrac{f(x)-f(1)}{x-1}$

%TCIMACRO{%
%\hyperref{\fbox{\textbf{master 30007}}}{}{\fbox{\textbf{master 30007}}}{}}%
%BeginExpansion
\msihyperref{\fbox{\textbf{master 30007}}}{}{\fbox{\textbf{master 30007}}}{}%
%EndExpansion
\end{ExerciseList}

\begin{instructions}
\FRAME{itbpF}{235.6875pt}{5.5pt}{0pt}{}{}{dots.wmf}{\special{language
"Scientific Word";type "GRAPHIC";maintain-aspect-ratio TRUE;display
"USEDEF";valid_file "F";width 235.6875pt;height 5.5pt;depth
0pt;original-width 3.9167in;original-height 0.0735in;cropleft "0";croptop
"0.9772";cropright "0.8327";cropbottom "0.0226";filename
'graphics/dots.wmf';file-properties "XNPEU";}}\vspace{-9pt}
\end{instructions}

\begin{instructions}
\QTR{SpanExer}{27--31}{\small 
%TCIMACRO{\TeXButton{SQR}{\hskip .5em\rule{4pt}{4pt}\hskip .5em}}%
%BeginExpansion
\hskip .5em\rule{4pt}{4pt}\hskip .5em%
%EndExpansion
Find the domain of the function.}
\end{instructions}

\begin{ExerciseList}
\item[$\hfill $27.] $f(x)=\dfrac{x}{3x-1}$

%TCIMACRO{%
%\hyperref{ANSWER}{}{\textbf{ANSWER:}\ $\left\{ x\,|\,x\neq \tfrac{1}{3}\right\} =\left( -\infty ,\tfrac{1}{3}\right) \cup \left( \tfrac{1}{3},\infty \right) $}{}}%
%BeginExpansion
\msihyperref{ANSWER}{}{\textbf{ANSWER:}\ $\left\{ x\,|\,x\neq \tfrac{1}{3}\right\} =\left( -\infty ,\tfrac{1}{3}\right) \cup \left( \tfrac{1}{3},\infty \right) $}{}%
%EndExpansion

%TCIMACRO{%
%\hyperref{\fbox{\textbf{master 00023}}}{}{\fbox{\textbf{master 00023}}}{}}%
%BeginExpansion
\msihyperref{\fbox{\textbf{master 00023}}}{}{\fbox{\textbf{master 00023}}}{}%
%EndExpansion

\item[$\hfill $28.] $f(x)=\dfrac{5x+4}{x^{2}+3x+2}$

%TCIMACRO{%
%\hyperref{\fbox{\textbf{master 00024}}}{}{\fbox{\textbf{master 00024}}}{}}%
%BeginExpansion
\msihyperref{\fbox{\textbf{master 00024}}}{}{\fbox{\textbf{master 00024}}}{}%
%EndExpansion

\item[$\hfill $29.] $f(t)=\sqrt{t}+\sqrt[3]{t}$

%TCIMACRO{%
%\hyperref{ANSWER}{}{\textbf{ANSWER:}\ $\left[ 0,\infty \right) $}{}}%
%BeginExpansion
\msihyperref{ANSWER}{}{\textbf{ANSWER:}\ $\left[ 0,\infty \right) $}{}%
%EndExpansion

%TCIMACRO{%
%\hyperref{\fbox{\textbf{master 00025}}}{}{\fbox{\textbf{master 00025}}}{}}%
%BeginExpansion
\msihyperref{\fbox{\textbf{master 00025}}}{}{\fbox{\textbf{master 00025}}}{}%
%EndExpansion

\item[$\hfill $30.] $g(u)=\sqrt{u}+\sqrt{4-u}$

%TCIMACRO{%
%\hyperref{\fbox{\textbf{master 00026}}}{}{\fbox{\textbf{master 00026}}}{}}%
%BeginExpansion
\msihyperref{\fbox{\textbf{master 00026}}}{}{\fbox{\textbf{master 00026}}}{}%
%EndExpansion

\item[{$\hfill $\protect\fbox{\hspace{-2pt}31.\hspace{-2pt}}}] $h(x)=\dfrac{1%
}{\sqrt[4]{x^{2}-5x}}$

%TCIMACRO{%
%\hyperref{ANSWER}{}{\textbf{ANSWER:}\ $\left( -\infty ,0\right) \cup \left( 5,\infty \right) $}{}}%
%BeginExpansion
\msihyperref{ANSWER}{}{\textbf{ANSWER:}\ $\left( -\infty ,0\right) \cup \left( 5,\infty \right) $}{}%
%EndExpansion

%TCIMACRO{%
%\hyperref{\fbox{\textbf{master 00027}}}{}{\fbox{\textbf{master 00027}}}{}}%
%BeginExpansion
\msihyperref{\fbox{\textbf{master 00027}}}{}{\fbox{\textbf{master 00027}}}{}%
%EndExpansion
\end{ExerciseList}

\begin{instructions}
\FRAME{itbpF}{235.6875pt}{5.5pt}{0pt}{}{}{dots.wmf}{\special{language
"Scientific Word";type "GRAPHIC";maintain-aspect-ratio TRUE;display
"USEDEF";valid_file "F";width 235.6875pt;height 5.5pt;depth
0pt;original-width 3.9167in;original-height 0.0735in;cropleft "0";croptop
"0.9772";cropright "0.8327";cropbottom "0.0226";filename
'graphics/dots.wmf';file-properties "XNPEU";}}\vspace{-3pt}
\end{instructions}

\begin{ExerciseList}
\item[$\hfill $32.] Find the domain and range and sketch the graph of the 
\newline
function $h(x)=\sqrt{4-x^{2}}$.

%TCIMACRO{%
%\hyperref{\fbox{\textbf{master 00028}}}{}{\fbox{\textbf{master 00028}}}{}}%
%BeginExpansion
\msihyperref{\fbox{\textbf{master 00028}}}{}{\fbox{\textbf{master 00028}}}{}%
%EndExpansion
\end{ExerciseList}

\begin{instructions}
\QTR{SpanExer}{33--44}{\small 
%TCIMACRO{\TeXButton{SQR}{\hskip .5em\rule{4pt}{4pt}\hskip .5em}}%
%BeginExpansion
\hskip .5em\rule{4pt}{4pt}\hskip .5em%
%EndExpansion
Find the domain and sketch the graph of the function.}
\end{instructions}

\begin{ExerciseList}
\item[$\hfill $33.] $f(x)=5$

%TCIMACRO{%
%\hyperref{ANSWER}{}{\textbf{ANSWER:}\ $\left( -\infty ,\infty \right) $ \FRAME{itbpF}{1.1805in}{1.0136in}{1.0032in}{}{}{5et010129.wmf}{\special{language "Scientific Word";type "GRAPHIC";maintain-aspect-ratio TRUE;display "USEDEF";valid_file "F";width 1.1805in;height 1.0136in;depth 1.0032in;original-width 1.1805in;original-height 1.0136in;cropleft "0";croptop "1";cropright "1";cropbottom "0";filename 'graphics/5et010129.wmf';file-properties "XNPEU";}}}{}}%
%BeginExpansion
\msihyperref{ANSWER}{}{\textbf{ANSWER:}\ $\left( -\infty ,\infty \right) $ \FRAME{itbpF}{1.1805in}{1.0136in}{1.0032in}{}{}{5et010129.wmf}{\special{language "Scientific Word";type "GRAPHIC";maintain-aspect-ratio TRUE;display "USEDEF";valid_file "F";width 1.1805in;height 1.0136in;depth 1.0032in;original-width 1.1805in;original-height 1.0136in;cropleft "0";croptop "1";cropright "1";cropbottom "0";filename 'graphics/5et010129.wmf';file-properties "XNPEU";}}}{}%
%EndExpansion

%TCIMACRO{%
%\hyperref{\fbox{\textbf{master 00029}}}{}{\fbox{\textbf{master 00029}}}{}}%
%BeginExpansion
\msihyperref{\fbox{\textbf{master 00029}}}{}{\fbox{\textbf{master 00029}}}{}%
%EndExpansion

\item[$\hfill $34.] $F(x)=\frac{1}{2}(x+3)$

%TCIMACRO{%
%\hyperref{\fbox{\textbf{master 00030}}}{}{\fbox{\textbf{master 00030}}}{}}%
%BeginExpansion
\msihyperref{\fbox{\textbf{master 00030}}}{}{\fbox{\textbf{master 00030}}}{}%
%EndExpansion

\item[$\hfill $35.] $f(t)=t^{2}-6t$

%TCIMACRO{%
%\hyperref{ANSWER}{}{\textbf{ANSWER:}\ $\left( -\infty ,\infty \right) $ \FRAME{itbpF}{1.1805in}{1.1805in}{1.0032in}{}{}{5et010131.wmf}{\special{language "Scientific Word";type "GRAPHIC";maintain-aspect-ratio TRUE;display "USEDEF";valid_file "F";width 1.1805in;height 1.1805in;depth 1.0032in;original-width 1.1805in;original-height 1.1805in;cropleft "0";croptop "1";cropright "1";cropbottom "0";filename 'graphics/5et010131.wmf';file-properties "XNPEU";}}}{}}%
%BeginExpansion
\msihyperref{ANSWER}{}{\textbf{ANSWER:}\ $\left( -\infty ,\infty \right) $ \FRAME{itbpF}{1.1805in}{1.1805in}{1.0032in}{}{}{5et010131.wmf}{\special{language "Scientific Word";type "GRAPHIC";maintain-aspect-ratio TRUE;display "USEDEF";valid_file "F";width 1.1805in;height 1.1805in;depth 1.0032in;original-width 1.1805in;original-height 1.1805in;cropleft "0";croptop "1";cropright "1";cropbottom "0";filename 'graphics/5et010131.wmf';file-properties "XNPEU";}}}{}%
%EndExpansion

%TCIMACRO{%
%\hyperref{\fbox{\textbf{master 00031}}}{}{\fbox{\textbf{master 00031}}}{}}%
%BeginExpansion
\msihyperref{\fbox{\textbf{master 00031}}}{}{\fbox{\textbf{master 00031}}}{}%
%EndExpansion

\item[$\hfill $36.] $H(t)=\dfrac{4-t^{2}}{2-t}$

%TCIMACRO{%
%\hyperref{\fbox{\textbf{master 00032}}}{}{\fbox{\textbf{master 00032}}}{}}%
%BeginExpansion
\msihyperref{\fbox{\textbf{master 00032}}}{}{\fbox{\textbf{master 00032}}}{}%
%EndExpansion

\item[$\hfill $37.] $g(x)=\sqrt{x-5}$

%TCIMACRO{%
%\hyperref{ANSWER}{}{\textbf{ANSWER:}\ $\left[ 5,\infty \right) $\quad \FRAME{itbpF}{1.5696in}{0.7204in}{0.6529in}{}{}{4a010131.wmf}{\special{language "Scientific Word";type "GRAPHIC";maintain-aspect-ratio TRUE;display "USEDEF";valid_file "F";width 1.5696in;height 0.7204in;depth 0.6529in;original-width 0pt;original-height 0pt;cropleft "0";croptop "1";cropright "1";cropbottom "0";filename 'graphics/4a010131.wmf';file-properties "XNPEU";}}}{}}%
%BeginExpansion
\msihyperref{ANSWER}{}{\textbf{ANSWER:}\ $\left[ 5,\infty \right) $\quad \FRAME{itbpF}{1.5696in}{0.7204in}{0.6529in}{}{}{4a010131.wmf}{\special{language "Scientific Word";type "GRAPHIC";maintain-aspect-ratio TRUE;display "USEDEF";valid_file "F";width 1.5696in;height 0.7204in;depth 0.6529in;original-width 0pt;original-height 0pt;cropleft "0";croptop "1";cropright "1";cropbottom "0";filename 'graphics/4a010131.wmf';file-properties "XNPEU";}}}{}%
%EndExpansion

%TCIMACRO{%
%\hyperref{\fbox{\textbf{master 00033}}}{}{\fbox{\textbf{master 00033}}}{}}%
%BeginExpansion
\msihyperref{\fbox{\textbf{master 00033}}}{}{\fbox{\textbf{master 00033}}}{}%
%EndExpansion

\item[$\hfill $38.] $F(x)=\left| 2x+1\right| $

%TCIMACRO{%
%\hyperref{\fbox{\textbf{master 00034}}}{}{\fbox{\textbf{master 00034}}}{}}%
%BeginExpansion
\msihyperref{\fbox{\textbf{master 00034}}}{}{\fbox{\textbf{master 00034}}}{}%
%EndExpansion

\item[{$\hfill $\protect\fbox{\hspace{-2pt}39.\hspace{-2pt}}}] $G(x)=\dfrac{%
3x+\left| x\right| }{x}$

%TCIMACRO{%
%\hyperref{ANSWER}{}{\textbf{ANSWER:}\ $\left( -\infty ,0\right) \cup \left( 0,\infty \right) $\FRAME{itbpF}{0.9029in}{0.9029in}{0in}{}{}{ca010131.wmf}{\special{language "Scientific Word";type "GRAPHIC";maintain-aspect-ratio TRUE;display "USEDEF";valid_file "F";width 0.9029in;height 0.9029in;depth 0in;original-width 0pt;original-height 0pt;cropleft "0";croptop "1";cropright "1";cropbottom "0";filename 'graphics/ca010131.wmf';file-properties "XNPEU";}}}{}}%
%BeginExpansion
\msihyperref{ANSWER}{}{\textbf{ANSWER:}\ $\left( -\infty ,0\right) \cup \left( 0,\infty \right) $\FRAME{itbpF}{0.9029in}{0.9029in}{0in}{}{}{ca010131.wmf}{\special{language "Scientific Word";type "GRAPHIC";maintain-aspect-ratio TRUE;display "USEDEF";valid_file "F";width 0.9029in;height 0.9029in;depth 0in;original-width 0pt;original-height 0pt;cropleft "0";croptop "1";cropright "1";cropbottom "0";filename 'graphics/ca010131.wmf';file-properties "XNPEU";}}}{}%
%EndExpansion

%TCIMACRO{%
%\hyperref{\fbox{\textbf{master 00035}}}{}{\fbox{\textbf{master 00035}}}{}}%
%BeginExpansion
\msihyperref{\fbox{\textbf{master 00035}}}{}{\fbox{\textbf{master 00035}}}{}%
%EndExpansion

\item[$\hfill $40.] $g(x)=\dfrac{\left| x\right| }{x^{2}}$

%TCIMACRO{%
%\hyperref{\fbox{\textbf{master 00036}}}{}{\fbox{\textbf{master 00036}}}{}}%
%BeginExpansion
\msihyperref{\fbox{\textbf{master 00036}}}{}{\fbox{\textbf{master 00036}}}{}%
%EndExpansion

\item[$\hfill $41.] $f(x)=\left\{ 
\begin{array}{ll}
x+2 & \quad \text{if }x<0 \\ 
1-x & \quad \text{if }x\geq 0%
\end{array}%
\right. $

%TCIMACRO{%
%\hyperref{ANSWER}{}{\textbf{ANSWER:} $\left( -\infty ,\infty \right) $\quad \FRAME{itbpF}{0.9029in}{0.9029in}{0.8026in}{}{}{3ca010141.wmf}{\special{language "Scientific Word";type "GRAPHIC";maintain-aspect-ratio TRUE;display "USEDEF";valid_file "F";width 0.9029in;height 0.9029in;depth 0.8026in;original-width 0.9029in;original-height 0.9029in;cropleft "0";croptop "1";cropright "1";cropbottom "0";filename 'graphics/3cA010141.wmf';file-properties "XNPEU";}}}{}}%
%BeginExpansion
\msihyperref{ANSWER}{}{\textbf{ANSWER:} $\left( -\infty ,\infty \right) $\quad \FRAME{itbpF}{0.9029in}{0.9029in}{0.8026in}{}{}{3ca010141.wmf}{\special{language "Scientific Word";type "GRAPHIC";maintain-aspect-ratio TRUE;display "USEDEF";valid_file "F";width 0.9029in;height 0.9029in;depth 0.8026in;original-width 0.9029in;original-height 0.9029in;cropleft "0";croptop "1";cropright "1";cropbottom "0";filename 'graphics/3cA010141.wmf';file-properties "XNPEU";}}}{}%
%EndExpansion

%TCIMACRO{%
%\hyperref{\fbox{\textbf{master 30008}}}{}{\fbox{\textbf{master 30008}}}{}}%
%BeginExpansion
\msihyperref{\fbox{\textbf{master 30008}}}{}{\fbox{\textbf{master 30008}}}{}%
%EndExpansion

\item[$\hfill $42.] $f(x)=\left\{ 
\begin{array}{ll}
3-\frac{1}{2}x & \quad \text{if }x\leq 2 \\ 
2x-5 & \quad \text{if }x>2%
\end{array}%
\right. $

%TCIMACRO{%
%\hyperref{\fbox{\textbf{master 30009}}}{}{\fbox{\textbf{master 30009}}}{}}%
%BeginExpansion
\msihyperref{\fbox{\textbf{master 30009}}}{}{\fbox{\textbf{master 30009}}}{}%
%EndExpansion

\item[{$\hfill $\protect\fbox{\hspace{-2pt}43.\hspace{-2pt}}}] $f(x)=\left\{ 
\begin{array}{ll}
x+2 & \quad \text{if }x\leq -1 \\ 
x^{2} & \quad \text{if }x>-1%
\end{array}%
\right. $

%TCIMACRO{%
%\hyperref{ANSWER}{}{\textbf{ANSWER:}\ $\left( -\infty ,\infty \right) $\quad \FRAME{itbpF}{1.0836in}{0.9158in}{0.7524in}{}{}{4a010139.wmf}{\special{language "Scientific Word";type "GRAPHIC";maintain-aspect-ratio TRUE;display "USEDEF";valid_file "F";width 1.0836in;height 0.9158in;depth 0.7524in;original-width 0pt;original-height 0pt;cropleft "0";croptop "1";cropright "1";cropbottom "0";filename 'graphics/4a010139.wmf';file-properties "XNPEU";}}}{}}%
%BeginExpansion
\msihyperref{ANSWER}{}{\textbf{ANSWER:}\ $\left( -\infty ,\infty \right) $\quad \FRAME{itbpF}{1.0836in}{0.9158in}{0.7524in}{}{}{4a010139.wmf}{\special{language "Scientific Word";type "GRAPHIC";maintain-aspect-ratio TRUE;display "USEDEF";valid_file "F";width 1.0836in;height 0.9158in;depth 0.7524in;original-width 0pt;original-height 0pt;cropleft "0";croptop "1";cropright "1";cropbottom "0";filename 'graphics/4a010139.wmf';file-properties "XNPEU";}}}{}%
%EndExpansion

%TCIMACRO{%
%\hyperref{\fbox{\textbf{master 00039}}}{}{\fbox{\textbf{master 00039}}}{}}%
%BeginExpansion
\msihyperref{\fbox{\textbf{master 00039}}}{}{\fbox{\textbf{master 00039}}}{}%
%EndExpansion

\item[$\hfill $44.] $f(x)=\left\{ 
\begin{array}{ll}
x+9 & \quad \text{if }x<-3 \\ 
-2x & \quad \text{if }\left\vert x\right\vert \leq 3 \\ 
-6 & \quad \text{if }x>3%
\end{array}%
\right. $

%TCIMACRO{%
%\hyperref{\fbox{\textbf{master 80001}}}{}{\fbox{\textbf{master 80001}}}{}}%
%BeginExpansion
\msihyperref{\fbox{\textbf{master 80001}}}{}{\fbox{\textbf{master 80001}}}{}%
%EndExpansion
\vspace{-12pt}
\end{ExerciseList}

\begin{instructions}
\FRAME{itbpF}{235.6875pt}{5.5pt}{0pt}{}{}{dots.wmf}{\special{language
"Scientific Word";type "GRAPHIC";maintain-aspect-ratio TRUE;display
"USEDEF";valid_file "F";width 235.6875pt;height 5.5pt;depth
0pt;original-width 3.9167in;original-height 0.0735in;cropleft "0";croptop
"0.9772";cropright "0.8327";cropbottom "0.0226";filename
'graphics/dots.wmf';file-properties "XNPEU";}}\vspace{-9pt}
\end{instructions}

\begin{instructions}
\QTR{SpanExer}{45--50}{\small 
%TCIMACRO{\TeXButton{SQR}{\hskip .5em\rule{4pt}{4pt}\hskip .5em}}%
%BeginExpansion
\hskip .5em\rule{4pt}{4pt}\hskip .5em%
%EndExpansion
Find an expression for the function whose graph is the given curve.}
\end{instructions}

\begin{ExerciseList}
\item[$\hfill $45.] The line segment joining the points $(1,-3)$ and $(5,7)$

%TCIMACRO{%
%\hyperref{ANSWER}{}{\textbf{ANSWER:} $f(x)=\frac{5}{2}x-\frac{11}{2}$, $1\leq x\leq 5$}{}}%
%BeginExpansion
\msihyperref{ANSWER}{}{\textbf{ANSWER:} $f(x)=\frac{5}{2}x-\frac{11}{2}$, $1\leq x\leq 5$}{}%
%EndExpansion

%TCIMACRO{%
%\hyperref{\fbox{\textbf{master 80002}}}{}{\fbox{\textbf{master 80002}}}{}}%
%BeginExpansion
\msihyperref{\fbox{\textbf{master 80002}}}{}{\fbox{\textbf{master 80002}}}{}%
%EndExpansion

\item[$\hfill $46.] The line segment joining the points $(-5,10)$ and $%
(7,-10)$

%TCIMACRO{%
%\hyperref{\fbox{\textbf{master 80003}}}{}{\fbox{\textbf{master 80003}}}{}}%
%BeginExpansion
\msihyperref{\fbox{\textbf{master 80003}}}{}{\fbox{\textbf{master 80003}}}{}%
%EndExpansion

\item[{$\hfill $\protect\fbox{\hspace{-2pt}47.\hspace{-2pt}}}] The bottom
half of the parabola $x+(y-1)^{2}=0$

%TCIMACRO{\hyperref{ANSWER}{}{\textbf{ANSWER:}\ $f(x)=1-\sqrt{-x}$}{}}%
%BeginExpansion
\msihyperref{ANSWER}{}{\textbf{ANSWER:}\ $f(x)=1-\sqrt{-x}$}{}%
%EndExpansion

%TCIMACRO{%
%\hyperref{\fbox{\textbf{master 00043}}}{}{\fbox{\textbf{master 00043}}}{}}%
%BeginExpansion
\msihyperref{\fbox{\textbf{master 00043}}}{}{\fbox{\textbf{master 00043}}}{}%
%EndExpansion

\item[$\hfill $48.] The top half of the circle $x^{2}+(y-2)^{2}=4$

%TCIMACRO{%
%\hyperref{\fbox{\textbf{master 80004}}}{}{\fbox{\textbf{master 80004}}}{}}%
%BeginExpansion
\msihyperref{\fbox{\textbf{master 80004}}}{}{\fbox{\textbf{master 80004}}}{}%
%EndExpansion

\item[$\hfill $49.] \ \FRAME{itbpF}{1.3889in}{1.0404in}{0.9037in}{}{}{%
6et0101x49.ai}{\special{language "Scientific Word";type
"GRAPHIC";maintain-aspect-ratio TRUE;display "USEDEF";valid_file "F";width
1.3889in;height 1.0404in;depth 0.9037in;original-width
1.3612in;original-height 1.0127in;cropleft "0";croptop "1";cropright
"1";cropbottom "0";filename 'graphics/6et0101x49.ai';file-properties
"XNPEU";}}

%TCIMACRO{%
%\hyperref{ANSWER}{}{\textbf{ANSWER:} $f(x)=\left\{ %
%\begin{array}{ll}
%-x+3 & \quad \text{if }0\leq x\leq 3 \\ 
%2x-6 & \quad \text{if }3<x\leq 5%
%\end{array}\right. $}{}}%
%BeginExpansion
\msihyperref{ANSWER}{}{\textbf{ANSWER:} $f(x)=\left\{ %
\begin{array}{ll}
-x+3 & \quad \text{if }0\leq x\leq 3 \\ 
2x-6 & \quad \text{if }3<x\leq 5%
\end{array}\right. $}{}%
%EndExpansion

%TCIMACRO{%
%\hyperref{\fbox{\textbf{master 80005}}}{}{\fbox{\textbf{master 80005}}}{}}%
%BeginExpansion
\msihyperref{\fbox{\textbf{master 80005}}}{}{\fbox{\textbf{master 80005}}}{}%
%EndExpansion

\item[$\hfill $50.] \ \FRAME{itbpF}{1.3612in}{1.0127in}{0.9037in}{}{}{%
6et0101x50.ai}{\special{language "Scientific Word";type
"GRAPHIC";maintain-aspect-ratio TRUE;display "USEDEF";valid_file "F";width
1.3612in;height 1.0127in;depth 0.9037in;original-width
1.3612in;original-height 1.0127in;cropleft "0";croptop "1";cropright
"1";cropbottom "0";filename 'graphics/6et0101x50.ai';file-properties
"XNPEU";}}

%TCIMACRO{%
%\hyperref{ANSWER}{}{\textbf{ANSWER:}\ \textbf{New answer to come}}{}}%
%BeginExpansion
\msihyperref{ANSWER}{}{\textbf{ANSWER:}\ \textbf{New answer to come}}{}%
%EndExpansion

%TCIMACRO{%
%\hyperref{\fbox{\textbf{master 80006}}}{}{\fbox{\textbf{master 80006}}}{}}%
%BeginExpansion
\msihyperref{\fbox{\textbf{master 80006}}}{}{\fbox{\textbf{master 80006}}}{}%
%EndExpansion
\vspace{-12pt}
\end{ExerciseList}

\begin{instructions}
\FRAME{itbpF}{235.6875pt}{5.5pt}{0pt}{}{}{dots.wmf}{\special{language
"Scientific Word";type "GRAPHIC";maintain-aspect-ratio TRUE;display
"USEDEF";valid_file "F";width 235.6875pt;height 5.5pt;depth
0pt;original-width 3.9167in;original-height 0.0735in;cropleft "0";croptop
"0.9772";cropright "0.8327";cropbottom "0.0226";filename
'graphics/dots.wmf';file-properties "XNPEU";}}\vspace{-9pt}
\end{instructions}

\begin{instructions}
\QTR{SpanExer}{51--55}{\small 
%TCIMACRO{\TeXButton{SQR}{\hskip .5em\rule{4pt}{4pt}\hskip .5em}}%
%BeginExpansion
\hskip .5em\rule{4pt}{4pt}\hskip .5em%
%EndExpansion
Find a formula for the described function and state its }\newline
{\small domain.}
\end{instructions}

\begin{ExerciseList}
\item[$\hfill $51.] A rectangle has perimeter 20 m. Express the area of the
rect-angle as a function of the length of one of its sides.

%TCIMACRO{%
%\hyperref{ANSWER}{}{\textbf{ANSWER:}\ $A(L)=10L-L^{2}$, $0<L<10$}{}}%
%BeginExpansion
\msihyperref{ANSWER}{}{\textbf{ANSWER:}\ $A(L)=10L-L^{2}$, $0<L<10$}{}%
%EndExpansion

%TCIMACRO{%
%\hyperref{\fbox{\textbf{master 00047}}}{}{\fbox{\textbf{master 00047}}}{}}%
%BeginExpansion
\msihyperref{\fbox{\textbf{master 00047}}}{}{\fbox{\textbf{master 00047}}}{}%
%EndExpansion

\item[$\hfill $52.] A rectangle has area 16 m$^{2}$. Express the perimeter
of the rect-angle as a function of the length of one of its sides.

%TCIMACRO{%
%\hyperref{\fbox{\textbf{master 00048}}}{}{\fbox{\textbf{master 00048}}}{}}%
%BeginExpansion
\msihyperref{\fbox{\textbf{master 00048}}}{}{\fbox{\textbf{master 00048}}}{}%
%EndExpansion

\item[$\hfill $53.] Express the area of an equilateral triangle as a
function of the length of a side.

%TCIMACRO{%
%\hyperref{ANSWER}{}{\textbf{ANSWER:}\ $A(x)=\sqrt{3}x^{2}/4$, $x>0$}{}}%
%BeginExpansion
\msihyperref{ANSWER}{}{\textbf{ANSWER:}\ $A(x)=\sqrt{3}x^{2}/4$, $x>0$}{}%
%EndExpansion

%TCIMACRO{%
%\hyperref{\fbox{\textbf{master 00049}}}{}{\fbox{\textbf{master 00049}}}{}}%
%BeginExpansion
\msihyperref{\fbox{\textbf{master 00049}}}{}{\fbox{\textbf{master 00049}}}{}%
%EndExpansion

\item[$\hfill $54.] Express the surface area of a cube as a function of its
volume.

%TCIMACRO{%
%\hyperref{\fbox{\textbf{master 00050}}}{}{\fbox{\textbf{master 00050}}}{}}%
%BeginExpansion
\msihyperref{\fbox{\textbf{master 00050}}}{}{\fbox{\textbf{master 00050}}}{}%
%EndExpansion

\item[{$\hfill $\protect\fbox{\hspace{-2pt}55.\hspace{-2pt}}}] An open
rectangular box with volume 2 m$^{3}$ has a square base. Express the surface
area of the box as a function of the length of a side of the base.

%TCIMACRO{%
%\hyperref{ANSWER}{}{\textbf{ANSWER:}\ $S(x)=x^{2}+\left( 8/x\right) $, $x>0$}{}}%
%BeginExpansion
\msihyperref{ANSWER}{}{\textbf{ANSWER:}\ $S(x)=x^{2}+\left( 8/x\right) $, $x>0$}{}%
%EndExpansion

%TCIMACRO{%
%\hyperref{\fbox{\textbf{master 00051}}}{}{\fbox{\textbf{master 00051}}}{}}%
%BeginExpansion
\msihyperref{\fbox{\textbf{master 00051}}}{}{\fbox{\textbf{master 00051}}}{}%
%EndExpansion
\vspace{-15pt}
\end{ExerciseList}

\begin{instructions}
\FRAME{itbpF}{235.6875pt}{5.5pt}{0pt}{}{}{dots.wmf}{\special{language
"Scientific Word";type "GRAPHIC";maintain-aspect-ratio TRUE;display
"USEDEF";valid_file "F";width 235.6875pt;height 5.5pt;depth
0pt;original-width 3.9167in;original-height 0.0735in;cropleft "0";croptop
"0.9716";cropright "0.8327";cropbottom "0.0282";filename
'graphics/dots.wmf';file-properties "XNPEU";}}\pagebreak
\end{instructions}

\begin{ExerciseList}
\item[$\hfill $56.] A Norman window has the shape of a rectangle surmounted
by a semicircle. If the perimeter of the window is 30 ft, express the area $%
A $ of the window as a function of the width $x$ of the window.\bigskip 
\FRAME{dtbpF}{1.0404in}{1.5627in}{0pt}{}{}{6et0101x56_00048.tif}{\special%
{language "Scientific Word";type "GRAPHIC";maintain-aspect-ratio
TRUE;display "USEDEF";valid_file "F";width 1.0404in;height 1.5627in;depth
0pt;original-width 1.0127in;original-height 1.535in;cropleft "0";croptop
"1";cropright "1";cropbottom "0";filename
'graphics/6et0101x56_00048.tif';file-properties "XNPEU";}}

%TCIMACRO{%
%\hyperref{\fbox{\textbf{master 00052}}}{}{\fbox{\textbf{master 00052}}}{}}%
%BeginExpansion
\msihyperref{\fbox{\textbf{master 00052}}}{}{\fbox{\textbf{master 00052}}}{}%
%EndExpansion

\item[$\hfill $57.] A box with an open top is to be constructed from a
rectangular piece of cardboard with dimensions 12 in. by 20 in. by cutting
out equal squares of side $x$ at each corner and then folding up the sides
as in the figure. Express the volume $V$ of the box as a function of $x$%
.\bigskip

\FRAME{dtbpF}{2.847in}{1.0603in}{0pt}{}{}{3c0101x57.wmf}{\special{language
"Scientific Word";type "GRAPHIC";maintain-aspect-ratio TRUE;display
"USEDEF";valid_file "F";width 2.847in;height 1.0603in;depth
0pt;original-width 2.847in;original-height 1.0603in;cropleft "0";croptop
"1";cropright "1";cropbottom "0";filename
'graphics/3c0101x57.wmf';file-properties "XNPEU";}}

%TCIMACRO{%
%\hyperref{ANSWER}{}{\textbf{ANSWER:}\ $V(x)=4x^{3}-64x^{2}+240x$, $0<x<6$}{}}%
%BeginExpansion
\msihyperref{ANSWER}{}{\textbf{ANSWER:}\ $V(x)=4x^{3}-64x^{2}+240x$, $0<x<6$}{}%
%EndExpansion

%TCIMACRO{%
%\hyperref{\fbox{\textbf{master 00053}}}{}{\fbox{\textbf{master 00053}}}{}}%
%BeginExpansion
\msihyperref{\fbox{\textbf{master 00053}}}{}{\fbox{\textbf{master 00053}}}{}%
%EndExpansion

\item[$\hfill $58.] A taxi company charges two dollars for the first mile
(or part of a mile) and 20 cents for each succeeding tenth of a mile (or
part). Express the cost $C$ (in dollars) of a ride as a function of the
distance $x$ traveled (in miles) for $0<x<2$, and sketch the graph of this
function.

%TCIMACRO{%
%\hyperref{\fbox{\textbf{master 00054}}}{}{\fbox{\textbf{master 00054}}}{}}%
%BeginExpansion
\msihyperref{\fbox{\textbf{master 00054}}}{}{\fbox{\textbf{master 00054}}}{}%
%EndExpansion

\item[{$\hfill $\protect\fbox{\hspace{-2pt}59.\hspace{-2pt}}}] In a certain
country, income tax is assessed as follows. There is no tax on income up to
\$10,000. Any income over \$10,000 is taxed at a rate of 10\%, up to an
income of \$20,000. Any income over \$20,000 is taxed at 15\%.

\begin{ExerciseList}
\item[(a)] Sketch the graph of the tax rate \textit{R} as a function of the
income \textit{I}.

%TCIMACRO{%
%\hyperref{\fbox{\textbf{master 00055a}}}{}{\fbox{\textbf{master 00055a}}}{}}%
%BeginExpansion
\msihyperref{\fbox{\textbf{master 00055a}}}{}{\fbox{\textbf{master 00055a}}}{}%
%EndExpansion

\item[(b)] How much tax is assessed on an income of \$14,000? On \$26,000?

%TCIMACRO{%
%\hyperref{\fbox{\textbf{master 00055b}}}{}{\fbox{\textbf{master 00055b}}}{}}%
%BeginExpansion
\msihyperref{\fbox{\textbf{master 00055b}}}{}{\fbox{\textbf{master 00055b}}}{}%
%EndExpansion

\item[(c)] Sketch the graph of the total assessed tax \textit{T} as a
function of the income \textit{I}.

%TCIMACRO{%
%\hyperref{\fbox{\textbf{master 00055c}}}{}{\fbox{\textbf{master 00055c}}}{}}%
%BeginExpansion
\msihyperref{\fbox{\textbf{master 00055c}}}{}{\fbox{\textbf{master 00055c}}}{}%
%EndExpansion
\end{ExerciseList}

%TCIMACRO{%
%\hyperref{ANSWER}{}{\textbf{ANSWER:}\ (a)~\FRAME{itbpF}{1.7642in}{0.9426in}{0.8026in}{}{}{4a010155a.wmf}{\special{language "Scientific Word";type "GRAPHIC";maintain-aspect-ratio TRUE;display "USEDEF";valid_file "F";width 1.7642in;height 0.9426in;depth 0.8026in;original-width 0pt;original-height 0pt;cropleft "0";croptop "1";cropright "1";cropbottom "0";filename 'graphics/4a010155a.wmf';file-properties "XNPEU";}}\newline
%(b)~\$400, \$1900\newline
%(c)~\FRAME{itbpF}{2.0418in}{0.9418in}{0.8536in}{}{}{4a010155c.wmf}{\special{language "Scientific Word";type "GRAPHIC";maintain-aspect-ratio TRUE;display "USEDEF";valid_file "F";width 2.0418in;height 0.9418in;depth 0.8536in;original-width 0pt;original-height 0pt;cropleft "0";croptop "1";cropright "1";cropbottom "0";filename 'graphics/4a010155c.wmf';file-properties "XNPEU";}}}{}}%
%BeginExpansion
\msihyperref{ANSWER}{}{\textbf{ANSWER:}\ (a)~\FRAME{itbpF}{1.7642in}{0.9426in}{0.8026in}{}{}{4a010155a.wmf}{\special{language "Scientific Word";type "GRAPHIC";maintain-aspect-ratio TRUE;display "USEDEF";valid_file "F";width 1.7642in;height 0.9426in;depth 0.8026in;original-width 0pt;original-height 0pt;cropleft "0";croptop "1";cropright "1";cropbottom "0";filename 'graphics/4a010155a.wmf';file-properties "XNPEU";}}\newline
(b)~\$400, \$1900\newline
(c)~\FRAME{itbpF}{2.0418in}{0.9418in}{0.8536in}{}{}{4a010155c.wmf}{\special{language "Scientific Word";type "GRAPHIC";maintain-aspect-ratio TRUE;display "USEDEF";valid_file "F";width 2.0418in;height 0.9418in;depth 0.8536in;original-width 0pt;original-height 0pt;cropleft "0";croptop "1";cropright "1";cropbottom "0";filename 'graphics/4a010155c.wmf';file-properties "XNPEU";}}}{}%
%EndExpansion

\item[$\hfill $60.] The functions in Example~%
%TCIMACRO{\TeXButton{10}{10}}%
%BeginExpansion
10%
%EndExpansion
~and Exercises~%
%TCIMACRO{\TeXButton{58 and 59(a)}{58 and 59(a)}}%
%BeginExpansion
58 and 59(a)%
%EndExpansion
~are called \textit{step functions} because their graphs look like stairs.
Give two other examples of step functions that arise in everyday life.

%TCIMACRO{%
%\hyperref{\fbox{\textbf{master 00056}}}{}{\fbox{\textbf{master 00056}}}{}}%
%BeginExpansion
\msihyperref{\fbox{\textbf{master 00056}}}{}{\fbox{\textbf{master 00056}}}{}%
%EndExpansion
\end{ExerciseList}

\begin{instructions}
\QTR{SpanExer}{61--62}{\small 
%TCIMACRO{\TeXButton{SQR}{\hskip .5em\rule{4pt}{4pt}\hskip .5em}}%
%BeginExpansion
\hskip .5em\rule{4pt}{4pt}\hskip .5em%
%EndExpansion
Graphs of $f$ and $g$ are shown. Decide whether each function is even, odd,
or neither. Explain your reasoning.}
\end{instructions}

\begin{ExerciseList}
\item[$\hfill $61.] \ \FRAME{itbpF}{1.222in}{1.222in}{1.0542in}{}{}{%
5et0101x57.wmf}{\special{language "Scientific Word";type
"GRAPHIC";maintain-aspect-ratio TRUE;display "USEDEF";valid_file "F";width
1.222in;height 1.222in;depth 1.0542in;original-width 1.222in;original-height
1.222in;cropleft "0";croptop "1";cropright "1";cropbottom "0";filename
'graphics/5et0101x57.wmf';file-properties "XNPEU";}}

%TCIMACRO{\hyperref{ANSWER}{}{\textbf{ANSWER:} $f$ is odd, $g$ is even}{}}%
%BeginExpansion
\msihyperref{ANSWER}{}{\textbf{ANSWER:} $f$ is odd, $g$ is even}{}%
%EndExpansion

%TCIMACRO{%
%\hyperref{\fbox{\textbf{master 00057}}}{}{\fbox{\textbf{master 00057}}}{}}%
%BeginExpansion
\msihyperref{\fbox{\textbf{master 00057}}}{}{\fbox{\textbf{master 00057}}}{}%
%EndExpansion

\item[$\hfill $62.] \ \FRAME{itbpF}{1.2358in}{1.222in}{1.0542in}{}{}{%
5et0101x58.wmf}{\special{language "Scientific Word";type
"GRAPHIC";maintain-aspect-ratio TRUE;display "USEDEF";valid_file "F";width
1.2358in;height 1.222in;depth 1.0542in;original-width
1.2358in;original-height 1.222in;cropleft "0";croptop "1";cropright
"1";cropbottom "0";filename 'graphics/5et0101x58.wmf';file-properties
"XNPEU";}}

%TCIMACRO{%
%\hyperref{\fbox{\textbf{master 00058}}}{}{\fbox{\textbf{master 00058}}}{}}%
%BeginExpansion
\msihyperref{\fbox{\textbf{master 00058}}}{}{\fbox{\textbf{master 00058}}}{}%
%EndExpansion
\end{ExerciseList}

\begin{instructions}
\FRAME{itbpF}{235.6875pt}{5.5pt}{0pt}{}{}{dots.wmf}{\special{language
"Scientific Word";type "GRAPHIC";maintain-aspect-ratio TRUE;display
"USEDEF";valid_file "F";width 235.6875pt;height 5.5pt;depth
0pt;original-width 3.9167in;original-height 0.0735in;cropleft "0";croptop
"0.9772";cropright "0.8327";cropbottom "0.0226";filename
'graphics/dots.wmf';file-properties "XNPEU";}}
\end{instructions}

\begin{ExerciseList}
\item[$\hfill $63.] 

\begin{ExerciseList}
\item[(a)] If the point $(5,3)$ is on the graph of an even function, what
other point must also be on the graph?

%TCIMACRO{%
%\hyperref{\fbox{\textbf{master 00059a}}}{}{\fbox{\textbf{master 00059a}}}{}}%
%BeginExpansion
\msihyperref{\fbox{\textbf{master 00059a}}}{}{\fbox{\textbf{master 00059a}}}{}%
%EndExpansion

\item[(b)] If the point $(5,3)$ is on the graph of an odd function, what
other point must also be on the graph?

%TCIMACRO{%
%\hyperref{\fbox{\textbf{master 00059b}}}{}{\fbox{\textbf{master 00059b}}}{}}%
%BeginExpansion
\msihyperref{\fbox{\textbf{master 00059b}}}{}{\fbox{\textbf{master 00059b}}}{}%
%EndExpansion
\end{ExerciseList}

%TCIMACRO{%
%\hyperref{ANSWER}{}{\textbf{ANSWER:}\ (a)~$\left( -5,3\right) $\quad (b)~$\left( -5,-3\right) $}{}}%
%BeginExpansion
\msihyperref{ANSWER}{}{\textbf{ANSWER:}\ (a)~$\left( -5,3\right) $\quad (b)~$\left( -5,-3\right) $}{}%
%EndExpansion
{}

\item[$\hfill $64.] A function $f$ has domain $[-5,5]$ and a portion of its
graph is shown.

\begin{ExerciseList}
\item[(a)] Complete the graph of $f$ if it is known that $f$ is even.

%TCIMACRO{%
%\hyperref{\fbox{\textbf{master 00060a}}}{}{\fbox{\textbf{master 00060a}}}{}}%
%BeginExpansion
\msihyperref{\fbox{\textbf{master 00060a}}}{}{\fbox{\textbf{master 00060a}}}{}%
%EndExpansion

\item[(b)] Complete the graph of $f$ if it is known that $f$ is odd.\FRAME{%
dtbpF}{136.5pt}{75.5pt}{0pt}{}{}{r0101x54.wmf}{\special{language "Scientific
Word";type "GRAPHIC";maintain-aspect-ratio TRUE;display "USEDEF";valid_file
"F";width 136.5pt;height 75.5pt;depth 0pt;original-width
1.8887in;original-height 1.3223in;cropleft "0";croptop "1";cropright
"1";cropbottom "0.2152";filename 'graphics/r0101x54.wmf';file-properties
"XNPEU";}}

%TCIMACRO{%
%\hyperref{\fbox{\textbf{master 00060b}}}{}{\fbox{\textbf{master 00060b}}}{}}%
%BeginExpansion
\msihyperref{\fbox{\textbf{master 00060b}}}{}{\fbox{\textbf{master 00060b}}}{}%
%EndExpansion
\end{ExerciseList}
\end{ExerciseList}

\begin{instructions}
\QTR{SpanExer}{65--70}{\small 
%TCIMACRO{\TeXButton{SQR}{\hskip .5em\rule{4pt}{4pt}\hskip .5em}}%
%BeginExpansion
\hskip .5em\rule{4pt}{4pt}\hskip .5em%
%EndExpansion
Determine whether $f$ is even, odd, or neither. If you have a graphing
calculator, use it to check your answer visually.}
\end{instructions}

\begin{ExerciseList}
\item[$\hfill $65.] $f(x)=\dfrac{x}{x^{2}+1}$

%TCIMACRO{\hyperref{ANSWER}{}{\textbf{ANSWER:} Odd}{}}%
%BeginExpansion
\msihyperref{ANSWER}{}{\textbf{ANSWER:} Odd}{}%
%EndExpansion

%TCIMACRO{%
%\hyperref{\fbox{\textbf{master 30010}}}{}{\fbox{\textbf{master 30010}}}{}}%
%BeginExpansion
\msihyperref{\fbox{\textbf{master 30010}}}{}{\fbox{\textbf{master 30010}}}{}%
%EndExpansion

\item[$\hfill $66.] $f(x)=\dfrac{x^{2}}{x^{4}+1}$

%TCIMACRO{%
%\hyperref{\fbox{\textbf{master 30011}}}{}{\fbox{\textbf{master 30011}}}{}}%
%BeginExpansion
\msihyperref{\fbox{\textbf{master 30011}}}{}{\fbox{\textbf{master 30011}}}{}%
%EndExpansion

\item[$\hfill $67.] $f(x)=\dfrac{x}{x+1}$

%TCIMACRO{\hyperref{ANSWER}{}{\textbf{ANSWER:} Neither}{}}%
%BeginExpansion
\msihyperref{ANSWER}{}{\textbf{ANSWER:} Neither}{}%
%EndExpansion

%TCIMACRO{%
%\hyperref{\fbox{\textbf{master 30012}}}{}{\fbox{\textbf{master 30012}}}{}}%
%BeginExpansion
\msihyperref{\fbox{\textbf{master 30012}}}{}{\fbox{\textbf{master 30012}}}{}%
%EndExpansion

\item[$\hfill $68.] $f(x)=x\left| x\right| $

%TCIMACRO{%
%\hyperref{\fbox{\textbf{master 30013}}}{}{\fbox{\textbf{master 30013}}}{}}%
%BeginExpansion
\msihyperref{\fbox{\textbf{master 30013}}}{}{\fbox{\textbf{master 30013}}}{}%
%EndExpansion

\item[$\hfill $69.] $f(x)=1+3x^{2}-x^{4}$

%TCIMACRO{\hyperref{ANSWER}{}{\textbf{ANSWER:} Even}{}}%
%BeginExpansion
\msihyperref{ANSWER}{}{\textbf{ANSWER:} Even}{}%
%EndExpansion

%TCIMACRO{%
%\hyperref{\fbox{\textbf{master 30014}}}{}{\fbox{\textbf{master 30014}}}{}}%
%BeginExpansion
\msihyperref{\fbox{\textbf{master 30014}}}{}{\fbox{\textbf{master 30014}}}{}%
%EndExpansion

\item[$\hfill $70.] $f(x)=1+3x^{3}-x^{5}$

%TCIMACRO{%
%\hyperref{\fbox{\textbf{master 30015}}}{}{\fbox{\textbf{master 30015}}}{}}%
%BeginExpansion
\msihyperref{\fbox{\textbf{master 30015}}}{}{\fbox{\textbf{master 30015}}}{}%
%EndExpansion
\end{ExerciseList}

\begin{instructions}
\FRAME{itbpF}{235.6875pt}{5.5pt}{0pt}{}{}{dots.wmf}{\special{language
"Scientific Word";type "GRAPHIC";maintain-aspect-ratio TRUE;display
"USEDEF";valid_file "F";width 235.6875pt;height 5.5pt;depth
0pt;original-width 3.9167in;original-height 0.0735in;cropleft "0";croptop
"0.9772";cropright "0.8327";cropbottom "0.0226";filename
'graphics/dots.wmf';file-properties "XNPEU";}}
\end{instructions}

%TCIMACRO{%
%\TeXButton{e2col}{\end{multicols}
%\advance \leftskip by 165pt
%\advance\hsize by -165pt
%\advance\linewidth by -165pt}}%
%BeginExpansion
\end{multicols}
\advance \leftskip by 165pt
\advance\hsize by -165pt
\advance\linewidth by -165pt%
%EndExpansion

\end{document}
