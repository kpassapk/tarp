
\documentclass{sebase}
%%%%%%%%%%%%%%%%%%%%%%%%%%%%%%%%%%%%%%%%%%%%%%%%%%%%%%%%%%%%%%%%%%%%%%%%%%%%%%%%%%%%%%%%%%%%%%%%%%%%%%%%%%%%%%%%%%%%%%%%%%%%%%%%%%%%%%%%%%%%%%%%%%%%%%%%%%%%%%%%%%%%%%%%%%%%%%%%%%%%%%%%%%%%%%%%%%%%%%%%%%%%%%%%%%%%%%%%%%%%%%%%%%%%%%%%%%%%%%%%%%%%%%%%%%%%
\usepackage{amsmath}
\usepackage{float}
\usepackage{makeidx}
\usepackage{SECALCUL}
\usepackage{lie}

\setcounter{MaxMatrixCols}{10}
%TCIDATA{OutputFilter=LATEX.DLL}
%TCIDATA{Version=5.50.0.2953}
%TCIDATA{<META NAME="SaveForMode" CONTENT="1">}
%TCIDATA{BibliographyScheme=Manual}
%TCIDATA{Created=Mon Dec 15 16:20:00 1997}
%TCIDATA{LastRevised=Tuesday, September 22, 2009 17:18:01}
%TCIDATA{<META NAME="ViewSettings" CONTENT="25">}
%TCIDATA{<META NAME="GraphicsSave" CONTENT="32">}
%TCIDATA{CSTFile=SECALCUL.cst}

\input tcilatex
\newenvironment{instructions}{\STARTINSTR}{\ENDINSTR}
\begin{document}


%TCIMACRO{%
%\TeXButton{Int Fix}{\def\dint{\displaystyle \int}\def\diint{\displaystyle \iint}\def\diiint{\displaystyle \iiint}
%\def\tint{\textstyle \int}\def\tiint{\textstyle \iint }\def\tiiint{{\textstyle \iiint }\def\tiiiint{\textstyle \iiiint }}}}%
%BeginExpansion
\def\dint{\displaystyle \int}\def\diint{\displaystyle \iint}\def\diiint{\displaystyle \iiint}
\def\tint{\textstyle \int}\def\tiint{\textstyle \iint }\def\tiiint{{\textstyle \iiint }\def\tiiiint{\textstyle \iiiint }}%
%EndExpansion
%TCIMACRO{\TeXButton{noCCC}{\noCCC}}%
%BeginExpansion
\noCCC%
%EndExpansion
%TCIMACRO{\TeXButton{setET}{\renewcommand{\ET}{1}}}%
%BeginExpansion
\renewcommand{\ET}{1}%
%EndExpansion
%TCIMACRO{\TeXButton{setRM}{\renewcommand{\RM}{1}}}%
%BeginExpansion
\renewcommand{\RM}{1}%
%EndExpansion
%TCIMACRO{\TeXButton{noRM}{\renewcommand{\RM}{0}}}%
%BeginExpansion
\renewcommand{\RM}{0}%
%EndExpansion
%TCIMACRO{\TeXButton{set page 73}{\setcounter{page}{73}}}%
%BeginExpansion
\setcounter{page}{73}%
%EndExpansion

\chapter{%
%TCIMACRO{\TeXButton{8/ 7}{\ifnum\ET=0 8\else 7\fi}}%
%BeginExpansion
\ifnum\ET=0 8\else 7\fi%
%EndExpansion
\quad Techniques of Integration}

\section{%
%TCIMACRO{\TeXButton{8/ 7}{\ifnum\ET=0 8\else 7\fi}}%
%BeginExpansion
\ifnum\ET=0 8\else 7\fi%
%EndExpansion
\quad Review\protect\vspace*{-21pt}}

\QTP{MultColDiv}
CONCEPT CHECK\vspace{-18pt}

%TCIMACRO{\TeXButton{graphicSr}{\vspace{12pt}\hskip-170pt\hfil}}%
%BeginExpansion
\vspace{12pt}\hskip-170pt\hfil%
%EndExpansion
\vspace{-6pt}$\rule{7.15in}{0.02in}$%
%TCIMACRO{\TeXButton{graphicE}{\vspace{12pt}\hfil}}%
%BeginExpansion
\vspace{12pt}\hfil%
%EndExpansion

\vspace*{-9pt}%
%TCIMACRO{%
%\TeXButton{s2col}{\setlength{\columnsep}{24pt}
%\advance \leftskip by -165pt
%\advance\hsize by 165pt
%\advance\linewidth by 165pt
%\begin{multicols}{2}}}%
%BeginExpansion
\setlength{\columnsep}{24pt}
\advance \leftskip by -165pt
\advance\hsize by 165pt
\advance\linewidth by 165pt
\begin{multicols}{2}%
%EndExpansion

\begin{ExerciseList}
\item[\hfill 1.] State the rule for integration by parts. In practice, how
do you use it?

%TCIMACRO{%
%\hyperref{\fbox{\textbf{master 03890}}}{}{\fbox{\textbf{master 03890}}}{}}%
%BeginExpansion
\msihyperref{\fbox{\textbf{master 03890}}}{}{\fbox{\textbf{master 03890}}}{}%
%EndExpansion

\item[\hfill 2.] How do you evaluate $\tint \,\sin ^{m}x\,\cos ^{n}x\,dx$ if 
$m$ is odd? What if $n$ is odd? What if $m$ and $n$ are both even?

%TCIMACRO{%
%\hyperref{\fbox{\textbf{master 03891}}}{}{\fbox{\textbf{master 03891}}}{}}%
%BeginExpansion
\msihyperref{\fbox{\textbf{master 03891}}}{}{\fbox{\textbf{master 03891}}}{}%
%EndExpansion

\item[\hfill 3.] If the expression $\sqrt{a^{2}-x^{2}}$ occurs in an
integral, what substitution might you try? What if $\sqrt{a^{2}+x^{2}}$
occurs? What if $\sqrt{x^{2}-a^{2}}$ occurs?

%TCIMACRO{%
%\hyperref{\fbox{\textbf{master 03892}}}{}{\fbox{\textbf{master 03892}}}{}}%
%BeginExpansion
\msihyperref{\fbox{\textbf{master 03892}}}{}{\fbox{\textbf{master 03892}}}{}%
%EndExpansion

\item[\hfill 4.] What is the form of the partial fraction expansion of a
rational function $P\left( x\right) /Q\left( x\right) $ if the degree of $P$
is less than the degree of $Q$ and $Q\left( x\right) $ has only distinct
linear factors? What if a linear factor is repeated? What if $Q\left(
x\right) $ has an irreducible quadratic factor (not repeated)? What if the
quadratic factor is repeated?

%TCIMACRO{%
%\hyperref{\fbox{\textbf{master 03893}}}{}{\fbox{\textbf{master 03893}}}{}}%
%BeginExpansion
\msihyperref{\fbox{\textbf{master 03893}}}{}{\fbox{\textbf{master 03893}}}{}%
%EndExpansion

\item[\hfill 5.] State the rules for approximating the definite integral 
\newline
$\int_{a}^{b}\,\,f\left( x\right) \,dx$ with the Midpoint Rule, the
Trapezoidal Rule, and Simpson's Rule. Which would you expect to give the
best estimate? How do you approximate the error for each rule?

%TCIMACRO{%
%\hyperref{\fbox{\textbf{master 03894}}}{}{\fbox{\textbf{master 03894}}}{}}%
%BeginExpansion
\msihyperref{\fbox{\textbf{master 03894}}}{}{\fbox{\textbf{master 03894}}}{}%
%EndExpansion

\item[\hfill 6.] Define the following improper integrals.

\begin{ExerciseList}
\item[(a)] $\int_{a}^{\infty }\,\,f\left( x\right) \,dx$\hspace{12pt}\textbf{%
(b)}\hspace{3pt}$\int_{-\infty }^{b}\,\,f\left( x\right) \,dx$\hspace{12pt}%
\textbf{(c)}\hspace{3pt}$\int_{-\infty }^{\infty }\,\,f\left( x\right) \,dx$
\end{ExerciseList}

%TCIMACRO{%
%\hyperref{\fbox{\textbf{master 03895 for all 3 parts}}}{}{\fbox{\textbf{master 03895 for all 3 parts}}}{}}%
%BeginExpansion
\msihyperref{\fbox{\textbf{master 03895 for all 3 parts}}}{}{\fbox{\textbf{master 03895 for all 3 parts}}}{}%
%EndExpansion

\item[\hfill 7.] Define the improper integral $\int_{a}^{b}\,\,f\left(
x\right) \,dx$ for each of the following cases.

\begin{ExerciseList}
\item[(a)] $f$ has an infinite discontinuity at $a$.

\item[(b)] $f$ has an infinte discontinuity at $b$.

\item[(c)] $f$ has an infinite discontinuity at $c$, where $a<c<b$.
\end{ExerciseList}

%TCIMACRO{%
%\hyperref{\fbox{\textbf{master 03896 for all 3 parts}}}{}{\fbox{\textbf{master 03896 for all 3 parts}}}{}}%
%BeginExpansion
\msihyperref{\fbox{\textbf{master 03896 for all 3 parts}}}{}{\fbox{\textbf{master 03896 for all 3 parts}}}{}%
%EndExpansion

\item[\hfill 8.] State the Comparison Theorem for improper integrals.

%TCIMACRO{%
%\hyperref{\fbox{\textbf{master 03897}}}{}{\fbox{\textbf{master 03897}}}{}}%
%BeginExpansion
\msihyperref{\fbox{\textbf{master 03897}}}{}{\fbox{\textbf{master 03897}}}{}%
%EndExpansion
\end{ExerciseList}

%TCIMACRO{%
%\TeXButton{e2col}{\end{multicols}
%\advance \leftskip by 165pt
%\advance\hsize by -165pt
%\advance\linewidth by -165pt}}%
%BeginExpansion
\end{multicols}
\advance \leftskip by 165pt
\advance\hsize by -165pt
\advance\linewidth by -165pt%
%EndExpansion
\vspace*{-9pt}

\QTP{MultColDiv}
TRUE-FALSE QUIZ\vspace{-18pt}

%TCIMACRO{\TeXButton{graphicSr}{\vspace{12pt}\hskip-170pt\hfil}}%
%BeginExpansion
\vspace{12pt}\hskip-170pt\hfil%
%EndExpansion
\vspace{-6pt}$\rule{7.15in}{0.02in}$%
%TCIMACRO{\TeXButton{graphicE}{\vspace{12pt}\hfil}}%
%BeginExpansion
\vspace{12pt}\hfil%
%EndExpansion

\vspace*{-9pt}%
%TCIMACRO{%
%\TeXButton{s2col}{\setlength{\columnsep}{24pt}
%\advance \leftskip by -165pt
%\advance\hsize by 165pt
%\advance\linewidth by 165pt
%\begin{multicols}{2}}}%
%BeginExpansion
\setlength{\columnsep}{24pt}
\advance \leftskip by -165pt
\advance\hsize by 165pt
\advance\linewidth by 165pt
\begin{multicols}{2}%
%EndExpansion

\begin{instructions}
{\small Determine whether the statement is true or false. If it is true,
explain why. If it is false, explain why or give an example that disproves
the statement.}
\end{instructions}

\begin{ExerciseList}
\item[\hfill 1.] $\dfrac{x(x^{2}+4)}{x^{2}-4}$ can be put in the form $%
\dfrac{A}{x+2}+\dfrac{B}{x-2}$.

%TCIMACRO{\hyperref{ANSWER}{}{\textbf{ANSWER:} False}{}}%
%BeginExpansion
\msihyperref{ANSWER}{}{\textbf{ANSWER:} False}{}%
%EndExpansion

%TCIMACRO{%
%\hyperref{\fbox{\textbf{master 03898}}}{}{\fbox{\textbf{master 03898}}}{}}%
%BeginExpansion
\msihyperref{\fbox{\textbf{master 03898}}}{}{\fbox{\textbf{master 03898}}}{}%
%EndExpansion

\item[\hfill 2.] $\dfrac{x^{2}+4}{x(x^{2}-4)}$ can be put in the form $%
\dfrac{A}{x}+\dfrac{B}{x+2}+\dfrac{C}{x-2}$.

%TCIMACRO{%
%\hyperref{\fbox{\textbf{master 03899}}}{}{\fbox{\textbf{master 03899}}}{}}%
%BeginExpansion
\msihyperref{\fbox{\textbf{master 03899}}}{}{\fbox{\textbf{master 03899}}}{}%
%EndExpansion

\item[\hfill 3.] $\dfrac{x^{2}+4}{x^{2}(x-4)}$ can be put in the form $%
\dfrac{A}{x^{2}}+\dfrac{B}{x-4}$.

%TCIMACRO{\hyperref{ANSWER}{}{\textbf{ANSWER:} False}{}}%
%BeginExpansion
\msihyperref{ANSWER}{}{\textbf{ANSWER:} False}{}%
%EndExpansion

%TCIMACRO{%
%\hyperref{\fbox{\textbf{master 03900}}}{}{\fbox{\textbf{master 03900}}}{}}%
%BeginExpansion
\msihyperref{\fbox{\textbf{master 03900}}}{}{\fbox{\textbf{master 03900}}}{}%
%EndExpansion

\item[\hfill 4.] $\dfrac{x^{2}-4}{x(x^{2}+4)}$ can be put in the form $%
\dfrac{A}{x}+\dfrac{B}{x^{2}+4}$.

%TCIMACRO{%
%\hyperref{\fbox{\textbf{master 03901}}}{}{\fbox{\textbf{master 03901}}}{}}%
%BeginExpansion
\msihyperref{\fbox{\textbf{master 03901}}}{}{\fbox{\textbf{master 03901}}}{}%
%EndExpansion

\item[\hfill 5.] $\dint\nolimits_{0}^{4}\,\,\dfrac{x}{x^{2}-1}\,dx=\tfrac{1}{%
2}\,\ln \,15$

%TCIMACRO{\hyperref{ANSWER}{}{\textbf{ANSWER:} False}{}}%
%BeginExpansion
\msihyperref{ANSWER}{}{\textbf{ANSWER:} False}{}%
%EndExpansion

%TCIMACRO{%
%\hyperref{\fbox{\textbf{master 03902}}}{}{\fbox{\textbf{master 03902}}}{}}%
%BeginExpansion
\msihyperref{\fbox{\textbf{master 03902}}}{}{\fbox{\textbf{master 03902}}}{}%
%EndExpansion

\item[\hfill 6.] $\dint\nolimits_{1}^{\infty }\,\,\dfrac{1}{x^{\sqrt{2}}}%
\,dx $ is convergent.

%TCIMACRO{%
%\hyperref{\fbox{\textbf{master 03903}}}{}{\fbox{\textbf{master 03903}}}{}}%
%BeginExpansion
\msihyperref{\fbox{\textbf{master 03903}}}{}{\fbox{\textbf{master 03903}}}{}%
%EndExpansion

\item[\hfill 7.] If $f$ is continuous, then $\tint_{-\infty }^{\infty
}\,\,f(x)\,dx=\lim\nolimits_{t\rightarrow \infty
}\,\,\tint_{-t}^{t}\,\,f(x)\,dx$.

%TCIMACRO{\hyperref{ANSWER}{}{\textbf{ANSWER:} False}{}}%
%BeginExpansion
\msihyperref{ANSWER}{}{\textbf{ANSWER:} False}{}%
%EndExpansion

%TCIMACRO{%
%\hyperref{\fbox{\textbf{master 03904}}}{}{\fbox{\textbf{master 03904}}}{}}%
%BeginExpansion
\msihyperref{\fbox{\textbf{master 03904}}}{}{\fbox{\textbf{master 03904}}}{}%
%EndExpansion

\item[\hfill 8.] The Midpoint Rule is always more accurate than the \newline
Trapezoidal Rule.

%TCIMACRO{%
%\hyperref{\fbox{\textbf{master 03905}}}{}{\fbox{\textbf{master 03905}}}{}}%
%BeginExpansion
\msihyperref{\fbox{\textbf{master 03905}}}{}{\fbox{\textbf{master 03905}}}{}%
%EndExpansion

\item[\hfill 9.] 

\begin{ExerciseList}
\item[(a)] Every elementary function has an elementary derivative.

%TCIMACRO{\hyperref{ANSWER}{}{\textbf{ANSWER: }True}{}}%
%BeginExpansion
\msihyperref{ANSWER}{}{\textbf{ANSWER: }True}{}%
%EndExpansion

%TCIMACRO{%
%\hyperref{\fbox{\textbf{master 03906}}}{}{\fbox{\textbf{master 03906}}}{}}%
%BeginExpansion
\msihyperref{\fbox{\textbf{master 03906}}}{}{\fbox{\textbf{master 03906}}}{}%
%EndExpansion

\item[(b)] Every elementary function has an elementary anti-\newline
derivative.

%TCIMACRO{\hyperref{ANSWER}{}{\textbf{ANSWER} False}{}}%
%BeginExpansion
\msihyperref{ANSWER}{}{\textbf{ANSWER} False}{}%
%EndExpansion

%TCIMACRO{%
%\hyperref{\fbox{\textbf{master 40578}}}{}{\fbox{\textbf{master 40578}}}{}}%
%BeginExpansion
\msihyperref{\fbox{\textbf{master 40578}}}{}{\fbox{\textbf{master 40578}}}{}%
%EndExpansion
\end{ExerciseList}

\item[\hfill 10.] If $f$ is continuous on $[0,\infty )$ and $%
\int_{1}^{\infty }f\left( x\right) \,dx$ is convergent, then $%
\int_{0}^{\infty }f\left( x\right) \,dx$ is convergent.

%TCIMACRO{%
%\hyperref{\fbox{\textbf{master 03907}}}{}{\fbox{\textbf{master 03907}}}{}}%
%BeginExpansion
\msihyperref{\fbox{\textbf{master 03907}}}{}{\fbox{\textbf{master 03907}}}{}%
%EndExpansion

\item[\hfill 11.] If $f$ is a continuous, decreasing function on $[1,\infty
) $ and

$\lim_{x\rightarrow \infty }f(x)=0$, then $\int_{1}^{\infty }f\left(
x\right) \,dx$ is convergent.

%TCIMACRO{\hyperref{ANSWER}{}{\textbf{ANSWER: }False}{}}%
%BeginExpansion
\msihyperref{ANSWER}{}{\textbf{ANSWER: }False}{}%
%EndExpansion

%TCIMACRO{%
%\hyperref{\fbox{\textbf{master 03908}}}{}{\fbox{\textbf{master 03908}}}{}}%
%BeginExpansion
\msihyperref{\fbox{\textbf{master 03908}}}{}{\fbox{\textbf{master 03908}}}{}%
%EndExpansion

\item[\hfill 12.] If $\int_{a}^{\infty }f\left( x\right) \,dx$ and $%
\int_{a}^{\infty }g\left( x\right) \,dx$ are both convergent, then $%
\int_{a}^{\infty }[f\left( x\right) +g(x)]\,\,dx$ is convergent.

%TCIMACRO{%
%\hyperref{\fbox{\textbf{master 03909}}}{}{\fbox{\textbf{master 03909}}}{}}%
%BeginExpansion
\msihyperref{\fbox{\textbf{master 03909}}}{}{\fbox{\textbf{master 03909}}}{}%
%EndExpansion

\item[\hfill 13.] If $\int_{a}^{\infty }f\left( x\right) \,dx$ and $%
\int_{a}^{\infty }g\left( x\right) \,dx$ are both divergent, then

$\int_{a}^{\infty }[f\left( x\right) +g(x)]\,\,dx$ is divergent.

%TCIMACRO{\hyperref{ANSWER}{}{\textbf{ANSWER: }False}{}}%
%BeginExpansion
\msihyperref{ANSWER}{}{\textbf{ANSWER: }False}{}%
%EndExpansion

%TCIMACRO{%
%\hyperref{\fbox{\textbf{master 03910}}}{}{\fbox{\textbf{master 03910}}}{}}%
%BeginExpansion
\msihyperref{\fbox{\textbf{master 03910}}}{}{\fbox{\textbf{master 03910}}}{}%
%EndExpansion

\item[\hfill 14.] If $f\left( x\right) \leq g\left( x\right) $ and $%
\int_{0}^{\infty }g\left( x\right) \,dx$ diverges, then $\int_{0}^{\infty
}f\left( x\right) \,dx$ also diverges.

%TCIMACRO{%
%\hyperref{\fbox{\textbf{master 03911}}}{}{\fbox{\textbf{master 03911}}}{}}%
%BeginExpansion
\msihyperref{\fbox{\textbf{master 03911}}}{}{\fbox{\textbf{master 03911}}}{}%
%EndExpansion
\end{ExerciseList}

%TCIMACRO{%
%\TeXButton{e2col}{\end{multicols}
%\advance \leftskip by 165pt
%\advance\hsize by -165pt
%\advance\linewidth by -165pt}}%
%BeginExpansion
\end{multicols}
\advance \leftskip by 165pt
\advance\hsize by -165pt
\advance\linewidth by -165pt%
%EndExpansion
\vspace*{-9pt}

\QTP{MultColDiv}
EXERCISES\vspace{-18pt}

%TCIMACRO{\TeXButton{graphicSr}{\vspace{12pt}\hskip-170pt\hfil}}%
%BeginExpansion
\vspace{12pt}\hskip-170pt\hfil%
%EndExpansion
\vspace{-6pt}$\rule{7.15in}{0.02in}$%
%TCIMACRO{\TeXButton{graphicE}{\vspace{12pt}\hfil}}%
%BeginExpansion
\vspace{12pt}\hfil%
%EndExpansion

\vspace*{-9pt}%
%TCIMACRO{%
%\TeXButton{s2col}{\setlength{\columnsep}{24pt}
%\advance \leftskip by -165pt
%\advance\hsize by 165pt
%\advance\linewidth by 165pt
%\begin{multicols}{2}}}%
%BeginExpansion
\setlength{\columnsep}{24pt}
\advance \leftskip by -165pt
\advance\hsize by 165pt
\advance\linewidth by 165pt
\begin{multicols}{2}%
%EndExpansion

\begin{instructions}
\QTR{PartLetter}{Note:}{\small \quad Additional practice in techniques of
integration is provided in Exercises 
%TCIMACRO{\TeXButton{8.5/ 7.5}{\ifnum\ET=0 8.5\else 7.5\fi}}%
%BeginExpansion
\ifnum\ET=0 8.5\else 7.5\fi%
%EndExpansion
.}\vspace{-12pt}
\end{instructions}

\begin{instructions}
\QTR{SpanExer}{1--40}{\small 
%TCIMACRO{\TeXButton{SQR}{\hskip .5em\rule{4pt}{4pt}\hskip .5em}}%
%BeginExpansion
\hskip .5em\rule{4pt}{4pt}\hskip .5em%
%EndExpansion
Evaluate the integral.}
\end{instructions}

\begin{ExerciseList}
\item[\hfill 1.] $\dint\nolimits_{1}^{2}\,\,\dfrac{(x+1)^{2}}{x}\,dx$

%TCIMACRO{\hyperref{\fbox{\textbf{NEW}}}{}{\fbox{\textbf{NEW}}}{}}%
%BeginExpansion
\msihyperref{\fbox{\textbf{NEW}}}{}{\fbox{\textbf{NEW}}}{}%
%EndExpansion

\item[\hfill 2.] $\dint\nolimits_{1}^{2}\,\dfrac{x}{(x+1)^{2}}\,dx$

%TCIMACRO{\hyperref{\fbox{\textbf{NEW}}}{}{\fbox{\textbf{NEW}}}{}}%
%BeginExpansion
\msihyperref{\fbox{\textbf{NEW}}}{}{\fbox{\textbf{NEW}}}{}%
%EndExpansion

\item[\hfill 3.] $\dint\nolimits_{0}^{\pi /2}\,\,\sin \theta ~e^{\cos \theta
}\,d\theta $

%TCIMACRO{\hyperref{\fbox{\textbf{NEW}}}{}{\fbox{\textbf{NEW}}}{}}%
%BeginExpansion
\msihyperref{\fbox{\textbf{NEW}}}{}{\fbox{\textbf{NEW}}}{}%
%EndExpansion

\item[\hfill 4.] $\dint\nolimits_{1}^{\pi /6}\,\,t\sin 2t~dt$

%TCIMACRO{\hyperref{\fbox{\textbf{NEW}}}{}{\fbox{\textbf{NEW}}}{}}%
%BeginExpansion
\msihyperref{\fbox{\textbf{NEW}}}{}{\fbox{\textbf{NEW}}}{}%
%EndExpansion

\item[\hfill 5.] $\dint \,\,\dfrac{dt}{2t^{2}+3t+1}$

%TCIMACRO{\hyperref{\fbox{\textbf{NEW}}}{}{\fbox{\textbf{NEW}}}{}}%
%BeginExpansion
\msihyperref{\fbox{\textbf{NEW}}}{}{\fbox{\textbf{NEW}}}{}%
%EndExpansion

\item[\hfill 6.] $\dint\nolimits_{1}^{2}\,\,x^{5}\ln x\,dx$

%TCIMACRO{\hyperref{\fbox{\textbf{NEW}}}{}{\fbox{\textbf{NEW}}}{}}%
%BeginExpansion
\msihyperref{\fbox{\textbf{NEW}}}{}{\fbox{\textbf{NEW}}}{}%
%EndExpansion

\item[\hfill 7.] $\dint_{0}^{\pi /2}\,\sin ^{3}\theta ~\cos ^{2}\theta
\,d\theta $

%TCIMACRO{\hyperref{ANSWER}{}{\textbf{ANSWER: }$\frac{2}{15}$}{}}%
%BeginExpansion
\msihyperref{ANSWER}{}{\textbf{ANSWER: }$\frac{2}{15}$}{}%
%EndExpansion

%TCIMACRO{%
%\hyperref{\fbox{\textbf{master 80770}}}{}{\fbox{\textbf{master 80770}}}{}}%
%BeginExpansion
\msihyperref{\fbox{\textbf{master 80770}}}{}{\fbox{\textbf{master 80770}}}{}%
%EndExpansion

\item[\hfill 8.] $\dint \,\,\dfrac{dx}{\sqrt{e^{x}-1}}$

%TCIMACRO{%
%\hyperref{\fbox{\textbf{master 80771}}}{}{\fbox{\textbf{master 80771}}}{}}%
%BeginExpansion
\msihyperref{\fbox{\textbf{master 80771}}}{}{\fbox{\textbf{master 80771}}}{}%
%EndExpansion

\item[\hfill 9.] $\dint \,\,\dfrac{\sin (\ln t)}{t}\,dt$

%TCIMACRO{\hyperref{ANSWER}{}{\textbf{ANSWER: }$-\cos (\ln t)+C$}{}}%
%BeginExpansion
\msihyperref{ANSWER}{}{\textbf{ANSWER: }$-\cos (\ln t)+C$}{}%
%EndExpansion

%TCIMACRO{%
%\hyperref{\fbox{\textbf{master 03918}}}{}{\fbox{\textbf{master 03918}}}{}}%
%BeginExpansion
\msihyperref{\fbox{\textbf{master 03918}}}{}{\fbox{\textbf{master 03918}}}{}%
%EndExpansion

\item[\hfill 10.] $\dint\nolimits_{0}^{1}\,\,\dfrac{\sqrt{\arctan x}}{1+x^{2}%
}\,dx$

%TCIMACRO{%
%\hyperref{\fbox{\textbf{master 03921}}}{}{\fbox{\textbf{master 03921}}}{}}%
%BeginExpansion
\msihyperref{\fbox{\textbf{master 03921}}}{}{\fbox{\textbf{master 03921}}}{}%
%EndExpansion

\item[\hfill 11.] $\dint\nolimits_{1}^{2}\,\,\dfrac{\sqrt{x^{2}-1}}{x}\,dx$

%TCIMACRO{%
%\hyperref{ANSWER}{}{\textbf{ANSWER:} $\sqrt{3}-\left( \pi /3\right) $}{}}%
%BeginExpansion
\msihyperref{ANSWER}{}{\textbf{ANSWER:} $\sqrt{3}-\left( \pi /3\right) $}{}%
%EndExpansion

%TCIMACRO{%
%\hyperref{\fbox{\textbf{master 03922}}}{}{\fbox{\textbf{master 03922}}}{}}%
%BeginExpansion
\msihyperref{\fbox{\textbf{master 03922}}}{}{\fbox{\textbf{master 03922}}}{}%
%EndExpansion

\item[\hfill 12.] $\dint_{{}}^{{}}\,\dfrac{e^{2x}}{1+e^{4x}}\,dx$

%TCIMACRO{\hyperref{\fbox{\textbf{NEW}}}{}{\fbox{\textbf{NEW}}}{}}%
%BeginExpansion
\msihyperref{\fbox{\textbf{NEW}}}{}{\fbox{\textbf{NEW}}}{}%
%EndExpansion

\item[\hfill 13.] $\dint \,\,e^{\sqrt[3]{x}}~dx$

%TCIMACRO{%
%\hyperref{ANSWER}{}{\textbf{ANSWER:} $3e^{\sqrt[3]{x}}(x^{2/3}-2x^{1/3}+2)+C$}{}}%
%BeginExpansion
\msihyperref{ANSWER}{}{\textbf{ANSWER:} $3e^{\sqrt[3]{x}}(x^{2/3}-2x^{1/3}+2)+C$}{}%
%EndExpansion

%TCIMACRO{%
%\hyperref{\fbox{\textbf{master 30286}}}{}{\fbox{\textbf{master 30286}}}{}}%
%BeginExpansion
\msihyperref{\fbox{\textbf{master 30286}}}{}{\fbox{\textbf{master 30286}}}{}%
%EndExpansion

\item[\hfill 14.] $\dint \,\,\dfrac{x^{2}+2}{x+2}\,dx$

%TCIMACRO{%
%\hyperref{\fbox{\textbf{master 03925}}}{}{\fbox{\textbf{master 03925}}}{}}%
%BeginExpansion
\msihyperref{\fbox{\textbf{master 03925}}}{}{\fbox{\textbf{master 03925}}}{}%
%EndExpansion

\item[\hfill 15.] $\dint \,\,\dfrac{x-1}{x^{2}+2x}\,dx$

%TCIMACRO{%
%\hyperref{ANSWER}{}{\textbf{ANSWER:} $-\frac{1}{2}\ln \left\vert x\right\vert +\frac{3}{2}\ln \left\vert x+2\right\vert +C$}{}}%
%BeginExpansion
\msihyperref{ANSWER}{}{\textbf{ANSWER:} $-\frac{1}{2}\ln \left\vert x\right\vert +\frac{3}{2}\ln \left\vert x+2\right\vert +C$}{}%
%EndExpansion

%TCIMACRO{%
%\hyperref{\fbox{\textbf{master 80773}}}{}{\fbox{\textbf{master 80773}}}{}}%
%BeginExpansion
\msihyperref{\fbox{\textbf{master 80773}}}{}{\fbox{\textbf{master 80773}}}{}%
%EndExpansion

\item[\hfill 16.] $\dint \,\,\dfrac{\sec ^{6}\theta }{\tan ^{2}\theta }%
\,d\theta $

%TCIMACRO{%
%\hyperref{\fbox{\textbf{master 03927}}}{}{\fbox{\textbf{master 03927}}}{}}%
%BeginExpansion
\msihyperref{\fbox{\textbf{master 03927}}}{}{\fbox{\textbf{master 03927}}}{}%
%EndExpansion

\item[\hfill 17.] $\dint \,\,x\,\sec x\,\tan x\,dx$

%TCIMACRO{%
%\hyperref{ANSWER}{}{\textbf{ANSWER:} $x\sec x-\ln \,\left\vert \sec x+\tan x\right\vert +C$}{}}%
%BeginExpansion
\msihyperref{ANSWER}{}{\textbf{ANSWER:} $x\sec x-\ln \,\left\vert \sec x+\tan x\right\vert +C$}{}%
%EndExpansion

%TCIMACRO{%
%\hyperref{\fbox{\textbf{master 03928}}}{}{\fbox{\textbf{master 03928}}}{}}%
%BeginExpansion
\msihyperref{\fbox{\textbf{master 03928}}}{}{\fbox{\textbf{master 03928}}}{}%
%EndExpansion

\item[\hfill 18.] $\dint \,\,\dfrac{x^{2}+8x-3}{x^{3}+3x^{2}}\,dx$

%TCIMACRO{%
%\hyperref{\fbox{\textbf{master 03929}}}{}{\fbox{\textbf{master 03929}}}{}}%
%BeginExpansion
\msihyperref{\fbox{\textbf{master 03929}}}{}{\fbox{\textbf{master 03929}}}{}%
%EndExpansion

\item[\hfill 19.] $\dint \,\,\dfrac{x+1}{9x^{2}+6x+5}\,dx$

%TCIMACRO{%
%\hyperref{ANSWER}{}{\textbf{ANSWER: }$\tfrac{1}{18}\ln (9x^{2}+6x+5)+\tfrac{1}{9}\tan ^{-1}\left( \tfrac{1}{2}(3x+1)\right) +C$}{}}%
%BeginExpansion
\msihyperref{ANSWER}{}{\textbf{ANSWER: }$\tfrac{1}{18}\ln (9x^{2}+6x+5)+\tfrac{1}{9}\tan ^{-1}\left( \tfrac{1}{2}(3x+1)\right) +C$}{}%
%EndExpansion

%TCIMACRO{%
%\hyperref{\fbox{\textbf{master 03930}}}{}{\fbox{\textbf{master 03930}}}{}}%
%BeginExpansion
\msihyperref{\fbox{\textbf{master 03930}}}{}{\fbox{\textbf{master 03930}}}{}%
%EndExpansion

\item[\hfill 20.] $\dint \,\,\tan ^{5}\theta ~\sec ^{3}\theta ~d\theta $

%TCIMACRO{%
%\hyperref{\fbox{\textbf{master 80774}}}{}{\fbox{\textbf{master 80774}}}{}}%
%BeginExpansion
\msihyperref{\fbox{\textbf{master 80774}}}{}{\fbox{\textbf{master 80774}}}{}%
%EndExpansion

\item[\hfill 21.] $\dint \,\,\dfrac{dx}{\sqrt{x^{2}-4x}}$

%TCIMACRO{%
%\hyperref{ANSWER}{}{\textbf{ANSWER:} $\ln \,\left\vert x-2+\sqrt{x^{2}-4x}\right\vert +C$}{}}%
%BeginExpansion
\msihyperref{ANSWER}{}{\textbf{ANSWER:} $\ln \,\left\vert x-2+\sqrt{x^{2}-4x}\right\vert +C$}{}%
%EndExpansion

%TCIMACRO{%
%\hyperref{\fbox{\textbf{master 03932}}}{}{\fbox{\textbf{master 03932}}}{}}%
%BeginExpansion
\msihyperref{\fbox{\textbf{master 03932}}}{}{\fbox{\textbf{master 03932}}}{}%
%EndExpansion

\item[\hfill 22.] $\dint \,\,te^{\sqrt{t}}\,dt$

%TCIMACRO{%
%\hyperref{\fbox{\textbf{master 80775}}}{}{\fbox{\textbf{master 80775}}}{}}%
%BeginExpansion
\msihyperref{\fbox{\textbf{master 80775}}}{}{\fbox{\textbf{master 80775}}}{}%
%EndExpansion

\item[\hfill 23.] $\dint \,\,\dfrac{dx}{x\sqrt{x^{2}+1}}\,$

%TCIMACRO{%
%\hyperref{ANSWER}{}{\textbf{ANSWER:} $\ln \left\vert \dfrac{\sqrt{x^{2}+1}-1}{x}\right\vert +C$}{}}%
%BeginExpansion
\msihyperref{ANSWER}{}{\textbf{ANSWER:} $\ln \left\vert \dfrac{\sqrt{x^{2}+1}-1}{x}\right\vert +C$}{}%
%EndExpansion

%TCIMACRO{%
%\hyperref{\fbox{\textbf{master 80776}}}{}{\fbox{\textbf{master 80776}}}{}}%
%BeginExpansion
\msihyperref{\fbox{\textbf{master 80776}}}{}{\fbox{\textbf{master 80776}}}{}%
%EndExpansion

\item[\hfill 24.] $\dint \,\,e^{x}\,\cos x\,dx$

%TCIMACRO{%
%\hyperref{\fbox{\textbf{master 03935}}}{}{\fbox{\textbf{master 03935}}}{}}%
%BeginExpansion
\msihyperref{\fbox{\textbf{master 03935}}}{}{\fbox{\textbf{master 03935}}}{}%
%EndExpansion

\item[\hfill 25.] $\dint \,\,\dfrac{3x^{3}-x^{2}+6x-4}{(x^{2}+1)(x^{2}+2)}%
\,dx$

%TCIMACRO{%
%\hyperref{ANSWER}{}{\textbf{ANSWER:} \newline
%$\tfrac{3}{2}\ln \left( x^{2}+1\right) -3\tan ^{-1}x+\sqrt{2}\tan ^{-1}\left( x/\sqrt{2}\right) +C$}{}}%
%BeginExpansion
\msihyperref{ANSWER}{}{\textbf{ANSWER:} \newline
$\tfrac{3}{2}\ln \left( x^{2}+1\right) -3\tan ^{-1}x+\sqrt{2}\tan ^{-1}\left( x/\sqrt{2}\right) +C$}{}%
%EndExpansion

%TCIMACRO{%
%\hyperref{\fbox{\textbf{master 03936}}}{}{\fbox{\textbf{master 03936}}}{}}%
%BeginExpansion
\msihyperref{\fbox{\textbf{master 03936}}}{}{\fbox{\textbf{master 03936}}}{}%
%EndExpansion

\item[\hfill 26.] $\dint \,\,x~\sin x~\cos x~dx$

%TCIMACRO{%
%\hyperref{\fbox{\textbf{master 80777}}}{}{\fbox{\textbf{master 80777}}}{}}%
%BeginExpansion
\msihyperref{\fbox{\textbf{master 80777}}}{}{\fbox{\textbf{master 80777}}}{}%
%EndExpansion

\item[\hfill 27.] $\dint\nolimits_{0}^{\pi /2}\,\,\cos ^{3}x\,\sin \,2x\,dx$

%TCIMACRO{\hyperref{ANSWER}{}{\textbf{ANSWER:} $\tfrac{2}{5}$}{}}%
%BeginExpansion
\msihyperref{ANSWER}{}{\textbf{ANSWER:} $\tfrac{2}{5}$}{}%
%EndExpansion

%TCIMACRO{%
%\hyperref{\fbox{\textbf{master 03938}}}{}{\fbox{\textbf{master 03938}}}{}}%
%BeginExpansion
\msihyperref{\fbox{\textbf{master 03938}}}{}{\fbox{\textbf{master 03938}}}{}%
%EndExpansion

\item[\hfill 28.] $\dint \,\,\dfrac{\sqrt[3]{x}+1}{\sqrt[3]{x}-1}\,dx$

%TCIMACRO{%
%\hyperref{\fbox{\textbf{master 03939}}}{}{\fbox{\textbf{master 03939}}}{}}%
%BeginExpansion
\msihyperref{\fbox{\textbf{master 03939}}}{}{\fbox{\textbf{master 03939}}}{}%
%EndExpansion

\item[\hfill 29.] $\dint\nolimits_{-3}^{3}\,\,\dfrac{x}{1+\left\vert
x\right\vert }\,dx$

%TCIMACRO{\hyperref{\fbox{\textbf{NEW}}}{}{\fbox{\textbf{NEW}}}{}}%
%BeginExpansion
\msihyperref{\fbox{\textbf{NEW}}}{}{\fbox{\textbf{NEW}}}{}%
%EndExpansion

\item[\hfill 30.] $\dint \,\,\dfrac{dx}{e^{x}\sqrt{1-e^{-2x}}}$

%TCIMACRO{%
%\hyperref{\fbox{\textbf{master 03941}}}{}{\fbox{\textbf{master 03941}}}{}}%
%BeginExpansion
\msihyperref{\fbox{\textbf{master 03941}}}{}{\fbox{\textbf{master 03941}}}{}%
%EndExpansion

\item[\hfill 31.] $\dint\nolimits_{0}^{\ln 10}\,\,\dfrac{e^{x}\sqrt{e^{x}-1}%
}{e^{x}+8}\,dx$

%TCIMACRO{\hyperref{ANSWER}{}{\textbf{ANSWER: }$6-3\pi /2$}{}}%
%BeginExpansion
\msihyperref{ANSWER}{}{\textbf{ANSWER: }$6-3\pi /2$}{}%
%EndExpansion

%TCIMACRO{%
%\hyperref{\fbox{\textbf{master 03942}}}{}{\fbox{\textbf{master 03942}}}{}}%
%BeginExpansion
\msihyperref{\fbox{\textbf{master 03942}}}{}{\fbox{\textbf{master 03942}}}{}%
%EndExpansion

\item[\hfill 32.] $\dint\nolimits_{0}^{\pi /4}\,\,\dfrac{x\sin x}{\cos ^{3}x}%
\,dx$

%TCIMACRO{\hyperref{ANSWER}{}{\textbf{ANSWER: New answer to come}}{}}%
%BeginExpansion
\msihyperref{ANSWER}{}{\textbf{ANSWER: New answer to come}}{}%
%EndExpansion

%TCIMACRO{%
%\hyperref{\fbox{\textbf{master 03943}}}{}{\fbox{\textbf{master 03943}}}{}}%
%BeginExpansion
\msihyperref{\fbox{\textbf{master 03943}}}{}{\fbox{\textbf{master 03943}}}{}%
%EndExpansion

\item[\hfill 33.] $\dint \,\,\dfrac{x^{2}}{\left( 4-x^{2}\right) ^{3/2}}\,dx$

%TCIMACRO{%
%\hyperref{ANSWER}{}{\textbf{ANSWER:} $\left( x/\sqrt{4-x^{2}}\right) -\sin ^{-1}\left( x/2\right) +C$}{}}%
%BeginExpansion
\msihyperref{ANSWER}{}{\textbf{ANSWER:} $\left( x/\sqrt{4-x^{2}}\right) -\sin ^{-1}\left( x/2\right) +C$}{}%
%EndExpansion

%TCIMACRO{%
%\hyperref{\fbox{\textbf{master 03944}}}{}{\fbox{\textbf{master 03944}}}{}}%
%BeginExpansion
\msihyperref{\fbox{\textbf{master 03944}}}{}{\fbox{\textbf{master 03944}}}{}%
%EndExpansion

\item[\hfill 34.] $\dint \,\,(\arcsin \,x)^{2}\,dx$

%TCIMACRO{%
%\hyperref{\fbox{\textbf{master 03945}}}{}{\fbox{\textbf{master 03945}}}{}}%
%BeginExpansion
\msihyperref{\fbox{\textbf{master 03945}}}{}{\fbox{\textbf{master 03945}}}{}%
%EndExpansion

\item[\hfill 35.] $\dint \,\,\dfrac{1}{\sqrt{x+x^{3/2}}}\,dx$

%TCIMACRO{\hyperref{ANSWER}{}{\textbf{ANSWER: }$4\sqrt{1+\sqrt{x}}+C$}{}}%
%BeginExpansion
\msihyperref{ANSWER}{}{\textbf{ANSWER: }$4\sqrt{1+\sqrt{x}}+C$}{}%
%EndExpansion

%TCIMACRO{%
%\hyperref{\fbox{\textbf{master 03946}}}{}{\fbox{\textbf{master 03946}}}{}}%
%BeginExpansion
\msihyperref{\fbox{\textbf{master 03946}}}{}{\fbox{\textbf{master 03946}}}{}%
%EndExpansion

\item[\hfill 36.] $\dint \,\,\dfrac{1-\tan \theta }{1+\tan \theta }\,d\theta 
$

%TCIMACRO{%
%\hyperref{\fbox{\textbf{master 03947}}}{}{\fbox{\textbf{master 03947}}}{}}%
%BeginExpansion
\msihyperref{\fbox{\textbf{master 03947}}}{}{\fbox{\textbf{master 03947}}}{}%
%EndExpansion

\item[\hfill 37.] $\dint \,\,(\cos x+\sin x)^{2}\cos \,2x\,dx$

%TCIMACRO{%
%\hyperref{ANSWER}{}{\textbf{ANSWER:} $\tfrac{1}{2}\sin 2x-\tfrac{1}{8}\cos 4x+C$}{}}%
%BeginExpansion
\msihyperref{ANSWER}{}{\textbf{ANSWER:} $\tfrac{1}{2}\sin 2x-\tfrac{1}{8}\cos 4x+C$}{}%
%EndExpansion

%TCIMACRO{%
%\hyperref{\fbox{\textbf{master 03948}}}{}{\fbox{\textbf{master 03948}}}{}}%
%BeginExpansion
\msihyperref{\fbox{\textbf{master 03948}}}{}{\fbox{\textbf{master 03948}}}{}%
%EndExpansion

\item[\hfill 38.] $\dint \,\,\dfrac{2^{\sqrt{x}}}{\sqrt{x}}\,dx$

%TCIMACRO{\hyperref{\fbox{\textbf{NEW}}}{}{\fbox{\textbf{NEW}}}{}}%
%BeginExpansion
\msihyperref{\fbox{\textbf{NEW}}}{}{\fbox{\textbf{NEW}}}{}%
%EndExpansion

\item[\hfill 39.] $\dint\nolimits_{0}^{1/2}\,\,\dfrac{xe^{2x}}{(1+2x)^{2}}%
\,dx$

%TCIMACRO{%
%\hyperref{ANSWER}{}{\textbf{ANSWER: }$\tfrac{1}{8}e-\tfrac{1}{4}$}{}}%
%BeginExpansion
\msihyperref{ANSWER}{}{\textbf{ANSWER: }$\tfrac{1}{8}e-\tfrac{1}{4}$}{}%
%EndExpansion

%TCIMACRO{%
%\hyperref{\fbox{\textbf{master 03950}}}{}{\fbox{\textbf{master 03950}}}{}}%
%BeginExpansion
\msihyperref{\fbox{\textbf{master 03950}}}{}{\fbox{\textbf{master 03950}}}{}%
%EndExpansion

%TCIMACRO{\TeXButton{longpage}{\enlargethispage{\baselineskip}}}%
%BeginExpansion
\enlargethispage{\baselineskip}%
%EndExpansion

\item[\hfill 40.] $\dint\nolimits_{\pi /4}^{\pi /3}\,\,\dfrac{\sqrt{\tan
\theta }}{\sin 2\theta }\,d\theta $

%TCIMACRO{%
%\hyperref{\fbox{\textbf{master 03951}}}{}{\fbox{\textbf{master 03951}}}{}}%
%BeginExpansion
\msihyperref{\fbox{\textbf{master 03951}}}{}{\fbox{\textbf{master 03951}}}{}%
%EndExpansion
\vspace*{-9pt}
\end{ExerciseList}

\begin{instructions}
\FRAME{itbpF}{237.375pt}{6.6875pt}{0pt}{}{}{dots.wmf}{\special{language
"Scientific Word";type "GRAPHIC";maintain-aspect-ratio TRUE;display
"USEDEF";valid_file "F";width 237.375pt;height 6.6875pt;depth
0pt;original-width 3.9167in;original-height 0.0735in;cropleft "0";croptop
"0.9532";cropright "0.8216";cropbottom "0.0468";filename
'graphics/dots.wmf';file-properties "XNPEU";}}
\end{instructions}

\begin{instructions}
\QTR{SpanExer}{41--50}{\small 
%TCIMACRO{\TeXButton{SQR}{\hskip .5em\rule{4pt}{4pt}\hskip .5em}}%
%BeginExpansion
\hskip .5em\rule{4pt}{4pt}\hskip .5em%
%EndExpansion
Evaluate the integral or show that it is divergent.}
\end{instructions}

\begin{ExerciseList}
\item[\hfill 41.] $\dint\nolimits_{1}^{\infty }\,\,\dfrac{1}{(2x+1)^{3}}\,dx$

%TCIMACRO{\hyperref{ANSWER}{}{\textbf{ANSWER:} $\tfrac{1}{36}$}{}}%
%BeginExpansion
\msihyperref{ANSWER}{}{\textbf{ANSWER:} $\tfrac{1}{36}$}{}%
%EndExpansion

%TCIMACRO{%
%\hyperref{\fbox{\textbf{master 03952}}}{}{\fbox{\textbf{master 03952}}}{}}%
%BeginExpansion
\msihyperref{\fbox{\textbf{master 03952}}}{}{\fbox{\textbf{master 03952}}}{}%
%EndExpansion

\item[\hfill 42.] $\dint\nolimits_{1}^{\infty }\,\,\dfrac{\ln x}{x^{4}}\,dx$

%TCIMACRO{%
%\hyperref{\fbox{\textbf{master 80779}}}{}{\fbox{\textbf{master 80779}}}{}}%
%BeginExpansion
\msihyperref{\fbox{\textbf{master 80779}}}{}{\fbox{\textbf{master 80779}}}{}%
%EndExpansion

\item[\hfill 43.] $\dint\nolimits_{2}^{\infty }\,\,\dfrac{dx}{x\ln x}$

%TCIMACRO{\hyperref{ANSWER}{}{\textbf{ANSWER: }D}{}}%
%BeginExpansion
\msihyperref{ANSWER}{}{\textbf{ANSWER: }D}{}%
%EndExpansion

%TCIMACRO{%
%\hyperref{\fbox{\textbf{master 03954}}}{}{\fbox{\textbf{master 03954}}}{}}%
%BeginExpansion
\msihyperref{\fbox{\textbf{master 03954}}}{}{\fbox{\textbf{master 03954}}}{}%
%EndExpansion

\item[\hfill 44.] $\dint\nolimits_{2}^{6}\,\,\dfrac{y}{\sqrt{y-2}}\,dy$

%TCIMACRO{%
%\hyperref{\fbox{\textbf{master 03955}}}{}{\fbox{\textbf{master 03955}}}{}}%
%BeginExpansion
\msihyperref{\fbox{\textbf{master 03955}}}{}{\fbox{\textbf{master 03955}}}{}%
%EndExpansion

\item[\hfill 45.] $\dint\nolimits_{0}^{4}\,\,\dfrac{\ln x}{\sqrt{x}}\,dx$

%TCIMACRO{\hyperref{ANSWER}{}{\textbf{ANSWER:} $4\ln 4-8$}{}}%
%BeginExpansion
\msihyperref{ANSWER}{}{\textbf{ANSWER:} $4\ln 4-8$}{}%
%EndExpansion

%TCIMACRO{%
%\hyperref{\fbox{\textbf{master 03956}}}{}{\fbox{\textbf{master 03956}}}{}}%
%BeginExpansion
\msihyperref{\fbox{\textbf{master 03956}}}{}{\fbox{\textbf{master 03956}}}{}%
%EndExpansion

\item[\hfill 46.] $\dint\nolimits_{0}^{1}\,\,\dfrac{1}{2-3x}\,dx$

%TCIMACRO{%
%\hyperref{\fbox{\textbf{master 03957}}}{}{\fbox{\textbf{master 03957}}}{}}%
%BeginExpansion
\msihyperref{\fbox{\textbf{master 03957}}}{}{\fbox{\textbf{master 03957}}}{}%
%EndExpansion

\item[\hfill 47.] $\dint\nolimits_{0}^{1}\,\,\dfrac{x-1}{\sqrt{x}}~dx$

%TCIMACRO{\hyperref{ANSWER}{}{\textbf{ANSWER:} $-\frac{4}{3}$}{}}%
%BeginExpansion
\msihyperref{ANSWER}{}{\textbf{ANSWER:} $-\frac{4}{3}$}{}%
%EndExpansion

%TCIMACRO{%
%\hyperref{\fbox{\textbf{master 80780}}}{}{\fbox{\textbf{master 80780}}}{}}%
%BeginExpansion
\msihyperref{\fbox{\textbf{master 80780}}}{}{\fbox{\textbf{master 80780}}}{}%
%EndExpansion

\item[\hfill 48.] $\dint\nolimits_{-1}^{1}\,\,\dfrac{dx}{x^{2}-2x}$

%TCIMACRO{%
%\hyperref{\fbox{\textbf{master 80781}}}{}{\fbox{\textbf{master 80781}}}{}}%
%BeginExpansion
\msihyperref{\fbox{\textbf{master 80781}}}{}{\fbox{\textbf{master 80781}}}{}%
%EndExpansion

\item[\hfill 49.] $\dint\nolimits_{-\infty }^{\infty }\,\,\dfrac{dx}{%
4x^{2}+4x+5}$

%TCIMACRO{\hyperref{ANSWER}{}{\textbf{ANSWER:} $\pi /4$}{}}%
%BeginExpansion
\msihyperref{ANSWER}{}{\textbf{ANSWER:} $\pi /4$}{}%
%EndExpansion

%TCIMACRO{%
%\hyperref{\fbox{\textbf{master 03960}}}{}{\fbox{\textbf{master 03960}}}{}}%
%BeginExpansion
\msihyperref{\fbox{\textbf{master 03960}}}{}{\fbox{\textbf{master 03960}}}{}%
%EndExpansion

\item[\hfill 50.] $\dint\nolimits_{1}^{\infty }\,\,\dfrac{\tan ^{-1}x}{x^{2}}%
\,dx$

%TCIMACRO{%
%\hyperref{\fbox{\textbf{master 03961}}}{}{\fbox{\textbf{master 03961}}}{}}%
%BeginExpansion
\msihyperref{\fbox{\textbf{master 03961}}}{}{\fbox{\textbf{master 03961}}}{}%
%EndExpansion
\vspace*{-9pt}
\end{ExerciseList}

\begin{instructions}
\FRAME{itbpF}{237.375pt}{6.6875pt}{0pt}{}{}{dots.wmf}{\special{language
"Scientific Word";type "GRAPHIC";maintain-aspect-ratio TRUE;display
"USEDEF";valid_file "F";width 237.375pt;height 6.6875pt;depth
0pt;original-width 3.9167in;original-height 0.0735in;cropleft "0";croptop
"0.9532";cropright "0.8216";cropbottom "0.0468";filename
'graphics/dots.wmf';file-properties "XNPEU";}}\pagebreak
\end{instructions}

\begin{instructions}
%TCIMACRO{\TeXButton{GCALCI}{\GCALCI}}%
%BeginExpansion
\GCALCI%
%EndExpansion
\QTR{SpanExer}{51--52}{\small 
%TCIMACRO{\TeXButton{SQR}{\hskip .5em\rule{4pt}{4pt}\hskip .5em}}%
%BeginExpansion
\hskip .5em\rule{4pt}{4pt}\hskip .5em%
%EndExpansion
Evaluate the indefinite integral. Illustrate and check that your answer is
reasonable by graphing both the function and its antiderivative (take $C=0).$%
}
\end{instructions}

\begin{ExerciseList}
\item[\hfill 51.] $\dint \,\,\ln (x^{2}+2x+2)\,dx$

%TCIMACRO{%
%\hyperref{ANSWER}{}{\textbf{ANSWER:} \newline
%$\left( x+1\right) \ln \left( x^{2}+2x+2\right) +2\arctan \left( x+1\right) -2x+C$}{}}%
%BeginExpansion
\msihyperref{ANSWER}{}{\textbf{ANSWER:} \newline
$\left( x+1\right) \ln \left( x^{2}+2x+2\right) +2\arctan \left( x+1\right) -2x+C$}{}%
%EndExpansion

%TCIMACRO{%
%\hyperref{\fbox{\textbf{master 03962}}}{}{\fbox{\textbf{master 03962}}}{}}%
%BeginExpansion
\msihyperref{\fbox{\textbf{master 03962}}}{}{\fbox{\textbf{master 03962}}}{}%
%EndExpansion

\item[\hfill 52.] $\dint \,\,\dfrac{x^{3}}{\sqrt{x^{2}+1}}\,dx$

%TCIMACRO{%
%\hyperref{\fbox{\textbf{master 12157}} - same as 03963}{}{\fbox{\textbf{master 12157}} - same as 03963}{}}%
%BeginExpansion
\msihyperref{\fbox{\textbf{master 12157}} - same as 03963}{}{\fbox{\textbf{master 12157}} - same as 03963}{}%
%EndExpansion
\vspace*{-9pt}
\end{ExerciseList}

\begin{instructions}
\FRAME{itbpF}{237.375pt}{6.6875pt}{0pt}{}{}{dots.wmf}{\special{language
"Scientific Word";type "GRAPHIC";maintain-aspect-ratio TRUE;display
"USEDEF";valid_file "F";width 237.375pt;height 6.6875pt;depth
0pt;original-width 3.9167in;original-height 0.0735in;cropleft "0";croptop
"1";cropright "0.8216";cropbottom "0";filename
'graphics/dots.wmf';file-properties "XNPEU";}}
\end{instructions}

\begin{ExerciseList}
\item[\hfill 53.] 
%TCIMACRO{\TeXButton{GCALCX}{\GCALCX}}%
%BeginExpansion
\GCALCX%
%EndExpansion
Graph the function $f(x)=\cos ^{2}x\,\sin ^{3}x$ and use the graph to guess
the value of the integral $\int_{0}^{2\pi }f(x)\,dx$. Then evaluate the
integral to confirm your guess.

%TCIMACRO{\hyperref{ANSWER}{}{\textbf{ANSWER:} $0$}{}}%
%BeginExpansion
\msihyperref{ANSWER}{}{\textbf{ANSWER:} $0$}{}%
%EndExpansion

%TCIMACRO{%
%\hyperref{\fbox{\textbf{master 12158}} - same as 03964}{}{\fbox{\textbf{master 12158}} - same as 03964}{}}%
%BeginExpansion
\msihyperref{\fbox{\textbf{master 12158}} - same as 03964}{}{\fbox{\textbf{master 12158}} - same as 03964}{}%
%EndExpansion

\item[\hfill 54.] 

\begin{ExerciseList}
\item[(a)] 
%TCIMACRO{\TeXButton{CASXT}{\CASXT}}%
%BeginExpansion
\CASXT%
%EndExpansion
How would you evaluate $\int \,\,x^{5}e^{-2x}\,dx$ by hand? \newline
(Don't actually carry out the integration.)

%TCIMACRO{%
%\hyperref{\fbox{\textbf{master 03965}}}{}{\fbox{\textbf{master 03965}}}{}}%
%BeginExpansion
\msihyperref{\fbox{\textbf{master 03965}}}{}{\fbox{\textbf{master 03965}}}{}%
%EndExpansion

\item[(b)] How would you evaluate $\int \,\,x^{5}e^{-2x}\,dx$ using tables? 
\newline
(Don't actually do it.)

%TCIMACRO{%
%\hyperref{\fbox{\textbf{master 06154}}}{}{\fbox{\textbf{master 06154}}}{}}%
%BeginExpansion
\msihyperref{\fbox{\textbf{master 06154}}}{}{\fbox{\textbf{master 06154}}}{}%
%EndExpansion

\item[(c)] Use a CAS to evaluate $\int \,\,x^{5}e^{-2x}\,dx$.

%TCIMACRO{%
%\hyperref{\fbox{\textbf{master 06155}}}{}{\fbox{\textbf{master 06155}}}{}}%
%BeginExpansion
\msihyperref{\fbox{\textbf{master 06155}}}{}{\fbox{\textbf{master 06155}}}{}%
%EndExpansion

\item[(d)] Graph the integrand and the indefinite integral on the same
screen.

%TCIMACRO{%
%\hyperref{\fbox{\textbf{master 06156}}}{}{\fbox{\textbf{master 06156}}}{}}%
%BeginExpansion
\msihyperref{\fbox{\textbf{master 06156}}}{}{\fbox{\textbf{master 06156}}}{}%
%EndExpansion
\vspace*{-9pt}
\end{ExerciseList}
\end{ExerciseList}

\begin{instructions}
\QTR{SpanExer}{55--58}{\small 
%TCIMACRO{\TeXButton{SQR}{\hskip .5em\rule{4pt}{4pt}\hskip .5em}}%
%BeginExpansion
\hskip .5em\rule{4pt}{4pt}\hskip .5em%
%EndExpansion
Use the Table of Integrals on the Reference Pages to evaluate the integral.}
\end{instructions}

\begin{ExerciseList}
\item[\hfill 55.] $\dint \sqrt{4x^{2}-4x-3}\,dx$

%TCIMACRO{%
%\hyperref{ANSWER}{}{\textbf{ANSWER:} $\frac{1}{4}(2x-1)\sqrt{4x^{2}-4x-3}-\newline
%\ln \left\vert 2x-1+\sqrt{4x^{2}-4x-3}\right\vert +C$}{}}%
%BeginExpansion
\msihyperref{ANSWER}{}{\textbf{ANSWER:} $\frac{1}{4}(2x-1)\sqrt{4x^{2}-4x-3}-\newline
\ln \left\vert 2x-1+\sqrt{4x^{2}-4x-3}\right\vert +C$}{}%
%EndExpansion

%TCIMACRO{%
%\hyperref{\fbox{\textbf{master 80782}}}{}{\fbox{\textbf{master 80782}}}{}}%
%BeginExpansion
\msihyperref{\fbox{\textbf{master 80782}}}{}{\fbox{\textbf{master 80782}}}{}%
%EndExpansion

\item[\hfill 56.] $\dint \csc ^{5}t\,dt$

%TCIMACRO{%
%\hyperref{\fbox{\textbf{master 03967}}}{}{\fbox{\textbf{master 03967}}}{}}%
%BeginExpansion
\msihyperref{\fbox{\textbf{master 03967}}}{}{\fbox{\textbf{master 03967}}}{}%
%EndExpansion

\item[\hfill 57.] $\dint \cos x~\sqrt{4+\sin ^{2}x}\,dx$

%TCIMACRO{%
%\hyperref{ANSWER}{}{\textbf{ANSWER:} $\frac{1}{2}\sin x\sqrt{4+\sin ^{2}x}+2\ln \!\left( \sin x+\sqrt{4+\sin ^{2}x}\right) +C$}{}}%
%BeginExpansion
\msihyperref{ANSWER}{}{\textbf{ANSWER:} $\frac{1}{2}\sin x\sqrt{4+\sin ^{2}x}+2\ln \!\left( \sin x+\sqrt{4+\sin ^{2}x}\right) +C$}{}%
%EndExpansion

%TCIMACRO{%
%\hyperref{\fbox{\textbf{master 80783}}}{}{\fbox{\textbf{master 80783}}}{}}%
%BeginExpansion
\msihyperref{\fbox{\textbf{master 80783}}}{}{\fbox{\textbf{master 80783}}}{}%
%EndExpansion

\item[\hfill 58.] $\dint \dfrac{\cot \,x}{\sqrt{1+2\sin \,x}}\,dx$

%TCIMACRO{%
%\hyperref{\fbox{\textbf{master 03969}}}{}{\fbox{\textbf{master 03969}}}{}}%
%BeginExpansion
\msihyperref{\fbox{\textbf{master 03969}}}{}{\fbox{\textbf{master 03969}}}{}%
%EndExpansion
\vspace*{-9pt}
\end{ExerciseList}

\begin{instructions}
\FRAME{itbpF}{237.375pt}{6.6875pt}{0pt}{}{}{dots.wmf}{\special{language
"Scientific Word";type "GRAPHIC";maintain-aspect-ratio TRUE;display
"USEDEF";valid_file "F";width 237.375pt;height 6.6875pt;depth
0pt;original-width 3.9167in;original-height 0.0735in;cropleft "0";croptop
"0.9532";cropright "0.8216";cropbottom "0.0468";filename
'graphics/dots.wmf';file-properties "XNPEU";}}
\end{instructions}

\begin{ExerciseList}
\item[\hfill 59.] Verify Formula 33 in the Table of Integrals 
\QTR{PartLetter}{(a)} by differentiation and \QTR{PartLetter}{(b)} by using
a trigonometric substitution.

%TCIMACRO{\hyperref{ANSWER}{}{\textbf{ANSWER:} \textit{No answer}}{}}%
%BeginExpansion
\msihyperref{ANSWER}{}{\textbf{ANSWER:} \textit{No answer}}{}%
%EndExpansion

%TCIMACRO{\hyperref{(a) = }{}{(a) = }{}}%
%BeginExpansion
\msihyperref{(a) = }{}{(a) = }{}%
%EndExpansion
%TCIMACRO{%
%\hyperref{\fbox{\textbf{master 03970}}}{}{\fbox{\textbf{master 03970}}}{}}%
%BeginExpansion
\msihyperref{\fbox{\textbf{master 03970}}}{}{\fbox{\textbf{master 03970}}}{}%
%EndExpansion
\quad 
%TCIMACRO{\hyperref{(b) = }{}{(b) = }{}}%
%BeginExpansion
\msihyperref{(b) = }{}{(b) = }{}%
%EndExpansion
%TCIMACRO{%
%\hyperref{\fbox{\textbf{master 40582}}}{}{\fbox{\textbf{master 40582}}}{}}%
%BeginExpansion
\msihyperref{\fbox{\textbf{master 40582}}}{}{\fbox{\textbf{master 40582}}}{}%
%EndExpansion

\item[\hfill 60.] Verify Formula 62 in the Table of Integrals.

%TCIMACRO{%
%\hyperref{\fbox{\textbf{master 03971}}}{}{\fbox{\textbf{master 03971}}}{}}%
%BeginExpansion
\msihyperref{\fbox{\textbf{master 03971}}}{}{\fbox{\textbf{master 03971}}}{}%
%EndExpansion

\item[\hfill 61.] Is it possible to find a number $n$ such that $%
\int_{0}^{\infty }\,\,x^{n}\,dx$ is \newline
convergent?

%TCIMACRO{\hyperref{ANSWER}{}{\textbf{ANSWER:} No}{}}%
%BeginExpansion
\msihyperref{ANSWER}{}{\textbf{ANSWER:} No}{}%
%EndExpansion

%TCIMACRO{%
%\hyperref{\fbox{\textbf{master 03972}}}{}{\fbox{\textbf{master 03972}}}{}}%
%BeginExpansion
\msihyperref{\fbox{\textbf{master 03972}}}{}{\fbox{\textbf{master 03972}}}{}%
%EndExpansion

\item[\hfill 62.] For what values of $a$ is $\int_{0}^{\infty }e^{ax}\cos
x\,dx$ convergent? Evaluate the integral for those values of $a$.

%TCIMACRO{%
%\hyperref{\fbox{\textbf{master 03973}}}{}{\fbox{\textbf{master 03973}}}{}}%
%BeginExpansion
\msihyperref{\fbox{\textbf{master 03973}}}{}{\fbox{\textbf{master 03973}}}{}%
%EndExpansion
\vspace*{-6pt}
\end{ExerciseList}

\begin{instructions}
\QTR{SpanExer}{63--64}{\small 
%TCIMACRO{\TeXButton{SQR}{\hskip .5em\rule{4pt}{4pt}\hskip .5em}}%
%BeginExpansion
\hskip .5em\rule{4pt}{4pt}\hskip .5em%
%EndExpansion
Use }\QTR{PartLetter}{(a)}{\small \ the Trapezoidal Rule, }\QTR{PartLetter}{%
(b)}{\small \ the Midpoint Rule, and }\QTR{PartLetter}{(c)}{\small \
Simpson's Rule with $n=10$ to approximate the given integral. Round your
answers to six decimal places.}
\end{instructions}

\begin{ExerciseList}
\item[\hfill 63.] $\dint\nolimits_{2}^{4}\,\,\dfrac{1}{\ln x}\,dx$

%TCIMACRO{%
%\hyperref{ANSWER}{}{\textbf{ANSWER:} (a)$\QTR{sechead}{~1.925444}$\quad (b)$\QTR{sechead}{~1.920915}$\quad (c)$\QTR{sechead}{~1.922470}$}{}}%
%BeginExpansion
\msihyperref{ANSWER}{}{\textbf{ANSWER:} (a)$\QTR{sechead}{~1.925444}$\quad (b)$\QTR{sechead}{~1.920915}$\quad (c)$\QTR{sechead}{~1.922470}$}{}%
%EndExpansion

%TCIMACRO{%
%\hyperref{\fbox{\textbf{master 80784}}}{}{\fbox{\textbf{master 80784}}}{}}%
%BeginExpansion
\msihyperref{\fbox{\textbf{master 80784}}}{}{\fbox{\textbf{master 80784}}}{}%
%EndExpansion

\item[\hfill 64.] $\dint\nolimits_{1}^{4}\,\,\sqrt{x}\,\cos x\,dx$

%TCIMACRO{%
%\hyperref{\fbox{\textbf{master 80785}}}{}{\fbox{\textbf{master 80785}}}{}}%
%BeginExpansion
\msihyperref{\fbox{\textbf{master 80785}}}{}{\fbox{\textbf{master 80785}}}{}%
%EndExpansion
\vspace*{-9pt}
\end{ExerciseList}

\begin{instructions}
\FRAME{itbpF}{237.375pt}{6.6875pt}{0pt}{}{}{dots.wmf}{\special{language
"Scientific Word";type "GRAPHIC";maintain-aspect-ratio TRUE;display
"USEDEF";valid_file "F";width 237.375pt;height 6.6875pt;depth
0pt;original-width 3.9167in;original-height 0.0735in;cropleft "0";croptop
"0.9532";cropright "0.8216";cropbottom "0.0468";filename
'graphics/dots.wmf';file-properties "XNPEU";}}%
%TCIMACRO{\TeXButton{longpage}{\enlargethispage{12pt}}}%
%BeginExpansion
\enlargethispage{12pt}%
%EndExpansion
\end{instructions}

\begin{ExerciseList}
\item[\hfill 65.] Estimate the errors involved in Exercise 63, parts (a) and
(b). How large should $n$ be in each case to guarantee an error of less than 
$0.00001$?

%TCIMACRO{%
%\hyperref{ANSWER}{}{\textbf{ANSWER:} (a)$\QTR{sechead}{~0.01348}$, $n\geq 368$\quad (b)$\QTR{sechead}{~0.00674}$, $n\geq 260$}{}}%
%BeginExpansion
\msihyperref{ANSWER}{}{\textbf{ANSWER:} (a)$\QTR{sechead}{~0.01348}$, $n\geq 368$\quad (b)$\QTR{sechead}{~0.00674}$, $n\geq 260$}{}%
%EndExpansion

%TCIMACRO{%
%\hyperref{\fbox{\textbf{master 80786}}}{}{\fbox{\textbf{master 80786}}}{}}%
%BeginExpansion
\msihyperref{\fbox{\textbf{master 80786}}}{}{\fbox{\textbf{master 80786}}}{}%
%EndExpansion

\item[\hfill 66.] Use Simpson's Rule with $n=6$ to estimate the area under
the curve $y=e^{x}/x$ from $x=1$ to $x=4$.

%TCIMACRO{%
%\hyperref{\fbox{\textbf{master 03977}}}{}{\fbox{\textbf{master 03977}}}{}}%
%BeginExpansion
\msihyperref{\fbox{\textbf{master 03977}}}{}{\fbox{\textbf{master 03977}}}{}%
%EndExpansion

\item[\hfill 67.] The speedometer reading ($v$) on a car was observed at 
\newline
1-minute intervals and recorded in the chart. Use Simpson's Rule to estimate
the distance traveled by the car.\\[6pt]
$\hspace*{\fill}%
\begin{tabular}[t]{|c|c||c|c|}
\hline
\ $t$ (min)\  & \ $v$ (mi$/$h)\  & \ $t$ (min)\  & \ $v$ (mi$/$h) \\ \hline
0 & 40 & \hspace{3pt}6 & 56 \\ 
1 & 42 & \hspace{3pt}7 & 57 \\ 
2 & 45 & \hspace{3pt}8 & 57 \\ 
3 & 49 & \hspace{3pt}9 & 55 \\ 
4 & 52 & 10 & 56 \\ 
5 & 54 &  &  \\ \hline
\end{tabular}%
\ \hspace*{\fill}$

%TCIMACRO{\hyperref{ANSWER}{}{\textbf{ANSWER:} 8.6 mi}{}}%
%BeginExpansion
\msihyperref{ANSWER}{}{\textbf{ANSWER:} 8.6 mi}{}%
%EndExpansion

%TCIMACRO{%
%\hyperref{\fbox{\textbf{master 03978}}}{}{\fbox{\textbf{master 03978}}}{}}%
%BeginExpansion
\msihyperref{\fbox{\textbf{master 03978}}}{}{\fbox{\textbf{master 03978}}}{}%
%EndExpansion

\item[\hfill 68.] A population of honeybees increased at a rate of $r\left(
t\right) $ bees per week, where the graph of $r$ is as shown. Use Simpson's
Rule with six subintervals to estimate the increase in the bee population
during the first 24 weeks.\\[4pt]
\hspace*{\fill}\FRAME{itbpF}{2.7501in}{1.7807in}{0in}{}{}{4e08rx68.wmf}{%
\special{language "Scientific Word";type "GRAPHIC";maintain-aspect-ratio
TRUE;display "USEDEF";valid_file "F";width 2.7501in;height 1.7807in;depth
0in;original-width 2.7501in;original-height 1.7807in;cropleft "0";croptop
"1";cropright "1";cropbottom "0";filename
'graphics/4e08rx68.wmf';file-properties "XNPEU";}}\hspace*{\fill}\vspace*{%
-6pt}

%TCIMACRO{%
%\hyperref{\fbox{\textbf{master 03979}}}{}{\fbox{\textbf{master 03979}}}{}}%
%BeginExpansion
\msihyperref{\fbox{\textbf{master 03979}}}{}{\fbox{\textbf{master 03979}}}{}%
%EndExpansion

\item[\hfill 69.] 

\begin{ExerciseList}
\item[(a)] 
%TCIMACRO{\TeXButton{CASXT}{\CASXT}}%
%BeginExpansion
\CASXT%
%EndExpansion
If $f(x)=\sin (\sin \,x)$, use a graph to find an upper bound for $%
\left\vert f^{(4)}(x)\right\vert $.

%TCIMACRO{\hyperref{ANSWER}{}{\textbf{ANSWER:} (a) $3.8$}{}}%
%BeginExpansion
\msihyperref{ANSWER}{}{\textbf{ANSWER:} (a) $3.8$}{}%
%EndExpansion

%TCIMACRO{%
%\hyperref{\fbox{\textbf{master 03980}}}{}{\fbox{\textbf{master 03980}}}{}}%
%BeginExpansion
\msihyperref{\fbox{\textbf{master 03980}}}{}{\fbox{\textbf{master 03980}}}{}%
%EndExpansion

\item[(b)] Use Simpson's Rule with $n=10$ to approximate \newline
$\int_{0}^{\pi }f(x)\,dx$ and use part (a) to estimate the error.

%TCIMACRO{\hyperref{ANSWER}{}{\textbf{ANSWER:} (b) $1.7867$, $0.000646$}{}}%
%BeginExpansion
\msihyperref{ANSWER}{}{\textbf{ANSWER:} (b) $1.7867$, $0.000646$}{}%
%EndExpansion

%TCIMACRO{%
%\hyperref{\fbox{\textbf{master 06152}}}{}{\fbox{\textbf{master 06152}}}{}}%
%BeginExpansion
\msihyperref{\fbox{\textbf{master 06152}}}{}{\fbox{\textbf{master 06152}}}{}%
%EndExpansion

\item[(c)] How large should $n$ be to guarantee that the size of the error
in using $S_{n}$ is less than $0.00001$?

%TCIMACRO{\hyperref{ANSWER}{}{\textbf{ANSWER:} (c) $n\geq 30$}{}}%
%BeginExpansion
\msihyperref{ANSWER}{}{\textbf{ANSWER:} (c) $n\geq 30$}{}%
%EndExpansion

%TCIMACRO{%
%\hyperref{\fbox{\textbf{master 06153}}}{}{\fbox{\textbf{master 06153}}}{}}%
%BeginExpansion
\msihyperref{\fbox{\textbf{master 06153}}}{}{\fbox{\textbf{master 06153}}}{}%
%EndExpansion
\end{ExerciseList}

\item[\hfill 70.] Suppose you are asked to estimate the volume of a
football. You measure and find that a football is 28 cm long. You use a
piece of string and measure the circumference at its widest point to be $53$
cm. The circumference $7$ cm from each end is $45$ cm. Use Simpson's Rule to
make your estimate.\\[4pt]
\hspace*{\fill}\FRAME{itbpF}{1.5835in}{1.2375in}{0in}{}{}{4e08rx70.wmf}{%
\special{language "Scientific Word";type "GRAPHIC";maintain-aspect-ratio
TRUE;display "USEDEF";valid_file "F";width 1.5835in;height 1.2375in;depth
0in;original-width 1.5835in;original-height 1.2375in;cropleft "0";croptop
"1";cropright "1";cropbottom "0";filename
'graphics/4e08rx70.wmf';file-properties "XNPEU";}}\hspace*{\fill}

%TCIMACRO{%
%\hyperref{\fbox{\textbf{master 03981}}}{}{\fbox{\textbf{master 03981}}}{}}%
%BeginExpansion
\msihyperref{\fbox{\textbf{master 03981}}}{}{\fbox{\textbf{master 03981}}}{}%
%EndExpansion

\item[\hfill 71.] Use the Comparison Theorem to determine whether the
integral is convergent or divergent.

\begin{ExerciseList}
\item[(a)] $\dint\nolimits_{1}^{\infty }\,\,\dfrac{2+\sin x}{\sqrt{x}}\,dx$

\item[(b)] $\dint\nolimits_{1}^{\infty }\,\,\dfrac{1}{\sqrt{1+x^{4}}}\,dx$
\end{ExerciseList}

%TCIMACRO{\hyperref{\fbox{\textbf{NEW}}}{}{\fbox{\textbf{NEW}}}{}}%
%BeginExpansion
\msihyperref{\fbox{\textbf{NEW}}}{}{\fbox{\textbf{NEW}}}{}%
%EndExpansion

\item[\hfill 72.] Find the area of the region bounded by the hyperbola 
\newline
$y^{2}-x^{2}=1$ and the line $y=3$.

%TCIMACRO{%
%\hyperref{\fbox{\textbf{master 03983}}}{}{\fbox{\textbf{master 03983}}}{}}%
%BeginExpansion
\msihyperref{\fbox{\textbf{master 03983}}}{}{\fbox{\textbf{master 03983}}}{}%
%EndExpansion

\item[\hfill 73.] Find the area bounded by the curves $y=\cos \,x$ and $%
y=\cos ^{2}x$ between $x=0$ and $x=\pi $.

%TCIMACRO{\hyperref{ANSWER}{}{\textbf{ANSWER:} $2$}{}}%
%BeginExpansion
\msihyperref{ANSWER}{}{\textbf{ANSWER:} $2$}{}%
%EndExpansion

%TCIMACRO{%
%\hyperref{\fbox{\textbf{master 03984}}}{}{\fbox{\textbf{master 03984}}}{}}%
%BeginExpansion
\msihyperref{\fbox{\textbf{master 03984}}}{}{\fbox{\textbf{master 03984}}}{}%
%EndExpansion

\item[\hfill 74.] Find the area of the region bounded by the curves \newline
$y=1/(2+\sqrt{x})$, $y=1/(2-\sqrt{x})$, and $x=1$.

%TCIMACRO{%
%\hyperref{\fbox{\textbf{master 03985}}}{}{\fbox{\textbf{master 03985}}}{}}%
%BeginExpansion
\msihyperref{\fbox{\textbf{master 03985}}}{}{\fbox{\textbf{master 03985}}}{}%
%EndExpansion

\item[\hfill 75.] The region under the curve $y=\cos ^{2}x,0\leq x\leq \pi /2
$, is rotated about the $x$-axis. Find the volume of the resulting solid.

%TCIMACRO{\hyperref{ANSWER}{}{\textbf{ANSWER:} $\tfrac{3}{16}\pi ^{2}$}{}}%
%BeginExpansion
\msihyperref{ANSWER}{}{\textbf{ANSWER:} $\tfrac{3}{16}\pi ^{2}$}{}%
%EndExpansion

%TCIMACRO{%
%\hyperref{\fbox{\textbf{master 03986}}}{}{\fbox{\textbf{master 03986}}}{}}%
%BeginExpansion
\msihyperref{\fbox{\textbf{master 03986}}}{}{\fbox{\textbf{master 03986}}}{}%
%EndExpansion

\item[\hfill 76.] The region in Exercise 75 is rotated about the $y$-axis.
Find the volume of the resulting solid.

%TCIMACRO{%
%\hyperref{\fbox{\textbf{master 03987}}}{}{\fbox{\textbf{master 03987}}}{}}%
%BeginExpansion
\msihyperref{\fbox{\textbf{master 03987}}}{}{\fbox{\textbf{master 03987}}}{}%
%EndExpansion

\item[\hfill 77.] If $f^{\prime }$ is continuous on $[0,\infty )$ and $%
\lim_{x\rightarrow \infty }f(x)=0$, show that\\[4pt]
$\hspace*{\fill}\dint\nolimits_{0}^{\infty }\,\,f^{\prime }(x)\,dx=-f(0)%
\hspace*{\fill}$

%TCIMACRO{\hyperref{ANSWER}{}{\textbf{ANSWER: }\textit{No answer}}{}}%
%BeginExpansion
\msihyperref{ANSWER}{}{\textbf{ANSWER: }\textit{No answer}}{}%
%EndExpansion

%TCIMACRO{%
%\hyperref{\fbox{\textbf{master 03988}}}{}{\fbox{\textbf{master 03988}}}{}}%
%BeginExpansion
\msihyperref{\fbox{\textbf{master 03988}}}{}{\fbox{\textbf{master 03988}}}{}%
%EndExpansion

\item[\hfill 78.] We can extend our definition of average value of a
continuous function to an infinite interval by defining the average value of 
$f$ on the interval $\left[ a,\infty \right) $ to be \\[4pt]
$\hspace*{\fill}\underset{t\rightarrow \infty }{\lim }\,\dfrac{1}{t-a}%
\,\dint\nolimits_{a}^{t}\,\,f\left( x\right) \,dx$\vspace{6pt}$\hspace*{\fill%
}$

\begin{ExerciseList}
\item[(a)] Find the average value of $y=\tan ^{-1}x$ on the interval $\left[
0,\infty \right) $.

%TCIMACRO{%
%\hyperref{\fbox{\textbf{master 03989}}}{}{\fbox{\textbf{master 03989}}}{}}%
%BeginExpansion
\msihyperref{\fbox{\textbf{master 03989}}}{}{\fbox{\textbf{master 03989}}}{}%
%EndExpansion

\item[(b)] If $f\left( x\right) \geq 0$ and $\int_{a}^{\infty }\,f(x)\,dx$
is divergent, show that the average value of $f$ on the interval $\left[
a,\infty \right) $ is $\lim\nolimits_{x\rightarrow \infty }\,f\left(
x\right) $, if this limit exists.

%TCIMACRO{%
%\hyperref{\fbox{\textbf{master 40589}}}{}{\fbox{\textbf{master 40589}}}{}}%
%BeginExpansion
\msihyperref{\fbox{\textbf{master 40589}}}{}{\fbox{\textbf{master 40589}}}{}%
%EndExpansion

\item[(c)] If $\int_{a}^{\infty }\,f(x)\,dx$ is convergent, what is the
average value of $f$ on the interval $\left[ a,\infty \right) $?

%TCIMACRO{%
%\hyperref{\fbox{\textbf{master 40590}}}{}{\fbox{\textbf{master 40590}}}{}}%
%BeginExpansion
\msihyperref{\fbox{\textbf{master 40590}}}{}{\fbox{\textbf{master 40590}}}{}%
%EndExpansion

\item[(d)] Find the average value of $y=\sin x$ on the interval $\left[
0,\infty \right) $.

%TCIMACRO{%
%\hyperref{\fbox{\textbf{master 40591}}}{}{\fbox{\textbf{master 40591}}}{}}%
%BeginExpansion
\msihyperref{\fbox{\textbf{master 40591}}}{}{\fbox{\textbf{master 40591}}}{}%
%EndExpansion
\end{ExerciseList}

\item[\hfill 79.] Use the substitution $u=1/x$ to show that\\[4pt]
$\hspace*{\fill}\dint\nolimits_{0}^{\infty }\dfrac{\ln x}{1+x^{2}}\,dx=0%
\hspace*{\fill}$

%TCIMACRO{\hyperref{ANSWER}{}{\textbf{ANSWER: }\textit{No answer}}{}}%
%BeginExpansion
\msihyperref{ANSWER}{}{\textbf{ANSWER: }\textit{No answer}}{}%
%EndExpansion

%TCIMACRO{%
%\hyperref{\fbox{\textbf{master 03990}}}{}{\fbox{\textbf{master 03990}}}{}}%
%BeginExpansion
\msihyperref{\fbox{\textbf{master 03990}}}{}{\fbox{\textbf{master 03990}}}{}%
%EndExpansion

\item[\hfill 80.] The magnitude of the repulsive force between two point
charges with the same sign, one of size 1 and the other of size $q$, is\\[4pt%
]
$\hspace*{\fill}F=\dfrac{q}{4\pi \varepsilon _{0}r^{2}}\hspace*{\fill}$\\[4pt%
]
where $r$ is the distance between the charges and $\varepsilon _{0}$ is a
constant. The \textit{potential }$V$ at a point $P$ due to the charge $q$ is
defined to be the work expended in bringing a unit charge to $P$ from
infinity along the straight line that joins $q$ and $P$. Find a formula for $%
V$.

%TCIMACRO{%
%\hyperref{\fbox{\textbf{master 03991}}}{}{\fbox{\textbf{master 03991}}}{}}%
%BeginExpansion
\msihyperref{\fbox{\textbf{master 03991}}}{}{\fbox{\textbf{master 03991}}}{}%
%EndExpansion
\end{ExerciseList}

%TCIMACRO{%
%\TeXButton{e2col}{\end{multicols} \advance \leftskip by 165pt \advance\hsize by -165pt \advance\linewidth by -165pt }}%
%BeginExpansion
\end{multicols} \advance \leftskip by 165pt \advance\hsize by -165pt \advance\linewidth by -165pt %
%EndExpansion

\end{document}
