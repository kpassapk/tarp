
\documentclass{sebase}
%%%%%%%%%%%%%%%%%%%%%%%%%%%%%%%%%%%%%%%%%%%%%%%%%%%%%%%%%%%%%%%%%%%%%%%%%%%%%%%%%%%%%%%%%%%%%%%%%%%%%%%%%%%%%%%%%%%%%%%%%%%%%%%%%%%%%%%%%%%%%%%%%%%%%%%%%%%%%%%%%%%%%%%%%%%%%%%%%%%%%%%%%%%%%%%%%%%%%%%%%%%%%%%%%%%%%%%%%%%%%%%%%%%%%%%%%%%%%%%%%%%%%%%%%%%%
\usepackage{amssymb}
\usepackage{amsmath}
\usepackage{float}
\usepackage{makeidx}
\usepackage{SECALCUL}
\usepackage{lie}

\setcounter{MaxMatrixCols}{10}
%TCIDATA{OutputFilter=LATEX.DLL}
%TCIDATA{Version=5.50.0.2953}
%TCIDATA{<META NAME="SaveForMode" CONTENT="1">}
%TCIDATA{BibliographyScheme=Manual}
%TCIDATA{Created=Mon Dec 15 16:20:00 1997}
%TCIDATA{LastRevised=Tuesday, October 06, 2009 15:49:03}
%TCIDATA{<META NAME="ViewSettings" CONTENT="25">}
%TCIDATA{<META NAME="GraphicsSave" CONTENT="32">}
%TCIDATA{CSTFile=SECALCUL.cst}

\input tcilatex
\newenvironment{instructions}{\STARTINSTR}{\ENDINSTR}
\begin{document}


%TCIMACRO{%
%\TeXButton{Int Fix}{\def\dint{\displaystyle \int}\def\diint{\displaystyle \iint}\def\diiint{\displaystyle \iiint}
%\def\tint{\textstyle \int}\def\tiint{\textstyle \iint }\def\tiiint{{\textstyle \iiint }\def\tiiiint{\textstyle \iiiint }}}}%
%BeginExpansion
\def\dint{\displaystyle \int}\def\diint{\displaystyle \iint}\def\diiint{\displaystyle \iiint}
\def\tint{\textstyle \int}\def\tiint{\textstyle \iint }\def\tiiint{{\textstyle \iiint }\def\tiiiint{\textstyle \iiiint }}%
%EndExpansion
%TCIMACRO{\TeXButton{noCCC}{\noCCC}}%
%BeginExpansion
\noCCC%
%EndExpansion
%TCIMACRO{\TeXButton{setET}{\renewcommand{\ET}{1}}}%
%BeginExpansion
\renewcommand{\ET}{1}%
%EndExpansion
%TCIMACRO{\TeXButton{noRM}{\renewcommand{\RM}{0}}}%
%BeginExpansion
\renewcommand{\RM}{0}%
%EndExpansion
%TCIMACRO{\TeXButton{setRM}{\renewcommand{\RM}{1}}}%
%BeginExpansion
\renewcommand{\RM}{1}%
%EndExpansion
%TCIMACRO{%
%\TeXButton{set page 96 / 77}{\ifnum\RM=1 \setcounter{page}{96}\else \setcounter{page}{77}\fi}}%
%BeginExpansion
\ifnum\RM=1 \setcounter{page}{96}\else \setcounter{page}{77}\fi%
%EndExpansion
$\,\,$

\chapter[Problems Plus]{}

\section[Problems Plus]{Problems Plus\vspace*{-21pt}}

%TCIMACRO{\TeXButton{graphicSpp}{\vspace{12pt}\hskip-160pt\hfil}}%
%BeginExpansion
\vspace{12pt}\hskip-160pt\hfil%
%EndExpansion
\vspace{-6pt}$\rule{7in}{0.02in}$%
%TCIMACRO{\TeXButton{graphicE}{\vspace{12pt}\hfil}}%
%BeginExpansion
\vspace{12pt}\hfil%
%EndExpansion
\vspace*{-9pt}

\begin{Example}[1]
\thinspace 
\marginpar{
Cover up the solution to the example and try it yourself first.}

\begin{enumerate}
\item[(a)] 
%TCIMACRO{%
%\hyperref{\fbox{{\tiny EX 00437}}\quad }{}{\fbox{{\tiny EX 00437}}\quad }{}}%
%BeginExpansion
\msihyperref{\fbox{{\tiny EX 00437}}\quad }{}{\fbox{{\tiny EX 00437}}\quad }{}%
%EndExpansion
Prove that if $f$ is a continuous function, then 
\begin{equation*}
\dint\nolimits_{0}^{a}\,\,f(x)\,dx=\dint\nolimits_{0}^{a}\,\,f(a-x)\,dx
\end{equation*}

\item[(b)] 
%TCIMACRO{%
%\hyperref{\fbox{{\tiny EX 00438}}\quad }{}{\fbox{{\tiny EX 00438}}\quad }{}}%
%BeginExpansion
\msihyperref{\fbox{{\tiny EX 00438}}\quad }{}{\fbox{{\tiny EX 00438}}\quad }{}%
%EndExpansion
Use part (a) to show that 
\begin{equation*}
\dint\nolimits_{0}^{\pi /2}\frac{\sin ^{n}x}{\sin ^{n}x+\cos ^{n}x}\,dx=%
\frac{\pi }{4}
\end{equation*}%
for all positive numbers $n$.
\end{enumerate}
\end{Example}

\begin{Solution}
\thinspace

\begin{enumerate}
\item[(a)] 
\marginpar{\vspace{24pt}
\par
The principles of problem solving are discussed on page 76.}At first sight,
the given equation may appear somewhat baffling. How is it possible to
connect the left side to the right side? Connections can often be made
through one of the principles of problem solving: \textit{introduce
something extra.} Here the extra ingredient is a new variable. We often
think of introducing a new variable when we use the Substitution Rule to
integrate a specific function. But that technique is still useful in the
present circumstance in which we have a general function $f$.

\quad Once we think of making a substitution, the form of the right side
suggests that it should be $u=a-x$. Then $du=-dx$. When $x=0$, $u=a$; when $%
x=a$, $u=0$. So 
\begin{equation*}
\dint\nolimits_{0}^{a}\,\,f(a-x)\,dx=-\dint\nolimits_{a}^{0}\,\,f(u)\,du=%
\dint\nolimits_{0}^{a}\,\,f(u)\,du
\end{equation*}
But this integral on the right side is just another way of writing $%
\tint\nolimits_{0}^{a}\,\,f(x)\,dx$. So the given equation is proved.

\item[(b)] 
\marginpar{
The computer graphs in Figure 1\ make it seem plausible that all of the
integrals in the example have the same value. The graph of each integrand is
labeled with the corresponding value of $n$.\FRAME{dtbpFU}{1.6527in}{1.2384in%
}{0pt}{\Qcb{\label{F08pp01}\QTR{FigureNumber}{FIGURE 1}}}{}{4e08pp01.wmf}{%
\special{language "Scientific Word";type "GRAPHIC";maintain-aspect-ratio
TRUE;display "USEDEF";valid_file "F";width 1.6527in;height 1.2384in;depth
0pt;original-width 1.6527in;original-height 1.2384in;cropleft "0";croptop
"1";cropright "1";cropbottom "0";filename
'graphics/4e08pp01.wmf';file-properties "XNPEU";}}}If we let the given
integral be $I$ and apply part (a) with $a=\pi /2$, we get 
\begin{equation*}
I=\dint\nolimits_{0}^{\pi /2}\,\,\frac{\sin ^{n}x}{\sin ^{n}x+\cos ^{n}x}%
\,dx=\dint\nolimits_{0}^{\pi /2}\,\,\frac{\sin ^{n}(\pi /2-x)}{\sin ^{n}(\pi
/2-x)+\cos ^{n}(\pi /2-x)}\,dx
\end{equation*}
A well-known trigonometric identity tells us that $\sin (\pi /2-x)=\cos x$
and $\cos (\pi /2-x)=\sin x$, so we get 
\begin{equation*}
I=\dint\nolimits_{0}^{\pi /2}\,\,\frac{\cos ^{n}x}{\cos ^{n}x+\sin ^{n}x}\,dx
\end{equation*}
Notice that the two expressions for $I$ are very similar. In fact, the
integrands have the same denominator. This suggests that we should add the
two expressions. If we do so, we get 
\begin{equation*}
2I=\dint\nolimits_{0}^{\pi /2}\,\,\frac{\sin ^{n}x+\cos ^{n}x}{\sin
^{n}x+\cos ^{n}x}\,dx=\dint\nolimits_{0}^{\pi /2}\,\,1\,dx=\frac{\pi }{2}
\end{equation*}
Therefore, $I=\pi /4$.$\blacksquare $
\end{enumerate}
\end{Solution}

\subsubsection{Problems\protect\vspace*{-9pt}}

\begin{enumerate}
\item[1.] 
%TCIMACRO{\TeXButton{GCALCXTpp}{\GCALCXTpp}}%
%BeginExpansion
\GCALCXTpp%
%EndExpansion
\marginpar{\FRAME{dtbpFU}{1.3889in}{1.6094in}{0pt}{\Qcb{\label{F08pp02}%
\QTR{FigureNumber}{FIGURE FOR PROBLEM 1}}}{}{4e08ppx01.wmf}{\special%
{language "Scientific Word";type "GRAPHIC";maintain-aspect-ratio
TRUE;display "USEDEF";valid_file "F";width 1.3889in;height 1.6094in;depth
0pt;original-width 1.3889in;original-height 1.6094in;cropleft "0";croptop
"1";cropright "1";cropbottom "0";filename
'graphics/4e08ppx01.wmf';file-properties "XNPEU";}}}Three mathematics
students have ordered a 14-inch pizza. Instead of slicing it in the
traditional way, they decide to slice it by parallel cuts, as shown in the
figure. Being mathematics majors, they are able to determine where to slice
so that each gets the same amount of pizza. Where are the cuts made?

%TCIMACRO{%
%\hyperref{ANSWER}{}{\textbf{ANSWER:} About 1.85 inches from the center}{}}%
%BeginExpansion
\msihyperref{ANSWER}{}{\textbf{ANSWER:} About 1.85 inches from the center}{}%
%EndExpansion

%TCIMACRO{%
%\hyperref{\fbox{\textbf{master 04105}}}{}{\fbox{\textbf{master 04105}}}{}}%
%BeginExpansion
\msihyperref{\fbox{\textbf{master 04105}}}{}{\fbox{\textbf{master 04105}}}{}%
%EndExpansion

\item[2.] Evaluate \hspace*{\fill}$\dint \,\,\dfrac{1}{x^{7}-x}\,dx$\hspace*{%
\fill}\\[6pt]

The straightforward approach would be to start with partial fractions, but
that would be brutal. Try a substitution.

%TCIMACRO{%
%\hyperref{\fbox{\textbf{master 04106}}}{}{\fbox{\textbf{master 04106}}}{}}%
%BeginExpansion
\msihyperref{\fbox{\textbf{master 04106}}}{}{\fbox{\textbf{master 04106}}}{}%
%EndExpansion

\item[3.] Evaluate $\dint\nolimits_{0}^{1}\,\,\left( \sqrt[3]{1-x^{7}}-\sqrt[%
7]{1-x^{3}}\right) dx$.

%TCIMACRO{\hyperref{ANSWER}{}{\textbf{ANSWER:} 0}{}}%
%BeginExpansion
\msihyperref{ANSWER}{}{\textbf{ANSWER:} 0}{}%
%EndExpansion

%TCIMACRO{%
%\hyperref{\fbox{\textbf{master 04107}}}{}{\fbox{\textbf{master 04107}}}{}}%
%BeginExpansion
\msihyperref{\fbox{\textbf{master 04107}}}{}{\fbox{\textbf{master 04107}}}{}%
%EndExpansion

\item[4.] The centers of two disks with radius 1 are one unit apart. Find
the area of the union of the two disks.

%TCIMACRO{%
%\hyperref{\fbox{\textbf{master 80767}}}{}{\fbox{\textbf{master 80767}}}{}}%
%BeginExpansion
\msihyperref{\fbox{\textbf{master 80767}}}{}{\fbox{\textbf{master 80767}}}{}%
%EndExpansion

\item[5.] An ellipse is cut out of a circle with radius $a$. The major axis
of the ellipse coincides with a diameter of the circle and the minor axis
has length $2b$. Prove that the area of the remaining part of the circle is
the same as the area of an ellipse with semiaxes $a$ and $a-b$.

%TCIMACRO{\hyperref{ANSWER}{}{\textbf{ANSWER:} \textit{No text answer}}{}}%
%BeginExpansion
\msihyperref{ANSWER}{}{\textbf{ANSWER:} \textit{No text answer}}{}%
%EndExpansion

%TCIMACRO{%
%\hyperref{\fbox{\textbf{master 80768}}}{}{\fbox{\textbf{master 80768}}}{}}%
%BeginExpansion
\msihyperref{\fbox{\textbf{master 80768}}}{}{\fbox{\textbf{master 80768}}}{}%
%EndExpansion

\item[6.] 
\marginpar{\FRAME{dtbpFU}{2.0972in}{1.4468in}{0pt}{\Qcb{\QTR{FigureNumber}{%
FIGURE FOR PROBLEM 6}}}{}{4e08ppx04.wmf}{\special{language "Scientific
Word";type "GRAPHIC";maintain-aspect-ratio TRUE;display "USEDEF";valid_file
"F";width 2.0972in;height 1.4468in;depth 0pt;original-width
29.1779in;original-height 2.0496in;cropleft "0";croptop "1";cropright
"1";cropbottom "0";filename 'graphics/4e08ppx04.wmf';file-properties
"XNPEU";}}}A man initially standing at the point $O$ walks along a pier
pulling a rowboat by a rope of length $L$. The man keeps the rope straight
and taut. The path followed by the boat is a curve called a \textit{tractrix}
and it has the property that the rope is always tangent to the curve (see
the figure).

\begin{enumerate}
\item[(a)] Show that if the path followed by the boat is the graph of the
function $y=f(x)$, then \\[4pt]
\hspace*{\fill}$f^{\prime }(x)=\dfrac{dy}{dx}=\dfrac{-\sqrt{L^{2}-x^{2}}}{x}$%
\hspace*{\fill}

%TCIMACRO{%
%\hyperref{\fbox{\textbf{master 04108}}}{}{\fbox{\textbf{master 04108}}}{}}%
%BeginExpansion
\msihyperref{\fbox{\textbf{master 04108}}}{}{\fbox{\textbf{master 04108}}}{}%
%EndExpansion

\item[(b)] Determine the function $y=f(x)$.

%TCIMACRO{%
%\hyperref{\fbox{\textbf{master 06164}}}{}{\fbox{\textbf{master 06164}}}{}}%
%BeginExpansion
\msihyperref{\fbox{\textbf{master 06164}}}{}{\fbox{\textbf{master 06164}}}{}%
%EndExpansion
\end{enumerate}

\item[7.] A function $f$ is defined by \\[4pt]
\hspace*{\fill}$f(x)=\dint\nolimits_{0}^{\pi }\,\,\cos t\,\cos (x-t)\,dt$%
\qquad $0\leq x\leq 2\pi $\hspace*{\fill}\\[4pt]
Find the minimum value of $f$.

%TCIMACRO{%
%\hyperref{ANSWER}{}{\textbf{ANSWER:} $f\left( \pi \right) =-\pi /2$}{}}%
%BeginExpansion
\msihyperref{ANSWER}{}{\textbf{ANSWER:} $f\left( \pi \right) =-\pi /2$}{}%
%EndExpansion

%TCIMACRO{%
%\hyperref{\fbox{\textbf{master 04109}}}{}{\fbox{\textbf{master 04109}}}{}}%
%BeginExpansion
\msihyperref{\fbox{\textbf{master 04109}}}{}{\fbox{\textbf{master 04109}}}{}%
%EndExpansion

\item[8.] If $n$ is a positive integer, prove that $\dint\nolimits_{0}^{1}\,%
\,(\ln x)^{n}\,dx=(-1)^{n}n!$

%TCIMACRO{%
%\hyperref{\fbox{\textbf{master 04110}}}{}{\fbox{\textbf{master 04110}}}{}}%
%BeginExpansion
\msihyperref{\fbox{\textbf{master 04110}}}{}{\fbox{\textbf{master 04110}}}{}%
%EndExpansion

\item[9.] Show that $\dint\nolimits_{0}^{1}\,\,(1-x^{2})^{n}\,dx=\dfrac{%
2^{2n}(n!)^{2}}{(2n+1)!}$.\newline
\textit{Hint:} Start by showing that if $I_{n}$ denotes the integral, then $%
I_{k+1}=\dfrac{2k+2}{2k+3}\,I_{k}$

%TCIMACRO{\hyperref{ANSWER}{}{\textbf{ANSWER: }\textit{No text answer}}{}}%
%BeginExpansion
\msihyperref{ANSWER}{}{\textbf{ANSWER: }\textit{No text answer}}{}%
%EndExpansion

%TCIMACRO{%
%\hyperref{\fbox{\textbf{master 04111}}}{}{\fbox{\textbf{master 04111}}}{}}%
%BeginExpansion
\msihyperref{\fbox{\textbf{master 04111}}}{}{\fbox{\textbf{master 04111}}}{}%
%EndExpansion

\item[10.] 
%TCIMACRO{\TeXButton{GCALCXpp}{\GCALCXpp}}%
%BeginExpansion
\GCALCXpp%
%EndExpansion
Suppose that $f$ is a positive function such that $f^{\prime }$ is
continuous.

\begin{enumerate}
\item[(a)] How is the graph of $y=f(x)\sin nx$ related to the graph of $%
y=f(x)$? What happens as $n\rightarrow \infty $?

%TCIMACRO{%
%\hyperref{\fbox{\textbf{master 04112}}}{}{\fbox{\textbf{master 04112}}}{}}%
%BeginExpansion
\msihyperref{\fbox{\textbf{master 04112}}}{}{\fbox{\textbf{master 04112}}}{}%
%EndExpansion

\item[(b)] Make a guess as to the value of the limit $\lim\limits_{n%
\rightarrow \infty }\,\,\dint\nolimits_{0}^{1}\,\,f(x)\sin nx\,dx$ based on
graphs of the integrand.

%TCIMACRO{%
%\hyperref{\fbox{\textbf{master 40593}}}{}{\fbox{\textbf{master 40593}}}{}}%
%BeginExpansion
\msihyperref{\fbox{\textbf{master 40593}}}{}{\fbox{\textbf{master 40593}}}{}%
%EndExpansion

\item[(c)] Using integration by parts, confirm the guess that you made in
part (b). [Use the fact that, since $f^{\prime }$ is continuous, there is a
constant $M$ such that $\left\vert f^{\prime }(x)\right\vert \leq M$ for $%
0\leq x\leq 1$.]

%TCIMACRO{%
%\hyperref{\fbox{\textbf{master 40594}}}{}{\fbox{\textbf{master 40594}}}{}}%
%BeginExpansion
\msihyperref{\fbox{\textbf{master 40594}}}{}{\fbox{\textbf{master 40594}}}{}%
%EndExpansion
\end{enumerate}

\item[11.] If $0<a<b$, find $\lim\limits_{t\rightarrow 0}\,\,\left\{
\dint\nolimits_{0}^{1}\,\,[bx+a(1-x)]^{t}\,dx\right\} ^{1/t}$.

%TCIMACRO{%
%\hyperref{ANSWER}{}{\textbf{ANSWER:} $\left( b^{b}a^{-a}\right) ^{1/\left( b-a\right) }e^{-1}$}{}}%
%BeginExpansion
\msihyperref{ANSWER}{}{\textbf{ANSWER:} $\left( b^{b}a^{-a}\right) ^{1/\left( b-a\right) }e^{-1}$}{}%
%EndExpansion

%TCIMACRO{%
%\hyperref{\fbox{\textbf{master 04113}}}{}{\fbox{\textbf{master 04113}}}{}}%
%BeginExpansion
\msihyperref{\fbox{\textbf{master 04113}}}{}{\fbox{\textbf{master 04113}}}{}%
%EndExpansion

\item[12.] 
%TCIMACRO{\TeXButton{GCALCXpp}{\GCALCXpp}}%
%BeginExpansion
\GCALCXpp%
%EndExpansion
Graph $f(x)=\sin (e^{x})$ and use the graph to estimate the value of $t$
such that $\tint\nolimits_{t}^{t+1}\,\,f(x)\,dx$ is a maximum. Then find the
exact value of $t$ that maximizes this integral.

%TCIMACRO{%
%\hyperref{\fbox{\textbf{master 04114}}}{}{\fbox{\textbf{master 04114}}}{}}%
%BeginExpansion
\msihyperref{\fbox{\textbf{master 04114}}}{}{\fbox{\textbf{master 04114}}}{}%
%EndExpansion

\item[13.] Evaluate $\dint\nolimits_{-1}^{\infty }\,\,\left( \dfrac{x^{4}}{%
1+x^{6}}\right) ^{2}~dx$.

%TCIMACRO{%
%\hyperref{ANSWER}{}{\textbf{ANSWER:} $\frac{1}{8}\pi -\frac{1}{12}$}{}}%
%BeginExpansion
\msihyperref{ANSWER}{}{\textbf{ANSWER:} $\frac{1}{8}\pi -\frac{1}{12}$}{}%
%EndExpansion

%TCIMACRO{%
%\hyperref{\fbox{\textbf{master 82091}}}{}{\fbox{\textbf{master 82091}}}{}}%
%BeginExpansion
\msihyperref{\fbox{\textbf{master 82091}}}{}{\fbox{\textbf{master 82091}}}{}%
%EndExpansion

\item[14.] Evaluate $\dint ~\sqrt{\tan x}~dx$.

%TCIMACRO{%
%\hyperref{\fbox{\textbf{master 83341}}}{}{\fbox{\textbf{master 83341}}}{}}%
%BeginExpansion
\msihyperref{\fbox{\textbf{master 83341}}}{}{\fbox{\textbf{master 83341}}}{}%
%EndExpansion

\item[15.] 
\marginpar{\FRAME{dtbpFU}{1.7218in}{1.2912in}{0pt}{\Qcb{\QTR{FigureNumber}{%
FIGURE FOR PROBLEM 15}}}{}{5et07ppx11.wmf}{\special{language "Scientific
Word";type "GRAPHIC";maintain-aspect-ratio TRUE;display "USEDEF";valid_file
"F";width 1.7218in;height 1.2912in;depth 0pt;original-width
1.7218in;original-height 1.2912in;cropleft "0";croptop "1";cropright
"1";cropbottom "0";filename 'graphics/5ET07ppx11.wmf';file-properties
"XNPEU";}}}The circle with radius 1 shown in the figure touches the curve $%
y=\left\vert 2x\right\vert $ twice. Find the area of the region that lies
between the two curves.

%TCIMACRO{%
%\hyperref{ANSWER}{}{\textbf{ANSWER: }$2-\sin ^{-1}\left( 2/\sqrt{5}\right) $}{}}%
%BeginExpansion
\msihyperref{ANSWER}{}{\textbf{ANSWER: }$2-\sin ^{-1}\left( 2/\sqrt{5}\right) $}{}%
%EndExpansion

%TCIMACRO{%
%\hyperref{\fbox{\textbf{master 04115}}}{}{\fbox{\textbf{master 04115}}}{}}%
%BeginExpansion
\msihyperref{\fbox{\textbf{master 04115}}}{}{\fbox{\textbf{master 04115}}}{}%
%EndExpansion

\item[16.] A rocket is fired straight up, burning fuel at the constant rate
of $b$ kilograms per second. Let $v=v(t)$ be the velocity of the rocket at
time $t$ and suppose that the velocity $u$ of the exhaust gas is constant.
Let $M=M(t)$ be the mass of the rocket at time $t$ and note that $M$
decreases as the fuel burns. If we neglect air resistance, it follows from
Newton's Second Law that \\[4pt]
\hspace*{\fill}$F=M\,\,\dfrac{dv}{dt}-ub$\hspace*{\fill}\\[4pt]
where the force $F=-Mg$. Thus\\[4pt]
\QTR{BOXHEAD}{(1)}$\hspace*{\fill}M\,\,\dfrac{dv}{dt}-ub=-Mg\hspace*{\fill}$%
\\[4pt]
Let $M_{1}$ be the mass of the rocket without fuel, $M_{2}$ the initial mass
of the fuel, and $M_{0}=M_{1}+M_{2}$. Then, until the fuel runs out at time $%
t=M_{2}b$, the mass is $M=M_{0}-bt$.

\begin{enumerate}
\item[(a)] Substitute $M=M_{0}-bt$ into Equation 1 and solve the resulting
equation for $v$. Use the initial condition $v(0)=0$ to evaluate the
constant.

%TCIMACRO{%
%\hyperref{\fbox{\textbf{master 04116}}}{}{\fbox{\textbf{master 04116}}}{}}%
%BeginExpansion
\msihyperref{\fbox{\textbf{master 04116}}}{}{\fbox{\textbf{master 04116}}}{}%
%EndExpansion

\item[(b)] Determine the velocity of the rocket at time $t=M_{2}/b$. This is
called the \textit{burnout velocity}.

%TCIMACRO{%
%\hyperref{\fbox{\textbf{master 40595}}}{}{\fbox{\textbf{master 40595}}}{}}%
%BeginExpansion
\msihyperref{\fbox{\textbf{master 40595}}}{}{\fbox{\textbf{master 40595}}}{}%
%EndExpansion

\item[(c)] Determine the height of the rocket $y=y(t)$ at the burnout time.

%TCIMACRO{%
%\hyperref{\fbox{\textbf{master 40596}}}{}{\fbox{\textbf{master 40596}}}{}}%
%BeginExpansion
\msihyperref{\fbox{\textbf{master 40596}}}{}{\fbox{\textbf{master 40596}}}{}%
%EndExpansion

\item[(d)] Find the height of the rocket at any time $t$.

%TCIMACRO{%
%\hyperref{\fbox{\textbf{master 40597}}}{}{\fbox{\textbf{master 40597}}}{}}%
%BeginExpansion
\msihyperref{\fbox{\textbf{master 40597}}}{}{\fbox{\textbf{master 40597}}}{}%
%EndExpansion
\end{enumerate}
\end{enumerate}

\end{document}
